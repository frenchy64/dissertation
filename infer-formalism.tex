\Dchapter{Formalism}
\label{infer:sec:formalism}

We present \lambdatrack{}, an untyped $\lambda$-calculus
that describes the essense of our approach to automatic annotations.
Our model is split into two phases: the collection phase 
\collectOp{}
that runs an instrumented program and collects observations, and
an inference phase 
\inferanns{}
that derives type annotations from these observations
that can be used to automatically annotate the program.

We define the top-level driver function \annotateOp{} that connects
both pieces.
It says, given a program \e{}
and top-level variables $\ova{\x{}}$ to infer annotations for,
return an annotation environment \atenv{} with possible entries for
$\ova{\x{}}$ based on observations from evaluating
an instrumented \e{}.
%
%\begin{mathpar}
%\infer[]
%{ \collectnoalign{\e{}}{\ova{\x{}}}{\res{}}
%  \\
%  \inferannsnoalign{\res{}}{\atenv{}}
%}
%{ \annotatenoalign{\e{}}{\ova{\x{}}}{\atenv{}} }
%\end{mathpar}
\begin{mathpar}
  \begin{array}{lllll}
    \annotateOp{} : \e{}, {\ova{\x{}}} \rightarrow \atenv{}\\
    \annotateOp{} = \inferanns{} \circ \collectOp{}
  \end{array}
\end{mathpar}

To contextualize the presentation of these phases, we begin a running example:
inferring the type of a top-level function $f$, that takes a map and
returns its {\makekw{a}} entry, 
based on the following usage.
%
\begin{Verbatim}[commandchars=\\\{\}, codes={\catcode`$=3\catcode`^=7}]
define $f$ = \uabs{m}{\getexp{m}{\makekw{a}}}

\appexp{f}{\curlymap{\makekw{a} 42}}
=> 42
\end{Verbatim}
%
Plugging this example into our driver function
we get a candidate annotation for $f$:
$$
\annotatenoalign{\appexp{f}{\curlymap{\makekw{a}\ 42}}}{[f]}{\{\hastype{f}{[\{\makekw{a}\ \IntT{}\} \rightarrow \IntT{}]}\}}
$$

\Dsection{Collection phase}
\label{infer:sec:formal:collection-phase}

Now that we have a high-level picture of how these phases interact,
we describe the syntax and semantics of \lambdatrack{}, before
presenting the details of \collectOp{}.
%
\figref{infer:fig:syntax} presents the syntax of \lambdatrack{}.
Values \v{} consist of numbers \num{}, Clojure-style keywords {\kw{}},
closures {\closure{\uabs{\x{}}{\e{}}}{\openv{}}}, constants \const{},
and keyword keyed hash maps {\curlymapvaloverrightnoarrow{\kw{}}{\val{}}}.

Expressions \e{} consist of variables \x{}, values,
functions, hash-maps, and function applications.
The special form
\trackE{\e{}}{\inferpath{}}
observes {\e{}} as related to path {\inferpath{}}.
Paths \inferpath{} 
record the source of a runtime value with respect
to a sequence of path elements \inferpth{}, which always starts with
a variable \x{}, and are read left-to-right.
Other path elements are
a function domain \dompe{}, 
a function range \rngpe{},
and a map entry {\inferkeype{\ova{\kw{1}}}{\kw{2}}}
which represents the result of looking up key {\kw{2}}
in a map with keyset ${\ova{\kw{1}}}$.

Inference results \restwoarrow{\inferpath{}}{\t{}}
are pairs of paths {\inferpath{}} and types \t{}
that say the path \inferpath{} was observed to be 
type \t{}.
Types \t{} are numbers \IntT{}, function types \arrow{\t{}}{\t{}},
ad-hoc union types \Union{\t{}}{\t{}},
type aliases \alias{},
%top type \Top{},
and unknown type \UnknownT{} that represents
a temporary lack of knowledge during the inference process.
Heterogeneous keyword map types \HMappretty{\ova{\kw{}\ \t{}}}
for now represent a series of required keyword entries---we will extend
them to have optional entries in later phases.

\begin{figure}
  \ifdefined\PAPER
  \footnotesize
  \fi
  $$
  \begin{altgrammar}
    \val{} &::=& \num{}
       \alt {\kw{}}
       \alt {\closure{\uabs{\x{}}{\e{}}}{\openv{}}}
       \alt {\curlymap{\ova{\kw{}\ {\val{}}}}}
       \alt {\const{}}
       &\mbox{Values} \\
   \e{} &::=& \x{}
       \alt \val
       \alt \trackE{\e{}}{\inferpath{}}
       \alt {\uabs{\x{}}{\e{}}}
       \\
       &\alt& {\curlymapvaloverrightnoarrow{\e{}}{\e{}}}
       \alt {\appexp{\e{}}{\ova{\e{}}}}
       &\mbox{Expressions} \\
    \openv{} &::=& \{\ova{x \mapsto \val{}}\}
       &\mbox{Runtime environments} \\
   \inferpth{}
      &::=& \x{}
       \alt \dompe{}
       \alt \rngpe{}
       \alt {\inferkeype{\HMapreq{}}{\kw{}}}
       &\mbox{Path Elements} \\
   \inferpath{} &::=& \ova{\inferpth{}}
       &\mbox{Paths} \\
       \res{}
      &::=& \restwoarrow{\inferpath{}}{\tau{}}
      &\mbox{Inference results} \\
    \t{}, \s{}
      &::=& \IntT{}
       \alt \arrow{\t{}}{\t{}}
       %\alt \HMappretty{\ova{\kw{}\ \t{}}}
       \alt \HMaptwo{\HMapreq{}}{\HMapopt{}}
       \alt \Unionsplice{\ova{\t{}}}
       \\
       &\alt& \alias{} % type alias
       \alt \kw{}
       \alt \Keyword{}
       \alt \Top{}
       \alt \IPersistentMap{\t{}}{\t{}}
       \alt \UnknownT{}
      &\mbox{Types} \\
    \tenv{} &::=& \{\ova{\hastype{\x{}}{\t{}}}\}
      &\mbox{Type environments} \\
    \HMapreq{}, \HMapopt{}
      &::=& \{ \ova{\kw{}\ {\t{}}} \}
      &\mbox{HMap entries} \\
    \aenv{} &::=& \{\ova{\alias{} \mapsto \tau}\}
      &\mbox{Type alias environments} \\
    \atenv{} &::=& (\aenv{}, \tenv{})
      &\mbox{Annotation environments} \\
  \end{altgrammar}
  $$
\caption{Syntax of Terms, Types, Inference results, and Environments for \lambdatrack{}}
\label{infer:fig:syntax}
\end{figure}

The big-step operational semantics
{\opsemtrack{\openv{}}{\e{}}{\v{}}{\res{}}}
(\figref{infer:fig:semantics})
says under runtime environment \openv{}
expression \e{} evaluates to value \v{}
with inference results \res{}.
Most rules are standard, with extensions to correctly
propagate inference results \res{}.
B-Track is the only interesting rule, which instruments
its fully-evaluated argument with the \trackmetaOp{}
metafunction.

The metafunction \trackmeta{\v{}}{\inferpath{}}{\vp{}}{\res{}} (\figref{infer:fig:trackmeta})
says if value \v{} occurs at path {\inferpath{}}, then return a possibly-instrumented
\vp{} paired with inference results {\res{}} that can be immediately derived
from the knowledge that \v{} occurs at path {\inferpath{}}.
It has a case for every kind of value.
The first three cases records the number input as type {\IntT{}}.
The fourth case, for closures, returns a wrapped value
resembling higher-order function contracts~\cite{findler2002contracts},
but we track the domain and range rather than verify them.
The remaining rules case, for maps, recursively tracks each map value,
and returns a map with possibly wrapped values.
Immediately accessible inference results are combined
and returned.
A specific rule for the empty map is needed because we otherwise only rely on
recursive calls to \trackEOp{} to gather inference results---in the empty case,
we have no data to recur on.

\begin{figure*}
  \ifdefined\PAPER
  \footnotesize
  \fi
\begin{mathpar}
  \begin{array}{llll}
    \infer [B-Track]
    { \opsemtrack{\openv{}}{\e{}}{\v{}}{\res{}} \\\\
    \trackmeta{\v{}}{\inferpath{}}{\vp{}}{\resp{}}}
    { \opsemtrack{\openv{}}{\trackE{\e{}}{\inferpath{}}}{\vp{}}{\unionres{\res{}}{\resp{}}}
    }
  \end{array}

\infer [B-App]
{ \opsemtrack{\openv{}}{\e{1}}{\closure{\uabs{\x{}}{\e{}}}{\openvp{}}}{\res{1}} \\\\
  \opsemtrack{\openv{}}{\e{2}}{\v{}}{\res{2}} \\\\
  \opsemtrack{\extendopenv{\openvp{}}{\x{}}{\v{}}}{\e{}}{\vp{}}{\res{3}} \\
}
{ \opsemtrack{\openv{}}{\appexp{\e{1}}{\e{2}}}{\vp{}}{\bigunionres{\ova{\res{i}}}}
}

  \begin{array}{lll}
    \infer [B-Clos]
    {}
    { \opsemtrack{\openv{}}{\uabs{\x{}}{\e{}}}{\closure{\uabs{\x{}}{\e{}}}{\openv{}}}{\emptyres{}}}
    \\\\
    \infer [B-Val]
    {}
    { \opsemtrack{\openv{}}{\val{}}{\val{}}{\emptyres{}} }
    \ \ \ \ \ \ 

    \infer [B-Var]
    {}
    { \opsemtrack{\openv{}}{\xvar{}}{\inopenvnoeq{\openv{}}{\xvar{}}}{\emptyres{}}
    }
  \end{array}

\infer [B-Delta]
{ \opsemtrack{\openv{}}{\e{}}{\const{}}{\res{1}}
  \\
  \overrightarrow{\opsemtrack {\openv{}}{\ep{}}{\v{}}{\resp{}}}
  \\\\
  \inferconstantopsem{\const{}}{\ova{\v{}}}{\vp{}}{\res{2}}
}
{ \opsemtrack {\openv{}}
              {\appexp {\e{}} {\overrightarrow{\ep{}}}}
              {\vp{}}
              {\overrightarrow{\unionres{\res{}}{\resp{}}}}
       }

\end{mathpar}
\caption{Operational Semantics for \lambdatrack{}}
\label{infer:fig:semantics}
\end{figure*}

\begin{figure}
  \ifdefined\PAPER
  \footnotesize
  \fi
\begin{mathpar}
  \arraycolsep=1.4pt
  \begin{array}{lllll}
    %\trackmeta{\v{}}{\inferpath{}}{\v{}}{\res{}}\\\\

    \trackmetaalign{\num{}}{\inferpath{}}{\num{}}{\singletonres{\inferpath{}}{\IntT{}}}
    \\
    \trackmetaalign{\kw{}}{\inferpath{}}{\kw{}}
                   {\singletonres{\inferpath{}}
                                 {\Keyword{}}}
    \\
    \trackmetaalign{\const{}}{\inferpath{}}{\const{}}{\emptyres{}}
    \\
    \trackmetalhs{\closure{\uabs{\x{}}{\e{}}}{\openv{}}}
                 {\inferpath{}}
                 &=&
                 \trackmetarhs
                   {\closure{\ep{}}{\openv{}}}
                   {\emptyres{}}
                   % let's treat functions as opaque wrt arity as they are in clojure
                   %{\singletonres{\inferpath{}}
                   %              {\arrow{\UnknownT{}}{\UnknownT{}}}}
         \\
    &&
    \begin{array}{@{}llll}
      \text{where } \y{} \text{ is fresh},\\
       \text{ }    \begin{array}{@{}llll}
                      \ep{} =
                        \uabs{\y{}}{\trackE{&\appexp{(\uabs{\x{}}{\e{}})}{\trackE{\yvar{}}{\appendone{\inferpath{}}{\dompe{}}}}}
                                           {\\&\appendone{\inferpath{}}{\rngpe{}}}}
                   \end{array}
    \end{array}
    \\
    \trackmetaalign{\{\}}
                   {\inferpath{}}
                   {\{\}}
                   {\singletonres{\inferpath{}}
                                 {\HMappretty{}}}
    \\
    \trackmetaalign{\{\ova{\kw{1}\ {\kw{2}}}\ 
                      \ova{\kw{}\ {\v{}}}
                    \}}
                   {\inferpath{}}
                   {\{\ova{\kw{1}\ {\kw{2}}}\ 
                      \ova{\kw{}\ {\vp{}}}
                    \}}
                   {\bigunionres{\res{}}}
    \\
    &&
    \text{where } \ova{\trackmeta{\v{}}
                                            {\appendone{\inferpath{}}
                                                       {\inferkeype{\{\ova{\kw{1}\ {\kw{2}}}\ 
                                                                      \ova{\kw{}\ {\UnknownT{}}}
                                                                    \}}
                                                                   {\kw{}}}}
                                            {\vp{}}
                                            {\res{}}}
    \\
  \end{array}
  
  \arraycolsep=1.4pt
  \begin{array}{lllr}
    \inferconstantopsemalign{\assocliteral{}}{\curlymap{\ova{\kw{}\ \v{}}}, \kwp{}, \vp{}}{\updatemap{\curlymap{\ova{\kw{}\ \v{}}}}{\kwp{}}{\vp{}}}
                            {\emptyres{}}\\
    \inferconstantopsemalign{\getliteral{}}{\curlymap{\kw{}\ \v{}, \ova{\kwp{}\ \vp{}}}, \kw{}}{\v{}}
                            {\emptyres{}}\\
    \inferconstantopsemalign{\dissocliteral{}}{\curlymap{\kw{}\ \v{}, \ova{\kwp{}\ \vp{}}}, \kw{}}{\curlymap{\ova{\kwp{}\ \vp{}}}}
                            {\emptyres{}}\\
  \end{array}
\end{mathpar}
\caption{Definition of \trackmeta{\v{}}{\inferpath{}}{\v{}}{\res{}} and constants}
\label{infer:fig:trackmeta}
\end{figure}

Now we have sufficient pieces to describe the initial collection phase of our model.
Given an expression \e{} and variables ${\ova{\x{}}}$ to track,
\instrumentnoalign{\e{}}{\ova{\x{}}}{\ep{}}
returns an instrumented expression \ep{}
that tracked usages of $\ova{\x{}}$.
It is defined via capture-avoiding substitution:
$$
\instrumentnoalign{\e{}}{\ova{\x{}}}{\replacefor{\e{}}{\ova{\trackE{\x{}}{[\x{}]}}}{\ova{\x{}}}}
$$

Then, the overall collection phase 
\collectnoalign{\e{}}{\ova{\x{}}}{\res{}}
says, given an expression \e{}
and variables
$\ova{\x{}}$
to track,
returns inference results {\res{}}
that are the results of evaluating \e{}
with instrumented occurrences of $\ova{\x{}}$.
It is defined as:
%
$$
\collectnoalign{\e{}}{\ova{\x{}}}{\res{}}, \text{ where }
  \opsemtrack{}{\instrument{\e{}}{\ova{\x{}}}}{\v{}}{\res{}}
$$

For our running example
of collecting for the program \appexp{f}{\curlymap{\makekw{a}\ 42}},
we instrument the program by wrapping occurrences of $f$ with \trackEOp{}
with path $[f]$.
$$
\instrumentnoalign{\appexp{f}{\curlymap{\makekw{a}\ 42}}}{[f]}{\appexp{\trackE{f}{[f]}}{\curlymap{\makekw{a}\ 42}}}
$$

Then we evaluate the instrumented program and derive two inference results
(colored in red for readability):
$$
\opsemtrack{}{\appexp{\trackE{f}{[f]}}{\curlymap{\makekw{a}\ 42}}}{42}{\resflatcolor{\resentry{[f, \dompe{}, \inferkeypenokeyset{[\makekw{a}]}{\makekw{a}}]}{\IntT{}}, \resentry{[f, \rngpe{}]}{\IntT{}}}}
$$

Here is the full derivation:
\begin{Verbatim}[commandchars=\\\{\}, codes={\catcode`$=3\catcode`^=7}]
=> \appexp{\trackE{f}{[f]}}{\curlymap{\makekw{a}\ 42}}
=> \trackE{\getexp{\trackE{\curlymap{\makekw{a}\ 42}}{[f, \dompe{}]}}{\makekw{a}}}{[f, \rngpe{}]}
=> \trackE{\getexp{{\curlymap{\makekw{a}\ 42}} ; \resflatcolor{\resentry{[f, \dompe{}, \inferkeypenokeyset{[\makekw{a}]}{\makekw{a}}]}{\IntT{}}}}{\makekw{a}}}{[f, \rngpe{}]}
=> \trackE{42 ; \resflatcolor{\resentry{[f, \dompe{}, \inferkeypenokeyset{[\makekw{a}]}{\makekw{a}}]}{\IntT{}}}}{[f, \rngpe{}]}
=> $42 ; \resflatcolor{\resentry{[f, \dompe{}, \inferkeypenokeyset{[\makekw{a}]}{\makekw{a}}]}{\IntT{}}, \resentry{[f, \rngpe{}]}{\IntT{}}}$
\end{Verbatim}

Notice that intermediate values can have inference results (colored) attached to them with a semicolon,
and the final value has inference results about both $f$'s domain and range.

\Dsection{Inference phase}
\label{infer:sec:formal:inference-phase}

After the collection phase, we have a collection of inference results \res{}
which can be passed to the 
metafunction \inferanns{}(\res{}) = \atenv{} to produce an annotation environment:
\begin{mathpar}
  \begin{array}{lllll}
    \inferanns{} : \res{} \rightarrow \atenv{}\\
    \inferanns{} = \inferrecOp{} \circ \generatetenv{}\\\\
  \end{array}
\end{mathpar}

The first pass $\generatetenv{} (\res{}) = \tenv{}$ generates an initial type environment
from inference results \res{}.
%It is defined (in \figref{infer:fig:generatetenv})
%as a fold over \res{}, building a \tenv{} incrementally via the \inferupdateOp{}
%metafunction.
%
The second pass
$$\squashlocal{}(\tenv{}) = \atenvp{}$$
creates individual type aliases
for each HMap type in \tenv{} and then merges aliases that both occur inside the same
nested type into possibly recursive types. % (\figref{infer:fig:squashlocal}).
%
The third pass $\squashglobal{} (\atenv{}) = \atenvp{}$
merges type aliases in \atenv{} based on their similarity. % (\figref{infer:fig:squashglobal}).

\Dsubsection{Pass 1: Generating initial type environment}
\label{infer:sec:formal:inference-phase:genenv}

The first pass is given in \figref{infer:fig:generatetenv}.
The entry point \generatetenv{} folds over inference results
to create an initial type environment via \inferupdateOp{}.
This style is inspired by occurrence typing~\cite{TF10},
from which we also borrow the concepts of paths into types.

We process paths right-to-left in \inferupdateOp{}, building
up types from leaves to root, before joining the fully constructed type with the existing
type environment via \joinOp{}.
The first case handles the \keypeOp{} path element.
The extra map of type information preserves both keyset
information and any entries that might represent tags
(populated by the final case of \trackEOp{}, \figref{infer:fig:trackmeta}).
This information helps us avoid prematurely collapsing tagged maps,
by the side condition of the HMap \joinOp{} case.
The \joinHMapOp{} metafunction aggressively combines two HMaps---required
keys in both maps are joined and stay required, otherwise keys
become optional.

The second and third \inferupdateOp{} cases update the domain and range of a function type,
respectively.
The \joinOp{} case for function types joins covariantly on the domain to yield more useful
annotations. For example, if a function accepts \IntT{} and \Keyword{},
it will have type
\joinnoalign{\arrow{\IntT{}}{\UnknownT{}}}{\arrow{\Keyword{}}{\UnknownT{}}}
{\arrow{\Union{\IntT{}}{\Keyword{}}}{\UnknownT{}}}.

Returning to our running example, we now want to convert our inference results
$$
\res{} = \{\resentry{[f, \dompe{}, \inferkeypenokeyset{[\makekw{a}]}{\makekw{a}}]}{\IntT{}}, \resentry{[f, \rngpe{}]}{\IntT{}}\}.
$$
into a type environment. Via $\generatetenv{}(\res{})$, we start to trace
\inferupdate{\{\}}{[f, \dompe{}, \inferkeypenokeyset{[\makekw{a}]}{\makekw{a}}]}{\IntT{}}

%\begin{figure*}
%\begin{mathpar}
%  \begin{array}{lllll}
%    \inferanns{} : \res{} \rightarrow \atenv{}\\
%    \inferanns{} = \squashglobal{} \circ \squashlocal{} \circ \generatetenv{}\\
%  \end{array}
%\end{mathpar}
%\caption{Algorithm summary: Generate an initial type environment from inference results.
%Then generate an updated type environment paired with a type alias environment by 1: creating
%a recursive type if a HMap contains another HMap whose keysets have a non-empty intersection, 2:
%globally merging type aliases based on identical HMap keysets, 3: cleaning up redundant aliases.}
%\label{infer:fig:inferanns}
%\end{figure*}

\begin{figure*}
\begin{mathpar}
  \begin{array}{rllll}
    \joinOp{} : \t{}, \t{} \rightarrow \t{}
    \\
    \joinalign{\Unionsplice{\ova{\s{}}}}{\t{}}{\Unionsplice{\ova{\joinexpression{\s{}}{\t{}}}}}
    \\
    \joinalign{\t{}}{\Unionsplice{\ova{\s{}}}}{\Unionsplice{\ova{\joinexpression{\s{}}{\t{}}}}}
    \\
    \joinalign{\UnknownT{}}{\t{}}{\t{}}
    \\
    \joinalign{\t{}}{\UnknownT{}}{\t{}}
    \\
    \joinalign{\arrow{\t{1}}{\s{1}}}{\arrow{\t{2}}{\s{2}}}
              {\arrow{\joinexpression{\t{1}}{\t{2}}}
                     {\joinexpression{\s{1}}{\s{2}}}}
    \\
    \joinalign{\HMaptwo{\HMapreq{1}}{\HMapopt{1}}}
              {\HMaptwo{\HMapreq{2}}{\HMapopt{2}}}
              {\joinHMapexpression{\HMaptwo{\HMapreq{1}}{\HMapopt{1}}}
                                  {\HMaptwo{\HMapreq{2}}{\HMapopt{2}}}}
                                  \text{,}
                                  \\
    &&\ova{(\kw{}, \kw{i}) \in {\HMapreq{i}}} \Rightarrow \ova{\kw{i-1} = \kw{i}}
    \\
    \joinalign{\t{}}{\s{}}{\Union{\t{}}{\s{}}} \text{, otherwise}
  \end{array}
%
  \begin{array}{lllll}
    \joinHMapnoalign{\HMaptwo{\HMapreq{1}}{\HMapopt{1}}}{\HMaptwo{\HMapreq{2}}{\HMapopt{2}}}{\HMaptwo{\HMapreq{}}{\HMapopt{}}}
    \\
    \begin{array}{lllll}
      \text{where}
          &\mathsf{req}  = \bigcup \ova{\textsf{dom}({\HMapreq{i}})} \\
          &\mathsf{opt}  = \bigcup \ova{\textsf{dom}({\HMapopt{i}})} \\
          &\ova{\kw{}^r} = \bigcap \ova{\textsf{dom}({\HMapreq{i}})} \setminus \mathsf{opt}\\
          &\ova{\kw{}^o} = \mathsf{opt} \cup (\mathsf{req} \setminus \ova{\kw{}^r})\\
          &\HMapreq{}    = \{\ova{\kw{}^r\ \joinstarexpression{\ova{\HMapreq{i}[\kw{}^r]}}} \} \\
          &\HMapopt{}    = \{\ova{\kw{}^o\ \joinstarexpression{\ova{\HMapreq{i}[\kw{}^o], \HMapopt{i}[\kw{}^o]}}} \}
    \end{array}
  \end{array}

  \begin{array}{lllll}
    \textsf{fold} : \forall \alpha, \beta. (\alpha, \beta \rightarrow \alpha), \alpha, \ova{\beta} \rightarrow \alpha\\
    \textsf{fold}(\textsf{f}, \textsf{a}_0, \ova{\textsf{b}}^n) = \textsf{a}_n\\
    \begin{array}{llll}
      \text{where } \ova{\textsf{a}_i = \textsf{f}(\textsf{a}_{i-1}, \textsf{b}_{i})}^{1 \leq i \leq n}\\
    \end{array}
    \\\\
    \generatetenv{} : \res{} \rightarrow \tenv{}\\
    \generatetenv{} (\res{}) = \textsf{fold}(\inferupdateOp{}, \{\}, \res{})\\
  \end{array}
  \begin{array}{lllll}
    \inferupdateOp : \tenv{}, \resentry{\inferpath{}}{\t{}} \rightarrow  \tenv{} 
    \\
    \inferupdatealign{\tenv{}}{\appendone{\inferpath{}}{\inferkeype{\{\ova{\kwp{}\ \s{}} \}}{\kw{}}}}{\t{}}
            {\inferupdate{\tenv{}}{\inferpath{}}{\{\ova{\kwp{}\ \s{}}\ \kw{}\ \t{} \}}}
    \\
    \inferupdatealign{\tenv{}}{\appendone{\inferpath{}}{\dompe{}}}{\t{}}
                     {\inferupdate{\tenv{}}{\inferpath{}}{\arrow{\t{}}{\UnknownT{}}}}
    \\
    \inferupdatealign{\tenv{}}{\appendone{\inferpath{}}{\rngpe{}}}{\t{}}
                {\inferupdate{\tenv{}}{\inferpath{}}{\arrow{\UnknownT{}}{\t{}}}}
    \\
    \inferupdatealign{\updatemap{\tenv{}}{\x{}}{\s{}}}{[x]}{\t{}}
                     {\updatemap{\tenv{}}
                                {\x{}}
                                {\joinexpression{\t{}}{\s{}}}
                                 }
    \\
    \inferupdatealign{\tenv{}}{[\xvar{}]}{\t{}}{\updatemap{\tenv{}}{\x{}}{\t{}}}
    \\
  \end{array}

\end{mathpar}
\caption{Definition of $\generatetenv{}(\res{}) = \tenv{}$}
\label{infer:fig:generatetenv}
\end{figure*}

\Dsubsection{Pass 2: Squash locally}
\label{infer:sec:formal:inference-phase:squash-local}

We now describe the algorithm for generating recursive type aliases.
The first step \squashlocal{} creates recursive types from directly nested types.
It folds over each type in the type environment, first
creating aliases with \aliashmap{}, and then
attempting to merge these aliases by \squashall{}.

A type is aliased by \aliashmap{} either if it is a union containing a HMap,
or a HMap that is not a member of a union.
While we will use the structure of HMaps to determine when to create a recursive
type, keeping surrounding type information close to HMaps helps create more
compact and readable recursive types.
The implementation uses a post-order traversal via \textsf{postwalk},
which also threads an annotation environment as it applies
the provided function.

Then, \squashall{} follows each alias \alias{i} reachable from the type environment
and attempts to merge it with any alias reachable from \alias{i}.
The \squash{} function maintains 
a set of already visited aliases to avoid infinite loops.

The logic for merging aliases is contained in \mergealiases{}.
Merging \alias{2} into \alias{1} involves mapping \alias{2}
to \alias{1} and \alias{1} to the join of both definitions.
Crucially, before joining, we rename occurrences of 
\alias{2} to \alias{1}. This avoids a linear increase in the
width of union types, proportional to the number of merged aliases.
%For example,
%without this, if we further merge \alias{3} into \alias{1},
%like \Unionsplice{{\alias{1}}\ {\alias{2}}\ {\alias{3}}} occur
%instead of simply \alias{1}.
The running time of our algorithm is proportional to the
width of union types (due to the quadratic combination of
unions in the join function) and this optimization greatly
helped the running time of several benchmarks.
To avoid introducing infinite types,
top-level references to other aliases we are merging with
are erased with the helper \textsf{f}.

The \shouldmergeOp{} function determines whether two types are related enough
to warrant being merged. We present our current implementation, which is simplistic,
but is fast and effective in practice, but many variations are possible.
Aliases are merged if they are all HMaps (not contained in unions), that
contain a keyword key in common, with possibly disjoint mapped values.
For example, our opening example has the \clj{:op} key mapped to either
\clj{:leaf} or \clj{:node}, and so aliases for each map would be merged.
Notice again, however, the join operator does not collapse differently-tagged
maps, so they will occur recursively in the resulting alias, but separated by union.

Even though this implementation of \shouldmergeOp{} does not directly utilize the aliased
union types carefully created by \aliashmap{}, they still affect the final types.
For example, squashing \clj{T} in
\begin{cljlisting}
(defalias T 
  (U nil
     '{:op :node :left '{:op :leaf ...} ...}))
\end{cljlisting}
results in 
\begin{cljlisting}
(defalias T 
  (U nil
     '{:op :node :left T ...}
     '{:op :leaf ...}))
\end{cljlisting}
rather than
\begin{cljlisting}
(defalias T2 (U '{:op :node :left T ...}
                '{:op :leaf ...}))
(defalias T (U nil T2))
\end{cljlisting}
An alternative implementation of \shouldmergeOp{} we experimented with included computing sets of keysets
for each alias, and merging if the keysets overlapped. This, and many of our early experimentations,
required expensive computations of keyset combinations and traversals over them that could be emulated
with cruder heuristics like the current implementation.

\begin{figure*}
\begin{mathpar}
  \begin{array}{lllll}
    \aliashmap{} : \atenv{}, \t{} \rightarrow (\atenv{}, \t{})\\
    \aliashmap{}(\atenv{}, \t{}) = \textsf{postwalk}(\atenv{}, \t{}, \textsf{f})\\
    \begin{array}{lllll}
      \text{where} %&\textsf{f} : \atenv{}, \t{} \rightarrow (\atenv{}, \t{})\\
                   &\textsf{f}(\atenv{}, {\HMaptwo{\HMapreq{1}}{\HMapreq{2}}}) = \register{}(\atenv{},\HMaptwo{\HMapreq{1}}{\HMapreq{2}})\\
                   &\textsf{f}(\atenv{}, \Unionsplice{\ova{\t{}}}) = \register{}(\atenv{},\Unionsplice{\ova{\fullyresolve{}(\t{})}}) \text{,}\\
                   %&\text{ if } \exists\t{}(\ova{\s{}}^m) \in\ova{\t{}} .\ m > 0\\
                   &\text{ if } \alias{} \in\ova{\t{}}\\
                   &\textsf{f}(\atenv{}, \t{}) = (\atenv{}, \t{}), \text{otherwise}
    \end{array}
  \end{array}
  \begin{array}{lllll}
    \register{} : \atenv{}, \t{} \rightarrow (\atenv{}, \t{})\\
    \register{}(\atenv{}, \t{}) = (\updatemap{\atenv{}}{\alias{}}{\t{}}, \alias{}), \text{ where } \alias{} \text{ is fresh}
    \\\\
    \fullyresolve{} : \atenv{}, \t{} \rightarrow \t{}\\
    \fullyresolve{}(\atenv{}, \alias{}) = \fullyresolve{}(\atenv{}[\alias{}])\\
    \fullyresolve{}(\atenv{}, \t{}) = \t{}  \text{, otherwise}
  \end{array}

  \begin{array}{lllll}
    \aliasesin{} : \t{} \rightarrow \ova{\alias{}}\\
    \aliasesin{}(\alias{}) = [\alias{}]\\
    \aliasesin{}(\t{}(\ova{\s{}})) = \bigcup{\ova{\aliasesin{}(\s{})}}
    \\\\
    \textsf{postwalk} : \atenv{}, \t{}, (\atenv{}, \t{} \rightarrow (\atenv{}, \t{})) \rightarrow (\atenv{}, \t{})\\
    \textsf{postwalk}(\atenv{0}, \t{}(\ova{\s{}}^n), \textsf{w}) = \textsf{w}(\atenv{n}, \t{}(\ova{\sp{}}))\\
    \begin{array}{lllll}
      \text{where}
        &\ova{(\atenv{i}, \sp{i}) = \textsf{postwalk}(\atenv{i-1}, \s{i}, \textsf{w})}\\
    \end{array}
    \\\\
    \mergealiases{} : \atenv{}, \ova{\alias{}} \rightarrow \atenv{}\\
    \mergealiases{}(\atenv{}, []) = \atenv{}\\
    \mergealiases{}(\atenv{}, [\alias{1} ... \alias{n}]) =
        \updatemap{\updatemapmulti{\atenv{}}{\alias{i}}{\alias{1}}}{\alias{1}}{\s{}}\\\
    \begin{array}{llll}
      \text{where } &\s{} = \joinstarexpression{\ova{\replacefor{\textsf{f}(\fullyresolve{}(\atenv{},\alias{i}))}{\alias{1}}{\alias{i}}}}
                    \\
                    &\textsf{f}(\aliasp{}) = \EmptyUnion{}, \text{ if } \aliasp{} \in \ova{\alias{}}\\
                    &\textsf{f}(\Unionsplice{\ova{\t{}}}) = \Unionsplice{\ova{\textsf{f}(\t{})}}\\
                    &\textsf{f}(\t{}) = \t{} \text{, otherwise}
    \end{array}
    \\\\
    %\trymergealias{} : \atenv{}, \alias{}, \alias{} \rightarrow \atenv{}
    %\\
    %\trymergealias{}(\atenv{}, \alias{1}, \alias{2}) =\\
    %\begin{array}{llll}
    %  \textsf{if } \neg \shouldmergeOp{}(\ova{\fullyresolve{}(\atenv{}, \alias{})}) \textsf{ then } \atenv{}\\
    %  \textsf{else } \updatemap{\updatemap{\atenv{}}{\alias{2}}{\alias{1}}}
    %                           {\alias{1}}
    %                           % this fixes the "too much garbage" issue
    %                           {\joinstarexpression{\ova{\replacefor{\atenv{}[\alias{i}]}{\alias{1}}{\alias{2}}}}}\\
    %\end{array}\\
      %fold-based version
    \squashlocal{} : \tenv{} \rightarrow \atenv{}\\
    \squashlocal{}(\tenv{}) = \textbf{fold}(\steptwohelper{}, \emptyatenv{}, \tenv{})\\
    \begin{array}{lllll}
      \text{where} &\steptwohelper{} (\atenv{}, \hastype{\x{}}{\t{}}) = \updatemap{\atenv{2}}{\x{}}{\t{2}}\\
                   &\begin{array}{lllll}
                      \text{where}
                        &(\atenv{1}, \t{1}) = \aliashmap{}(\atenv{}, \t{})\\
                        &(\atenv{2}, \t{2}) = \squashall{}(\atenv{1}, \t{1})\\
                    \end{array}
    \end{array}
      %nonfold-based version
    %\squashlocal{} : \atenv{} \rightarrow \atenv{}\\
    %\squashlocal{}(\atenv{0}) = \atenv{n}\\
    %\begin{array}{lllll}
    %  \text{where}\\
    %  \begin{array}{lllll}
    %    \steptwohelper{} (\atenv{}, \x{}, \t{}) = \replacefor{\atenv{2}}{\t{2}}{\x{}}\\
    %    \begin{array}{lllll}
    %      \text{where}
    %      &(\atenv{1}, \t{1}) = \aliashmap{}(\atenv{}, \t{})\\
    %      &(\atenv{2}, \t{2}) = \squashall{}(\atenv{1}, \t{1})\\
    %    \end{array}
    %    \\
    %    \ova{\hastype{\x{}}{\t{}}}^n = \atenv{0}[\tenv{}]\\
    %    \ova{\atenv{i} = \steptwohelper{}(\atenv{i-1},{\x{i}},{\t{i}})} \\
    %  \end{array}
    %\end{array}
  \end{array}
  \begin{array}{llll}
    \squashall{} : \atenv{}, \t{} \rightarrow \atenv{}\\
    \squashall{}(\atenv{0}, \t{}) = \atenv{n} \\
    \begin{array}{llll}
      \text{where }
      &\ova{\alias{}}^n = \aliasesin{}(\t{})\\
      &\ova{\atenv{i} = \squash{}(\atenv{i-1}, [\alias{i}], [])}\\
    \end{array}
    \\\\
    \squash : \atenv{}, \ova{\alias{}}, \ova{\alias{}} \rightarrow \atenv{}\\
    \squash(\atenv{}, [], \textsf{d}) = \atenv{}\\
    \squash(\atenv{}, \alias{1} :: \textsf{w}, \textsf{d}) = 
      \squash(\atenvp{}, \textsf{w} \cup \textsf{as}, \textsf{d} \cup \{\alias{1}\})\\
    \begin{array}{lllll}
      \begin{array}{@{}llll}
        \text{where }
        &\textsf{as} = \aliasesin{}(\atenv{}[\alias{1}]) \setminus \textsf{d}\\
        &\textsf{ap} = \textsf{d} \setminus \{\alias{1}\}\\
        &\begin{array}{@{}llll}
           \textsf{f}(\atenv{}, \alias{2}) = 
             &\textsf{if } \neg \shouldmergeOp{}(\ova{\fullyresolve{}(\atenv{}, \alias{})}),\\
             &\textsf{then } \atenv{}\\
             &\textsf{else } \mergealiases{}(\atenv{}, \ova{\alias{i}})\\
        \end{array}\\
        &\begin{array}{@{}llll}
          \atenvp{} = &\textsf{if } \alias{} \in \textsf{d} \textsf{, then } \atenv{} \text{,}\\
          &\textsf{else } \textsf{fold}(\textsf{f}, \atenv{}, \textsf{ap} \cup \textsf{as})
        \end{array}
      \end{array}
    \end{array}
    \\\\
    \shouldmergeOp{} : \ova{\t{}} \rightarrow \textbf{Bool}\\
    \shouldmergeOp{}(\ova{\HMaptwo{\HMapreq{i}}{\HMapopt{i}}}) = \exists\kw{}. \ova{(\kw{}, \kw{i}) \in \HMapreq{i}}\\
    \shouldmergeOp{}(\ova{\t{}}) = \textbf{F}, \text{ otherwise}
  \end{array}

\end{mathpar}
\caption{Definition of $\squashlocal{}(\tenv{}) = \atenv{}$
%\aliasesin{}(\t{}) returns the set of aliases that syntactically occur in \t{}.
%  Step 2 summary: Create aliases for HMaps (graph nodes), then squash recursive types locally
%(don't try to merge data examples from different paths). Omitted: follow-aliases call, that erases
%redundant aliases.
  }
  \label{infer:fig:squashlocal}
\end{figure*}

\Dsubsection{Pass 3: Squash globally}
\label{infer:sec:formal:inference-phase:squash-global}

The final step combines aliases
without restriction on whether they occur ``together''.
This step combines type information between different positions
(such as in different arguments or functions) so that any deficiencies
in unit testing coverage are massaged away.

The \squashglobal{} function is the entry point in this pass,
and is similar in structure to the previous pass.
It first creates aliases for each HMap via \aliassinglehmap{}.
Then, HMap aliases are grouped and merged in \squashhorizonally{}.

The \aliassinglehmap{} function first traverses the type environment
and creates alises for any HMaps via \singlehmap{}, and binds the resulting
envionment as \atenvp{}.
Then, the we update alias environment entries with \textsf{f}, whose first
case prevents re-aliasing a top-level HMap before we call \singlehmap{}
(\singlehmap{}'s second argument works with both \x{} and \alias{}).
The \t{}(\ova{\s{}}) syntax represents a type \t{} whose constructor
takes types \ova{\s{}}.

After that, \squashhorizonally{} creates groups of related
aliases with \textsf{groupSimilarReq}.
Each group contains HMap aliases whose required keysets are similar,
but are never differently-tagged.
The code creates a map \textsf{r} from keysets to groups of HMap
aliases with that (required) keyset.
Then, for every keyset \ova{\kw{}}, \textsf{similarReq} adds aliases to the group
whose keysets are a subset of \ova{\kw{}}. The number of missing
keys permitted is determined by \textsf{thres}, for which we do not provide a
definition.
Finally, \textsf{remDiffTag} removes differently-tagged HMaps from each group,
and the groups are merged via \mergealiases{} as before.

\Dsubsection{Implementation}

Further passes are used in the implementation.
In particular, we trim unreachable aliases and remove aliases
that simply point to another alias (like \alias{2} in \mergealiases{})
between each pass.

\begin{figure*}
\begin{mathpar}
  \begin{array}{lllll}
    \textsf{req}: \atenv{}, \alias{} \rightarrow \HMapreq{}\\
    \textsf{req}(\atenv{}, \alias{}) = \textsf{req}(\atenv{}, \alias{})\\
    \textsf{req}(\atenv{}, \HMaptwo{\HMapreq{}}{\HMapopt{}}) = \HMapreq{}\\
    \\\\
    %\textsf{req}: \atenv{}, \alias{} \rightarrow \ova{\HMapreq{}}\\
    %\textsf{req}(\atenv{}, \alias{}) = \textsf{req}(\atenv{}, \alias{})\\
    %\textsf{req}(\atenv{}, \HMaptwo{\HMapreq{}}{\HMapopt{}}) = [\HMapreq{}]\\
    %\textsf{req}(\atenv{}, \Unionsplice{\ova{\t{}}}) = \bigcup\ova{\textsf{req}(\atenv{},\t{})}\\
    %\textsf{req}(\atenv{}, \t{}) = [] \text{, otherwise}
    \squashhorizonally{} : \atenv{} \rightarrow \atenv{}\\
    \squashhorizonally{}(\atenv{}) =
      \textsf{fold}(\mergealiases{}, \atenv{}, \textsf{groupSimilarReq}(\atenv{}))\\
    \\\\
    \squashglobal{} : \atenv{} \rightarrow \atenv{}\\
    \squashglobal{} = \squashhorizonally{} \circ \aliassinglehmap{}
  \end{array}
  \begin{array}{lllll}
    \aliassinglehmap{} : \atenv{} \rightarrow \atenv{}\\
    \aliassinglehmap{}(\atenv{}) = \textsf{fold}(\textsf{f}, \atenvp{}, \atenvp{}[\aenv{}])\\
    \begin{array}{llll}
      \text{where } &\atenvp{} = \textsf{fold}(\singlehmap{}, \atenv{}, \atenv{}[\tenv{}])\\
                    &\textsf{f}(\atenv{0}, \t{}(\ova{\s{}}^n)) = (\atenv{n}, \t{}(\sp{})) \text{, if } \t{} = \HMaptwo{\HMapreq{}}{\HMapopt{}}\\
                    &\begin{array}{lllll}
                      \text{where } \ova{(\atenv{i}, \s{i}) = \singlehmap{}(\atenv{i-1}, \s{i})}
                     \end{array}
                    \\
                    &\textsf{f}(\atenv{}, \t{}) = \singlehmap{}(\atenv{}, \t{}) \text{, otherwise}
    \end{array}
    \\\\
    \singlehmap{} : \forall \alpha. \atenv{}, (\alpha, \t{}) \rightarrow \atenv{}\\
    \singlehmap{}(\atenv{}, (\textsf{x}, \t{})) = \updatemap{\atenv{}}{\textsf{x}}{\s{}}\\
    \begin{array}{lllll}
      \text{where} &(\atenvp{}, \s{}) = \textsf{postwalk}(\atenv{}, \t{}, \textsf{f})\\
                   &\textsf{f}(\atenv{}, {\HMaptwo{\HMapreq{1}}{\HMapreq{2}}}) = \register{}(\atenv{},\HMaptwo{\HMapreq{1}}{\HMapreq{2}})\\
                   &\textsf{f}(\atenv{}, \t{}) = (\atenv{}, \t{}), \text{otherwise}
    \end{array}
    %\\\\
    %\inferrecOp{} : \atenv{} \rightarrow \atenv{}\\
    %\inferrecOp{} = \squashglobal{} \circ \squashlocal{}
  \end{array}

  \begin{array}{lllll}
    \textsf{groupSimilarReq} : \atenv{} \rightarrow \ova{\ova{\alias{}}}\\
    \textsf{groupSimilarReq}(\atenv{}) = 
                   [\ova{\alias{}} | \ova{\kw{}} \in \textsf{dom}(\textsf{r}),
                                     \ova{\alias{}} = \textsf{remDiffTag}(\textsf{similarReq}(\ova{\kw{}}))
                                     ]\\
    \begin{array}{lllll}
      \text{where} &\textsf{r} = \{(\ova{\kw{}}, \ova{\alias{}}) | \HMaptwo{\{\ova{\kw{}\ \t{}}\}}{\HMapopt{}} \in \textsf{rng}(\atenv{}[\aenv{}]),
                                                                              \ova{\alias{}} = \textsf{matchingReq}(\ova{\kw{}})\}\\
                   &\textsf{matchingReq}(\ova{\kw{}}) = [\alias{} | (\alias{}, \HMaptwo{\HMapreq{}}{\HMapopt{}}) \in \atenv{}]\\
                   &\textsf{similarReq}(\ova{\kw{}}) = [\alias{} | \ova{\kwp{}}^n \subseteq \ova{\kw{}}^m,
                                                                   m-n \leq \textsf{thres}(m),
                                                                   \alias{} \in \textsf{r}[\ova{\kwp{}}]]\\
                   &\textsf{remDiffTag}(\ova{\alias{}}) = [\aliasp{} | \aliasp{} \in \ova{\alias{}},
                                                                       \text{ if } (\kw{}, \kwp{}) \in \textsf{req}(\atenv{}, \aliasp{})
                                                                       \text{ and } 
                                                                       \bigvee\ova{(\kw{}, \kwpp{}) \in \textsf{req}(\atenv{}, \alias{})}
                                                                       \text{ then }
                                                                       \ova{\kwp{} = \kwpp{}}
                                                                      ]
    \end{array}
  \end{array}
\end{mathpar}
\caption{
    Definition of $\squashglobal{}(\atenv{}) = \atenvp{}$\\
%  Step 3 summary: First ensure all HMaps correspond to an alias. Then merge
%aliases that point to a HMaps with identical required keysets (aliases must point to exactly one
%top-level HMap, no unions).
  }
  \label{infer:fig:squashglobal}
\end{figure*}

% Things we could prove:
% - update is commutative
