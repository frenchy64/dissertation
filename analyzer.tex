\chapter{Background}

Clojure is a dialect of Lisp, and so supports metaprogramming
via macros.
This immediately poses an interesting problem for Clojure
type systems: how do we check a macro call?
Ideally, we don't want to require special typing rules for each
macro, since that imposes additional burden on the programmer
to define special rules for their own macros.
On the other hand, sometimes its helpful to write custom rules
for customized error messages, or a higher-level specification
for a macro's usage.

In this part we explore several solutions to this problem,
from the standard approach of expanding macros to
primitive forms before checking, to more involved solutions
that allow extensible typing rules for each macro.

Several constraints guide us through our designs.
There is a question of soundness: does what we actually
check match up with the code being evaluated?
There is a natural tension between soundness and user extensibility.
Allowing custom rules for macros gives a kind of flexibility
that makes it hard to relate type checking semantics with
the running code---which is the whole idea behind a soundness result.
On the other hand, expanding code before checking ensures
we check the actual code being run.
In all of these cases, wrappers that communicate information
to the type system are needed, but they interact
with evaluated code differently.

We also consider the experience of using these solutions.
Error messages can be unrelated to the source problem
if pre-expanding code, but we may miss actual errors
by using a poorly written typing rule.
We are interested in the difficultly of extending each system,
including any additional annotation burden,
additional knowledge needed to manage evaluation semantics
in typing rules, and additional type system knowledge required
to write typing rules. Finally, we also consider implications
to type checking performance and amenability to iterative development.

The following chapters present several designs of Typed Clojure,
their extensibility stories, and general implementation concerns for
Clojure type system designers.

%{
%\singlespacing
%\begin{verbatim}
%- Problem
%  - Clojure is a Lisp with macros
%  - don't want to write typing rules for each macro
%    - don't want to burden users
%    - so we expand them before checking
%    - but sometimes it's really helpful to write custom rules
%- Possible solutions
%  - pre expansion
%  - interleaved analysis and evaluation
%- Constraints
%  - Soundness?
%    - tension between soundness and user extensibility
%  - advantages of pre-expanding
%    - can't have "wrong" expansion in pre-expanded code
%      - we check the actual expansion that gets wr
%  - wrapper macros needed in all of these systems
%\end{verbatim}
%}

\chapter{Expand before checking}

Typed Clojure's initial design was inspired by Typed Racket,
which checks Racket code by first expanding until it consists
of only primitives, and then checking using fixed rules for each
primitive.
This chapter goes into this design in more detail, starting
with our choice of analyzer and then how to handle extensibility.

\section{Upfront Analysis with \texttt{tools.analyzer}}

Instead of using Clojure's compiler to analyze code,
we opted to use \texttt{tools.analyzer}, a standalone nano-pass
analyzer providing an idiomatic map-based AST format providing
passes for hygienic transformations and Java reflection resolution.

%{
%\singlespacing
%\begin{verbatim}
%- Pros to expanding up front:
%  - Separation of concerns (expander does expansion)
%- Cons to expanding up front:
%  - Lose contextual information from unexpanded macros while type checking.
%  - Requires wrapper macros which pollute runtime expansions and
%    often require copying implementation details (brittle).
%\end{verbatim}
%}

\figref{fig:analyzer:control-flow-pre-expand} demonstrates 
how Typed Clojure checks code using the pre-expansion approach.
To simplify presentation we assume \texttt{tools.analyzer}
uses only 2 passes. The first pass \texttt{analyze} creates
a bare AST with no platform specific information.
The second pass is composed of two tree traversals.
The first is a pre-traversal \texttt{pre-passes} which
is called before we visit the children of an AST node.
The second is a post-traversal \texttt{post-passes} which
is called after we visit the children of an AST node.

This arrangement is convenient as a type system implementer,
insofar as there is a clean separation of concerns: the analyzer
handles expansion and evaluation, while the type system
merely checks.
However, much contextual information is lost from the expansion
process that is needed for checking.
We now present how we surmount this challenge while still
preserving the pre-expanded checking model.

\begin{figure}
\singlespacing
$$
  \begin{array}{r||l|l|l|}
    \text{Time} & \text{\clj{(let [...]}} & \text{\clj{(cond ...}} & \text{\clj{(+ ...)))}}\\
    \hline
     0          & \text{\clj{analyze}}^{>}    &                             &                      \\
     1          &                             & \text{\clj{analyze}}^{>}    &                      \\
     2          &                             &                             & \text{\clj{analyze}}^{>} \\
     3          &                             &                             & \text{\clj{analyze}}^{<} \\
     4          &                             & \text{\clj{analyze}}^{<}    &                      \\
     5          & \text{\clj{analyze}}^{<}    &                             &                      \\
     6          & \text{\clj{pre-passes}}^{>} &                             &                      \\
     7          &                             & \text{\clj{pre-passes}}^{>} &                      \\
     8          &                             &                             & \text{\clj{pre-passes}}^{>} \\
     9          &                             &                             & \text{\clj{post-passes}}^{<} \\
     10         &                             & \text{\clj{post-passes}}^{<}&                      \\
     11         & \text{\clj{post-passes}}^{<}&                             &                      \\
     12         & \text{\clj{check}}^{>}      &                             &                      \\
     13         &                             & \text{\clj{check}}^{>}      &                      \\
     14         &                             &                             & \text{\clj{check}}^{>} \\
     15         &                             &                             & \text{\clj{check}}^{<} \\
     16         &                             & \text{\clj{check}}^{<}      &                      \\
     17         & \text{\clj{check}}^{<}      &                             &                      \\
  \end{array}
$$
%\begin{verbatim}
%time | (let [...]  | (cond ...   | (+ ...)))
% |   | ---------------------------------------
% v   | analyze    >|             |
%     |             | analyze >   |
%     |             |             | analyze    >
%     |             |             |<analyze
%     |             |<analyze     |
%     |<analyze     |             |
%     | pre-passes >|             |
%     |             | pre-passes >|
%     |             |             | pre-passes >
%     |             |             |<post-passes
%     |             |<post-passes |
%     |<post-passes |             |
%     | check>      |             |
%     |             | check>      |
%     |             |             | check>
%     |             |             |<check
%     |             |<check       |
%     |<check       |             |
%\end{verbatim}
  \caption{Illustrative control flow when
  using \texttt{tools.analyzer} to expand code via \clj{analyze} and several passes,
  followed by Typed Clojure checking.
  The partial expression \clj{(let [...] (cond ... (+ ...)))}
  was chosen since it has at least 3 levels of nesting.
  Many more levels will be revealed after expansion by \clj{analyze}, which we do not picture.
  ${}^>$ and ${}^<$ indicate work done to a node before and after processing its children, respectively.
  }
  \label{fig:analyzer:control-flow-pre-expand}
\end{figure}

\section{Extensibility}

%{
%\singlespacing
%\begin{verbatim}
%- Problem
%  - need to communicate between type system and Clojure runtime
%- Constraints
%  - a "typed" program must evaluate unchanged via normal Clojure compilation
%    - extensions must be done via macros provided by Typed Clojure
%      - imported and used as normal by Clojure programmers
%    - in contrast to #lang system
%      - which always guarantees the type system is in charge of expanding
%      - (both approaches use macros for extension and to share information)
%- how to communicate to type system via expanded code?
%  - eg. tc-ignore, ann-form
%  - in Racket you would use syntax properties, or side effects
%  - Clojure has metadata, but not as robust as syntax properties
%    - how metadata is compiled is implementation dependent (I forgot how?)
%    - we decided to emit special `do` forms to communicate with type system
%      - (do :special-form ...)
%      - "variable protocol" in Advanced Macrology
%  - side effects
%    - Clojure's compilation strategy is straightforward
%      - files are just sequences of top-level forms
%      - evaluate each in turn
%      - side effects of expanding/evaluating a previous form
%        can be used to compile a subsequent form
%      - members of top-level `do` forms are also top-level forms, and thus
%        are evaluated in turn
%    - Typed Clojure collects global type annotations by evaluation side effects
%      - macroexpansion side effects not used in case AOT compiled
%- how to define custom rules?
%  - Approach 1: custom expansions for embedding typing rules in expansion
%  - Approach 2: "typing rules by analogy"
%    - lose ability to check actual expansion
%\end{verbatim}
%}

Now that we have outlined how we use \texttt{tools.analyzer} to pre-expand code before type checking,
we describe Typed Clojure's approach to sharing information between the programs it checks
and the type system.
We deviate significantly from Typed Racket's approach~\cite{Culpepper07advancedmacrology}
mostly because of differences in compilation models between Clojure and Racket.

One constraint we must consider in Typed Clojure is that a ``typed'' Clojure program must
evaluate unchanged under normal Clojure compilation. In Racket, we could instead specify
the language under which a module is compiled using the \texttt{\#lang} directive---this is Typed
Racket's approach. 
In Clojure, there is just one language and no built-in facilities to extend the compilation
process, so Typed Clojure provides a suite of macros for communicating with the type system that
users must explicitly load and use.

These macros come in several flavors:

\begin{itemize}
  \item syntax-based communication to type checker,
  \item side-effectful communication to type checker, and
  \item wrappers for existing untyped macros.
    %to avoid checking complex expansions
    %or provide .
\end{itemize}

We discuss each in the following sections.

\subsection{Syntax-based communication}

A simple macro provided by Typed Clojure that communicates to the checker
via syntax is \clj{tc-ignore}, which takes a number of forms, places
them in a \clj{do} form, and tells the checker to ignore the resulting
form and assign it type \clj{Any}.

\begin{figure*}
\begin{cljlisting}
(defmacro tc-ignore 
  "Ignore forms in body during type checking"
  [& body]
  `(do :clojure.core.typed.special-form/special-form
       :clojure.core.typed/tc-ignore
       ~@(or body [nil])))
\end{cljlisting}
  \caption{Public facing macro definition for \clj{tc-ignore}.}
  \label{fig:analyzer:tc-ignore}
\end{figure*}

\figref{fig:analyzer:tc-ignore} shows the implementation of the \clj{tc-ignore} macro.
It demonstrates the \clj{do}-special-form protocol:
if the first member of a \clj{do} is the keyword
\clj{:clojure.core.typed.special-form/special-form},
the following keyword names a special typing rule to use
to check the entire form.
A corresponding typing rule must then be registered with the type checker under this name,
like in \figref{fig:analyzer:tc-ignore-do-op}.

\begin{figure*}
\begin{cljlisting}
(defmethod internal-special-form :clojure.core.typed/tc-ignore
  [expr expected]
  (tc-ignore/check-tc-ignore check-expr expr expected))
\end{cljlisting}
  \caption{Registering a corresponding typing rule for \clj{tc-ignore} via the \clj{do}-special-form protocol.}
  \label{fig:analyzer:tc-ignore-do-op}
\end{figure*}

Clojure's compilation and runtime models make \clj{do} statements an excellent candidate for the basis of
an extensible syntax-based communication protocol.
First, it naturally inherits the top-level characteristics of \clj{do}, which is key to defining
wrapper macros that operate at the top-level.
A usage of \clj{tc-ignore} that relies on this is demonstrated in \figref{fig:analyzer:tc-ignore-usage}.
Second, it avoids the need to pre-expand its arguments to attach information, or
have special cases for particular arguments.
On the other hand, a communication protocol based on attaching metadata properties
would require pre-expanding arguments, since metadata is lost on macroexpansion,
and in some cases would not be possible, since many common Clojure forms do not support metadata
(such as keywords, numbers, and nil).
Third, the information can be compiled away using standard techniques,
since they are constant statements---extra information can be provided via a map of constant values
placed after the typing rule name, as in
the definition of \clj{ann-form} (\figref{fig:analyzer:ann-form-definition}).

\begin{figure*}
\begin{cljlisting}
(defmacro ann-form
  "Annotate a form with an expected type."
  [form ty]
  `(do :clojure.core.typed.special-form/special-form
       :clojure.core.typed/ann-form
       {:type '~ty}
       ~form))
\end{cljlisting}
  \caption{The definition of \clj{ann-form} shows how to communicate extra information to the type checker}
  \label{fig:analyzer:ann-form-definition}
\end{figure*}

While a strong choice, there are some downsides to basing our communication protocol on \clj{do}
statements.
There is no guarantee the information will be compiled away at runtime, and
thus may contribute to bloating the runtime.
On the other hand, \clj{tools.analyzer} must be carefully configured to not erase these constant
values before Typed Clojure can access them.

Alternative \clj{do}-based protocols could be similarly effective
such as attaching metadata directly to the symbol \clj{do} or list \clj{(do ...)}.
We felt embedding the information directly in programs had the best chance of forward-compatibility,
since the interaction between metadata and compilation is not well documented and
can be platform-dependent (in our experience ClojureScript has handled some cases differently,
like evaluating metadata instead of simply quoting it as in Clojure).

\begin{figure}
\begin{cljlisting}
(tc-ignore
  (defmacro reverse-app [a f] `(~f ~a))
  (reverse-app 1 inc)) ;=> 2
\end{cljlisting}
  \caption{Example top-level usage of \clj{tc-ignore}
           where the second form must expand after the first evaluates.
  It works because \clj{tc-ignore} wraps only with \clj{do}.}
  \label{fig:analyzer:tc-ignore-usage}
\end{figure}

\subsection{Side-effectful communication}

Racket has a sophisticated system for managing compile-time side effects
to accompany its module system.
Clojure does not have a module system, and instead relies on conventions
and a simple compilation model to write effective programs.

The unit of compilation in Clojure is a top-level form. A top-level Clojure form
is guaranteed to have all previous top-level forms fully expanded
and evaluated before it is expanded and evaluated itself.
This blurs the lines between compile-time and runtime, compared to the
distinct phases of Racket compilation.

When checking a file with Typed Clojure, we have similar guarantees:
when checking a top-level form, we can depend on the fact that all
previous top-level forms have been expanded, evaluated, and checked,
and that the current form has been fully expanded.

Thus, we have a choice of (at least) three times to send side-effectful communication
to the type checker:
expansion-time, evaluation-time, and checking-time.
\figref{fig:analyzer:ann-definition} shows the most frequently used
side-effectful macro \clj{ann}, which registers the type of a var in the
global environment.
It expands to code that uses internal function \clj{ann*}, which does
the registering. This is a \emph{evaluation-time} side effect,
and we similarly perform most communication at this time.
We now elaborate on why this is a good choice.

A previous implementation of Typed Clojure (which was used by CircleCI
in \secref{sec:casestudy}) only collected top-level annotations
from \clj{ann} at checking-time. This forced Typed Clojure to recursively
check other files just to collection annotations.
We decided the natural behavior of rechecking a file would be to
recheck its dependencies so, among other benefits, top-level annotations
would be kept up-to-date.
Unfortunately, the checker was much slower at evaluating files
than the Clojure compiler, meaning iterative development was hampered.
To fix this, we made checking of transitive file dependencies optional, and
so dependencies containing top-level annotations would potentially only 
be evaluated by the Clojure compiler.
Evaluation-time was then the natural time to collect these annotations.

A side-effect of this design choice is that it is no longer a sound idea to
infer types for unannotated top-level bindings. In the aforementioned 
implementation, if the checker finds an unannotated top-level \clj{def}
like \clj{(def a 1)}, it will update the global environment with the 
inferred type of the right-hand-side.
Now that transitive dependencies are optionally checked, it is not guaranteed
the checker will infer these annotations, and so more top-level annotations
via \clj{ann} are needed to recover consistent checking behavior.
This unfortunately increases the annotation burden even more, however the rewards
are great.
We believe that Clojure programmers will enjoy the ability to rapidly recheck
small parts of their code base, just like they are used to in untyped Clojure.

Now, we discuss the merits of collection at evaluation-time over expansion-time.
We avoid expansion-time for side-effects because Clojure code can be
evaluated in two ways: from the original source code in on-the-fly compilation mode, and 
from precompiled JVM bytecode in ahead-of-time compilation mode.
In the latter, code is expanded ahead-of-time (potentially in a different environment)
and thus expansion-time side-effects are lost.
We applied the standard solution to this problem: remove the side-effect from
the macro itself and move it to the evaluation of the code it expands into.


%\begin{verbatim}
%- not forced to recursively check other files just to collect annotations
%  - problem identified with CircleCI
%  - simply need to evaluate a file normally
%  - in turn requires more annotations
%    - tradeof between annotation burden and performance
%- avoid relying on expansion-time side effects
%  - lost with AOT compilation
%- "staged at checking time": under Typed Racket AOT compilation, it stages global type annotations for eval time
%  - we don't have a similar mode
%    - compiling a Typed Clojure file does not require checking
%\end{verbatim}


\begin{figure*}
\begin{cljlisting}
(defmacro ann 
  "Register top-level var with type."
  [varsym typesyn]
  (let [qsym (qualify-in-current-ns varsym)
        opts (meta varsym)
        check? (not (:no-check opts))]
    `(tc-ignore (ann* '~qsym '~typesyn '~check? '~&form))))
(defn ann* 
  "Internal use only. Use ann."
  [qsym typesyn check? form]
  ; omitted - registers `qsym` at type `typesym`
  )
\end{cljlisting}
  \caption{Implementation of \clj{ann}, which expands to code that registers types at evaluation-time.}
  \label{fig:analyzer:ann-definition}
\end{figure*}

\subsection{Wrapper macros}

Several situations call for wrapper macros for existing untyped macros.
In practice, this often means the type system author provides an alternative
implementation for a macro, and the type system user
replaces any usages of the original macro in type-checked code with the alternative implementation.
Sometimes this choice is aesthetic, providing a prettier 
way to write annotations. For example, the \clj{fn} wrapper
enables writing annotations like
\clj{(fn [a :- Int] ...)}
instead of the more verbose
\clj{(ann-form (fn [a] ...) [Int -> Any])}.

The more pressing need for wrapper macros when checking pre-expanded
code is to manage complex expansions.
Some macro expansions are too complex for Typed Clojure to reason about,
so it becomes necessary to rewrite these expansions to be more palatable
for the checker.
For example, the \clj{for} macro is a lazy sequence builder using
a list-comprehension syntax---however it expands into local
loops using local mutable state, which are problematic to check.
The wrapper macro for \clj{for} expands (and thus evaluates) similarly, but inserts user-provided
type annotations strategically into the expansion so it more easily type checks.

The problem with this kind of wrapper macros is that large amounts
of implementation code must be copied to preserve the original semantics.
Instead of checking a higher-level specification of the macro's behavior,
we are tied closely to a particular implementation.
This has the advantage of checking the actual code that gets evaluated, but
unfortunately
requires the type system writer to closely follow the original implementations
(hampering both backwards- and forwards-compatibility with versions of the original macro).
Furthermore, users not only must use wrapper macros where necessary, but
also recognize when they are required---usually attempting to check a complex
expansion yields an incomprehensible error as Typed Clojure fails to check it.
It is rarely apparent that a wrapper macro is needed from such an error message.

\chapter{Interleaved expansion and checking}

The previous chapter outlined a design for Typed Clojure that fully expands code
before checking.
We identified several problems with the user experience of Typed Clojure's initial design,
including bad error messages, and excessive copying of macro implementations for wrapper
macros.
Additionally, we identified several issues with \texttt{tools.analyzer} that we have
not yet discussed.

First, \texttt{tools.analyzer}'s goals of being mostly platform-agnostic made analysis particularly 
slow, and so added an undesirable performance overhead to type checking.
In particular, a copy of the
global scope is maintained for every namespace. While it enables a convenient platform-agnostic API
for symbol resolution,
it comes at a performance cost since it must be updated (from scratch) frequently.
Furthermore, some macroexpansion side effects are not (yet) recognized by the analyzer
which means analysis sometimes deviates from Clojure compiler, an undesirable situation
since Typed Clojure intends to model how code runs \emph{outside} of type checking.
Unfortunately, fixing some of these differences would require even more frequent costly updates.

Second, it is impractical to recover contextual information lost via analysis.
This is both because \texttt{tools.analyzer} has no way of representing unanalyzed
code (so there is no choice but to expand immediately), and
because \texttt{tools.analyzer} uses at least 2 passes over the AST
(so there is no obvious place to recover contextual information since pre-traversal
passes run \emph{after} the entire program has been expanded).
For example, \figref{fig:analyzer:control-flow-pre-expand}
illustrates \texttt{tools.analyzer}'s control flow with just 2 traversals.
Say at time 1 we wished to take advantage of the unexpanded \clj{cond}
form with a special rule (before it expands and contextual information is lost).
In fact, \texttt{tools.analyzer} provides the extension point \clj{macroexpand-1}
for just this purpose, which allows the user to specify exactly how a form is expanded.
Unfortunately, time 0 introduced local bindings that are unhygienic, and the hygienic
transformation pass (required for checking because occurrence typing's propositions do not recognize variable shadowing)
happens at time 6 with \clj{pre-passes}.
So, there is no room for a checking rule for \clj{cond} until time 13, well
after the \clj{cond} is expanded away.

Fortunately, \texttt{tools.analyzer}'s design and implementation
is otherwise brilliant and innovative, and forms a great base to build a new Clojure analyzer better suited to help solve
many of the aforementioned analysis and checking problems---we did exactly that in \texttt{core.typed.analyzer}.

\section{Interleaved Analysis with \texttt{core.typed.analyzer}}

To replace \texttt{tools.analyzer}, we built \texttt{core.typed.analyzer}. In this section,
we describe how \texttt{core.typed.analyzer} works, and outline both the ideas we repurposed
from \texttt{tools.analyzer} and those specific to \texttt{core.typed.analyzer}.

\subsection{Overview}

The main feature of \texttt{core.typed.analyzer} is the ability to stop and resume
analysis at any point, while still supporting the essentials of a general-purpose Clojure analyzer.
Supporting this requires several key innovations and restrictions over \texttt{tools.analyzer}.
First, a new AST node type for partially expanded forms is needed to return a paused analysis.
Second, the analyzer must have the ability to incrementally perform a small amount of analysis
(on the order of expanding one macro) to provide fine-grained control over the AST.
Third, all AST traversals must be fused into one traversal to minimize
the bookkeeping needed to manage the AST.

To this end, \texttt{core.typed.analyzer} provides an API of 4 functions.
First, \clj{(unanalyzed form env)} creates an \clj{:unanalyzed} AST node
that pauses the analysis of \clj{form} in local environment \clj{env}.
Second, \clj{(analyze-outer ast)} analyzes the outermost form represented by \clj{ast}
further by roughly one macroexpansion if possible, otherwise it returns \clj{ast}.
Third, \clj{(run-pre-passes ast)} and \clj{(run-post-passes ast)}
decorate \clj{ast} with extra information, used before and after visiting its children,
respectively.

To sample how it feels to use this API to implement a type checker, we now
walk through checking \clj{(let [...] (cond ... (+ ...)))} in \figref{fig:analyzer:typed-analyzer-overview}.
To check the outermost \clj{let},
we use \clj{unanalyzed} to create an initial AST from a entire form at time 0.
Then at time 1, the checker calls \clj{analyze-outer} zero or more times, either 
until a special rule for partially expanded code is triggered
or to a fixed point.
Next at time 2 and 3 we decorate our AST node with \clj{run-pre-passes} (adding hygienic bindings)
before calling \clj{check}.
After checking its children during time 4-13, at time 14 and 15 we use \clj{run-post-passes}
to add the rest of the decorations (e.g., resolving interop reflection)
before any final checks from \clj{check}.
The interleaving of operations using \texttt{core.typed.analyzer} is clear to see when
compared to the same example using \texttt{tools.analyzer} 
(\figref{fig:analyzer:control-flow-pre-expand}).

Now with the interleaving analyzer, we can solve the problem we posed at the beginning of this chapter
of wanting a custom typing rule for \clj{cond}: we simply
limit the number of expansions done via \clj{analyze-outer} at time 4 
before calling \clj{check} (\figref{fig:analyzer:typed-analyzer-overview}).
The call to \clj{run-pre-passes} at time 2 will make any introduced let bindings hygienic,
and so it's safe to reason about them with occurrence typing, and thus Typed Clojure.


\begin{figure*}
\singlespacing
$$
  \begin{array}{r||l|l|l|}
    \text{Time} & \text{\clj{(let [...]}}            & \text{\clj{(cond ...}}          & \text{\clj{(+ ...)))}}          \\
    \hline
     0          & \text{\clj{unanalyzed}}^{>}        &                                 &                                 \\
     1          & \text{\clj{analyze-outer}}^{*}     &                                 &                                 \\
     2          & \text{\clj{run-pre-passes}}^{>}    &                                 &                                 \\
     3          & \text{\clj{check}}^{>}             &                                 &                                 \\
     4          &                                    & \text{\clj{analyze-outer}}^{*}  &                                 \\
     5          &                                    & \text{\clj{run-pre-passes}}^{>} &                                 \\
     6          &                                    & \text{\clj{check}}^{>}          &                                 \\
     7          &                                    &                                 & \text{\clj{analyze-outer}}^{*}  \\
     8          &                                    &                                 & \text{\clj{run-pre-passes}}^{>} \\
     9          &                                    &                                 & \text{\clj{check}}^{>}          \\
     10         &                                    &                                 & \text{\clj{run-post-passes}}^{<}\\
     11         &                                    &                                 & \text{\clj{check}}^{<}          \\
     12         &                                    & \text{\clj{run-post-passes}}^{<}&                                 \\
     13         &                                    & \text{\clj{check}}^{<}          &                                 \\
     14         & \text{\clj{run-post-passes}}^{<}   &                                 &                                 \\
     15         & \text{\clj{check}}^{<}             &                                 &                                 \\
  \end{array}
$$

  \caption{Illustrative control flow for interleaved checking and analysis using
  \texttt{core.typed.analyzer}. ${}^*$ denotes zero or more calls.
  }
  \label{fig:analyzer:typed-analyzer-overview}
\end{figure*}

\subsection{Implementation}

We now go into more detail about how \texttt{core.typed.analyzer}
is implemented as a modification of \texttt{tools.analyzer}
and the various tradeoffs that were chosen.

To support the requirement of \clj{analyze-outer} performing as little
analysis as possible, we converting the \clj{analyze} function from
a full AST traversal to a pre-traversal that only visits the current node.
This mostly involved substituting recursive
calls to \clj{analyze-form} with \clj{unanalyzed}, as we
can see from porting the \clj{parse-if} helper function
  in \figref{fig:analyze:parse-if-port}.

\begin{figure*}
\begin{cljlisting}
; tools.analyzer version
(defn parse-if
  "Convert a Clojure `(if <test> <then> <else>)` form to an AST."
  [[_ test then else :as form] env]
  {:op      :if
   :form     form
   :env      env
   :test     (__red>analyze-form<red__ test (assoc env :context :ctx/expr))
   :then     (__red>analyze-form<red__ then env)
   :else     (__red>analyze-form<red__ else env)
   :children [:test :then :else]})

; core.typed.analyzer version
(defn parse-if
  "Convert a Clojure `(if <test> <then> <else>)` form to an AST."
  [[_ test then else :as form] env]
  {:op      :if
   :form     form
   :env      env
   :test     (__red>unanalyzed<red__ test (assoc env :context :ctx/expr))
   :then     (__red>unanalyzed<red__ then env)
   :else     (__red>unanalyzed<red__ else env)
   :children [:test :then :else]})
\end{cljlisting}

  \caption{Example of porting a \texttt{tools.analyzer} function
  to \texttt{core.typed.analyzer} using \clj{unanalyzed} (differences highlighted in \textcolor{red}{red}).
  }
  \label{fig:analyze:parse-if-port}
\end{figure*}

Porting the nano-pass machinery was more involved, however
we have a similar goal: passes must perform the minimum possible
work so they can be easily composed as-needed.
Thankfully, passes in \texttt{tools.analyzer} are written modularly,
so we can straightforwardly pick a subset of them we need for \texttt{core.typed.analyzer}.
To connect the passes, metadata declares dependencies on other passes
and the traversal strategy.
We can see this in action for \clj{constant-lift} (\figref{fig:analyzer:constant-lift}),
which is declared to be part of a post-traversal
that must run after \clj{elide-meta} and \clj{analyze-host-expr}.

\begin{figure}
\begin{cljlisting}
(defn constant-lift
  "Like clojure.tools.analyzer.passes.constant-lifter/constant-lift but
   transforms also :var nodes where the var has :const in the metadata
   into :const nodes and preserves tag info"
  {:pass-info __red>{:walk :post, :depends #{},
               :after #{#'elide-meta #'analyze-host-expr}}<red__}
  [ast]
  (merge (constant-lift* ast)
         (select-keys ast [:tag :o-tag :return-tag :arglists])))
\end{cljlisting}
  \caption{Passes in \texttt{tools.analyzer} are defined as regular functions,
  with \clj{:pass-info} metadata (\textcolor{red}{red}) declaring dependencies on other passes and tree walking strategy.}
  \label{fig:analyzer:constant-lift}
\end{figure}

A scheduler compiles the passes according to this metadata into as few traversals as possible.
We reuse this setup of scheduled passes in \texttt{core.typed.analyzer},
with the restriction that all passes compile into one traversal.
We could convert many existing pre- and post-traversal passes without much modification.
Only the most crucial pass required much modification:
the hygienic transformation pass \clj{uniquify-locals}.
It must be a pre-traversal in \texttt{core.typed.analyzer} (for reasons we have already discussed),
and was modified from a full tree walk.

%Furthermore, passes are almost always extensible via Clojure's multimethods, so it is trivial to add
%support for new AST types, like an AST representation for unanalyzed code.

To help support \clj{:unanalyzed} AST nodes, a 
\clj{:clojure.core.typed.analyzer/config} 
entry (abbreviated \clj{::config}) was added to all nodes
to attach data that applies to AST nodes even after they are expanded.
For example, a top-level expression is still top-level after it is expanded.
The implementations of \clj{unanalyzed} and \clj{analyze-outer}
in \figref{fig:analyzer:config-inheritance} show their propagation---\clj{unanalyzed}
initializes \clj{::config} on line 9, and \clj{analyze-outer} propagates it on line 16 after further analysis.

\begin{figure}
  \lstset{numbers=left}
  \begin{cljlisting}
(defn unanalyzed
  "Create an unanalyzed AST node from form and env"
  [form env]
  {:op :unanalyzed
   :form form
   :env env
   ;; ::config will be inherited by whatever node
   ;; this :unanalyzed node becomes when analyzed
   __red>::config<red__ {}})

(defn analyze-outer
  "If ast is :unanalyzed, then call analyze-form on it, otherwise returns ast."
  [ast]
  (case (:op ast)
    :unanalyzed (__red>assoc<red__ (analyze-form (:form ast) (:env ast))
                       __red>::config (::config ast)<red__)
    ast))
\end{cljlisting}
  \caption{The initialization and propagation of \clj{::config} (relevant parts \textcolor{red}{highlighted})}
  \label{fig:analyzer:config-inheritance}
\end{figure}

Finally, we revised to platform-agnostic parts of the \texttt{tools.analyzer} API
to allow better performance.
Symbol and namespace resolution are now platform-dependent, which allows us to 
remove the global environment mirroring we identified as a performance issue
at the beginning of this chapter.
This added a slight burden to platform implementers of 
\texttt{core.typed.analyzer}---the JVM support added a dozen lines of code, although
it took several revisions and testing to recover the original behavior.

%{
%\singlespacing
%\begin{verbatim}
%- Goals
%  1. Build a better tools.analyzer
%     - too slow
%     - too many passes
%     - reuse the passes/scheduler/analysis
%       - and :unanalyzed
%         - instead of analyzing children, store context and return
%       - unforce one pass
%  2. Extensibility
%     - we want custom rules for syntax BEFORE expansion
%     - avoid need for wrapper macros
%       - avoid implementation-dependence
%       - better error messages for users
%     - but lose ability to check actual expansions
%- (This is the Turnstile approach)
%  - Except we don't have syntax objects, how to do it?
%- Create a single-pass tools.analyzer variant that can be paused in
%  the middle of analysis
%  - `analyze` now expands absolute minimum (usually 1 macro)
%- now `check` has access to the raw Clojure forms before they are expanded
%  - much power = much responsibility
%    - top-level evaluation side effects
%    - expansion side effects
%      - talk about that in a different chapter
%    - must manually manage local scope
%    - avoiding double macro expansion
%    - avoiding double evaluation
%    - double analysis is OK though, no side effects
%      - so we can "reinsert" a fully analyzed AST back into
%        a macro call so it can be expanded as usual.
%        - eg. (my-macro (unexpanded))
%            =>
%              (my-macro ~(check (unexpanded) ...))
%- Pros
%  - now have access to original macro forms for higher-level reasoning
%- Cons
%  - no longer checking the implementation of macros
%    - although were we ever, really?
%      - wrapper macros are copied implementation details
%  - must carefully manage compile-time side effects
%\end{verbatim}
%}



\section{Extensibility in Interleaved checking}

Now we present the most significant type system feature
enabled by \texttt{core.typed.analyzer}: custom typing rules.
We already hinted at how this support works in
\figref{fig:analyzer:typed-analyzer-overview}---in this section
we make that explicit with a small type system implementation.

\begin{figure}
  \lstset{numbers=left}
\begin{cljlisting}
(defn check
  "Check an analyzed AST node has the expected type."
  [expr expected]
  (case (:op expr)
    :if (let [ctest (check-expr (:test expr) (*@\emph{<omitted>}@*))](*@\label{analyzer:listing:typed:check-calls-check-expr}@*)
          (*@\emph{<omitted>}@*))
    :lambda (*@\emph{<omitted>}@*)
    (*@\emph{<omitted other cases>}@*)))
(defn check-expr(*@\label{analyzer:listing:typed:check-expr}@*)
  "Check an AST node has the expected type."
  [expr expected]
  (if (= :unanalyzed (:op expr))
    (case (*@\emph{<resolved-op-sym-for-expr>}@*)
      __red>clojure.core/cond (check-special-cond expr expected)<red__(*@\label{analyzer:listing:typed:check-expr:special-cond}@*)
      ; default case
      (check-expr (analyze-outer expr) expected))
    (run-post-passes
      (check (run-pre-passes expr)(*@\label{analyzer:listing:typed:check-expr-calls-check}@*)
             expected))))
(defn check-form(*@\label{analyzer:listing:typed:check-form}@*)
  "Check a Clojure expression has the expected type"
  [form expected]
  (check-expr (unanalyzed form (empty-env))
              expected))
\end{cljlisting}
  \caption{The driver function \clj{check-form} for a type system using \texttt{core.typed.analyzer},
  which dispatches to a special typing rule for an unexpanded \clj{cond} (\textcolor{red}{red}).}
  \label{fig:analyzer:core.typed.analyzer-driver}
\end{figure}

We now present the sample type system in \figref{fig:analyzer:core.typed.analyzer-driver}.
The main entry point is \clj{check-form} (line \ref{analyzer:listing:typed:check-form}),
and we can check our running example has type \clj{expected}
with:

\begin{cljlisting}
(check-form '(let [...] (cond ... (+ ...)))
            expected)
\end{cljlisting}

A pair of mutually recursive helpers assist the main driver: \clj{check-expr} (line \ref{analyzer:listing:typed:check-expr})
handles the analysis machinery along with unanalyzed forms,
and \clj{check}
which type checks an analyzed AST node.
Once \clj{check-expr} has found a fully analyzed AST, it calls
\clj{check} (line \ref{analyzer:listing:typed:check-expr-calls-check})
in between running the analyzer passes.
Correspondingly, any recursive checking of children performed in \clj{check}
could trigger a special rule for unanalyzed forms, and so
calls \clj{check-expr} (for example, checking \clj{:if}'s test on
line \ref{analyzer:listing:typed:check-calls-check-expr}).

Finally, custom typing rules are dispatched by \clj{check-expr}---we have included
an example dispatch to a \clj{cond} rule on line \ref{analyzer:listing:typed:check-expr:special-cond}.
The \clj{check-special-cond} function now has the ability
to define a robust typing rule for \clj{cond}: it has full access to both
the unexpanded \clj{cond} form and its hygienic type context.

This is a far cry from what was possible with \clj{tools.analyzer},
and so \clj{core.typed.analyzer} is a success in that light.
However, with great power comes great responsibility:
handing users
the ability to control the order of analysis
via custom typing rules
requires careful planning in the face of compile-time side effects.
The next chapter is dedicated to discussing this caveat.

\chapter{Managing Analysis Side effects}

To change Clojure's order-of-macroexpansion is to change the semantics
of Clojure---in theory.
This chapter will give an overview of Clojure's evaluation model
so that the full implications of giving Typed Clojure users the responsibility
to handle macroexpansion via custom typing rules becomes apparent.
We also present how both \texttt{tools.analyzer} and \texttt{core.typed.analyzer}
attempt to preserve these semantics.
We will then compare our issues with those in other systems that allow
typing rules.

\section{Clojure's Evaluation Model}

In this section, we describe the subtleties of evaluating Clojure code.
To evaluate a string of Clojure code, it is first parsed (via \clj{read}) into
a Clojure data representation and then macroexpanded until
it consists of only language primitives.
This is then compiled to JVM bytecode which is executed
to produce the result of evaluation.
Loading a file of Clojure code is mostly equivalent to
evaluating each form in the file from top-to-bottom.

A form is given a special status when considered
\emph{top-level}: it will be completely evaluated
before the next top-level form is expanded.
Under evaluation, a form is considered top-level 
unless it is nested under another form.
For example, \clj{(query)} in \clj{(cond (query) ...)}
is not considered top-level, and the entire \clj{cond} form
is top-level (unless nested in a larger form).
The exception to this rule is nesting under
\clj{do} expressions:
arguments of a top-level \clj{do} form inherits its top-level status.
That is, in the top-level expression \clj{(do (def a ...) (def b ...))},
\clj{a} will be completely defined before
\clj{b} is expanded.
This arrangement allows the expansion of one top-level form
to depend on the evaluation (and thus expansion) of all preceding top-level forms.

As described above, Clojure is always compiled (it has no interpreter).
Clojure offers two modes of compilation: on-the-fly and ahead-of-time.
The main distinction is that
on-the-fly mode discards the generated bytecode after executing it, whereas ahead-of-time mode
both executes and saves the bytecode (as JVM \texttt{.class} files) for later execution.
This is different from other Lisps like Chez Scheme~\cite{dybvig2018chez} and
Common Lisp~\cite{steele1990common},
which has distinct semantics for 
interpreted and compiled modes.
In these languages, there is an implicit assumption that expressions are only
compiled development machines,
and so
compilation mode in these languages skips
the evaluation of certain expressions
to avoid production-only side effects (e.g., initializing databases). Programmers must
use \texttt{eval-when} to opt-in to different behavior.
In contrast, Clojure evaluates all code during compilation (and Clojure is always compiled). Programmers
rely on
on Java-style \clj{main} methods (invoked from the command line) to trigger initialization steps only
applicable in production.

The most important consequence of Clojure's ahead-of-time compilation is that
macros are expanded in a different environment than the program is executed in, and
thus state is not necessarily preserved between them.
This is a well-known problem in most Lisps like Chez Scheme and Common Lisp---to work around it,
Steele~\cite{steele1990common} suggests the convention of moving compile-time
side effects into the code that the macro expands to.
This way, the side effects are evaluation-time, and thus always visible in
every mode of compilation.
Clojure also recommends this convention---without it,
it is possible to have accidental dependencies on expansion
side-effects that only cause bugs under ahead-of-time compilation (usually performed 
only as the last step of software deployment).
Racket's module system, on the other hand, avoids these latent bugs~\cite{flatt2002composable}
by erasing compile-time state before evaluation.
This emulates the conditions of ahead-of-time compilation in Racket's interpreted mode,
at the cost of repeated module reinitializations.

\section{Is order-of-expansion defined in Clojure?}

Order of evaluation in Clojure is usually specified where it makes sense~\cite{CljEvalDoc}.
For example, invocations \clj{(f arg*)} are evaluated left-to-right
starting from \clj{f}, whereas the order of evaluation for elements of unordered set literals \clj{#\{k*\}}
is undefined.
On the other hand, the order of \emph{expansion} is not addressed at all in the Clojure
documentation.
It would be extremely convenient for the writers and users of Typed Clojure
to avoid micromanaging the order of expansion, and would
make writing custom typing rules and other Typed Clojure
extensions more viable.
With those biases in mind, we now attempt to give a balanced account
of expansion order in Clojure.

It is worth distinguishing between order of expansion
of top-level forms and inner forms.
Common Lisp asserts~\cite{steele1990common} that the order
of macroexpansion for inner forms is unspecified.
This gives flexibility not only to both platform implementors
but also macro writers, because it grants macros the flexibility to
expand their arguments, a pattern
used by the Clojure core library \texttt{core.async}~\cite{CljCoreAsync}.
This seems to work in practice for \texttt{core.async} users
without any special instruction or warnings.
Also, \texttt{core.async} was designed
by the same team that develops Clojure itself,
so it gives us more confidence that changing expansion order
(by manually expanding a macro's arguments)
is a sound choice.

Even more reason to doubt the importance of expansion
order is its seeming lack of preservation 
across platforms for core macros.
For instance,
Clojure and ClojureScript share the same infrastructure for
writing macros, but only share a subset of core macro \emph{definitions}.
That is, some macros are redefined in ClojureScript to cater to the
JavaScript host---furthermore, some functions in Clojure are turned into macros
in ClojureScript.
While the order of evaluation must be preserved
for compatibility with Clojure,
it seems unlikely that any special measures were taken to preserve
expansion order of arguments---especially for more complicated,
platform-specific macros.
In practice, however, most macros probably do preserve expansion-order:
idiomatic macros do not expand their arguments
and merely forward them to more primitive operators
(often \clj{do} or \clj{let}) that have
more consistent expansions across platforms.
This might be coincidental, since
we are not aware of any special effort to force this style,
and might more be a consequence of following general Clojure idioms.

The popular general-purpose code analyzer \texttt{tools.analyzer} 
ignores particular expansion-time side-effects, without any
apparent downsides.
Specifically, changes to the current namespace are ignored
during macroexpansion. This is
a common \emph{runtime} side-effect in Clojure, and is crucial for
an analyzer to adhere to because analyzing a form in the
wrong namespace is incorrect.
Even so, we are not aware of any cases in practice in which this is a problem.
Given that \texttt{tools.analyzer} is thoroughly tested and used
in industry, it might then be reasonable to conclude that expansion-time side effects
are rare. On the other hand, changing namespaces is a very specific side-effect
whose conventions are perhaps not generalizable to other side-effects.
Usually, changing namespaces is only triggered by the expansion of the \clj{ns} macro,
and \clj{ns} is almost always used exactly once at the top of every Clojure file (to declare
namespace dependencies). There could be other, more common expansion-time
side-effects that are compatible with \texttt{tools.analyzer} that
we are not aware of.

Relatedly, to help measure the practicality of an alternative design of Typed Clojure
that expands macros multiple times,
we talked to Clojure and Racket developers about repeated macro expansions.
One Clojure developer felt
that avoiding repeated expansion was important in Clojure,
but could not show an example of real-world code
that would fail in these circumstances.
In contrast, a Racket researcher was quick to demonstrate complex
assumptions in their macros that would be violated in these
conditions (for example, global counters for identifying specific expansions).

Reflecting on these anecodes, Clojure developers routinely reload
code (REPL's can take minutes to start and so encourage developers to leave
them open for days)
and so are guided to write code that can be re-evaluated
in almost any order (after an initial load), encouraged further by the late-binding semantics
of Clojure Vars.
On the other hand, Racket's module system is much stricter and reinitializes
entire modules in their own sandboxes. Racket programmers can (and do) rely
on these restrictions to support complex top-level invariants.

In both Clojure and Racket, it is idiomatic to avoid ``double expansions''
when writing macros by binding intermediate results to names in a macro's
expansion, usually
for performance reasons.
This also prevents double \emph{evaluation}, an even more serious performance concern
since expressions are run many more times than they are
expanded.
The issue of a third party (like Typed Clojure) expanding macros
multiple times is tangentially related to this idiom, since
the cost of double expansions must be paid.
The main difference, at least in the design we proposed,
was that the extra expansions by the third party would eventually be discarded
and not evaluated, thus avoiding the cost of double evaluation.

It's interesting to note that the backgrounds and daily obligations of each groups varied significantly,
with the Racket programmers being mostly from academia (studying language extensibility)
and the Clojure developers mostly from industry.
While these opinions about repeated expansions
are useful to help contextualize our larger discussion of
order-of-expansion for inner forms,
there are important details to take into account
before prematurely
linking the two subjects.
We must avoid using the fact that Clojure programmers
reload expressions out-of-order as direct evidence 
to support changing the expansion order of inner forms.
This is because only \emph{top-level} expression are reloaded---intuitively,
this does not change the expansion order of inner forms.
We also note that reloading Clojure expressions can be notoriously
buggy in certain circumstances, resulting in desynchronization
between code on disk and code loaded into memory.
Many disparate Clojure libraries and conventions have been developed to
help manage these situations and there is no centralized solution.
On the other hand, this problem is recognized and addressed by Racket's module system.
We do not want Typed Clojure contributing yet another source of
desynchronization, which is the main reason behind this extended discussion.

%Given that information, combined with Clojure's strong Common Lisp heritage,
%we think it is reasonable to assert that the order of expansion
%(of inner forms) is also unspecified in Clojure.

%Clojure tracks the current \emph{namespace} with the dynamic
%variable \clj{*ns*}.

%{
%\singlespacing
%\begin{verbatim}
%- introduce Clojure's evaluation model
%  - `do` children treated as top-level
%    - eg. (do e1 e2)
%      - e1 is completely expanded+evaluated before e2 is expanded.
%      - e2 can depend on evaluation-time side effects of e1
%    - eg. (let [] (do e1 e2))
%      - e1 is ONLY expanded before e2
%      - e2 cannot depend on evaluation-time side effects of e1
%- introduce *ns*, global thread-local variable that holds the current namespace
%  - probably the most important evaluation-time side effect
%    - changing namespaces
%- introduce AOT-compilation
%  - macroexpansions are performed upfront
%    - so relying on expansion side-effects are uncommon and unreliable
%- since checking & evaluation are interleaved, important that
%  order-of-checking == order-of-evaluation
%- but what about order-of-macroexpansion?
%  - unaware of examples where order of macroexpansion side-effects are important
%    outside of top-level forms
%    - we have serious problems otherwise for any analysis tool
%      - eg. (fn [] (change-ns-at-mexpansion-time) (relys-on-previous-macro-to-resolve))
%        - this evaluates just fine in Clojure since there is
%          no special handling of *ns* and things never pause and always just happen
%          in order
%        - but seems intractable for things like typing rules
%          - custom typing rules must check forms in the order they are mexpanded!
%            - seems very restrictive and error prone.
%          - eg. ((fn [] ...) args ...)
%            - imagine delaying checking a `fn` before its arguments.
%              - but if args depend on fn body to expand, then we are forced
%                to fully expand (ie. CHECK) the fn before checking the args.
%                - very restrictive
%- Relevant reading:
%  - Flatt, "You want it, when?"
%  - SamTH, Scheme 2007 (good for lispy background), PLDI 2011 (general audience)
%- Chez
%  - https://www.scheme.com/csug8/system.html#./system:s76
%  - AOT compilation does not evaluate top-level forms by default
%    - special cases for `define-syntax`
%    - use eval-when for special cases where syntax depends on define's
%      - eg. (eval-when (compile) (defn helper ..))
%            (defmacro m ... (helper ...) ...)
%  - Compared to Clojure:
%    - AOT compilation in Clojure evaluates top-level forms
%      - must be more careful with top-level evaluation side effects then,
%        usually 
%      - usually use -main function for side effects to start a program
%- Common Lisp
%  - http://filonenko-mikhail.github.io/cltl2-doc/enpdf/cltl2.pdf
%  - section 5.3.3
%    - top-level `progn` members are also top-level
%    - order of macroexpansion (for inner forms) is unspecified
%    - convention: top-level macros should side-effect during eval, not expand
%\end{verbatim}
%}

\section{Preserving evaluation order during type checking}

Now that we have discussed some details of Clojure's evaluation model,
we demonstrate how to write a type checker that correctly preserves
these semantics.
Most of the details concern top-level expressions. In particular,
a top-level \clj{do} gives top-level status to its arguments, and
that top-level forms must be completely evaluated in order.

The most interesting case to consider is a top-level \clj{(do e1 e2)} form,
where the
\emph{expansion} of \clj{e2}
depends on the \emph{evaluation} of \clj{e1}.
For example,

{
\lstset{numbers=left}
\begin{cljlisting}
(do (defmacro mac [] 42)
    (mac))
\end{cljlisting}
}

defines the macro \clj{mac} and then uses it to expand
the macro call on the the last line (the final result is \clj{42}).
This will only execute correctly if top-level semantics of \clj{do} 
are faithfully preserved.
We will use this as a running example in our exploration of
each analyzer.

\subsection{\texttt{tools.analyzer}}

\begin{figure}
\singlespacing
$$
\begin{array}{r||l|l|l|l|}
\text{Time} & \text{\clj{(do}}               & \text{\clj{(defmacro mac [] 42)}} & \text{\clj{(mac))}}\\
\hline
            & \text{\clj{analyze+eval}}^{>}  &                      \\
            & \textcolor{red}{\text{\clj{macroexpand-1}}^{*}} &                      \\
            &                                & \text{\clj{analyze+eval}}^{>}   \\
            &                                & \textcolor{red}{\text{\clj{macroexpand-1}}^{*}}    \\
            &                                & \text{\clj{analyze+passes}}     \\
            &                                & \text{\clj{check}}              \\
            &                                & \textcolor{red}{\text{\clj{eval}}}                 \\
            &                                & \text{\clj{analyze+eval}}^{<}   \\
            &                                & & \text{\clj{analyze+eval}}^{>}   \\
            &                                & & \textcolor{red}{\text{\clj{macroexpand-1}}^{*}}    \\
            &                                & & \text{\clj{analyze+passes}}     \\
            &                                & & \text{\clj{check}}              \\
            &                                & & \textcolor{red}{\text{\clj{eval}}}                 \\
            &                                & & \text{\clj{analyze+eval}}^{<}   \\
            & \text{\clj{analyze+eval}}^{<}  &                      \\
\end{array}
$$
  \caption{Using \texttt{tools.analyzer}
  to interleave evaluation of
  top-level forms during the checking of \clj{(do (defmacro mac [] 42) (42))}.
  }
  \label{fig:analyzer:control-flow-pre-expand-side-effects}
\end{figure}

An \clj{analyze+eval} function handles top-level evaluation
concerns in \texttt{tools.analyzer}, as 
\figref{fig:analyzer:control-flow-pre-expand-side-effects}
demonstrates with our
running example.
The type checker's main obligation is to provide a \clj{check}
function that checks a completely analyzed AST before it is evaluated.
When passed to \clj{analyze+eval}, it will only call \clj{check}
on the smallest top-level forms by expanding macros until
a non-\clj{do} expression is reached.

There are several other details that must be handled
that led us to write our own \clj{analyze+eval} variant for Typed Clojure---for
example, expected type propagation and
handling the \clj{do}-special-form protocol. We
do not discuss those details here, and hope that
the original implementation of \clj{analyze+eval} combined with the
descriptions in this section should be sufficient to create other variants.

\subsection{\texttt{core.typed.analyzer}}

We have just described how to preserve evaluation order with \texttt{tools.analyzer}.
The main distinction to keep in mind is that \texttt{tools.analyzer} controls
analysis and checking for us, and so it handles its own internal bookkeeping 
to keep track of top-level forms.
On the other hand, this section describes the same problem with \texttt{core.typed.analyzer},
which has comparatively very little control of when analysis is performed
(since the provided \clj{analyze-outer} function only performs a single expansion).
Some other party---in this case, the type checker---must incorporate analysis into its
main loop.

To handle this, \texttt{core.typed.analyzer} uses markers to
distinguish between
different kinds of top-level expressions.
The first marker is performed by
\clj{mark-top-level} 
and is added to every kind of top-level expression.
The second marker, \clj{mark-eval-top-level},
further marks a top-level expression for evaluation.
The \clj{eval-top-level} function 
then evaluates AST nodes marked by \clj{mark-eval-top-level}
(with other corner cases we will describe later).

We use \figref{fig:analyzer:control-flow-incremental-side-effects}
to demonstrate this communication with our running example.
The entry point here is \clj{check-top-level} at time 0, which has been enhanced
to handle top-level evaluation order.
To set up the rest of checking,
at time 2 the outer \clj{do} expression is marked with \clj{mark-top-level}.
Then, once \clj{analyze-outer} determines the outer expression is a \clj{do},
its children are then marked with \clj{mark-top-level} (times 4 and 5).

For the \clj{defmacro} form, let us assume for simplicity that
it expands to a non-\clj{do} expression.
Under those circumstances, 
once \clj{analyze-outer} at time 8 reaches a fixed point, the resulting AST node is marked
for evaluation with \clj{mark-eval-top-level} at time 9.
Then (after it has been checked) at time 14 \clj{eval-top-level} evaluates the \clj{defmacro}
expression (because it was marked with \clj{mark-eval-top-level}).
This repeats similarly for the final expression \clj{(mac)}.

There are a few notable cases to where \clj{eval-top-level}
does \emph{not} trigger evaluation.
The common case where evaluation is skipped is when an AST node has no top-level markings,
like during time 11 where the main checking loop
should not recursively evaluate expressions.
A less-obvious case is a node marked with \clj{mark-top-level},
but \emph{not} \clj{mark-eval-top-level}.
For example, the outer-most \clj{do} is only marked by
\clj{mark-top-level} (time 2). Evaluation is then skipped at time 24 because
the top-level markers have been passed along to its children
(at times 4 and 5)
and so they themselves
have been evaluated
(at times 14 and 21). If we evaluated at time 24,
it would result in
an incorrect double-evaluation of the \clj{defmacro} and \clj{(mac)} expressions---hence
evaluation is skipped by \clj{eval-top-level} in this case.

\begin{figure*}
\singlespacing
$$
  \begin{array}{r||l|l|l|l|}
    \text{T} & \text{\clj{(do}}                                   & \text{\clj{(defmacro mac [] 42)}} & \text{\clj{(mac))}}\\
    \hline
     0       & \text{\clj{check-top-level}}^{>}                   & &                     \\
     1       & \text{\clj{unanalyzed}}                            & &\\
     2       & \textcolor{red}{\text{\clj{mark-top-level}}}       & &\\
     3       & \text{\clj{analyze-outer}}^{*}                     & &\\
     4       &                                                    & \textcolor{red}{\text{\clj{mark-top-level}}}&\\
     5       &                                                    & & \textcolor{red}{\text{\clj{mark-top-level}}}          \\
     6       & \text{\clj{run-pre-passes}}^{>}                    & &\\
     7       & \text{\clj{check}}^{>}                             & &\\
     8       &                                                    & \text{\clj{analyze-outer}}^{*}    &                           \\
     9       &                                                    & \textcolor{red}{\text{\clj{mark-eval-top-level}}}&\\
     10      &                                                    & \text{\clj{run-pre-passes}}^{>}   &                           \\
     11      &                                                    & \text{\clj{check}}^{>}            &                           \\
     12      &                                                    & \text{\clj{run-post-passes}}^{<}  &                           \\
     13      &                                                    & \text{\clj{check}}^{<}            &                           \\
     14      &                                                    & \textcolor{red}{\text{\clj{eval-top-level}}}^{<}       & \\
     15      &                                                    & & \text{\clj{analyze-outer}}^{*}                               \\
     16      &                                                    & & \textcolor{red}{\text{\clj{mark-eval-top-level}}}          \\
     17      &                                                    & & \text{\clj{run-pre-passes}}^{>}                              \\
     18      &                                                    & & \text{\clj{check}}^{>}                                       \\
     19      &                                                    & & \text{\clj{run-post-passes}}^{<}                             \\
     20      &                                                    & & \text{\clj{check}}^{<}                                       \\
     21      &                                                    & & \textcolor{red}{\text{\clj{eval-top-level}}}^{<}             \\
     22      & \text{\clj{run-post-passes}}^{<}                   & \\
     23      & \text{\clj{check}}^{<}                             & \\
     24      & \textcolor{red}{\text{\clj{eval-top-level}}^{<}}   & \\
     25      & \text{\clj{check-top-level}}^{<}                   & &                     \\
  \end{array}
$$
  \caption{Using \texttt{core.typed.analyzer}
  to interleave checking with the evaluation of
  top-level forms.}
  \label{fig:analyzer:control-flow-incremental-side-effects}
\end{figure*}

Incorporating correct top-level evaluation into a \texttt{core.typed.analyzer}-based
type checking loop is a mostly-straightforward extension, as demonstrated in
  \figref{fig:analyzer:core.typed.analyzer-top-level-driver}.
The main differences are highlighted.
First, \clj{unanalyzed-top-level} combines
\clj{unanalyzed}
and
\clj{mark-top-level} (times 1 and 2\ in our previous discussion).
Then, \clj{eval-top-level} should be added as the final operation in
the main checking loop.
%
Finally,
the handling of unanalyzed nodes 
(line \ref{analyzer:listing:cta-top-level-driver:unanalyzed-cases})
must carefully manage markers to prevent double evaluations.
In particular, \clj{eval-top-level}
has a corner case to handle AST nodes that are not fully analyzed:
nodes that are both \clj{:unanalyzed} and 
are marked with \clj{mark-top-level} \emph{are} evaluated.
The corresponding functions \clj{unmark-top-level}
\clj{unmark-eval-top-level} can be used
to undo the markers if these semantics are undesirable for particular
situations.

\begin{figure}
  \lstset{numbers=left}
\begin{cljlisting}
(defn check-expr
  "Check an AST node has the expected type."
  [expr expected]
  (if (= :unanalyzed (:op expr))
    __red>...<red__ (*@\label{analyzer:listing:cta-top-level-driver:unanalyzed-cases}@*)
    (__red>eval-top-level<red__
      (run-post-passes
        (check (run-pre-passes expr)
               expected)))))
(defn check-top-level [form expected]
  (check-expr (__red>unanalyzed-top-level<red__ form (empty-env))
              expected))
\end{cljlisting}
  \caption{Handling top-level forms in a checker based on \clj{core.typed.analyzer}}
  \label{fig:analyzer:core.typed.analyzer-top-level-driver}
\end{figure}
