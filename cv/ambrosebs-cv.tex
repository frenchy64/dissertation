% LaTeX source of my resume
% =========================

% Commented for easy reuse... ;)

% See the `README.md` file for more info.

% This file is licensed under the CC-NC-ND Creative Commons license.


% Start a document with the here given default font size and paper size.
\documentclass[11pt]{article}

% Set the page margins.
\usepackage[top=1in,bottom=1in,left=1in,right=1in]{geometry}

% Setup the language.
\usepackage[english]{babel}
\hyphenation{Some-long-word}

% Makes resume-specific commands available.
\usepackage{resume}

\begin{document}  % begin the content of the document
\sloppy  % this to relax whitespacing in favour of straight margins


% title on top of the document
\maintitle{CV: Ambrose Bonnaire-Sergeant}{Last update on \today}

\nobreakvspace{0.3em}  % add some page break averse vertical spacing

% \noindent prevents paragraph's first lines from indenting
% \mbox is used to obfuscate the email address
% \sbull is a spaced bullet
% \href well..
% \\ breaks the line into a new paragraph
\noindent\href{mailto:abonnairesergeant.at.gmail.dot.com}{abonnairesergeant\mbox{}[at]\mbox{}gmail.com}\sbull
%\textsmaller{+}31.646469087\sbull
\href{http://ambrosebs.com/}{ambrosebs.com}\sbull
\href{https://github.com/frenchy64}{github.com/frenchy64}
%\\
%Luddy Hall, Room 3025\sbull
%700 N Woodlawn Ave \sbull
%Bloomington \sbull 
%IN 47408 \sbull
%USA


%\spacedhrule{0.9em}{-0.4em}  % a horizontal line with some vertical spacing before and after

\roottitle{Education}

\headedsection
  {\href{http://www.indiana.edu}{Indiana University Bloomington}}
  {\textsc{Bloomington, Indiana}} {%
  \headedsubsection
    {Ph.D. in Computer Science 
    %(candidate)
    }
    {%Fall 
    2014 --- 
   % Spring 
    2019 %(expected)
    }
    {}
  \headedsubsection
    {Master of Science in Computer Science}
    {2014 ---
    2017}
    {}
}

\headedsection
  {\href{http://www.uwa.edu.au}{University of Western Australia}}
  {\textsc{Western Australia, Australia}} {%
  \headedsubsection
    {BSc in Computer Science with Honours}
    {2008 ---
    2013}
    {}
}

\roottitle{Employment}

\headedsection
  {Software Developer (Remote Intern)}
  {\textsc{Sparkfund, }}
  {%
      {\textit{May 2017 --- Febuary 2018}}
      {\bodytext{Studied verification techniques used in practice (Clojure).
      \begin{itemize}
        \item enhanced interactive pricing estimation tools using ClojureScript and om.next
        \item produced internal pricing reports in collaboration with sales team using Clojure and Google Sheets
        \item parallelized the CI builds of several projects with a migration to CircleCI 2.0 Workflows
        \item completed company-wide upgrade to a version of Clojure with breaking changes
        %\item 
          %and participated in daily standup, pair programming,
          %weekly 1-1's with managers,
          %and monthly planning meetings
        %\item worked remotely
        %\item used Clojure, ClojureScript, om.next, Google Sheets, Salesforce, Asana, Boot
      \end{itemize}
      }}
  }

\headedsection
  {{Research Assistant / Assistant Instructor}}
  {\textsc{Indiana University Bloomington, }} {%
    {\textit{Fall 2014---Spring 2019}}
    {
    \begin{itemize}
      \item published peer reviewed papers about Typed Clojure and clojure.spec
      \item taught undergraduates introductory programming based on How To Design Programs (C211) 
        and introductory data structures (C343)
    \end{itemize}
    }
}

%\headedsection
%  {{Research Assistant}}
%  {\textsc{Indiana University Bloomington}} {%
%  \headedsubsection
%    {%Advisor: Sam Tobin-Hochstadt
%    }
%    { Fall 2014---Summer 2015, Spring 2016---Fall 2016, 
%    Summer 2018 --- Fall 2018, Spring 2019}
%    {\bodytext{
%    \begin{itemize}
%      \item Produced papers about Typed Clojure and clojure.spec.
%    \end{itemize}
%    }
%     %\bodytext{
%     %   We have worked to formalise and publish a paper on the fundamentals
%     %   of Typed Clojure, and worked towards extending Typed Clojure with gradual typing.
%     %   More recently, we developed a
%     %   framework to infer type annotations for optional type systems
%     %   based on \href{http://ambrosebs.com/auto-ann.html}{runtime instrumentation}.
%     % }
%    }
%}

%\headedsection
%  {{Assistant Instructor}}
%  {\textsc{Indiana University Bloomington}} {%
%  \headedsubsection
%    {Instructors: Sam Tobin-Hochstadt, Suzanne Menzel, Chung-chieh Shan}
%    {F.~2015, F.~2016---Sp.~2018}
%    {
%    \bodytext{Taught undergraduates C211, an introductory programming
%      language course based on How To Design Programs.
%			Also C343, introductory data structures.
%    Conducted labs and office hours.
%    }
%    }
%}

\headedsection
  {Analyst Programmer}
  {\textsc{University of Western Australia, }}
  {%
    {\textit{2010 --- 2011}}
    {
      \begin{itemize}
        \item conducted interviews of library staff to identify
          the format and location of research data
        \item set up VIVO semantic web software, utilizing Java, XSLT,
          bash, and Linux
      \end{itemize}
    }
  }

\roottitle{Open Source Work}

\headedsection
  {Crowdfunded Open Source work}
  {\textsc{Perth, Western Australia}} {%

  \headedsubsection
  {\href{https://www.indiegogo.com/projects/typed-clojure-clojure-spec-auto-annotations}{Automatic
    Annotations for Typed Clojure and clojure.spec}
    (\$8,621 USD raised by 69 backers)
  }
    {2016}
    {\bodytext{
      This campaign concentrated on automating the manual labor of type and spec annotations.
      It helped me attend multiple industry conferences to meet real users
      of Typed Clojure and clojure.spec to discuss the needs of the community.
      }}

  \headedsubsection
  {\href{https://www.indiegogo.com/projects/gradual-typing-for-clojure}{Gradual Typing for Clojure}
    (\$11,695 USD raised by 199 backers)
  }
    {2015}
    {\bodytext{These funds helped me design a \emph{gradual typing} framework for Typed Clojure.
      %which is still in development. 
      It supports automatic contracts for
      global variables based on the typed-untyped boundary, as well as allowing
      arbitrary exporting of macros from typed namespaces---a shortcoming of previous systems.
      }}

  \headedsubsection
    {\href{https://www.indiegogo.com/projects/typed-clojure}{Typed Clojure}
    (\$35,254 USD raised by 545 backers)
  }
    {2013}
    {\bodytext{
      These funds helped support me as I worked on improving 
      and extending Typed Clojure's type system.
      \$5,000 of these funds were used to commission
      further open source work on tools.analyzer, now
      an important Clojure library.
    }}
}
\headedsection
  {\href{http://ambrosebs.com/\#gsoc}{Google Summer of Code}} {\textsc{Clojure organisation}} {
  \headedsubsection
    {Student} {2012, 2013}
    {}
    {\bodytext{
      I developed Typed Clojure, improving the support of
      Clojure idioms, supporting more expressive types,
      and adding documentation.
      }
    }
    {}
  \headedsubsection
    {Mentor} {2014 (3 students), 2015 (2 students)}
    {}
    {\bodytext{
        I have mentored 5 projects, including several adding
        advanced type system features to Typed Clojure,
        and work towards a ClojureScript compiler using
        tools.analyzer as a backend.
      }}
    {}
  \headedsubsection
    {Administrator} {2014, 2015}
    {}
    {\bodytext{
      My role as an administrator included proposing
      projects, preparing and reviewing application documents,
      and advertising for interested students.
      }}
    {}
}

\headedsection{Selected Open Source Contributions}{
\headedsection
  {\href{https://github.com/clojure/core.typed}{core.typed}}
  {\textsc{Lead Developer}} {%
    {
      \bodytext{
      Designed, implemented, and maintain
        the main library of Typed Clojure.
      }
    }
}
\headedsection
  {\href{https://github.com/clojure/core.match}{core.match}}
  {\href{https://github.com/clojure/core.match/commits?author=frenchy64}{\textsc{Contributor}}} {%
    {\bodytext{
        Studied optimising pattern matching literature and collaborated
        on a pattern matcher for Clojure.
      }
    }
}

\headedsection
  {\href{https://github.com/clojure/clojure}{Clojure}}
  {\href{https://github.com/clojure/clojure/commits?author=frenchy64}{\textsc{Contributor}}} {%
    {\bodytext{
      Contributed several minor patches improving error messages,
      fixing bugs, and adding features to the core Clojure language.
      }
    }
   }

\headedsection
  {\href{https://github.com/clojure/clojurescript}{ClojureScript}}
  {\href{https://github.com/clojure/clojurescript/commits?author=frenchy64}{\textsc{Contributor}}} {%
    {\bodytext{
      Converts the internal AST representation of ClojureScript's compiler
      to tools.analyzer format. Various bug fixes and performance enhancements.
      }
    }
   }
%\bodytext
%  {\href{https://github.com/clojure/clojure/commits?author=frenchy64}{Clojure},
%   \href{https://github.com/clojure/clojurescript/commits?author=frenchy64}{ClojureScript},
%   \href{https://github.com/clojure/core.match/commits?author=frenchy64}{core.match}
%  % ,
%   %\href{tools.analyzer.jvm}
%   }
   }

%\headedsection
%  {\href{https://github.com/clojure/clojurescript}{ClojureScript}}
%  {\href{https://github.com/clojure/clojurescript/commits?author=frenchy64}{\textsc{Contributor}}}




\roottitle{Research Groups}

\headedsection
  {\href{http://wonks.github.io/}{PL Wonks}}
  {\textsc{Indiana University Bloomington}} {%
}

\headedsection
  {\href{http://prl.ccs.neu.edu/gtp/index.html}{Gradual Typing Group}}
  {\textsc{Indiana University Bloomington}} {%
}


\roottitle{Conference Publications}

%\headedsection
%  {\href{https://frenchy64.github.io/papers/submitted-pldi19-squash-work-draft.pdf}{Squash the work! Inferring Recursive Type Annotations from Plain Data for Optional Type Systems}
%  (In submission)}
%  {\textsc{with Sam Tobin-Hochstadt}} {%
%    {}
%}

\headedsection
  {\href{http://frenchy64.github.io/papers/esop16-short.pdf}{Practical Optional Types for Clojure}
  (ESOP'16)}
  {\textsc{with Rowan Davies, Sam Tobin-Hochstadt}} {%
    {}
}

\roottitle{Theses}

\headedsection
  {\href{https://s3.amazonaws.com/github/downloads/frenchy64/papers/ambrose-honours.pdf}{A Practical Optional Type System for Clojure} }
  {\textsc{supervised by Rowan Davies}} {%
  \bodytext{Undergraduate Honours dissertation.}
}

%\spacedhrule{1.6em}{-0.4em}

%\vspace{0.5em}
%\inlineheadsection
%  {Natural languages:}
%  {Dutch \emph{(mother tongue)}, English \emph{(full professional proficiency)}, German \emph{(limited working proficiency)}, French \emph{(elementary proficiency)} and Mandarin Chinese \emph{(beginner)}.}
%

%\spacedhrule{1.6em}{-0.4em}



\roottitle{Selected Talks}

\headedsection
  {\href{https://www.youtube.com/watch?v=kcZVkvO1Dpo}{Tool-assisted spec Development}}
  {\textsc{Clojure/conj 2017}} {%
}
\headedsection
  {\href{http://ambrosebs.com/talks/esop16.pdf}{Practical Optional Types for Clojure}}
  {\textsc{ESOP 2016}} {%
  }
\headedsection
  {\href{https://www.youtube.com/watch?v=yG9CffLlXx0}{Typed Clojure: From Optional to Gradual Typing}}
  {\textsc{Strangeloop 2015}} {%
}
\headedsection
  {\href{https://www.youtube.com/watch?v=a0gT0syAXsY}{Typed Clojure in Practice}}
  {\textsc{Strangeloop 2014}} {%
}
\headedsection
  {\href{https://www.youtube.com/watch?v=a0gT0syAXsY}{Typed Clojure}}
  {\textsc{Clojure/conj 2012}} {%
}
\headedsection
  {\href{https://www.youtube.com/watch?v=irjP8BO1B8Y}{Introduction to Logic Programming}}
  {\textsc{Clojure/conj 2011}} {%
}

%\roottitle{Programming languages}
%
%\bodytext
%  {
%        Proficient in Clojure. Experience with Java, C, Racket, Coq, and Agda.
%        }
%
\roottitle{Service}

\bodytext
  {
\begin{itemize}
    %already mentioned above
  %\item Google Summer of Code Administrator for Clojure (2014-2015)
  \item Organizer of \href{http://wonks.github.io}{PL Wonks}, Indiana University's weekly Programming Languages seminar
    (2017-2019)
  \item President of Indiana University Graduate Computer Science Association
    (2017-2019)
\end{itemize}
    }

%\bodytext
%  {Experienced with open source development and maintenance, 
%    including Git, GitHub, and continuous integration.
%    Enjoys building static and dynamic
%    program verification tools for dynamically typed languages.
%    Experienced Clojure programmer.
%    Some experience in Java, C, Racket, and Scheme.
%  }

%\roottitle{Personal}
%
%\bodytext
%  {I live in Bloomington, Indiana and have a US work permit.
%  }


%\headedsection{Interests}
%{\bodytext
%  {Singing, guitar, reading, teaching.}
%}


\end{document}
