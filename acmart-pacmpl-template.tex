%% For double-blind review submission
\documentclass[acmlarge,review,anonymous]{acmart}\settopmatter{printfolios=true}
%% For single-blind review submission
%\documentclass[acmlarge,review]{acmart}\settopmatter{printfolios=true}
%% For final camera-ready submission
%\documentclass[acmlarge]{acmart}\settopmatter{}

%% Note: Authors migrating a paper from PACMPL format to traditional
%% SIGPLAN proceedings format should change 'acmlarge' to
%% 'sigplan,10pt'.


%% Some recommended packages.
%\usepackage{booktabs}   %% For formal tables:
%                        %% http://ctan.org/pkg/booktabs
%\usepackage{subcaption} %% For complex figures with subfigures/subcaptions
%                        %% http://ctan.org/pkg/subcaption
\usepackage{amsmath}
\usepackage{mathpartir}
\usepackage{alltt,fancyvrb}
\usepackage{mmm}
\usepackage{style}
\usepackage{tikz}
\usepackage{tikz-qtree}
\bibliographystyle{abbrvnat}
\include{bibliography.bib}

\newcommand{\cL}{{\cal L}}



\makeatletter\if@ACM@journal\makeatother
%% Journal information (used by PACMPL format)
%% Supplied to authors by publisher for camera-ready submission
\acmJournal{PACMPL}
\acmVolume{1}
\acmNumber{1}
\acmArticle{1}
\acmYear{2017}
\acmMonth{1}
\acmDOI{10.1145/nnnnnnn.nnnnnnn}
\startPage{1}
\else\makeatother
%% Conference information (used by SIGPLAN proceedings format)
%% Supplied to authors by publisher for camera-ready submission
\acmConference[PL'17]{ACM SIGPLAN Conference on Programming Languages}{January 01--03, 2017}{New York, NY, USA}
\acmYear{2017}
\acmISBN{978-x-xxxx-xxxx-x/YY/MM}
\acmDOI{10.1145/nnnnnnn.nnnnnnn}
\startPage{1}
\fi


%% Copyright information
%% Supplied to authors (based on authors' rights management selection;
%% see authors.acm.org) by publisher for camera-ready submission
\setcopyright{none}             %% For review submission
%\setcopyright{acmcopyright}
%\setcopyright{acmlicensed}
%\setcopyright{rightsretained}
%\copyrightyear{2017}           %% If different from \acmYear


%% Bibliography style
\bibliographystyle{ACM-Reference-Format}
%% Citation style
%% Note: author/year citations are required for papers published as an
%% issue of PACMPL.
\citestyle{acmauthoryear}   %% For author/year citations



\begin{document}

%% Title information
%\title[Short Title]{Full Title}         %% [Short Title] is optional;
%                                        %% when present, will be used in
%                                        %% header instead of Full Title.
%\title{Dynamic Type Inference for Recursive Structural Types}
%\title{ from unit tests to generative tests and types}
%\title{Generating generative tests}
%\title{Generating generative tests, then types}
%\title{From tests, to generative tests and types, automatically}
%\title{Squash the work}
\title{Squash the work: Generating useful optional types}
%\title{Generating useful types for optionally typed languages}
%\subtitle{Generating useful optional types}
%\subtitle{Ease the porting process}
%\titlenote{with title note}             %% \titlenote is optional;
%                                        %% can be repeated if necessary;
%                                        %% contents suppressed with 'anonymous'
%\subtitle{Subtitle}                     %% \subtitle is optional
%\subtitlenote{with subtitle note}       %% \subtitlenote is optional;
%                                        %% can be repeated if necessary;
%                                        %% contents suppressed with 'anonymous'


%% Author information
%% Contents and number of authors suppressed with 'anonymous'.
%% Each author should be introduced by \author, followed by
%% \authornote (optional), \orcid (optional), \affiliation, and
%% \email.
%% An author may have multiple affiliations and/or emails; repeat the
%% appropriate command.
%% Many elements are not rendered, but should be provided for metadata
%% extraction tools.

%% Author with single affiliation.
\author{Ambrose Bonnaire-Sergeant}
%\authornote{with author1 note}          %% \authornote is optional;
                                        %% can be repeated if necessary
%\orcid{nnnn-nnnn-nnnn-nnnn}             %% \orcid is optional
\affiliation{
  %\position{Position1}
  \department{Computer Science}              %% \department is recommended
  \institution{Indiana University}            %% \institution is required
  %\streetaddress{Street1 Address1}
  %\city{City1}
  %\state{State1}
  %\postcode{Post-Code1}
  \country{USA}
}
\email{abonnair@indiana.edu}          %% \email is recommended

%% Author with two affiliations and emails.
\author{Sam Tobin-Hochstadt}
%\authornote{with author2 note}          %% \authornote is optional;
                                        %% can be repeated if necessary
%\orcid{nnnn-nnnn-nnnn-nnnn}             %% \orcid is optional
\affiliation{
  %\position{Position2a}
  \department{Computer Science}             %% \department is recommended
  \institution{Indiana University}           %% \institution is required
  %\streetaddress{Street2a Address2a}
  %\city{City2a}
  %\state{State2a}
  %\postcode{Post-Code2a}
  \country{USA}
}
\email{samth@indiana.com}         %% \email is recommended
%\affiliation{
%  \position{Position2b}
%  \department{Department2b}             %% \department is recommended
%  \institution{Institution2b}           %% \institution is required
%  \streetaddress{Street3b Address2b}
%  \city{City2b}
%  \state{State2b}
%  \postcode{Post-Code2b}
%  \country{Country2b}
%}
%\email{first2.last2@inst2b.org}         %% \email is recommended


%% Paper note
%% The \thanks command may be used to create a "paper note" ---
%% similar to a title note or an author note, but not explicitly
%% associated with a particular element.  It will appear immediately
%% above the permission/copyright statement.
%\thanks{with paper note}                %% \thanks is optional
                                        %% can be repeated if necesary
                                        %% contents suppressed with 'anonymous'


%% Abstract
%% Note: \begin{abstract}...\end{abstract} environment must come
%% before \maketitle command
\begin{abstract}
This paper describes an approach helping programmers
move along the spectrum of dynamically checked
and statically checked programs. All that we
require is a way to run the program, via unit testing or
test runs.

We test our approach in Clojure, a dynamically typed
language with a culture for unit testing.
First we generate generative tests by running
your program and emitting clojure.spec annotations.
Then, we run your new generative tests
to exercise more parts of the program. Using the extra
information, we then generate Typed Clojure static type
annotations. We then use this information to type check
your program.
Manual intervention in needed in each step to supervise
the output. We investigate the amount needed in real programs.

This paper also describes a novel algorithm that recovers
recursively defined records, and evaluates it on real Clojure
programs.

%  The untyped-typed porting process is costly, but
%  tools to smooth this transition are scarce.
%  We isolate the process of writing static type
%  annotations for untyped top-level variables, often manual and tedious.
%  Programmers must first annotate their own variables.
%  Even then, annotate used libraries. 
%  Worse still, annotate variables in the macroexpansion of imported macros.
%
%  In this paper, we explore a
%  tool dyinfer to generate type annotations.
%  What makes a good annotation?
%  Annotations should be readable, compact,
%  and capture the essential structure of
%  the running program.
%  Annotations should help programmers
%  type check their programs
%  by capturing relevant usages in the current 
%  context.
%  We aim to generate mostly-good annotations
%  that require little engineering effort
%  to correct, of which most is driven by
%  static type errors.
%
%  Our tool instruments running programs
%  and summarises observed values, outputing type annotations.
%  We handle higher-order functions
%  and record-like constructs.
%  To improve readability and make annotations
%  more useful for type checking programs, we generate
%  recursive union types when appropriate.
%
%  We apply our algorithm to Clojure programs
%  to generate Typed Clojure annotations.
%  We show the resulting annotations are often
%  adequate to type check usages.
\end{abstract}


%% 2012 ACM Computing Classification System (CSS) concepts
%% Generate at 'http://dl.acm.org/ccs/ccs.cfm'.
%\begin{CCSXML}
%<ccs2012>
%<concept>
%<concept_id>10011007.10011006.10011008</concept_id>
%<concept_desc>Software and its engineering~General programming languages</concept_desc>
%<concept_significance>500</concept_significance>
%</concept>
%<concept>
%<concept_id>10003456.10003457.10003521.10003525</concept_id>
%<concept_desc>Social and professional topics~History of programming languages</concept_desc>
%<concept_significance>300</concept_significance>
%</concept>
%</ccs2012>
%\end{CCSXML}

%\ccsdesc[500]{Software and its engineering~General programming languages}
%\ccsdesc[300]{Social and professional topics~History of programming languages}
%% End of generated code


%% Keywords
%% comma separated list
%\keywords{keyword1, keyword2, keyword3}  %% \keywords is optional


%% \maketitle
%% Note: \maketitle command must come after title commands, author
%% commands, abstract environment, Computing Classification System
%% environment and commands, and keywords command.
\maketitle


% Paper outline
%
% Introduction
% - set the scene for inferring types
%   - Typed Clojure
%   - optional/gradual typing requires annotations
% - give introductory example
%   - generate types + specs
%   - show delta needed to typecheck
% - enumerate our contributions
% - signpost the rest of the paper
% Overview
% - show off the inference algorithm
% - simple example
% - higher order functions
% - talk about join
% - talk about track
% - talk about update
% - maps
%   - optional keys
%   - when to throw away info
%   - how to merge maps
% - recursive example
%   - common keys
%   - problems: multiple "tags"
% - names
%   - alias names
%     - map tags propagate down into names
%   - spec names (using function arguments)
% - visualise recursive algorithm
%   - horizonal + vertical squashing
% Formal model
% - semantics
% - track
% - union
% - join
% - update
% - type syntax
% Implementation
% - optimisations
%   - depth limit
% - handling pre-existing annotations
% - spec generation
% Experience
% - experiments
%   1. manual inspection
%   2. Generate types + type check 
%   3. Gen spec -> run spec -> run tests -> gen tests
%   4. Gen types -> gradual typing
% Related work
% - see runtime infer github
% Conclusion

\input{intro}

\input{grammar}


%\begin{verbatim}
%(let [lib (fn [x y] x)
%      id (track (fn [x] x) [id])]
%  (id (track lib [lib])))
%\end{verbatim}
%
\input{semantics}

\input{overview}

\input{track}

\section{The algorithm}

\subsection{Step 1}

Where: \emph{make-Union}

This step begins by merging HMap's 
and functions that are contained in the same union.

Algorithm: Traverse each union and merge members
of the same union with the following merge strategy.

HMaps with 100\% same keys are merged, with
entries being the union of the original HMaps.
In the special case where we can guess a common
dispatch key for all maps 
(a common entry mapped to a keyword),
we instead merge using the value of that entry.

If there is a mix of collection and non-collection
types, we upcast the entire union to Any.
From experience, these combinations are rare in
non-polymorphic functions, so instead we require
a polymorphic type variable, which is future work.

Functions with the same arities are merged, with
the new domains being the join of the domains
point-wise, and the new range being the join
of the old ranges.

Assumption: No optional keys. No recursive types.

\subsection{Step 2.a}

Where: \emph{alias-HMap}

Relabel graph nodes to keysets.

Traverse graph up from leaves.
Label each HMap as a singleton set containing
its keyset.
Label each union as a set of keysets of its
members. Delete HMap labels of its members.

There are effectively two types of edges.
"Union edges" don't really exist, since
they're on the same level as their members.
Other edges are normal, thus we don't delete
HMaps labels they point to.

Assumption: No recursive types.

\subsection{Step 2.b}

Where: \emph{squash-all}

Vertical recursive types.

Now recursive types are derived by
depth traversal of the types.

We start at the top of each branch and remember
the keysets of the current node.
We then traverse down depth-first and remember
the keyset of each subsequent node. 
If we find two nodes that have overlapping set of keysets,
we merge them.

The set of keysets of a HMap are every union of the keys
in its required entry set and the subsets of the
keys in its optional entry set.
The set of keysets of a type is the union of
all the keysets in its members (eg., for a union type,
take the union of the set of keysets of its members).

Here is the merge algorithm:

Given nodes m and n, where non-empty(intersection(keysets(m), keyset(n))),
and m was found before n.

1. Redirect all in/out edges on n to m.
2. Delete node n.
3. Let type(m) = join(type(m), type(n))
4. Let label(m) = union(label(m), label(n))

Question: This algorithm could run until a fixed point.
Should it? The next step does something similar that
merges types in different graph forests, but does
not handle recursive types.

Assumption: No recursive types before we start 
(?but surely as we go?)

\subsection{Step 3.a}

Where: \emph{alias-single-HMaps}

Relabel nodes to HMap keyset.

Relabel each node to its keyset if it is
a HMap, otherwise do not label.

\subsection{Step 3.b}

Where: \emph{squash-horizontally}

Horizontally merge similar types.

Given a collection of sets of labelled nodes (to HMap
keysets), in each set, choose just the types
that are not recursive (by traversing the type and
noticing if a cycle is made to the original label)
and merge these nodes into the same graph.

The merge is performed by joining the types
of all the candidate nodes to a "new" node. Then reroute all
in/out edges belonging to the candidate
node to the "new" node.

Note: This only merges non-recursive nodes.

\subsection{Step 4}

Where: \emph{follow-all}

Remove aliases that just point to other aliases.

\section{Formalism}

We create a $\lambda$-calculus with multi-arity functions,
$\lambda \overrightarrow{x}. e$,
plain maps
$\{\overrightarrow{e\ e}\}$
and
numbers $n$.
The \textbf{track} function
$(\text{track}\ e\ \overrightarrow{pth})$
takes a value and a path and returns 
a wrapped value.

The big-step operational semantics
$\rho \vdash e \Downarrow v\ ; res$
says under runtime environment $\rho$,
expression $e$ evaluates to value $v$
with inference results $res$.

\section{Evaluation}

We performed a qualitative evaluation of our algorithm
on several open source programs.

Our approach is to carry out three experiments.

\subsection{Experiment 1: Manual inspection}

In the first experiment, we simply run our 
type inferencer on each program after executing
their tests.
We then manually inspect the resulting
types and compare them to some ideal
annotations.

We rate the quality of generated annotations
on several axes.

\paragraph{Compactness} Type annotations should be succinct,
        but without sacrificing too much accuracy.
        Are our type aliases intelligently combined
        with good choices for optional keys?

  \paragraph{Accuracy} Would executing a program with these
      type annotations cause an error?
      Have we too eagerly erased information in favor
      of compactness?

  \paragraph{Organization} Have we chosen good recursive types?
      Do they have good names?

Our first program is an implementation of a
1971 Star Trek game.
It comes with minimal tests, so to complete this experiment,
we instead played the game for 30 seconds.

\subsection{Experiment 2: Delta needed to type check}

\subsection{Experiment 3: Generating generative tests}

% We should generate the card playing specs in this guide:
% http://clojure.org/guides/spec

% # How evaluate
% ## qualitative
% Does it make sense??
% 
% 1. Don't run, gen type, manual inspection
%   - done on something small but real
%   - star trek game?
% 
% - Try different eval methods on different programs
%   - try different projects on different methods
%
% 2. Generate types, try type checking programs
%   - record what changes needed to get it to
%     type check 
%   - (on a different program than 1.) 
% 
% 3. Generate spec, insert the spec, run the test
%    with the spec on, also generate tests
%   - does spec ignore the input??
%     or just generate tests
%   - best situation:
%     - spec all passes
%     - then types check with minimal changes
%   - Q: can we use spec's tests to improve
%        types, iteratively?
%        (could throw away exceptions, throw
%         away bad input etc., different options
%         here)
% (optional)
% 4. Generate types, use gradual typing


\input{related-work}

\section{Conclusion}

%\appendix
%\section{Appendix Title}
%
%This is the text of the appendix, if you need one.


\bibliography{bibliography}


\end{document}
