\section{Introduction}

% - set the scene for inferring types
%   - Typed Clojure
%   - optional/gradual typing requires annotations

This paper starts to address a major usability flaw
for gradually and optionally typed languages:
writing type annotations is a manual process.

Take \texttt{vertices} (Figure~\ref{vertices}),
written in Clojure.
As is good style, it comes with a unit test.
Our goal is to \textit{generate} Typed Clojure~\cite{typed-clojure}
annotations
for this function, relieving most of the annotation
burden.

Our approach features several stages.
First, we \textit{instrument} top-level functions
(Section \ref{instrument-TODO}),
then run the unit tests and \textit{track}
how they are used at runtime
(Section \ref{track-TODO}).
At this point, we have a preliminary
annotation:
%
\begin{Verbatim}
(ann vertices ['{:op ':node, :left '{:op ':leaf, :val Int}, :right '{:op ':leaf, :val Int}} -> Int]})
\end{Verbatim}
%
This type is too specific---trees are recursively
defined---we \textit{squash} types to be
recursive from example unrollings (Section \ref{recursive-TODO}):

\begin{Verbatim}
(defalias NodeLeaf 
  (U '{:op ':node :left NodeLeaf :right NodeLeaf}
     '{:op ':leaf :val Int}))
(ann vertices [NodeLeaf -> Int])
\end{Verbatim}
%
%\begin{Verbatim}
%(declare Node Leaf)
%(defalias NodeLeaf (U Node Leaf))
%(defalias Node 
%  '{:op ':node :left NodeLeaf :right NodeLeaf})
%(defalias Leaf '{:op ':leaf :val int})
%(ann verbatim [NodeLeaf -> Int]})
%\end{Verbatim}
%
If \texttt{NodeLeaf} is used in multiple positions
in the program, local recursive types are redundant.
In this paper, we name and \textit{merge} recursive
types, reusing them in annotations.
%
\begin{Verbatim}
(ann vertices [NodeLeaf -> Int])
(ann sum-tree [NodeLeaf NodeLeaf -> NodeLeaf])
\end{Verbatim}
%
If minor variants of the recursive types occur
across a program,
we use \textit{optional} entries~\cite{typed-clojure}
to reduce redundancy (Section \ref{optional-merge-TODO}).
%
\begin{Verbatim}
(defalias NodeLeaf 
  (U '{:op ':node :left NodeLeaf :right NodeLeaf}
     (HMap :mandatory {:op ':leaf :val Int}
           :optional {:label Str})))
\end{Verbatim}

After inserting these annotations, we can run the
type checker over them to check their usefulness.
We found annotations to be readable and minimize
redundancy compared to hand-written annotations
(Section \ref{experiment1}).
Minimal changes were needed to successfully type check
functions with the generated annotations,
mostly consisting of local function and loop annotations,
and renaming of type aliases
(Section \ref{experiment2}).
Generating and running \textit{tests} improved the quality
of type annotations by exercising more paths through the
program (Section \ref{experiment3}).

Several open questions remain.
Automatically
drawing the typed-untyped boundary in gradual typing
would mean less manual casts are needed.
(Section \ref{boundaries}).

%The Clojure programming language has several verification
%systems that require annotating your programs.
%Typed Clojure is a type system that supports many Clojure
%idioms. Here, we must provide type annotations for
%top level variables, local functions, and invoked libraries.
%Clojure.spec is a pseudo contract system
%that can also generate tests.
%Similarly, specifications (``specs'') must be provided
%for all top level variables.
%
%These annotations are useful for learning about our programs,
%but they can be burdensome to write and maintain.
%Currently, one must reverse engineer annotations
%by visual analysis of the source code.
%
%In this paper, we present a tool that automatically
%generates annotations, based on the tests already present
%in idiomatic Clojure programs.
%These annotations are readable, compact, feature good
%names, and recover recursively defined records.
%There is no guarantee the generated annotations will
%immediately type check, however.
%
%Our goal is to minimize the difference needed
%to type check programs from the generated annotations.
%We envision programmers running our tool, generating
%a few dozen lines of annotations, and only a fraction
%of them should need manual changing to actually type
%check a program.

% - give introductory example
%   - generate types + specs
%   - show delta needed to typecheck
% - enumerate our contributions
% - signpost the rest of the paper


\subsection*{Contributions}
\begin{itemize}
\item We outline a generalized approach to automatically
    generating type annotations.
%\item
%  Our main contribution is a robust, easy to use, open source tool that 
%  Clojure programmers can use to help learn about and specify 
%  their programs.
\item
  We describe a novel approach to reconstructing recursively
  defined structural records from fully unrolled examples.
\item
  We report our experience using this algorithm to generate
  types, tests, and contracts on several
  Clojure libraries and programs.
\item
  We include a formal model of our inference algorithm.
\end{itemize}

\begin{figure}
\begin{Verbatim}
(defn vertices [t]
  (case (:op t)
    :leaf 1
    :node (+ 1 (vertices (:left t))
               (vertices (:right t)))))
(deftest vertices-test
  (is (= 3 (vertices {:op :node 
                      :left {:op :leaf :val 2}
                      :right {:op :leaf :val 3}))))
\end{Verbatim}
\caption{A function counting vertices in a tree,
with a test.}
\label{vertices}
\end{figure}
