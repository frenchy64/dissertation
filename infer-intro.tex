\Dchapter{Introduction}

\emph{It is better to have 100 functions operate on one data structure than 10 functions on 10 data structures.}---Alan Perlis

Optional type systems~\cite{bracha2004pluggable} extend existing untyped languages
with type checking.
For example,
TypeScript~\cite{typescript} and Flow~\cite{flow} extend
JavaScript,
Hack~\cite{hack} extends PHP,
and
mypy~\cite{mypy} extends Python.
They typically support existing syntax and idioms of their target languages.

Transitioning to an optional type system
requires adding type annotations to existing code,
a significant manual burden.
There has been interest in 
tooling to help create~\cite{saftoiu2010jstrace,pyannotate,typette18,An10dynamicinference,pytype} and
evolve~\cite{kristensen2017inference}
these annotations.

Clojure~\cite{Hic08} is an untyped language that compiles to the Java
Virtual Machine. Compared to languages already mentioned, it
strongly encourages programming with plain data structures, and is
a good example of implementing Perlis' advise in our opening quote.
Clojure provides
many functions and idioms around persistent, immutable hash-maps,
including literal map syntax \clj{\{k v ...\}}, interned \emph{keywords}
suitable both for map keys (e.g., \clj{:a}, \clj{:b}) and
functions that look themselves up in a map (e.g., \clj{(:a \{:a 1\}) => 1}),
and a suite of functions to deeply transform, manipulate, and validate
maps, with multimethods providing open extension.
Typed Clojure~\cite{bonnaire2016practical} is an optional type system for Clojure
designed to recognize these idioms given sufficient type annotations---which our
tool assists the programmer in writing.

\begin{figure}
\begin{cljlisting}
(defn nodes
  "Returns the number of nodes in a binary tree."
  [t] (case (:op t)
        :leaf 1
        :node (+ 1 (nodes (:left t))
                   (nodes (:right t)))))
(assert (= 3 (nodes
               {:op :node,
                :left {:op :leaf, :val 2},
                :right {:op :leaf, :val 3}})))
\end{cljlisting}
\caption{A typical use of maps in Clojure
to represent records,
that requires type annotations
to check with Typed Clojure.
Our tool %\texttt{core.typed.annotator}
will automatically annotate this program
with a useful recursive type (\figref{fig:infer:nodestype}).
}
\label{fig:infer:nodes}
\end{figure}

Maps in Clojure often replace records or objects,
demonstrated in \figref{fig:infer:nodes}:
instead of representing a binary tree with \texttt{Node} and \texttt{Leaf} classes,
they are encoded in maps
with an explicit keyword \emph{dispatch entry} (e.g., \clj{:op}) to distinguish
cases---\clj{\{:op :leaf, :val ...\}} for instances of \texttt{Leaf}, and
\clj{\{:op :node, :left ..., :right ...\}} for \texttt{Node}.

This emphasis on maps has far-reaching
implications for
Typed Clojure. %~\cite{bonnaire2016practical}, an optional type system for Clojure.
Types for maps (written
\clj{'\{:op ':leaf, :val ...\}} and
\clj{'\{:op ':node, :left ..., :right ...\}}
for our examples) combine with
ad-hoc union types and
equirecursive type aliases
in the type:
\begin{cljlisting}
(defalias Tree
  (U '{:op ':node, :left Tree, :right Tree}
     '{:op ':leaf, :val Int}))
\end{cljlisting}

Porting \figref{fig:infer:nodes} to Typed Clojure involves writing a type definition
\clj{Tree}, and annotating \clj{nodes} as
\begin{cljlisting}
(ann nodes [Tree -> Int])
\end{cljlisting}
Existing approaches to automatically inferring this information
do not help annotating these kinds of programs.
In common with most of these tools is
object orientation, % (JavaScript, Ruby, and Python)
and accessible class declarations.
Recursive types are then naturally inferred from local class invariants,
like a \texttt{Tree} class containing fields of type \texttt{Tree}.
Unfortunately, these approaches depend on these names to infer
further structure to their types, and so
do not extend to annotating plain recursive data
such as JSON or heterogeneous dictionaries.

To demonstrate the current state-of-the-art in automatic annotation tools for optional type systems,
we transliterate \figref{fig:infer:nodes} to JavaScript using plain objects. 
We use TypeWiz~\cite{typewiz} to generate TypeScript annotations via dynamic analysis,
as it is well maintained and generates comparable annotations
to similar tools~\cite{saftoiu2010jstrace,pyannotate,typette18,An10dynamicinference,pytype,kristensen2017inference}.

\begin{figure}
\begin{lstlisting}[language=JavaScript]
function nodes(t: {left: {op: string, val: number},
                   op: string,
                   right: {op: string, val: number}}
                | {op: string, val: number}) ...
\end{lstlisting}
%\begin{lstlisting}[language=JavaScript]
%function nodes(t: @{left: {op: string, val: number},
%                   op: string,
%                   right: {op: string, val: number}}
%                | {op: string, val: number}@) {
%  switch t.op {
%    case "node": 
%      return 1 + nodes(t.left) + nodes(t.right);
%    case "leaf": return 1;
%    default: throw t.op;
%  }
%}
%nodes({op: "node",
%       left: {op: "leaf", val: 1},
%       right: {op: "leaf", val: 2}}); //test
%\end{lstlisting}
\caption{TypeWiz's TypeScript annotations for \figref{fig:infer:nodes}.}
\label{fig:infer:typewiz}
\end{figure}

\figref{fig:infer:typewiz} shows the actual output of TypeWiz for a JavaScript translation of \figref{fig:infer:nodes}.
Unfortunately, the annotation is too specific: it only accepts trees of height 1 or 2.
For fair comparison, even a class-based translation to JavaScript with
a common \js{nodes} method yields a shallow annotation:
\begin{lstlisting}[language=JavaScript]
class Node {               class Leaf { 
  public left: Leaf;         public data: number;
  public right: Leaf;
  ...                        ...
}                          }
\end{lstlisting}
%\begin{lstlisting}[language=JavaScript]
%class Node {                class Leaf { 
%  public left: @Leaf@;          public data: @number@;
%  public right: @Leaf@;
%  //constructors omitted
%  nodes() {                   nodes() {
%    return 1 +                  return 1;
%     this.left.nodes() +      }
%     this.right.nodes();    }
%  }
%}
%new Node(new Leaf(1), new Leaf(2)).nodes(); //test
%\end{lstlisting}

Since TypeWiz uses dynamic analysis, it is faithfully
providing the exact types that are observed at runtime.
Unfortunately, incrementally better test coverage does not quickly converge
to useful annotations.
For example, if we add an example of a nested \js{Node}
on the just the left branch
%\begin{lstlisting}[language=JavaScript]
%new Node(new Node(...), new Leaf(...)).nodes();
%\end{lstlisting}
the annotation for the plain objects version is still not recursive (alarmingly, it instead
grows linearly in the number of nodes), and the class-based
version helpfully updates the type of \js{left} to \js{Leaf|Node},
but \js{right} still remains \js{Leaf}.
%\begin{lstlisting}[language=JavaScript]
%class Node {               class Leaf { 
%  public left: Leaf@|Node@;    public data: number;
%  public right: Leaf;        ...
%  ...                      }
%}
%\end{lstlisting}

%In Python, the PyType tool for generating mypy generates an
%even less interesting annotation.
%\begin{lstlisting}[language=python]
%def nodes(t) -> int: ...
%\end{lstlisting}
%In contrast, encoding \clj{nodes} as a method gives PyType
%\begin{lstlisting}
%class Leaf:
%    val = ...  # type: Any
%    def __init__(self, val) -> None: ...
%    def nodes(self) -> int: ...
%class Node:
%    left = ...  # type: Any
%    right = ...  # type: Any
%    def __init__(self, left, right) -> None: ...
%    def nodes(self) -> Any: ...
%\end{lstlisting}

%
%Existing automated annotation approaches cannot derive
%any further structure to the input of \clj{nodes}, other
%than its explicit runtime representation.
%In Typed Clojure~\cite{bonnaire2016practical} notation,
%they infer:
%
%\begin{cljlisting}
%(ann nodes ['{:op Keyword,
%              :left '{:op Keyword :val Int},
%              :right '{:op Keyword :val Int}}
%            -> Int])
%\end{cljlisting}

On the other hand,
our approach recognizes \figref{fig:infer:nodes} as traversing recursively defined
data with two cases, distinguished by the \clj{:op}
entry (\figref{fig:infer:nodestype}).

\begin{figure}
\begin{cljlisting}
(defalias Op 
  (U '{:op ':node, :left Op, :right Op}
     '{:op ':leaf, :val Int}))
(ann nodes [Op -> Int])
\end{cljlisting}
  \caption{Our tool's Typed Clojure annotation of \figref{fig:infer:nodes}}
\label{fig:infer:nodestype}
\end{figure}

Our approach is sensitive enough to
to compute optional keys for each constructor.
Adding a unit test that includes a \clj{:val}
entry in a \clj{:node} yields the annotation
%For example, the following call uses a \clj{:val}
%entry for in a \clj{:node} map:
%\begin{cljlisting}
%(nodes {:op :node,
%        :left {:op :node, :val 42, ...},
%        :right {:op :leaf, ...}})
%\end{cljlisting}
\begin{cljlisting}
(defalias Op 
  (U (HMap :mandatory
           {:op ':node, :left Op, :right Op}
           :optional {:val Int})
     '{:op ':leaf, :val Int}))
\end{cljlisting}

Furthermore, we aggressively combine recursive data used
in the same file.
Given a 3-way node example with the key \clj{:node3},
used as inut to a distinct function \clj{nodes'},
our tool generates the combined type shown in \figref{fig:infer:node3}.
%Say \clj{nodes'}
%is similar to \clj{nodes},
%but
%whose unit tests also use a 3-way node \clj{:node3}.
%%\begin{cljlisting}
%%(nodes' {:op :node3,
%%         :left {:op :leaf, :val 1}
%%         :mid {:op :leaf, :val 2}
%%         :right {:op :leaf, :val 3}})
%%\end{cljlisting}
%The information derived from observing the executions
%of both functions are combined based on
%the common dispatch key (\figref{fig:infer:node3}).

\begin{figure}
\begin{cljlisting}
(defalias Op 
  (U (HMap :mandatory
           {:op ':node, :left Op, :right Op}
           :optional {:val Int})
     '{:op ':node3, :left Op, :mid Op, :right Op}
     '{:op ':leaf, :val Int}))
(ann nodes [Op -> Int])
(ann nodes' [Op -> Int])
\end{cljlisting}
\caption{Optional entries and combined information across functions.}
\label{fig:infer:node3}
\end{figure}

%, where plain
%heterogeneous hash-maps are idiomatic over predefined classes or records.
%Typed Clojure, in turn, has extensive support for specifying and checking
%heterogeneous maps.
%However, the high annotation burden has put off industrial users in practice.
%

% - set the scene for inferring types
%   - Typed Clojure
%   - optional/gradual typing requires annotations

% Typed Clojure has interesting idioms.
% Similarly we have to take into account specific idioms,
% Must deal with same realities as ESOP
% - need HMap's because we know.
% What does TypeScript auto ann do with `nodes`?
% - not class based
% - try python systems
% 
% - need to guess nominal names
% - need to guess grouping
% - nominal types: don't need to think about recursion
%  - just give a name, indirection is directed in a big table of names
%  - humans are used to referring things by name

% other 

% a class invariant is a recursive property "for free"
% - check locally, get recursion for free

% nothing for free in our setting

% we have names but not nominal types

%This paper starts to address a major usability flaw
%for gradually and optionally typed languages:
%writing type annotations is a manual process.
%
%Take \texttt{nodes} (\figref{infer:fig:nodes}),
%written in Clojure.
%As is good style, it comes with a unit test.
%Our goal is to \textit{generate} Typed Clojure~\cite{bonnaire2016practical}
%annotations
%for this function, relieving most of the annotation
%burden.

%\begin{figure}
%\begin{cljlisting}
%(defn nodes [t]
%  (case (:op t)
%    :leaf 1
%    :node (+ 1 (nodes (:left t))
%               (nodes (:right t)))))
%(assert (= 3 (nodes
%               {:op :node 
%                :left {:op :leaf :val 2}
%                :right {:op :leaf :val 3}})))
%\end{cljlisting}
%\caption{A Clojure function counting tree nodes, with test.}
%\label{infer:fig:nodes}
%\end{figure}

%\begin{figure}
%\begin{cljlisting}
%(defalias Op (U '{:op ':node ':left Op ':right Op}
%                '{:op ':leaf ':val Int}))
%(ann nodes [Op -> Int])
%\end{cljlisting}
%\caption{Automatically inferred Typed Clojure annotations.}
%\label{infer:fig:nodestype}
%\end{figure}

%Our approach features several stages.
%First, we \textit{instrument} top-level functions
%(Section \ref{instrument-TODO}),
%then run the unit tests and \textit{track}
%how they are used at runtime
%(Section \ref{track-TODO}).
%At this point, we have a preliminary
%annotation:
%
%\begin{cljlisting}
%(ann vertices ['{:op ':node,
%                 :left '{:op ':leaf :val Int},
%                 :right '{:op ':leaf :val Int}}
%               -> Int])
%\end{cljlisting}
%
%This type is too specific---trees are recursively
%defined---we \textit{squash} types to be
%recursive from example unrollings (Section \ref{recursive-TODO}):
%
%\begin{cljlisting}
%(defalias Op 
%  (U '{:op ':node, :left Op, :right Op}
%     '{:op ':leaf, :val Int}))
%(ann vertices [Op -> Int])
%\end{cljlisting}

%\begin{Verbatim}
%(declare Node Leaf)
%(defalias Op (U Node Leaf))
%(defalias Node 
%  '{:op ':node :left Op :right Op})
%(defalias Leaf '{:op ':leaf :val int})
%(ann verbatim [Op -> Int]})
%\end{Verbatim}
%
%If \texttt{Op} is used in multiple positions
%in the program, local recursive types are redundant.
%In this paper, we name and \textit{merge} recursive
%types, reusing them in annotations.
% 
%\begin{cljlisting}
%(ann vertices [Op -> Int])
%(ann sum-tree [Op Op -> Op])
%\end{cljlisting}
% 
%If minor variants of the recursive types occur
%across a program,
%we use \textit{optional} entries%~\cite{typed-clojure}
%to reduce redundancy (Section \ref{optional-merge-TODO}).
% 
%\begin{cljlisting}
%(defalias Op 
%  (U '{:op ':node, :left Op, :right Op}
%     (HMap :mandatory {:op ':leaf :val Int}
%           :optional {:label Str})))
%\end{cljlisting}
%
%After inserting these annotations, we can run the
%type checker over them to check their usefulness.
%We found annotations to be readable and minimize
%redundancy compared to hand-written annotations
%(Section \ref{experiment1}).
%Minimal changes were needed to successfully type check
%functions with the generated annotations,
%mostly consisting of local function and loop annotations,
%and renaming of type aliases
%(Section \ref{experiment2}).
%Generating and running \textit{tests} improved the quality
%of type annotations by exercising more paths through the
%program (Section \ref{experiment3}).

%Several open questions remain.
%Automatically
%drawing the typed-untyped boundary in gradual typing
%would mean less manual casts are needed.
%(Section \ref{boundaries}).

%The Clojure programming language has several verification
%systems that require annotating your programs.
%Typed Clojure is a type system that supports many Clojure
%idioms. Here, we must provide type annotations for
%top level variables, local functions, and invoked libraries.
%Clojure.spec is a pseudo contract system
%that can also generate tests.
%Similarly, specifications (``specs'') must be provided
%for all top level variables.
%
%These annotations are useful for learning about our programs,
%but they can be burdensome to write and maintain.
%Currently, one must reverse engineer annotations
%by visual analysis of the source code.
%
%In this paper, we present a tool that automatically
%generates annotations, based on the tests already present
%in idiomatic Clojure programs.
%These annotations are readable, compact, feature good
%names, and recover recursively defined records.
%There is no guarantee the generated annotations will
%immediately type check, however.
%
%Our goal is to minimize the difference needed
%to type check programs from the generated annotations.
%We envision programmers running our tool, generating
%a few dozen lines of annotations, and only a fraction
%of them should need manual changing to actually type
%check a program.

% - give introductory example
%   - generate types + specs
%   - show delta needed to typecheck
% - enumerate our contributions
% - signpost the rest of the paper


\Dsection*{Contributions}
\begin{itemize}
\item We outline a generalized approach to automatically
  generating type annotations (\secref{infer:sec:overview}).
%\item
%  Our main contribution is a robust, easy to use, open source tool that 
%  Clojure programmers can use to help learn about and specify 
%  their programs.
\item
  We describe a novel approach to reconstructing recursively
  defined structural records from fully unrolled examples
  in a formal model of our inference algorithm (\secref{infer:sec:formalism}).
\item
  We show how to extend our approach with space-efficient and lazy
  runtime tracking (\secref{infer:sec:extensions}).
\item
  We report our experience using this algorithm to generate
  types, tests, and contracts on several
  Clojure libraries and programs (\secref{infer:chap:evaluation}).
\end{itemize}
