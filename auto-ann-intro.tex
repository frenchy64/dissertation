\subsection{Automatic Type Annotations}

% - set the scene for inferring types
%   - Typed Clojure
%   - optional/gradual typing requires annotations

We now shift gears from introducing Typed Clojure to
addressing a major usability flaw that many
gradually and optionally typed languages share (including Typed Clojure):
writing type annotations is a \emph{manual} process.
%
Take \texttt{vertices} (below, written in Clojure)
a function that returnes the number of vertices
in a tree of tagged hash-maps.
As is good style, it comes with a unit test.
Our goal is to \textit{automatically generate} semi-accurate Typed Clojure %~\cite{bonnaire2016practical}
annotations
for this function, relieving most of the annotation
burden.

\begin{Verbatim}
(defn vertices [m]
  (case (:op m)
    :leaf 1
    :node (+ 1 (:left m) (:right m))))

(deftest test-vertices
  (is (= 3 {:op :node
            :left {:op :leaf :val 42} 
            :right {:op :leaf :val 24}})))
\end{Verbatim}

Our approach to automatic annotations features several stages.
First, we \textit{instrument} top-level functions.
Then, we \textit{exercise} the code by running its unit tests and \textit{observe}
the runtime behavior of the program.
If we pause at this point, we have collected enough data 
to generate a preliminary annotation:
%
\begin{Verbatim}
(ann vertices ['{:op ':node, :left '{:op ':leaf, :val Int}, :right '{:op ':leaf, :val Int}} -> Int]})
\end{Verbatim}
%
However, this type is too specific: trees are recursively defined
and the argument type is difficult to read and maintain.
To remedy this, we attempt to roll recursive-looking
types to be recursive from their example unrollings.
For example, below we have generalized the preliminary annotation's
depth 2 tree to the recursive \texttt{NodeLeaf}.
%
\begin{Verbatim}
(defalias NodeLeaf 
  (U '{:op ':node :left NodeLeaf :right NodeLeaf}
     '{:op ':leaf :val Int}))
(ann vertices [NodeLeaf -> Int])
\end{Verbatim}
%
%\begin{Verbatim}
%(declare Node Leaf)
%(defalias NodeLeaf (U Node Leaf))
%(defalias Node 
%  '{:op ':node :left NodeLeaf :right NodeLeaf})
%(defalias Leaf '{:op ':leaf :val int})
%(ann verbatim [NodeLeaf -> Int]})
%\end{Verbatim}
%
Now, if \texttt{NodeLeaf} is used in multiple positions
in the program, we don't want to repeat its definition multiple times.
Our type inference algorithm attempts merge recursive
types found throughout the program, reusing them in annotations.
For example,
if another function \texttt{sum-tree} accepts two
trees, we want reuse \texttt{NodeLeaf} in both annotations
like so:
%
\begin{Verbatim}
(ann vertices [NodeLeaf -> Int])
(ann sum-tree [NodeLeaf NodeLeaf -> NodeLeaf])
\end{Verbatim}

If minor variants of the recursive types occur
across a program,
we use optional \texttt{HMap} entries %~\cite{bonnaire2016practical}
to reduce redundancy.
%
\begin{Verbatim}
(defalias NodeLeaf 
  (U '{:op ':node :left NodeLeaf :right NodeLeaf}
     (HMap :mandatory {:op ':leaf :val Int}
           :optional {:label Str})))
\end{Verbatim}
%
After inserting these annotations, we can run the
type checker over them to check their usefulness.
Ideally, minimal changes will be needed to successfully type check
functions with the generated annotations,
mostly consisting of local function and loop annotations,
and renaming of type aliases.
Annotations should also be readable and minimize
redundancy, even when compared to hand-written annotations.
We will test this hypothesis with case studies
(\secref{sec:casestudy}).
%Generating and running \textit{tests} improved the quality
%of type annotations by exercising more paths through the
%program. %(Section \ref{experiment3})


