\chapter*{Typed Clojure in Theory and Practice}

Typed Clojure is an optional type system for the Clojure programming language
that aims to check idiomatic Clojure code.

\section*{Thesis Statement}

My thesis statement is:

\begin{quote}
Typed Clojure is a sound and practical optional type system for Clojure.
%Typed Clojure is a sound and practical optional type system for Clojure and the process of porting to Typed Clojure can be partially-automated, and this automation technology can be repurposed to further reveal how Clojure is used in real projects.
%Typed Clojure is a sound and practical optional type system for Clojure, and we can relieve the annotation burden for such verification systems by automatically annotating programs based on their runtime behavior.
%Typed Clojure is a sound and practical optional type system for Clojure whose annotation burden is partially-automatable, and this automation technology can be repurposed to generate clojure.spec annotations and test its effectiveness in hundreds of projects.
% to study how Clojure is used in real projects.
%Typed Clojure is sound, practical, and its annotation burden is partially-automatable,
%and we can repurpose this annotation technology to
%answer broad questions about how Clojure is used.
%Typed Clojure is a well-founded and practical optional type system for Clojure
%whose useability can be improved by
%developing a tool to automatically generate type annotations,
%and, by repurposing this tool, we can study general Clojure idioms and practices across
%hundreds of projects by generating, running, and exercising clojure.spec runtime specifications.
\end{quote}

%I will support this thesis statement with the following:
%

\section*{Structure of this thesis}

This document progresses in several parts that support my thesis statement, presented in chronological order
in when they were developed.

Part~\ref{part:types} motivates and presents the core type system of Typed Clojure.

\begin{itemize}
  \item \emph{Typed Clojure is sound} We formalize Typed Clojure, including
    its characteristic features like hash-maps, multimethods, and Java interoperability,
    and prove the model type sound.
  \item \emph{Typed Clojure is practical} 
    \begin{itemize}
      \item We present an empirical study of real-world Typed Clojure usage
        in over 19,000 lines of code, showing its features correspond to actual usage patterns.
    \end{itemize}
    % our annotation approach can be repurposed to generate clojure.spec which can be \emph{run}.
    % use results to explore both the quality of the tool and to explore what the inferred specs
    % tell us about clj programs.
\end{itemize}

As a response to the results of this work, I pursued several research directions.

Part~\ref{part:autoann} presents a solution to lower the annotation burden in real-world Typed Clojure programs.
We formalize and implement a tool to automatically annotate types for top-level
user and library definitions, and empirically study the manual changes needed for the generated annotations
to pass type checking.

Part~\ref{part:spec} investigates clojure.spec, Clojure's official runtime verification library.
Released 5 years after Typed Clojure, clojure.spec represents a significant development in Clojure
verification. I investigate clojure.spec's design and compare it to Typed Clojure
by producing a formal model of clojure.spec and repurposing the automatic annotation tool 
from Part~\ref{part:autoann} to automatically produce clojure.spec annotations for real-world Clojure programs.
%
%\begin{enumerate}
%  \item The process of porting to Typed Clojure can be partially-automated, and this automation technology can be repurposed to further reveal how Clojure is used in real projects.
%    \begin{itemize}
%      \item \emph{Repurpose automation technology}
%        We describe how to automatically generate clojure.spec annotations (``specs'') for existing programs by reusing
%        most of the the infrastructure for automatic Typed Clojure annotations.
%        We present a formal model of clojure.spec (an existing and popular runtime verification tool for Clojure)
%        and implement the model in Redex.
%      \item \emph{Study how Clojure is used in real projects}
%        We conduct a study of general Clojure idioms and practices by generating, enforcing, and exercising specs
%        across hundreds of projects, as well as analyzing design choices in Typed Clojure's type system,
%        clojure.spec's features, and our automatic annotation tool.
%      \item \emph{Test effectiveness of clojure.spec annotation generation}
%        We test the effectiveness of our generated specs by generating, enforcing, and exercising specs
%        across hundreds of projects, as well as analyze design choices in Typed Clojure's type system and
%        clojure.spec's features.
%    \end{itemize}
%\end{enumerate}

% Evan Chang (Boulder)
% - symbolic exec + type checking
% David Fisher (Olin Shivers)
% - Lazy Delegation JSP paper
% Matthew (STL talk)
% - let's write a expander
% - scope sets
% add this as an alternative to clojure spec's research direction
% - sell as: Programs hard to TC
% use (comp (map inc) (map dec)) as an example

\section*{Previously published work}

Part~\ref{part:types} has been published:
%
\begin{itemize}
  \item Ambrose Bonnaire-Sergeant, Rowan Davies, Sam Tobin-Hochstadt.
        Practical Optional Types for Clojure.
        In \textit{European Symposium on Programming Languages and Systems}, 2016
\end{itemize}
