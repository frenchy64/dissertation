\Dchapter{Extensions}

\Dsection{Polymorphic types}

\begin{figure*}
\begin{mathpar}
  \begin{altgrammar}
   \e{} &::=& ...
       \alt \trackpolyE{\e{}}{\inferpath{}}{\e{}}
       \alt \genE{}
  \end{altgrammar}

  \infer [B-Gen]
  { \num{}' = \num{} + 1
  }
  { \bigstepgen{\opsem{}}{\num{}}{\genE{}}{\num{}'}{\{\}}{\num{}'}}


  \infer [B-TrackPoly]
  { \bigstepgen{\opsem{}}{\num0}{\e{}}{\v{}}{\res{}}{\num1} 
    \\\\
    \bigstepgen{\opsem{}}{\num1}{\ep{}}{\num{}}{\res{}'}{\num2}
    \\\\
    \trackpolymeta{}(\v{}, \inferpath{}, {\num{}}) = \vp{}\ ; \res{}'' }
  { \bigstepgen{\opsem{}}{\num0}{\trackpolyE{\e{}}{\inferpath{}}{\ep{}}}{\vp{}}{\res{} \sqcup \res{}' \sqcup \res{}''}{\num2} }

  \begin{array}{lllll}
    \trackpolymeta{}(\v{}, \inferpath{}, \num{}) = \v{}\ ;\ \res{}\\\\

    \trackpolymeta{}(\num{}, \inferpath{}, \num{}')
    &=&
    n\ ; \{\inferpath{} : \IntT{}\}
    \\
    \trackpolymeta{}([\lambda \xvar{}. \e{}, \opsem{}], \inferpath{}, \num{})
    &=&
    [
    \lambda \yvar{}.
    \inferletliteral (\xvar{n} \genE{})
    \\&&
      \trackpolyE{((\lambda \xvar{}. \e{}) \trackpolyE{\yvar{}}{\appendone{\inferpath{}}{\dompe{}}}{\xvar{n}})}
             {\appendone{\inferpath{}}{\rngpe{}}}{\xvar{n}}
         , \opsem{}]
         \ ; \{\inferpath{} : [\UnknownT{} \rightarrow \UnknownT{}] \}
         \\
    &&
    \text{where}\ \yvar{} \text{ is fresh}
    \\
    \trackpolymeta{}(\{\overrightarrow{\val1\ \val2}\}, \inferpath{}, \num{})
    &=&
    \{\overrightarrow{\val1\ \val2{}'}\}
    \ ;\ \overrightarrow{\sqcup\ \res{}}
      \sqcup
    \{\inferpath{} : \{\overrightarrow{\val1\ \UnknownT{}}\} \}
    \\
    &&
    \text{where}\ \overrightarrow{\trackpolymeta{}(\val2, \appendone{\inferpath{}}{\inferkeype{\overrightarrow{\val1}}{\val1}}) = \val2{}'\ ;\ \res{}}
  \end{array}
\end{mathpar}
\caption{Polymorphic type tracking extensions}
\end{figure*}

\Dsection{Space efficient tracking}

To reduce the overhead of runtime tracking, we can borrow
the concept of ``space efficient'' contract checking from
the gradual typing literature.

Instead of tracking just one path at once, a space efficient
implementation of track threads through a set of paths.
When a tracked value flows into another tracked position,
we extract the unwrapped value, and then our new tracked value
tracks the paths that is the set of the old paths with the new path.

To model this, we introduce a new kind of value $\ProxyE{\val1{}}{\val2{}}{\overrightarrow{\inferpath{}}}$
that tracks old value $\val1{}$ as new value $\val2{}$ with the paths $\overrightarrow{\inferpath{}}$.
Proxy expressions are introduced when tracking functions, where instead of just returning
a new wrapped function, we return a $\ProxyE{}$.

\begin{figure*}
\begin{mathpar}
  \begin{altgrammar}
    \v{} &::=& ... \alt \ProxyE{\v{}}{\v{}}{\overrightarrow{\inferpath{}}}
       &\mbox{Values}
  \end{altgrammar}
  \\

  \begin{array}{lllll}
    \trackmetaOp{}([\lambda \xvar{}. \e{}, \opsem{}], \overrightarrow{\inferpath{}})
    &=&
    \ProxyE{[\lambda \xvar{}. \e{}, \opsem{}]}
    {[
    \lambda \yvar{}.
      \trackE{((\lambda \xvar{}. \e{}) \trackE{\yvar{}}{\overrightarrow{\appendone{\inferpath{}}{\dompe{}}}})}
             {\overrightarrow{\appendone{\inferpath{}}{\rngpe{}}}}
         , \opsem{}]}
         {\overrightarrow{\inferpath{}}}
         \ ; \overrightarrow{\{\inferpath{} : [\UnknownT{} \rightarrow \UnknownT{}] \}}
         \\
    &&
    \text{where}\ \yvar{} \text{ is fresh}
    \\
    \trackmetaOp{}(\ProxyE{\val1{}}{\val2{}}{\overrightarrow{\inferpath{}'}}, \overrightarrow{\inferpath{}})
    &=&
    \ProxyE{\val1{}}{\val3{}}{(\inferpath{} \cup \inferpath{}')}
         \ ; \res{}
         \\
    &&
    \text{where}\ \trackmetaOp{}(\val1{},\ \overrightarrow{\appendone{(\inferpath{} \cup \inferpath{}')}{\dompe{}}}) = \val3{} ;\ \res{}
  \end{array}

  \\
\infer [B-ProxyApp]
{ \opsemtrack{\opsem{}}{\exp1}{\ProxyE{\vpp{}}{\v{}}{\overrightarrow{\inferpath{}}}}{\res{1}} \\\\
  \opsemtrack{\opsem{}}{\appexp{\v{}}{\exp{2}}}{\vp{}}{\res{2}}\\
}
{ \opsemtrack{\opsem{}}{\appexp{\exp{1}}{\exp{2}}}{\vp{}}{\res{1} \sqcup \res{2}}}
\end{mathpar}
\caption{Space efficient tracking extensions}
\end{figure*}
