\Dchapter{Extensions}
\label{infer:sec:extensions}

%\Dsection{Polymorphic types}
%
%\begin{figure}
%  \ifdefined\PAPER
%  \footnotesize
%  \fi
%\begin{mathpar}
%  \begin{altgrammar}
%   \e{} &::=& ...
%       \alt \trackpolyE{\e{}}{\inferpath{}}{\e{}}
%       \alt \genE{}
%       &\mbox{Expressions}
%  \end{altgrammar}
%
%  \infer [B-Gen]
%  { \num{} \text{ is fresh}
%  }
%  { \opsemtrack{\openv{}}{\genE{}}{\num{}}{\emptyres{}}}
%
%  \infer [B-TrackPoly]
%  { \bigstepgen{\openv{}}{\num0}{\e{}}{\v{}}{\res{1}}{\num1} 
%    \\\\
%    \bigstepgen{\openv{}}{\num1}{\ep{}}{\num{}}{\res{2}}{\num2}
%    \\\\
%    \trackpolymeta{}(\v{}, \inferpath{}, {\num{}}) = \vp{}\ ; \res{3} }
%  { \bigstepgen{\openv{}}{\num0}{\trackpolyE{\e{}}{\inferpath{}}{\ep{}}}{\vp{}}{\bigunionres{\ova{\res{}}}}{\num2} }
%
%  \begin{array}{lllll}
%    \trackpolymeta{}(\v{}, \inferpath{}, \num{}) = \v{}\ ;\ \res{}\\\\
%
%    \trackpolymeta{}(\num{}, \inferpath{}, \num{}')
%    &=&
%    n\ ; \{\inferpath{} : \IntT{}\}
%    \\
%    \trackpolymeta{}([\lambda \xvar{}. \e{}, \openv{}], \inferpath{}, \num{})
%    &=&
%    [
%    \lambda \yvar{}.
%    \inferletliteral (\xvar{n} \genE{})
%    \\&&
%    FIXME
%      %\trackpolyE{((\lambda \xvar{}. \e{}) \trackpolyE{\yvar{}}{\appendone{\inferpath{}}{\dompe{}}}{\xvar{n}})}
%      %       {\appendone{\inferpath{}}{\rngpe{}}}{\xvar{n}}
%      %   , \openv{}]
%      %   \ ; \{\inferpath{} : [\UnknownT{} \rightarrow \UnknownT{}] \}
%         \\
%    &&
%    \text{where}\ \yvar{} \text{ is fresh}
%    \\
%    \trackpolymeta{}(\{\ova{\val1\ \val2}\}, \inferpath{}, \num{})
%    &=&
%    \{\ova{\val1\ \val2{}'}\}
%    \ ;\ \ova{\sqcup\ \res{}}
%      \sqcup
%    \{\inferpath{} : \{\ova{\val1\ \UnknownT{}}\} \}
%    \\
%    &&
%    \text{where}\ \ova{\trackpolymeta{}(\val2, \appendone{\inferpath{}}{\inferkeype{\ova{\val1}}{\val1}}) = \val2{}'\ ;\ \res{}}
%  \end{array}
%\end{mathpar}
%  \caption{Polymorphic type tracking extensions }
%\end{figure}

\Dsection{Space-efficient tracking}
\label{infer:sec:space-effecient-tracking}

To reduce the overhead of runtime tracking, we can borrow
the concept of ``space-efficient'' contract checking from
the gradual typing literature~\cite{Herman:2010}.
%
Instead of tracking just one path at once, a space-efficient
implementation of track threads through a set of paths.
When a tracked value flows into another tracked position,
we extract the unwrapped value, and then our new tracked value
tracks the paths that is the set of the old paths with the new path.

To model this, we introduce a new kind of value \ProxyV{\v{}}{\closure{\e{}}{\openv{}}}{\ova{\inferpath{}}}
that tracks old value \v{} as new value \closure{\e{}}{\openv{}} with the paths \ova{\inferpath{}}.
Proxy expressions are introduced when tracking functions, where instead of just returning
a new wrapped function, we return a proxy.
We can think of function proxies as a normal function with some extra metadata, so we
can reuse the existing semantics for function application---in fact we
can support space-efficient function tracking just by extending \trackEOp{}.

We present the extension in \figref{fig:infer:proxyext}.
The first two \trackEOp{} rules simply make inference
results for each of the paths.
The next rule says that a bare closure
reduces to a proxy that tracks the domain and range
of the closure with respect to the list of paths.
Attached to the proxy is everything needed to extend
it with more paths, which is the role of the
final rule. It extracts the original closure from the
proxy and creates a new proxy with updated paths
via the previous rule.

\begin{figure}
  \ifdefined\PAPER
  \footnotesize
  \fi
\begin{mathpar}
  \begin{altgrammar}
    \v{} &::=& ... \alt \ProxyVdiff{\closure{\uabs{\x{}}{\e{}}}{\openv{}}}{\closure{\uabs{\x{}}{\e{}}}{\openv{}}}{\ova{\inferpath{}}}
       &\mbox{Values}
  \end{altgrammar}

  \arraycolsep=1.4pt
  \begin{array}{lllll}
    \trackmetaalign{\num{}}{\ovadiff{\inferpath{}}}{\num{}}{\proxyextdiff{\bigunionres{\ovadiff{\proxyextsame{\singletonres{\inferpath{}}{\IntT{}}}}}}}\\
    \trackmetaalign{\kw{}}{\ovadiff{\inferpath{}}}{\kw{}}
                   {\proxyextdiff{\bigunionres{\ovadiff{\proxyextsame{\singletonres{\inferpath{}}{\Keyword{}}}}}}}\\
    \trackmetaalign{\closure{\uabs{\x{}}{\e{}}}{\openv{}}}
                   {\ovadiff{\inferpath{}}}
                   {\ProxyVdiff{\closure{\uabs{\x{}}{\e{}}}{\openv{}}}
                               {\closure{\ep{}}{\openv{}}}
                               {\ova{\inferpath{}}}}
                   {\emptyres{}}
         \\
    &&
    \begin{array}{@{}llll}
      \text{where } \yvar{} \text{ is fresh},\\
                    \begin{array}{lllll}
                        \ep{} =
                          \uabs{\y{}}{\trackE{&\appexp{(\uabs{\x{}}{\e{}})}{\trackE{\yvar{}}{\ovadiff{\appendone{\inferpath{}}{\dompe{}}}}}}
                                             {\\&\ovadiff{\appendone{\inferpath{}}{\rngpe{}}}}}
                     \end{array}
    \end{array}
                
    \\
    \trackmetaalignsplice{\ProxyV{\closure{\uabs{\x{}}{\e{}}}{\openv{}}}{\closure{\ep{}}{\openvp{}}}{\ova{\inferpathp{}}}}{\ova{\inferpath{}}}
                         {\trackmetalhs{\closure{\uabs{\x{}}{\e{}}}{\openv{}}}{\ova{\inferpath{}} \cup \ova{\inferpathp{}}}}
  \end{array}
\end{mathpar}
  \caption{Space-efficient tracking extensions (\textcolor{red}{changes})}
  \label{fig:infer:proxyext}
\end{figure}

\Dsection{Lazy tracking}
\label{infer:sec:lazy-tracking}

Building further on the extension of space-efficient functions,
we apply a similar idea for tracking maps.
In practice, eagerly walking data structures to gather inference
results is expensive.
Instead,
waiting until a data structure is used and tracking its contents
lazily can help ease this tradeoff, with the side-effect that
fewer inference results are discovered.

\figref{fig:infer:lazy} extends our system with lazy maps.
We add a new kind of value 
\MProxyV{\curlymap{\ova{\kw{}\ \v{}}}}{\curlymap{\ova{\kwp{}\ \curlymap{\ova{\HMapreq{}\ \ova{\inferpath{}}}}}}}
that wraps a map \curlymap{\ova{\kw{}\ \v{}}} with tracking information.
Keyword entries \kwp{} are associated with pairs of type information \HMapreq{}
with paths \ova{\inferpath{}}.
The first \trackEOp{} rule demonstrates how to create
a lazily tracked map.
We calculate the possibly tagged entries in our type information in advance,
much like the equivalent rule in \figref{infer:fig:trackmeta}, and store
them for later use. Notice that non-keyword entries are not yet traversed,
and thus no inference results are derived from them.
The second \trackEOp{} rule adds new paths to track.

The subtleties of lazily tracking maps lie in the \constantopsemliteral{}
rules.
The \assocliteral{} and \dissocliteral{} rules ensure we no longer track overwritten entries.
Then, the \getliteral{} rules perform the tracking that was deferred
from the \trackEOp{} rule for maps in \figref{infer:fig:trackmeta}
(if the entry is still tracked).

In our experience, some combination of lazy and eager tracking of
maps strikes a good balance between performance overhead and
quantity of inference results.
Intuitively, if a function does not access parts of its argument,
they should not contribute to that function's type signature.
However, our inference algorithm combines information 
\emph{across} function signatures to deduce useful, recursive
type aliases.
Some eager tracking helps normalize the quality of function annotations
with respect to unit test coverage.

For example, say functions \textsf{f} and \textsf{g} operate on the same
types of (deeply nested) arguments, and \textsf{f} has complete test coverage (but does not
traverse all of its arguments), and \textsf{g} has incomplete test coverage
(but fully traverses its arguments).
Eagerly tracking \textsf{f} would give better inference results,
but lazily tracking \textsf{g} is more efficient.
Forcing several layers of tracking helps strike this balance, which
our implementation exposes as a parameter.

This can be achieved in our formal system
by adding fuel arguments to \trackEOp{}
that contain depth and breadth tracking limits, and
defer to lazy tracking when out of fuel.

\begin{figure}
  \ifdefined\PAPER
  \footnotesize
  \fi
\begin{mathpar}
  \begin{altgrammar}
    \v{} &::=& ... \alt \MProxyVdiff{\curlymap{\ova{\kw{}\ \v{}}}}{\curlymap{\ova{\kw{}\ \curlymap{\ova{\HMapreq{}\ \ova{\inferpath{}}}}}}}
       &\mbox{Values}
  \end{altgrammar}

  \arraycolsep=1.4pt
  \begin{array}{lllll}
    \trackmetaalign{\curlymap{\ova{\kw{}\ \kwp{}}\ \ova{\kwpp{}\ \v{}}}}
                   {\ovadiff{\inferpath{}}}
                   {\MProxyVdiff{\curlymap{\ova{\kw{}\ \kwp{}}\ \ova{\kwpp{}\ \proxyextdiff{\v{}}}}}
                                {\curlymap{\ova{\kw{}\ \textsf{t}}\ \ova{\kwpp{}\ \textsf{t}}}}}
                   {\proxyextdiff{\emptyres{}}}
                   \\
                   &&
    \begin{array}{@{}llll}
      \text{where } \textsf{t} = \{\curlymap{\ova{\kw{}\ \kwp{}}\ \ova{\kwpp{}\ \UnknownT{}}}\ \ova{\inferpath{}}\}
    \end{array}
    \\
    \trackmetaalign{\MProxyVdiff{\curlymap{\ova{\kw{}\ \v{}}}}
                                {\curlymap{\ova{\kwp{}\ \{\HMapreq{}\ \ova{\inferpathp{}}\}}}}}
                   {\ovadiff{\inferpath{}}}
                   {\MProxyVdiff{\curlymap{\ova{\kw{}\ \v{}}}}
                                {\curlymap{\ova{\kwp{}\ \{\HMapreq{}\ (\ova{\inferpath{} \cup}\ \cup\ {\ova{\inferpathp{} \cup}})\}}}}}
                   {\emptyres{}}
  \end{array}

  \arraycolsep=1.4pt
  \begin{array}{lllr}
    \inferconstantopsemalignnospace
      {\assocliteral{}}
      {\MProxyVdiff{\curlymap{\ova{\kw{}\ \v{}}}}{\curlymap{\kwp{}\ \textsf{t}', \ova{\kwpp{}\ \textsf{t}}}}
       , \kwp{}
       , \vp{}}
      {\MProxyVdiff{\updatemap{\curlymap{\ova{\kw{}\ \v{}}}}{\kwp{}}{\vp{}}}
               {\curlymap{\ova{\kwpp{}\ \textsf{t}}}}}
      {\emptyres{}}\\
    \inferconstantopsemalignnospace
      {\assocliteral{}}
      {\MProxyVdiff{\curlymap{\ova{\kw{}\ \v{}}}}
                   {\curlymap{\ova{\kwpp{}\ \textsf{t}}}}
       , \kwp{}
       , \vp{}}
      {\MProxyVdiff{\updatemap{\curlymap{\ova{\kw{}\ \v{}}}}{\kwp{}}{\vp{}}}
                   {\curlymap{\ova{\kwpp{}\ \textsf{t}}}}}
      {\emptyres{}}\\
    \inferconstantopsemalignsplice
      {\getliteral{}}
      {\MProxyVdiff{\curlymap{\kw{}\ \v{}, \ova{\kwp{}\ \vp{}}}}
                   {\curlymap{\kw{}\ \textsf{t}, \ova{\kwpp{}\ \textsf{t}'}}}
       , \kw{}}
      {\proxyextdiff{\trackmetalhs{\proxyextsame{\v{}}}{\ova{\inferpath{}}}}}\\
      \begin{array}{lllll}
        \text{ where } \ova{\inferpath{}} = \big[\appendone{\inferpath{}}{\inferkeype{\HMapreq{}}{\kw{}}}
                                                \ |\
                                                (\HMapreq{}, \ova{\inferpath{}}) \in \textsf{t}, 
                                                \inferpath{} \in \ova{\inferpath{}}
                                           \big]
      \end{array}
      \\
    \inferconstantopsemalignsplice
      {\getliteral{}}
      {\MProxyVdiff{\curlymap{\kw{}\ \v{}, \ova{\kwp{}\ \vp{}}}}
                   {\curlymap{\ova{\kwpp{}\ \textsf{t}'}}}
       , \kw{}}
      {\v{}}\\
    \inferconstantopsemalign
      {\dissocliteral{}}
      {\MProxyVdiff{\curlymap{\kw{}\ \v{}, \ova{\kwp{}\ \vp{}}}}
                   {\curlymap{\kw{}\ \textsf{t}, \ova{\kwpp{}\ \textsf{t}'}}}
       , \kw{}}
      {\MProxyVdiff{\curlymap{\ova{\kwp{}\ \vp{}}}}
                   {\curlymap{\ova{\kwpp{}\ \textsf{t}'}}}}
      {\emptyres{}}
      \\
    \inferconstantopsemalign
      {\dissocliteral{}}
      {\MProxyVdiff{\curlymap{\kw{}\ \v{}, \ova{\kwp{}\ \vp{}}}}
                   {\curlymap{\ova{\kwpp{}\ \textsf{t}'}}}
       , \kw{}}
      {\MProxyVdiff{\curlymap{\ova{\kwp{}\ \vp{}}}}
                   {\curlymap{\ova{\kwpp{}\ \textsf{t}'}}}}
      {\emptyres{}}
  \end{array}
\end{mathpar}
  \caption{Lazy tracking extensions (\textcolor{red}{changes})}
  \label{fig:infer:lazy}
\end{figure}

\Dsection{Automatic contracts with clojure.spec}
\label{infer:sec:spec-extension}

While we originally designed our tool to generate Typed Clojure annotations,
it also supports generating ``specs'' for clojure.spec, Clojure's runtime
verification system.
There are key similarities between Typed Clojure and clojure.spec,
such as extensive support for potentially-tagged keyword maps,
however spec features a global registry of names
via \clj{s/def} and
an explicit way declare unions of maps with a common
dispatch key in \clj{s/multi-spec}.
These require differences in both type and name generation.

The following generated specs 
correspond to the first \clj{Op} case
of \figref{infer:fig:cljs}
(lines \ref{infer:listing:cljs:Op:op:bindingStart}-\ref{infer:listing:cljs:Op:op:bindingEnd}).
\begin{cljlistingnumbered}
  (defmulti op-multi-spec :op) ;dispatch on :op key
  (defmethod op-multi-spec :binding ;match :binding
    [_] ;s/keys matches keyword maps
    (s/keys :req-un [::op ...] ;required keys
            :opt-un [::column ...])) ;optional keys
  (s/def ::op #{:js :let ...}) ;:op key maps to keywords
  (s/def ::column int?) ;:column key maps to ints
  ; register ::Op as union dispatching on :op entry
  (s/def ::Op (s/multi-spec op-multi-spec :op))
  ; emit's first argument :ast has spec ::Op
  (s/fdef emit :args (s/cat :ast ::Op) :ret nil?)
\end{cljlistingnumbered}


%However, there are several non-trivial details to manage.
% specs have fundamental differences.

%As they are checked at runtime, specs are naturally more expressive than types.
%However, our generated specs are roughly the same as types in their precision.

%\paragraph{Registered Specs}
%Central to spec's design is a global registry of named specs.
%The form \clj{(s/def k s)} registers spec \clj{s} under name \clj{k}.
%To avoid conflicts, \emph{namespaced keywords} are used. The \clj{::a} reader syntax
%auto-resolves the keyword under the current namespace, like \clj{:this-ns/a}.
%
%\paragraph{HMaps}
%Heterogeneous map keyword specs are called ``entity maps'' in spec, 
%and are defined as collections
%of registered specs with \clj{s/keys}, and has quite different
%syntax and semantics than many other runtime verification systems.

%The Typed Clojure type
%\begin{cljlisting}
%(defalias M (HMap :mandatory {:a Int, ::a Bool}
%                  :optional  {:b Int, ::b Bool}))
%\end{cljlisting}
%is equivalent to
%\begin{cljlisting}
%(s/def :int/a integer?)
%(s/def ::a boolean?)
%(s/def :int/b integer?)
%(s/def ::b boolean?)
%(s/def ::M (s/keys :req-un [:int/a], :req [::a],
%                   :opt-un [:int/b], :opt [::b]))
%\end{cljlisting}
%The \clj{:req-un} and \clj{:opt-un} options
%(\clj{un}[qualified] keyword \clj{req}[uired] and \clj{opt}[ional] entries)
%take vectors of registered specs and, for
%each spec \clj{:ns/name}, uses the registered spec \clj{:ns/name}
%to verify values at \clj{:name}.
%
%The \clj{:req} and \clj{:opt} options
%(qualified keyword \clj{req}[uired] and \clj{opt}[ional] entries)
%take vectors of registered specs and, for
%each spec \clj{:ns/name}, uses the registered spec \clj{:ns/name}
%to verify values at \clj{:ns/name}.
%
%Spec has several other subtleties we must account for, but omit
%for brevity.

%\paragraph{Tagged maps}
%
%\paragraph{Functions}
