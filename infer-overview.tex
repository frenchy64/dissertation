\Dchapter{Overview}
\label{infer:sec:overview}

Now that we have introduced the problem,
we can flesh out our philosophy and overall
approach.
%
At a high level, there are two phases to
generating types---\textbf{collection} and
\textbf{inference}.
%
The collection phase (\secref{infer:sec:formal:collection-phase}),
gathers observations about a running program.
This is achieved by instrumenting the program and exercising
it, usually by running its unit tests.
%
The inference phase (\secref{infer:sec:formal:inference-phase})
uses these runtime observations to generate the final type annotations,
with recursive types, optional entries, and good names.

\figref{infer:fig:cljs} demonstrates sample output from our
tool. It shows the (inferred) recursively defined type
for the ClojureScript compiler's AST format.
Similar to our opening example, it uses the \clj{:op}
key to disambiguate between (16) cases, and has recursive
references (\clj{Op}).

We just present the first 4 cases.
The first case \clj{':binding} has 4 required
and 8 optional entries.
The \clj{:info} and \clj{:env} entries refer to
other \clj{HMap} type aliases generated by the tool.
Similar to \clj{:op},
the \clj{:local} entry maps to a keyword singleton
type,
however our tool wisely chose to cluster types 
based on the \clj{:op} entry since it is common to all cases.


%is split into 
%\textbf{instrumentation}, involves
%rewriting the code we wish to annotate such
%that we can record its runtime behavior.
%In this phase, we require the programmer to
%indicate which code we wish to generate types
%for, in advance.
%
%Once instrumented, we observe our running program
%via \textbf{runtime tracking}. To exercise our programs,
%we usually run their unit tests, generative tests,
%or just normally run the program (eg. to generate types for
%a game, we can simply play the game for a few minutes).
%We accumulate the results of tracking via \textbf{paths}.
%If we think of types as trees and supply a label
%for each branching path, our inference results
%specify the type down a particular path in this tree.
%Both phases are described in ,
%collectively as the \emph{collection phase}.
%
%Finally, the information collected during runtime tracking
%is combined into annotations by our \textbf{inference algorithm}.
%We first combine all inference result into a large tree of
%types. If we were to convert this tree into annotations directly,
%our annotations would be too specific---they would be too
%deep and fine-grained.
%Instead, our algorithm iterates over several passes to massage
%this tree, generating good names for the nodes, compacting similar
%types across the tree, and
%eventually converting the tree into a directed graph by reconstructing
%recursive types.
%This \emph{inference phase} is described in 
%\secref{infer:sec:formal:inference-phase}.

\begin{figure}
  % indented so line numbers can line up more tastefully
\begin{cljlistingnumbered}
  (defalias Op
    (U (HMap
        :mandatory
        {:op ':binding,
         :info (U NameShadowMap
                  FnScopeFnSelfNameNsMap),
         :local (U ':arg ':let ':fn),
         :name Sym}
        :optional
        {:arg-id Int,
         :binding-form? Boolean,
         :column Int,
         :env ColumnLineContextMap,
         :init Op,
         :line Int,
         :shadow (U nil Op),
         :tag Any})
      '{:op ':const,
        :env HMap49305,
        :form (U nil Int ':a),
        :tag Sym,
        :val (U nil Int ':a)}
      '{:op ':do,
        :body? Any,
        :children Any,
        :env HMap49305,
        :form (Coll Any),
        :ret Op,
        :statements (Vec Nothing),
        :tag Any}
      '{:op ':fn-method,
        :body Op,
        :children '[':params ':body],
        :env HMap49305,
        :fixed-arity Int,
        :form (Coll (Coll Any)),
        :name Op,
        :params '[Op],
        :recurs nil,
        :type nil,
        :variadic? false}
      ; omitted 10 cases
      ...))
\end{cljlistingnumbered}
\caption{While imperfect, this recursive type generated by our tool 
         is an invaluable starting
         point for further annotations.
         It describes the AST format for a compiler called cljs.compiler
         (\secref{infer:chap:evaluation}), and
         has 14 distinct operators, 5 inferred
         to have optional entries,
         with 22 recursive references.
}
\label{infer:fig:cljs}
%    '{:op ':host-call,
%      :args '[Op],
%      :children Any,
%      :env context-statement-tmp-HMap-alias20275,
%      :form (Coll Sym),
%      :method Sym,
%      :tag Any,
%      :target Op}
%    '{:op ':host-field,
%      :children '[':target],
%      :env context-statement-tmp-HMap-alias20275,
%      :field Sym,
%      :form (Coll Sym),
%      :tag Sym,
%      :target Op}
%    '{:op ':if,
%      :children '[':test ':then ':else],
%      :else Op,
%      :env context-statement-tmp-HMap-alias20275,
%      :form (Coll Any),
%      :tag (Set (U nil Sym)),
%      :test Op,
%      :then Op,
%      :unchecked Boolean}
%    '{:op ':invoke,
%      :args '[Op],
%      :children '[':fn ':args],
%      :env context-statement-tmp-HMap-alias20275,
%      :fn Op,
%      :form (Coll Any),
%      :tag Sym}
%    (HMap
%      :mandatory
%      {:op ':js,
%       :env context-statement-tmp-HMap-alias20275,
%       :form (Coll (U nil Str Sym)),
%       :js-op Sym,
%       :numeric nil,
%       :tag Sym}
%      :optional
%      {:args '[Op Op],
%       :children '[':args],
%       :code Str,
%       :segs (Coll Str)})
%    (HMap
%      :mandatory
%      {:op ':js-var, :name Sym, :ns Sym}
%      :optional
%      {:tag Sym})
%    '{:op ':let,
%      :bindings '[Op Op Any],
%      :body Any,
%      :children Any,
%      :env context-statement-tmp-HMap-alias20275,
%      :form Any,
%      :tag Any}
%    (HMap
%      :mandatory
%      {:op ':local,
%       :env context-statement-tmp-HMap-alias20275,
%       :form Sym,
%       :info Op,
%       :local (U ':arg ':let),
%       :name Sym}
%      :optional
%      {:arg-id Int, :init Op, :tag Sym})
%    '{:op ':map,
%      :children '[':keys ':vals],
%      :env context-statement-tmp-HMap-alias20275,
%      :form AMap,
%      :keys '[Op],
%      :tag Sym,
%      :vals '[Op]}
%    (HMap
%      :mandatory
%      {:op ':var, :name Sym, :ns Sym}
%      :optional
%      {:arglists (Coll Any),
%       :arglists-meta (Coll nil),
%       :column Int,
%       :doc Str,
%       :end-column Int,
%       :end-line Int,
%       :env context-statement-tmp-HMap-alias20275,
%       :file (U nil Str),
%       :fn-var Boolean,
%       :form Sym,
%       :info (U nil ColumnFileLineMap),
%       :line Int,
%       :max-fixed-arity Int,
%       :meta
%       (U
%         ColumnFileLineMap__0
%         FileArglistsColumnMap
%         ColumnEndColumnEndLineMap),
%       :method-params (Coll (Coll Sym)),
%       :protocol-impl nil,
%       :protocol-inline nil,
%       :ret-tag Sym,
%       :tag Sym,
%       :top-fn ArglistsArglistsMetaMaxFixedArityMap,
%       :variadic? Boolean})))
\end{figure}


An important question to answer is ``how accurate are these annotations?''.
Unlike previous work in this area~\cite{An10dynamicinference}, we do not aim for soundness guarantees
in our generated types. 
Our main contribution is a tool that Clojure programmers
can use to help learn about and specify their programs.
In that spirit, annotations should meet several criteria.

\paragraph{Good names}
Typed Clojure and clojure.spec annotations are abundant
with useful names for types. A good name often increases
readability.
Good names can sometimes be reconstructed from the program source,
like function or parameter names, and other times 
we can use the shape of a type to summarize it.

\paragraph{Compact}
Idiomatic Clojure code rarely mixes certain types in the same position,
unless the program is polymorphic. Using this knowledge---which we observed
by the annotations and specs assigned to idiomatic Clojure 
code---we can rule out certain combinations of types to compact our
resulting output, without losing information that would help us
type check our programs.

\paragraph{Recursive}
Maps in Clojure are often heterogeneous, and recursively defined.
Typed Clojure and clojure.spec supplies mechanisms for the most
common case: maps of known keyword entries.
We strategically \textbf{squash} flat types to be recursive
based on their unrolled shape.
For example, a recursively defined union of maps almost always
contains a known keyword ``tag'' mapped to a keyword.
By identifying this tag, we can reconstruct a good recursive
approximation of this type.

\Dsection{Naming}

For a type to be immediately useful to a programmer, it helps
to have a great name. We explored several avenues for
generating good names.

For types that occured as function arguments, the name of
the argument often indicated its role in the program.
Names like \clj{config} or \clj{env} are often used
for an environment being functionally threaded through
the program.

Similarly, types that occur as values in configuration
maps often have descriptive keys.
For example, one of our case studies, a Star Trek 
game written in Clojure,
features a configuration map with a \clj{:stardate}
entry containing
a map that of three number entries: 
\clj{:start},
\clj{:current}, and
\clj{:end}.

What if a type occurs in the return position of a function?
Sometimes these are named by \textbf{let} binding the result
of the computation.

Failing these heuristics, we fall back of several approaches
to naming.
First, if we are naming a keyword map which is part of a tagged
union, we use the tag as the name. For example, if the tagged entry
maps \textbf{:op} to \textbf{:fn}, we name this map \textbf{FnOp}.
Otherwise, for maps with less than three entries, we simply
enumerate its entries as the name.
Finally, for large keyword maps, we give an abbreviation
of its keyset as a name.

\Dsection{Algorithm}

A key hypothesis in our algorithm is that functions in the same
namespace operate on related data. Our inference is built
to be per-namespace (i.e., per-file),
and aggressively merges all apparently-related HMap types.
This often yields useful types in our benchmarks.
This approach does not work well for libraries providing polymorphic
functions, since unit tests may use sample data that are unrelated
to the details of the function, but our algorithm merges and displays
them as an ``important'' top-level type alias.
Our algorithm excels
with programs that mainly operate on one or two (possibly recursive) map-based
data representations,
which, outside of polymorphic libraries, are common in the Clojure ecosystem
in our experience.

%\begin{Verbatim}
%(defn f [x] (inc x))
%\end{Verbatim}
%
%\begin{figure}
%\begin{cljlisting}
%(defn vertices [m]
%  (case (:op m)
%    :leaf 1
%    :node (+ 1 (:left m)
%               (:right m))))
%\end{cljlisting}
%\label{code:vertices}
%\end{figure}

%%%% An old introductory example %%%%%%

%Consider the problem of inferring the type for
%the following Clojure program, a one
%argument function $f$ that increments numbers.
%
%\begin{verbatim}
%(defn f [x] (inc x))
%\end{verbatim}
%
%We might unit test this feature on the integers,
%to ensure incrementing $1$ gives $2$.
%
%\begin{verbatim}
%(deftest f-test
%  (is (= (f 1) 2)))
%\end{verbatim}
%
%We can instrument this program to observe its runtime behavior.
%
%\begin{verbatim}
%(deftest f-test
%  (is (= ((track f ['f]) 1) 2)))
%\end{verbatim}
%
%The $track$ function takes a value and a \emph{path},
%which tracks the subcomponent of the current environment
%the given value represents. For example, the path
%
%\begin{verbatim}
%  ['f :domain]
%\end{verbatim}
%
%represents the domain of $f$.
%
%After instrumentation we collect some inference results,
%associating paths with types: $path : \tau$.
%
%\begin{verbatim}
%  [['f :domain] Int]
%  [['f :range] Int]
%\end{verbatim}
%
%We combine this information into a type environment
%$\Gamma$ mapping variables to types: $\{x : \tau\}$.
%
%\begin{verbatim}
%  {x : [Int -> Int]}
%\end{verbatim}
%
%Now consider the case where we add a new unit test.
%
%\begin{verbatim}
%(deftest f-test
%  (is (= (f 1) 2))
%  (is (= (f 2.5) 3.5)))
%\end{verbatim}
%
%We now have a new set of inference results:
%
%\begin{verbatim}
%  [['f :domain] Num]
%  [['f :range] Num]
%\end{verbatim}
%
%which we want to combine with our previously inferred type environment
%
%\begin{verbatim}
%  {x : [Int -> Int]}
%\end{verbatim}
%
%How do join $[Int -> Int]$
%and $[Num -> Num]$?
%We have several options.
%
%
%\begin{verbatim}
%[(I Num Int) -> (U Num Int)]
%\end{verbatim}
%
%\begin{verbatim}
%[(U Num Int) -> (U Num Int)]
%\end{verbatim}
%
%\begin{verbatim}
%(IFn [Int -> Int]
%     [Num -> Num])
%\end{verbatim}

%\begin{figure}
%\begin{tikzpicture}[level distance=0.8cm, scale=0.9]
%\tikzstyle{every node}=[font=\small]
%\tikzset{grow'=down}
%\tikzset{every tree node/.style={align=center,anchor=north}}
%\Tree [.\node[draw](P0){\texttt{P0 = (Pairof P1 P2)}};
%        [.\node[draw](P2){\texttt{P2 = (Pairof P3 N)}};
%          [ \texttt{N} ]
%          [.\node[draw](P3){\texttt{P3 = (Pairof N N)}};
%            [ \texttt{N} ] [ \texttt{N} ]]
%]
%        [.\node[fill=gray!80](P1){\texttt{P1 = (Pairof N N)}};
%          [ \texttt{N} ] [ \texttt{N} ] ]
% ]
%\end{tikzpicture}
%%
%%\hspace{0.15cm}
%%
%\begin{tikzpicture}[level distance=0.8cm]
%\tikzstyle{every node}=[font=\small]
%\tikzset{grow'=down}
%\tikzset{every tree node/.style={align=center,anchor=north}}
%\Tree [.\node[fill=gray!80](P0){\texttt{P0 = (Pairof ($\cup$ N P0) ($\cup$ N P2))}};
%        [.\node[draw](P2){\texttt{P2 = (Pairof P3 N)}};
%          [ \texttt{N} ]
%          [.\node[draw]{\texttt{P3 = (Pairof N N)}};
%            [ \texttt{N} ] [ \texttt{N} ]]
%]
%          [ \texttt{N} ] [ \texttt{N} ] 
% ]
%%\draw[semithick,->] (P1)..controls +(west:1) and +(west:1)..(P0);
%\end{tikzpicture}
%\end{figure}
%
