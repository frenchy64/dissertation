\Dchapter{Overview}
\label{infer:sec:overview}

%\input{infer-old-pldi-overview}

We demonstrate our approach by synthesizing a Typed Clojure annotation for \clj{nodes}.
The following presentation is somewhat loose to keep from being bogged down by details---interested
readers may follow the pointers to subsequent sections where they are made precise.

We use dynamic analysis to observe the execution of functions, so we give an
explicit test suite for \clj{nodes}.

\begin{cljlisting}
(def t1 {:op :node, :left {:op :leaf, :val 1}, :right {:op :leaf, :val 2}})
(deftest nodes-test (is (= (nodes t1) 3)))
\end{cljlisting}

The first step is the instrumentation phase
(formalized in \secref{infer:sec:formal:collection-phase}), which 
monitors the inputs and outputs of \clj{nodes}
by redefining like so, where \clj{<nodes-body>} begins the \clj{case} expression of
the original \clj{nodes} definition.

\begin{cljlisting}
(def nodes (fn [t'] (track ((fn [t] <nodes-body>) (track t' ['nodes :dom]))
                           ['nodes :rng])))
\end{cljlisting}

The \clj{track} function (given later in \figref{infer:fig:trackmeta})
takes a value to track and a
\emph{path} that represents its origin, and returns an instrumented value
along with recording some runtime samples about the value.
A path is represented as a vector of \emph{path elements},
and describes the source of the value in question.
For example, \clj{(track 3 ['nodes :rng])}
returns \clj{3} and records the sample
\resentry{\clj{['nodes :rng]}}{\clj{Int}}
which says ``\clj{Int} was recorded at \clj{nodes}'s range.''
%
Running our test suite \clj{nodes-test} with an instrumented \clj{nodes}
results in more samples like this, most which use the path element
\clj{\{:key :kw\}} which represents a map lookup on the \clj{:kw} entry.

\inferrule[]
{}
{
\resentry{\clj{['nodes :dom \{:key :op\}]}}{\clj{':leaf}} \\
\resentry{\clj{['nodes :dom \{:key :op\}]}}{\clj{':node}} \\
\resentry{\clj{['nodes :dom \{:key :val\}]}}{\clj{?}} \\
\resentry{\clj{['nodes :dom \{:key :left\} \{:key :op\}]}}{\clj{':leaf}}\\
\resentry{\clj{['nodes :dom \{:key :right\} \{:key :op\}]}}{\clj{':leaf}}\\
\resentry{\clj{['nodes :dom \{:key :left\} \{:key :val\}]}}{\clj{Int}}\\
\resentry{\clj{['nodes :dom \{:key :right\} \{:key :val\}]}}{\clj{Int}}\\
\resentry{\clj{['nodes :rng]}}{\clj{Int}}
}

Now, our task is to transform these samples into a readable and useful annotation.
This the function of the inference phase (formalized in \secref{infer:sec:formal:inference-phase}),
which is split into two steps: first it generates a naive type from samples, then it
manipulates and combines those types to be compact and recursive.

We can reconcile these samples into a type that resembled TypeWiz's
``verbatim'' annotation given in \Dchapref{infer:chapter:intro}, except
the \clj{?} placeholder represents incomplete information about a path
(this process is formalized as \generatetenv{} in \figref{infer:fig:generatetenv}).

\begin{cljlisting}
(ann nodes [(U '{:op ':leaf, :val ?} '{:op ':node,
                                       :left '{:op ':leaf, :val Int},
                                       :right '{:op ':leaf, :val Int}}) -> Int])
\end{cljlisting}


\begin{cljlisting}
(defn visit-leaf "Updates :leaf nodes in tree t with function f."
  [f t] (case (:op t)
          :node (assoc t :left (visit-leaf f (:left t))
                         :right (visit-leaf f (:right t)))
          :leaf (f t)))
(def t1 {:op :node, :left {:op :leaf, :val 1}, :right {:op :leaf, :val 2}})
(nodes t1)  ;=>3
(visit-leaf (fn [leaf] (assoc leaf :val (inc (:val leaf)))) t1)
;=> {:op :node, :left {:op :leaf, :val 2}, :right {:op :leaf, :val 3}}
\end{cljlisting}

% initial annotation
\begin{cljlisting}
(defalias Op (U '{:op ':leaf, :val t/Int} '{:op ':node, :left Op, :right Op}))
(ann visit-leaf [[Op :-> Any] Op :-> Any])
\end{cljlisting}

\begin{cljlisting}
(*@\colorbox{pink}{(defalias Leaf}@*) '{:op ':leaf, :val Int}(*@\colorbox{pink}{)}@*)
(defalias Op (U (*@\colorbox{pink}{Leaf}@*) '{:op ':node, :left Op, :right Op}))
(ann visit-leaf [[(*@\colorbox{pink}{Leaf}@*) :-> (*@\colorbox{pink}{Op}@*)] Op :-> (*@\colorbox{pink}{Op}@*)])
\end{cljlisting}
