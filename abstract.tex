We present Typed Clojure, an optional type system for the Clojure programming language.
This thesis argues Typed Clojure is sound and practical,
that we can improve its usability by automatically generating type annotations based on runtime observations,
and then we can repurpose this annotation technology to generate clojure.spec runtime specifications
and evaluate these annotations in hundreds of projects.
%.
%and answer broad questions about how Clojure is used.

First, I will present Typed Clojure, an optional type system for Clojure.
I will develop a formal model of Typed Clojure that includes
key features like hash-maps, multimethods, Java interoperability, and occurrence typing,
and prove the model type sound.
Then, I will demonstrate that Typed Clojure's design is useful and corresponds to actual usage patterns
with an empirical study of real-world Typed Clojure usage in over 19,000 lines of code.

Second, we address a major usability flaw in Typed Clojure: users must \emph{manually}
write annotations.
To remedy this, 
I will present a tool that automatically generates Typed Clojure annotations based on observed
program behavior, including
a formal model of the tool, consisting of its runtime instrumentation phase that
collects samples from a running program, and type reconstruction phase
that creates useful annotations from these samples.
Then, I will give an overview of a practical implementation that generates Typed Clojure annotations for
real programs.
Next, I will study the effectiveness, accuracy, and usability of these annotations
by generating annotations for several projects, and then manually amending the annotations
until they type check.

Third, I will repurpose my automatic annotation tool to generate clojure.spec annotations (``specs'')
and subsequently test their effectiveness over hundreds of open source Clojure projects.
I will outline clojure.spec, the official runtime verification
library bundled with Clojure, and present a formal model of clojure.spec that highlights its
``generative testing'' function checking semantics.
Next, I will discuss how to extend my annotation tool to generate specs.
Finally, I will verify the effectiveness of generated specs in hundreds of open-source Clojure projects.
%Third, I will conduct a larger scale investigation of Clojure usage patterns by
%repurposing my automatic annotation tool to generate clojure.spec annotations (``specs'')
%and subsequently use them to enforce.
%I will outline clojure.spec, the official runtime verification
%library bundled with Clojure, and present a formal model of clojure.spec that highlights its
%``generative testing'' function checking semantics.
%Next, I will discuss how to extend my annotation tool to generate specs.
%Finally, I will automatically generate specs for hundreds of open-source Clojure projects,
%and use this data to investigate general questions like the effectiveness of unit and generative testing,
%the evolution of code over time, and the prevalence of idioms that Typed Clojure and clojure.spec
%have been designed around.

