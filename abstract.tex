%\chapter*{Abstract}

Typed Clojure is an optional type system for the Clojure programming language that aims to type check idiomatic Clojure code.
This dissertation presents the design of Typed Clojure, formalizes Typed Clojure's underlying theory, studies its effectiveness
in real-world code bases, and proposes several extensions to help address its shortcomings.

I develop a formal model of Typed Clojure that includes
key features like hash-maps, multimethods, Java interoperability, and occurrence typing,
and prove the model type sound.
Then, I demonstrate that Typed Clojure's design is useful and corresponds to actual usage patterns
with an empirical study of real-world Typed Clojure usage in over 19,000 lines of code.
This experience also revealed several usability shortcomings in Typed Clojure.

First, the top-level annotation burden needed to port untyped code is prohibitively high.
We present an automatic annotator for Typed Clojure to ease this burden, using runtime
observations to synthesize heterogeneous, recursive type annotations. We evaluate our
experience using the annotator by porting several open-source projects.
%First, I address a major usability flaw in Typed Clojure: users must \emph{manually}
%write annotations.
%To remedy this, 
%I present a tool that automatically generates Typed Clojure annotations based on observed
%program behavior, including
%a formal model of the tool, consisting of its runtime instrumentation phase that
%collects samples from a running program, and type reconstruction phase
%that creates useful annotations from these samples.
%Then, I give an overview of a practical implementation that generates Typed Clojure annotations for
%real programs.
%Next, I study the effectiveness, accuracy, and usability of these annotations
%by generating annotations for several projects, and then manually amending the annotations
%until they type check.

Second, pre-expanding macros before type checking makes type checking brittle.
We describe and implement a new analyzer for Clojure code that can provide the
foundation of an alternative approach where the user provides custom type rules for macros.

Third, too many local functions require annotations. We present a hybrid approach of symbolic 
execution and type checking that helps check some common higher-order Clojure idioms.

%The final part of this thesis will either:
%\begin{itemize}
%  \item increase the number of type checkable Clojure programs, especially those
%    combining polymorphic higher-order and anonymous functions, by combining
%    an extensible typing rule system with symbolic execution, and study its effectiveness
%    in reducing the changes needed to port Clojure programs to Typed Clojure, or
%  \item repurpose the automatic annotation tool to generate clojure.spec annotations,
%    study its effectiveness in generating good specs over several
%    hundred open source projects, and use it to help answer more
%    general questions about Clojure usage.
%\end{itemize}

%Third, we conduct a study of clojure.spec, the recently released runtime verification
%system that comes bundled with Clojure, and compare its feature set to Typed Clojure's.
%I present an empirical study of the use of Clojure's core.spec contract system in several
%real world code bases, observing which features are used, and the precision of
%specifications.
%I then present models of several subsets of clojure.spec, concentrating on its interesting
%handling of higher-order function checking, and precisely identifying its intentional
%unsoundness compared to traditional higher-order contract checking.
%
%I repurpose my automatic annotation tool to generate clojure.spec annotations (``specs'')
%and subsequently test their effectiveness over hundreds of open source Clojure projects.
%I outline clojure.spec, the official runtime verification
%library bundled with Clojure, and present a formal model of clojure.spec that highlights its
%``generative testing'' function checking semantics.
%Next, I discuss how to extend my annotation tool to generate specs.
%Finally, I verify the effectiveness of generated specs in hundreds of open-source Clojure projects.

%Third, I will conduct a larger scale investigation of Clojure usage patterns by
%repurposing my automatic annotation tool to generate clojure.spec annotations (``specs'')
%and subsequently use them to enforce.
%I will outline clojure.spec, the official runtime verification
%library bundled with Clojure, and present a formal model of clojure.spec that highlights its
%``generative testing'' function checking semantics.
%Next, I will discuss how to extend my annotation tool to generate specs.
%Finally, I will automatically generate specs for hundreds of open-source Clojure projects,
%and use this data to investigate general questions like the effectiveness of unit and generative testing,
%the evolution of code over time, and the prevalence of idioms that Typed Clojure and clojure.spec
%have been designed around.
