\documentclass[9pt]{extarticle}

\usepackage[margin=1in]{geometry}
\usepackage{savesym}
\savesymbol{r}
\savesymbol{AA}
\usepackage{esop-common}
\usepackage{infer-common}

\title{Thesis Proposal: Typed Clojure in Theory and Practice}
\author{Ambrose Bonnaire-Sergeant}

\usepackage[backend=bibtex]{biblatex}
\addbibresource{bibliography.bib}

\begin{document}

\maketitle

\begin{abstract}
  We present Typed Clojure, an optional type system for the Clojure programming language.
This thesis argues Typed Clojure is sound and practical.
%that we can improve its usability by automatically generating type annotations based on runtime observations,
%and then we can repurpose this annotation technology to generate clojure.spec runtime specifications
%and evaluate these annotations in hundreds of projects.
%.
%and answer broad questions about how Clojure is used.

First, I present Typed Clojure, an optional type system for Clojure.
I develop a formal model of Typed Clojure that includes
key features like hash-maps, multimethods, Java interoperability, and occurrence typing,
and prove the model type sound.
Then, I demonstrate that Typed Clojure's design is useful and corresponds to actual usage patterns
with an empirical study of real-world Typed Clojure usage in over 19,000 lines of code.

Second, we address a major usability flaw in Typed Clojure: users must \emph{manually}
write annotations.
To remedy this, 
I present a tool that automatically generates Typed Clojure annotations based on observed
program behavior, including
a formal model of the tool, consisting of its runtime instrumentation phase that
collects samples from a running program, and type reconstruction phase
that creates useful annotations from these samples.
Then, I give an overview of a practical implementation that generates Typed Clojure annotations for
real programs.
Next, I study the effectiveness, accuracy, and usability of these annotations
by generating annotations for several projects, and then manually amending the annotations
until they type check.

%The final part of this thesis will either:
%\begin{itemize}
%  \item increase the number of type checkable Clojure programs, especially those
%    combining polymorphic higher-order and anonymous functions, by combining
%    an extensible typing rule system with symbolic execution, and study its effectiveness
%    in reducing the changes needed to port Clojure programs to Typed Clojure, or
%  \item repurpose the automatic annotation tool to generate clojure.spec annotations,
%    study its effectiveness in generating good specs over several
%    hundred open source projects, and use it to help answer more
%    general questions about Clojure usage.
%\end{itemize}

Third, we conduct a study of clojure.spec, the recently released runtime verification
system that comes bundled with Clojure, and compare its feature set to Typed Clojure's.
I present an empirical study of the use of Clojure's core.spec contract system in several
real world code bases, observing which features are used, and the precision of
specifications.
I then present models of several subsets of clojure.spec, concentrating on its interesting
handling of higher-order function checking, and precisely identifying its intentional
unsoundness compared to traditional higher-order contract checking.

I repurpose my automatic annotation tool to generate clojure.spec annotations (``specs'')
and subsequently test their effectiveness over hundreds of open source Clojure projects.
I outline clojure.spec, the official runtime verification
library bundled with Clojure, and present a formal model of clojure.spec that highlights its
``generative testing'' function checking semantics.
Next, I discuss how to extend my annotation tool to generate specs.
Finally, I verify the effectiveness of generated specs in hundreds of open-source Clojure projects.

%Third, I will conduct a larger scale investigation of Clojure usage patterns by
%repurposing my automatic annotation tool to generate clojure.spec annotations (``specs'')
%and subsequently use them to enforce.
%I will outline clojure.spec, the official runtime verification
%library bundled with Clojure, and present a formal model of clojure.spec that highlights its
%``generative testing'' function checking semantics.
%Next, I will discuss how to extend my annotation tool to generate specs.
%Finally, I will automatically generate specs for hundreds of open-source Clojure projects,
%and use this data to investigate general questions like the effectiveness of unit and generative testing,
%the evolution of code over time, and the prevalence of idioms that Typed Clojure and clojure.spec
%have been designed around.

\end{abstract}

%\section{Clojure with static typing}

% current situation 

The popularity of dynamically-typed languages in software
development, combined with a recognition that types often improve
programmer productivity, software reliability, and performance, has
led to the recent development of a wide variety of optional and
gradual type systems aimed at checking existing programs written in
existing languages.  These include  TypeScript~\cite{typescript} and Flow~\cite{flow} for
JavaScript, Hack~\cite{hack} for PHP, and mypy~\cite{mypy}
for Python among the optional systems, and Typed Racket~\cite{TF08}, Reticulated
Python~\cite{Vitousek14}, and GradualTalk~\cite{gradualtalk} among gradually-typed systems.\footnote{We
  use ``gradual typing'' for systems like Typed Racket with sound
  interoperation between typed and untyped code; Typed Clojure or
 TypeScript which don't
  enforce type invariants we describe as ``optionally typed''.}

One key lesson of these systems, indeed a lesson known to early
developers of optional type systems such as StrongTalk, is that type
systems for existing languages must be designed to work with the
features and idioms of the target language. Often this takes the form
of a core language, be it of functions or classes and objects,
together with extensions to handle distinctive language features.


We synthesize these lessons to present \emph{Typed Clojure}, an
optional type system for Clojure. 
%
Clojure is a dynamically
typed language in the Lisp family---built on the Java Virtual
Machine (JVM)---which has recently gained popularity as an alternative
JVM language.  It offers the flexibility of a Lisp dialect, including
macros, emphasizes a functional style via
immutable data structures, and provides
interoperability with existing Java code, allowing programmers to use
existing Java libraries without leaving Clojure.
%
Since its initial release in 2007, Clojure has been widely adopted for
``backend'' development in places where its support for parallelism,
functional programming, and Lisp-influenced abstraction is desired on
the JVM. As a result, there is an extensive base of existing untyped
programs whose developers can benefit from Typed Clojure,
an experience we discuss in this paper.

Since Clojure is a language in the
Lisp family, we apply the lessons of Typed Racket, an existing gradual type
system for Racket, to the core of Typed Clojure, consisting of an extended
$\lambda$-calculus over a variety of base types shared between all Lisp systems.
%
Furthermore, Typed Racket's \emph{occurrence typing} has proved
necessary for type checking realistic Clojure programs.

\begin{figure*}[t!]
  \normalsize
\begin{lstlisting}
(*typed ann pname [(U File String) -> (U nil String)] typed*)
(defmulti pname class)  ; multimethod dispatching on class of argument
(defmethod pname String [s] (*invoke pname (*interop new File s interop*) invoke*)) ; String case 
(defmethod pname File [f] (*interop .getName f interop*)) ; File case, static null check
(*invoke pname "STAINS/JELLY" invoke*) ;=> "JELLY" :- (U nil Str)
\end{lstlisting}
\caption{A simple Typed Clojure program (delimiters: {\color{interop}Java interoperation (green)}, 
  {\color{types}type annotation (blue)},
  {\color{invoke}function invocation (black)}, {\color{red}collection literal (red)}, {\color{mygray}other (gray)})}
\label{fig:ex1}
\end{figure*}


However, Clojure goes beyond Racket in many ways, requiring several
new type system features which we detail in this paper.
%
Most significantly, Clojure supports, and Clojure developers use,
\textbf{multimethods} to structure their code in extensible
fashion. Furthermore, since Clojure is an untyped language, dispatch
within multimethods is determined by application of dynamic predicates
to argument values. 
%
Fortunately, the dynamic dispatch used by multimethods has surprising
symmetry with the conditional dispatch handled by occurrence
typing. Typed Clojure is therefore able to effectively handle complex
and highly dynamic dispatch as present in existing Clojure programs. 

But multimethods are not the only Clojure feature crucial to type
checking existing programs. As a language built on the Java Virtual
Machine, Clojure provides flexible and transparent access to existing
Java libraries, and \textbf{Clojure/Java interoperation} is found in almost
every significant Clojure code base. Typed Clojure therefore builds in
an understanding of the Java type system and handles interoperation
appropriately. Notably, \texttt{null} is a distinct type in Typed Clojure,
designed to automatically rule out null-pointer exceptions.

An example of these features is given in
\figref{fig:ex1}. Here, the \clj{pname} multimethod dispatches
on the \clj{class} of the argument---for \clj{String}s,
the first method implementation is called, for \clj{File}s, the
second. The \clj{String} method calls
a \clj{File} constructor, returning a non-nil \clj{File} instance---the 
\clj{getName} method 
on \clj{File} requires a non-nil target, returning a nilable
type.  
%Typed Clojure offers an opt-in mode that
%resolves JVM overloading, avoiding expensive runtime reflective calls.

Finally, flexible, high-performance immutable dictionaries
are the most common Clojure data structure.
Simply treating them as uniformly-typed
key-value mappings would be insufficient for existing
programs and programming styles. Instead, Typed Clojure provides a
flexible \textbf{heterogenous map} type, in which specific entries can be specified. 

While these features may seem disparate, they are unified in important
ways. First, they leverage the type system mechanisms
inherited from Typed Racket---multimethods when using 
dispatch via predicates, Java interoperation for handling
\texttt{null} tests, and heterogenous maps using union types and
reasoning about subcomponents of data. Second,
they are crucial features for handling Clojure code in
practice. Typed Clojure's use in real Clojure deployments would not be
possible without effective handling of these three Clojure features. 

%\subsection{Contributions}

Our main contributions are as follows:

\begin{enumerate}
  \item We motivate and describe  Typed Clojure, an optional
    type system for Clojure that understands existing Clojure idioms.
  \item We present a sound formal model for three crucial type
    system features: multi-methods, Java
    interoperability, and heterogenous maps.
  \item We evaluate the use of Typed Clojure features on existing
    Typed Clojure code, including both open source and in-house systems.
\end{enumerate}



%% \begin{figure}
%% \inputminted[firstline=5]{clojure}{code/demo/src/demo/parent2.clj}
%% \caption{A simple Typed Clojure program}
%% \label{fig:ex1}
%% \end{figure}

%% Figure~\ref{fig:ex1} presents a simple program demonstrating many
%% aspects of our system, from simple type annotations to explicit
%% handling of Java's \java{null} (written \clj{nil}) in interoperation, as well as an
%% extended form of occurrence typing and method resolution of
%% Java interoperability based on static type information.

%% The \clj{parent} function has the type 
%% $$
%% \clj{['{:file (U nil File)} -> (U nil Str)]}
%% $$
%% which means that it takes a hash table whose \clj{:file} key maps to either
%% \clj{nil} or a \clj{File}, and it produces either \clj{nil} or a
%% \clj{String}. The \clj{parent} function uses the \clj{:file} keyword
%% as an accessor to get the file, checks that it isn't \clj{nil}, and
%% then obtains the parent by making a Java method call.

\noindent
 The remainder of this paper begins with an example-driven
 presentation of the main type system features in
 \secref{sec:overview}. We then incrementally present a core calculus
 for Typed Clojure covering all of these features together in
 \secref{sec:formal} and prove type soundness
 (\secref{sec:metatheory}). We then 
 %discuss the full implementation of
 %Typed Clojure, \coretyped{}, which extends the formal model in many ways, 
 present an empirical analysis of significant code bases written
 in \coretyped{}---the full implementation of Typed Clojure---in \secref{sec:experience}. 
 Finally, we discuss related work and conclude.


\section{Overview of Typed Clojure}

\label{sec:overview}

We now begin a tour of the central features of Typed Clojure,
beginning with Clojure itself. Our presentation
uses the full Typed Clojure system to illustrate key type system
ideas,\footnote{Full examples: \url{https://github.com/typedclojure/esop16}} before studying the core features in detail in
\secref{sec:formal}.

\subsection{Clojure}

Clojure~\cite{Hic08} is a Lisp that runs on the
Java Virtual Machine with support for concurrent programming
and immutable data structures in a mostly-functional
style.
%, restricting imperative updates to a limited set of
%structures each with specific thread synchronization behaviour. 
%Fast implementations of immutable vectors, and hash tables are featured,
%and a means for defining new records.
%
Clojure provides easy interoperation with existing Java libraries, with Java values being like any other Clojure value. 
However, this smooth interoperability comes at the cost of pervasive \java{null}, which leads to the possibility of null pointer exceptions---a drawback we address in Typed Clojure.

%\paragraph{Clojure Syntax}
%
%We describe new syntax as they appear in each example, but
%begin with include the essential basics of Clojure syntax.
%
%\clj{nil} is exactly Java's \java{null}.
%Parentheses indicate \emph{applications}, brackets
%delimit
%\emph{vectors}, braces
%delimit
%\emph{hash-maps}
%and double quotes delimit \emph{Java strings}.
%\emph{Symbols} begin with an alphabetic character,
%and a colon prefixed symbol like \clj{:a} is a \emph{keyword}.
%
%\emph{Commas} are always \emph{whitespace}.

\subsection{Typed Clojure}

A simple one-argument function \clj{greet} is annotated with \clj{ann} to take and return strings.

\begin{lstlisting}
(*typed ann  greet [Str -> Str] typed*)
(defn greet [n] (*invoke str "Hello, " n "!" invoke*))
(*invoke greet "Grace" invoke*) ;=> "Hello, Grace!" :- Str
\end{lstlisting}
%
Providing \clj{nil} (exactly Java's \java{null})
is a static type error---\clj{nil} is not a string.
%
\begin{lstlisting}
(*invoke greet nil invoke*) ; Type Error: Expected Str, given nil
\end{lstlisting}

\paragraph{Unions} To allow \clj{nil}, we use \emph{ad-hoc unions} (\clj{nil} and \clj{false}
are logically false).
%
\begin{lstlisting}
(*typed ann  greet-nil [(U nil Str) -> Str] typed*)
(defn greet-nil [n] (*invoke str "Hello" (when n (*invoke str ", " n invoke*)) "!" invoke*))
(*invoke greet-nil "Donald" invoke*) ;=> "Hello, Donald!" :- Str 
(*invoke greet-nil nil invoke*)      ;=> "Hello!"         :- Str
\end{lstlisting}
%
%
Typed Clojure prevents well-typed code from dereferencing \clj{nil}.
%This is important for Clojure programs---\clj{nil}
%is treated like any other distinct datum in Clojure.

\paragraph{Flow analysis} Occurrence typing~\cite{TF10}
models type-based control flow.
In \clj{greetings}, a branch ensures \clj{repeat}
is never passed \clj{nil}.
%
\begin{lstlisting}
(*typed ann  greetings [Str (U nil Int) -> Str] typed*)
(defn greetings [n i]
  (*invoke str "Hello, " (when i (*invoke apply str (*invoke repeat i "hello, " invoke*) invoke*)) n "!" invoke*))
(*invoke greetings "Donald" 2 invoke*)  ;=> "Hello, hello, hello, Donald!" :- Str
(*invoke greetings "Grace" nil invoke*) ;=> "Hello, Grace!"                :- Str
\end{lstlisting}
%
Removing the branch is a static type error---\clj{repeat} cannot be passed \clj{nil}.
%
\begin{lstlisting}
(*typed ann  greetings-bad [Str (U nil Int) -> Str] typed*)
(defn greetings-bad [n i]           ; Expected Int, given (U nil Int)
  (*invoke str "Hello, " (*invoke apply str (*invoke repeat i "hello, " invoke*) invoke*) n "!" invoke*))
\end{lstlisting}


%\subsection{Type System Basics}
%
%\cite{TF10}
%presented Typed Racket with occurrence typing,
%a technique for deriving type information from conditional control flow.
%They introduced the concept of occurrence typing 
%with the following example.
%
%\inputminted[firstline=3]{racket}{code/tr/example1.rkt}
%
%This function takes a value that is either \emph{\#f} % mintinline really hates #
%or a number, represented by an \emph{untagged} union type.
%The `then' branch has an implicit invariant
%that \rkt{x} is a number, which is automatically inferred with occurrence typing
%and type checked without further annotations.
%
%We chose to build on the ideas and implementation
%of Typed Racket to implement a type system targeting Clojure for several reasons.
%Initially, the similarities between Racket and Clojure drew us to
%investigate the effectiveness of repurposing occurrence typing
%for a Clojure type system---both languages share a Lisp heritage,
%similar standard functions 
%(for instance \clj{map}
%in both languages is variable-arity)
%and idioms.
%While Typed Racket is gradually typed and has sophisticated
%dynamic semantics for cross-language interaction, we 
%chose to first implement
%the static semantics
%with the hope to extend Typed Clojure to be gradually typed at a future date.
%Finally,
%Typed Racket's combination of bidirectional checking
%and occurrence typing presents a successful model for 
%type checking dynamically typed programs without compromising
%soundness, which is appealing over success typing~\cite{Lindahl:2006:PTI}
%which cannot prove strong properties about programs
%and soft typing~\cite{CF91}
%which has proved too complicated in practice.
%
%Here is the above program in Typed Clojure.
%\begin{exmp}
%\inputminted[firstline=5]{clojure}{code/demo/src/demo/eg1.clj}
%\label{example:conditionalflow}
%\end{exmp}
%
%The \clj{fn} macro (provided by core.typed) supports optional annotations by 
%adding
%\clj{:-} and a type after a parameter
%position
%or binding vector 
%to annotate parameter types
%and return types respectively.
%\clj{number?} is
%a Java \java{instanceof} test of \clj{java.lang.Number}.
%As in Typed Racket, \clj{U} creates an \emph{untagged union} type, which can take
%any number of types.
%
%Typed Clojure can already check all of the examples in~\cite{TF10}---the 
%rest of this section describes the extensions necessary
%to check Clojure code.


\subsection{Java interoperability}
\label{sec:overviewjavainterop}

Clojure can interact with Java constructors, methods, and fields.
This program calls the \clj{getParent} on a constructed
\clj{File}
instance, returning a nullable string.

\begin{exmp}
\begin{lstlisting}
(*interop .getParent (*interop new File "a/b" interop*) interop*)  ;=> "a" :- (U nil Str)
\end{lstlisting}
\label{example:getparent-direct-constructor}
\end{exmp}
%
Typed Clojure can integrate with the Clojure compiler to avoid expensive reflective 
calls like \clj{getParent}, however if a specific overload cannot be found based on the
surrounding static context, a type error is thrown.
%
\begin{lstlisting}
(fn [f] (*interop .getParent f interop*)) ; Type Error: Unresolved interop: getParent
\end{lstlisting}
%
Function arguments default to \clj{Any}, which is similar to a union of all types. Ascribing
a parameter type allows Typed Clojure to find a specific method.

%Calls to Java methods and fields have prefix notation
%like \clj{(.method target args*)} and \clj{(.field target)} respectively,
%with method and field names prefixed with a dot and methods taking some number of arguments.

\begin{exmp}
\begin{lstlisting}
(*typed ann parent [(U nil File) -> (U nil Str)] typed*)
(defn parent [f] (if f (*interop .getParent f interop*) nil))
\end{lstlisting}
\label{example:parent-if}
\end{exmp}

%\begin{exmp}
%\inputminted[firstline=18,lastline=19]{clojure}{code/demo/src/demo/parent3.clj}
%\end{exmp}

The conditional guards from dereferencing \clj{nil}, and---as before---removing 
it is a static type error, as typed code could possibly dereference \clj{nil}.
\begin{lstlisting}
(defn parent-bad-in [f :- (U nil File)]
  (*interop .getParent f interop*)) ; Type Error: Cannot call instance method on nil.
\end{lstlisting}

Typed Clojure rejects programs that assume methods cannot return \clj{nil}.
%
\begin{lstlisting}
(defn parent-bad-out [f :- File] :- Str
  (*interop .getParent f interop*)) ; Type Error: Expected Str, given (U nil Str).
\end{lstlisting}
Method targets can never be \clj{nil}.
Typed Clojure also prevents passing \clj{nil} as Java method or
constructor arguments by default---this restriction can be
adjusted per method.

%
%Typed Clojure and Java treat \java{null} differently.
%In Clojure, where it is known as \clj{nil}, Typed Clojure assigns it an explicit type
%called \clj{nil}. In Java \java{null} is implicitly a member of any reference type.
%This means the Java static type \java{String} is equivalent to
%\clj{(U nil String)} in Typed Clojure.
%
%Reference types in Java are nullable, so to guarantee a method call does not
%leak \java{null} into a Typed Clojure program we
%must assume methods can return \clj{nil}.
%
In contrast, JVM invariants guarantee constructors return non-null.\footnote{\url{http://docs.oracle.com/javase/specs/jls/se7/html/jls-15.html#jls-15.9.4}}
%
\begin{exmp}
\begin{lstlisting}
(*invoke parent (*interop new File s interop*) invoke*)
\end{lstlisting}
\end{exmp}


\subsection{Multimethods}

\label{sec:multioverview}

\emph{Multimethods} are a kind of extensible function---combining a \emph{dispatch function} with 
one or more \emph{methods}---widely used to define Clojure operations.

\paragraph{Value-based dispatch}
This simple multimethod takes a keyword (\clj{Kw}) and says hello in different languages.%, as
%specified by a keyword argument.

\begin{exmp}
\begin{lstlisting}
(*typed ann hi [Kw -> Str] typed*) ; multimethod type
(defmulti hi identity) ; dispatch function `identity`
(defmethod hi :en [_] "hello") ; method for `:en`
(defmethod hi :fr [_] "bonjour") ; method for `:fr`
(defmethod hi :default [_] "um...") ; default method
\end{lstlisting}
\label{example:hi-multimethod}
\end{exmp}

When invoked, the arguments are first supplied to the dispatch function---\clj{identity}---yielding
a \emph{dispatch value}. A method is then chosen
based on the dispatch value, to which the arguments are then passed to return a value.
%
\begin{lstlisting}
(*invoke map hi [*vec :en :fr :bocce vec*] invoke*) ;=> (*list "hello" "bonjour" "um..." list*)
\end{lstlisting}
%
For example, 
\clj{(*invoke hi :en invoke*)} evaluates to \clj{"hello"}---it executes
the \clj{:en} method
because \clj{(*invoke = (*invoke identity :en invoke*) :en invoke*)} is true
and \clj{(*invoke = (*invoke identity :en invoke*) :fr invoke*)} is false.

Dispatching based on literal values enables certain forms of method
definition, but this is only part of the story for multimethod dispatch.

\paragraph{Class-based dispatch}
For class values, multimethods can choose methods based on subclassing
relationships.
%
Recall the multimethod from \figref{fig:ex1}. %, reproduced here.
%\begin{minted}{clj}
%(ann pname [(U File String) -> (U nil String)])
%(defmulti pname class)
%(defmethod pname String [s] (pname (new File s)))
%(defmethod pname File [f] (.getName f))
%\end{minted}
%
%Its dispatching function is
%\clj{class}, with two methods associated with dispatch values \clj{java.lang.String} and \clj{java.io.File}
%respectively.
%\noindent
The dispatch function \clj{class}
%---associated at multimethod creation with \clj{defmulti}---
dictates 
whether the \clj{String} or \clj{File} method is chosen.
%---both installed via \clj{defmethod}
%
The multimethod dispatch rules use
\clj{isa?}, a hybrid predicate which is both a subclassing check for classes and
an equality check for other values.

%(isa? (class "STAINS/JELLY") Object) ;=> true
%(isa? (identity :en) :fr) ;=> false
%(isa? (class (new File "JELLY")) String) ;=> false
\begin{lstlisting}
(*invoke isa? :en :en invoke*)       ;=> true
(*invoke isa? String Object invoke*) ;=> true
\end{lstlisting}
%
The current dispatch value and---in turn---each method's associated dispatch value
is supplied to \clj{isa?}. If exactly one method returns true, it is chosen.
%
For example,
the call
  \clj{(*invoke pname "STAINS/JELLY" invoke*)}
picks the \clj{String} method because \clj{(*invoke isa? String String invoke*)}
is true, and
\clj{(*invoke isa? String File invoke*)}
is not.
%---\clj{(class "STAINS/JELLY")}
%is \clj{String}.
%
%The \clj{String} method body
%\clj{(pname (new File "STAINS/JELLY"))}
%chooses the \clj{File} method for opposite reasons.

%The following Typed Clojure program is semantically identical to figure~\ref{fig:ex1}.
%
%\begin{minted}{clj}
%(ann pname [(U Str File) -> (U nil Str)])
%(defn pname [x]
%  ; dispatch value calculated by applying dispatch
%  ; function `class` to argument `x`.
%  (cond
%    ; if (class x) subclasses String, but not File
%    (and (isa? (class x) String)
%         (not (isa? (class x) File)))
%    ; then choose the String method
%    (pname (new File x))
%
%    ; else if (class x) subclasses File, but not String
%    (and (isa? (class x) File)         
%         (not (isa? (class x) String)))
%    ; then choose the File method
%    (.getName x)
%    :else (throw (Exception. "No match"))))
%\end{minted}
%
%An unambiguous match leads to the corresponding method being applied to the arguments,
%giving the final result.

%\subsection{Multimethods}
%
%A multimethod in Clojure is a function with a \emph{dispatch
%function} and a \emph{dispatch table} of methods. Multimethods are created with {\clj{defmulti}}.
%\inputminted[firstline=5,lastline=6]{clojure}{code/demo/src/demo/rep.clj}
%The multimethod \clj{path} has type \clj{[Any -> (U nil String)]}, an initially empty \emph{dispatch table}
%and \emph{dispatch function} \clj{class}, a function that
%returns the class of its argument or \clj{nil} if passed \clj{nil}.
%
%We can use {\clj{defmethod}} to install a method to \clj{path}.
%\inputminted[firstline=7,lastline=7]{clojure}{code/demo/src/demo/rep.clj}
%Now the dispatch table maps
%the \emph{dispatch value} \clj{String} to the function
%\clj{(fn [x] x)}. 
%We add another method
%which maps
%\clj{File} to the function
%\clj{(fn [x] (.getPath x))}
%in the dispatch table.
%\inputminted[firstline=8,lastline=8]{clojure}{code/demo/src/demo/rep.clj}
%
%After installing both methods, the call 
%$$
%\clj{(path (new File "dir/a"))}
%$$
%dispatches to the second method we installed because
%$$
%\clj{(isa? (class "dir/a") String)}
%$$
%is true, and finally returns 
%$$
%\clj{((fn [x] (.getPath x)) "dir/a")}.
%$$

%We include the above sequence of definitions as \egref{example:rep}.
%
%\begin{Code}
%\begin{exmp}
%\inputminted[firstline=5,lastline=10]{clojure}{code/demo/src/demo/rep.clj}
%\label{example:rep}
%\end{exmp}
%\end{Code}
%
%Typed Clojure does not predict if a runtime dispatch will be successful---\clj{(path :a)} 
%type checks because \clj{:a} agrees with the parameter type \clj{Any},
%but throws an error at runtime.

%\paragraph{Multiple dispatch} \clj{isa?} is special with vectors---vectors of the
%same length recursively call \clj{isa?} on the elements pairwise.
%\begin{minted}{clojure}
%  (isa? [Keyword Keyword] [Object Object]) ;=> true
%\end{minted}
%
%\inputminted[firstline=6,lastline=23]{clojure}{code/demo/src/demo/eg7.clj}
%
%\egref{example:multidispatch}
%simulates multiple dispatch by dispatching on
%a vector containing the class of both arguments. \clj{open}
%takes two arguments which can be strings or files and returns
%a new file that concatenates their paths.
%
%We call three different \clj{File} constructors, each known at compile-time
%via type hints.
%Multiple dispatch follows the same kind of reasoning as \egref{example:incmap},
%except we update multiple bindings simultaneously.

\subsection{Heterogeneous hash-maps}

%Beyond primitives and Java objects, 
The most common way to represent compound data in Clojure 
are immutable hash-maps, typicially with keyword keys.
%
Keywords double as functions that
look themselves up in a map, or return \clj{nil} if absent.
%
\begin{exmp}
\begin{lstlisting}
(def breakfast {*map :en "waffles" :fr "croissants" map*})
(*invoke :en breakfast invoke*)    ;=> "waffles" :- Str
(*invoke :bocce breakfast invoke*) ;=> nil       :- nil
\end{lstlisting}
\label{example:breakfastcomplete}
\end{exmp}

\emph{HMap types} describe the most common usages of
keyword-keyed maps.
\begin{lstlisting}
breakfast ; :- (HMap :mandatory {:en Str, :fr Str}, :complete? true)
\end{lstlisting}
This says
\clj{:en} and \clj{:fr} are known entries mapped to strings,
and the map is fully specified---that is, no other entries exist---by \clj{:complete?} being \clj{true}.

HMap types default to partial specification, with
\clj{'\{:en Str :fr Str\}} abbreviating \clj{(HMap :mandatory \{:en Str, :fr Str\})}.
%
\begin{exmp}
\begin{lstlisting}
(*typed ann lunch '{:en Str :fr Str} typed*)
(def lunch {*map :en "muffin" :fr "baguette" map*})
(*invoke :bocce lunch invoke*) ;=> nil :- Any ; less accurate type
\end{lstlisting}
\label{example:lunchpartial}
\end{exmp}
%(:en lunch)    ; :- Str
%;=> "muffin"
%(:fr lunch)    ; :- Str
%;=> "baguette"

\paragraph{HMaps in practice} The next example is extracted from a production system at CircleCI,
a company with a large production Typed Clojure system
(\secref{sec:casestudy} presents a case study and empirical
result from this code base).

\newpage

\begin{exmp}
\begin{lstlisting}
(*typed defalias RawKeyPair ; extra keys disallowed
  (HMap :mandatory {:pub RawKey, :priv RawKey}, 
        :complete? true) typed*)
(*typed defalias EncKeyPair ; extra keys disallowed
  (HMap :mandatory {:pub RawKey, :enc-priv EncKey}, :complete? true) typed*)

(*typed ann enc-keypair [RawKeyPair -> EncKeyPair] typed*)
(defn enc-keypair [kp]
  (*invoke assoc (*invoke dissoc kp :priv invoke*) :enc-priv (*invoke encrypt (*invoke :priv kp invoke*) invoke*) invoke*))
\end{lstlisting}
\label{example:circleci}
\end{exmp}
%
%(ann enc-keypair [RawKeyPair -> EncKeyPair])
%(defn enc-keypair [{pk :priv :as kp}] ; original map is kp
%  (assoc (dissoc kp :priv)       ; remove unencrypted private key
%         :enc-priv (encrypt pk))) ; add encrypted private key
%
%\inputminted[firstline=10,lastline=22]{clojure}{code/demo/src/demo/key.clj}
As \clj{EncKeyPair} is fully specified, we remove extra keys like \clj{:priv}
via \clj{dissoc}, which returns a new map that is the first argument without the
entry named by the second argument. Notice removing \clj{dissoc} causes a type error.
%
\begin{lstlisting}
(defn enc-keypair-bad [kp] ; Type error: :priv disallowed
  (*invoke assoc kp :enc-priv (*invoke encrypt (*invoke :priv kp invoke*) invoke*) invoke*))
\end{lstlisting}

%\clj{enc-keypair} takes an unencrypted keypair and returns an encrypted keypair by
%dissociating the raw \clj{:priv} entry with \clj{dissoc}
%and associating an encrypted private key
%as \clj{:enc-priv} on an immutable map with \clj{assoc}.
%The expression \clj{(:priv kp)} shows that keywords are also 
%functions that look themselves up in a map returning the associated value or \nil{} if the key is missing.
%Since \clj{EncKeyPair} is \clj{:complete?}, Typed Clojure enforces the return type
%does not contain an entry \clj{:priv}, and would complain if the \clj{dissoc}
%operation forgot to remove it.

%\egref{example:absentkeys}
%is like \egref{example:circleci}
%except the \clj{:absent-keys} HMap option is used
%instead of \clj{:complete?},
%which takes a \emph{set literal} of keywords that do not appear in the map, written 
%with \emph{\#}-prefixed braces.
%The syntax \clj{(fn [{pkey :priv, :as kp}] ...)}
%aliases \clj{kp} to the first argument and \clj{pkey} to \clj{(:priv m)}
%in the function body.
%
%\begin{exmp}
%\inputminted[firstline=10,lastline=21]{clojure}{code/demo/src/demo/key2.clj}
%\label{example:absentkeys}
%\end{exmp}
%
%Since this example enforces that \clj{:priv} must not appear
%in a \clj{EncKeyPair}
%Typed Clojure would still complain if we forgot to \clj{dissoc} \clj{:priv}
%from the return value.
%Now, however we could stash the raw private key in another entry
%like \clj{:secret-key} which is not mentioned by the partial HMap \clj{EncKeyPair}
%without Typed Clojure noticing.

%\paragraph{Branching on HMaps} Finally, testing on HMap properties
%allows us to refine its type down branches. \clj{dec-map} takes an
%\clj{Expr}, traverses to its nodes and decrements their values by \clj{dec}, then
%builds the \clj{Expr} back up with the decremented nodes.
%
%\begin{exmp}
%\inputminted[linenos,firstnumber=1,firstline=15,lastline=27]{clojure}{code/demo/src/demo/hmap.clj}
%\label{example:decmap}
%\end{exmp}
%
%If we go down the then branch (line 4), since \clj{(= (:op m) :if)} is true
%we remove
%the \clj{:do} and \clj{:const}
%Expr's from the type of \clj{m} (because their respective \clj{:op} entries disagrees with \clj{(= (:op m) :if)})
%and we are left with an \clj{:if} Expr.
%On line 8,
%we instead strike out the \clj{:if} Expr since it contradicts \clj{(= (:op m) :if)} being false. 
%Line 9 know we can
%remove the \clj{:const} Expr from the type of \clj{m} because it contradicts \clj{(= (:op m) :do)} being true,
%and we know \clj{m} is a \clj{:do} Expr.
%Line 12 we strike out \clj{:do} because \clj{(= (:op m) :do)} is false,
%so we are left with \clj{m} being a \clj{:const} Expr.
%
%Section~\ref{sec:formalpaths} discusses how this automatic reasoning is achieved.

\subsection{HMaps and multimethods, joined at the hip}

HMaps and multimethods are the primary ways for representing
and dispatching on data respectively, and so are intrinsically linked.
As type system designers, we must
search for a compositional approach that can anticipate
any combination of these features.

Thankfully, occurrence typing, originally designed for reasoning about
\clj{if} tests, provides the compositional approach we need.
By extending the system with
a handful of rules based on HMaps and other functions, 
we can automatically cover both easy cases and those
that compose rules in arbitrary ways.

Futhermore, this approach extends to multimethod dispatch by reusing
occurrence typing's approach to conditionals
%, whose branching
%mechanism may appear complex, but
%can be understood in terms of the humble \clj{if} conditional. 
and
encoding a small number of rules to handle
the \clj{isa?}-based dispatch.
In practice, conditional-based control flow typing
extends to multimethod dispatch, and vice-versa.

We first demonstrate a very common, simple dispatch style,
then move on to deeper structural dispatching where occurrence typing's
compositionality shines.

\paragraph{HMaps and unions} Partially specified HMap's with a common dispatch key
combine naturally with ad-hoc unions.
An \clj{Order} is one of three kinds of HMaps.

%FIXME define defalias above and keyword singletons

\begin{lstlisting}
(*typed defalias Order "A meal order, tracking dessert quantities."
  (U '{:Meal ':lunch, :desserts Int} '{:Meal ':dinner :desserts Int}
     '{:Meal ':combo :meal1 Order :meal2 Order}) typed*)
\end{lstlisting}

The \clj{:Meal} entry is common to each HMap, always mapped to a known keyword singleton
type.
It's natural to dispatch on the \clj{class} of an instance---it's similarly
natural to dispatch on a known entry like \clj{:Meal}.

\newpage

\begin{exmp}
\begin{lstlisting}
(*typed ann desserts [Order -> Int] typed*)
(defmulti desserts :Meal)  ; dispatch on :Meal entry
(defmethod desserts :lunch [o] (*invoke :desserts o invoke*))
(defmethod desserts :dinner [o] (*invoke :desserts o invoke*))
(defmethod desserts :combo [o] 
  (*invoke + (*invoke desserts (*invoke :meal1 o invoke*) invoke*) (*invoke desserts (*invoke :meal2 o invoke*) invoke*) invoke*))

(*invoke desserts {*map :Meal :combo, :meal1 {*map :Meal :lunch :desserts 1 map*}, 
           :meal2 {*map :Meal :dinner :desserts 2 map*} map*} invoke*) ;=> 3
\end{lstlisting}
\label{example:desserts-on-meal}
\end{exmp}
%
The \clj{:combo} method is verified to only structurally recur
on \clj{Order}s. This is achieved because we learn the argument \clj{o} must
be of type
\clj{'\{:Meal :combo\}}
since
\clj{(isa? (:Meal o) :combo)}
is true. Combining this
with the fact that \clj{o} is an \clj{Order}
eliminates possibility of \clj{:lunch} and \clj{:dinner}
orders, simplifying \clj{o} to
\clj{'\{:Meal ':combo :meal1 Order :meal2 Order\}}
which contains appropriate arguments for both recursive calls.

\paragraph{Nested dispatch}
A more exotic dispatch mechanism for \clj{desserts}
might be on the \clj{class} of the \clj{:desserts} key.
If the result is a number, then we know the \clj{:desserts}
key is a number, otherwise the input is a \clj{:combo} meal.
We have already seen dispatch on \clj{class} and on keywords
in isolation---occurrence typing automatically understands
control flow that combines its simple building blocks.

The first method has dispatch value \clj{Long}, a subtype
of \clj{Int}, and the second method has \clj{nil}, the sentinel value for a failed map lookup.
In practice, \clj{:lunch} and \clj{:dinner} meals will dispatch to the \clj{Long}
method, but Typed Clojure infers a slightly more general type due to the definition
of \clj{:combo} meals.

\begin{exmp}
\begin{lstlisting}
(*typed ann desserts' [Order -> Int] typed*)
(defmulti desserts' 
  (fn [o :- Order] (*invoke class (*invoke :desserts o invoke*) invoke*)))
(defmethod desserts' Long [o] 
;o :- (U '{:Meal (U ':dinner ':lunch), :desserts Int}
;       '{:Meal ':combo, :desserts Int, :meal1 Order, :meal2 Order})
  (*invoke :desserts o invoke*))
(defmethod desserts' nil [o]
  ; o :- '{:Meal ':combo, :meal1 Order, :meal2 Order}
  (*invoke + (*invoke desserts' (*invoke :meal1 o invoke*) invoke*) (*invoke desserts' (*invoke :meal2 o invoke*) invoke*) invoke*))
\end{lstlisting}
\label{example:desserts-on-class}
\end{exmp}
%
%(desserts' {:Meal :combo 
%            :meal1 {:Meal :lunch :desserts 1}
%            :meal2 {:Meal :dinner :desserts 2}})
%;=> 3

In the \clj{Long} method, Typed Clojure learns that
its argument is at least of type \clj{'\{:desserts Long\}}---since
\clj{(*invoke isa? (*invoke class (*invoke :desserts o invoke*) invoke*) Long invoke*)}
must be true.
%
%Knowing \clj{o} is also an
%\clj{Order},
Here the
\clj{:desserts} entry
\emph{must} be present and mapped to a \clj{Long}---even in a \clj{:combo} meal,
which does not specify \clj{:desserts}
as present or absent.

In the \clj{nil} method,
\clj{(*invoke isa? (*invoke class (*invoke :desserts o invoke*) invoke*) nil invoke*)}
must be true---which implies \clj{(*invoke class (*invoke :desserts o invoke*) invoke*)} is \clj{nil}.
%
Since lookups on missing keys return \clj{nil}, either
\begin{itemize}
  \item \clj{o} has a \clj{:desserts} entry to \clj{nil}, like \clj{{:desserts nil}}, or
  \item \clj{o} is missing a \clj{:desserts} entry.%, like \clj{{}}.
\end{itemize}
We can express this type with the \clj{:absent-keys} HMap option
%Equivalently, \clj{o} is of type
% Note: mintedinline doesn't work with hash characters #
\begin{lstlisting}
(U '{:desserts nil} (HMap :absent-keys #{:desserts}))
\end{lstlisting}
This eliminates non-\clj{:combo} meals
since their \clj{'\{:desserts Int\}} type does not agree
with this new information (because \clj{:desserts}
is neither \clj{nil} or absent).

%simplifies to a \clj{:combo} meal, 
%\begin{minted}{clojure}
%'{:Meal ':combo :meal1 Order :meal2 Order}
%\end{minted}
%thus allowing both recursive calls to type check.

\paragraph{From multiple to arbitrary dispatch}
Clojure multimethod dispatch, and Typed Clojure's handling of it, goes
even further, supporting dispatch on multiple arguments via vectors.
%
Dispatch on multiple arguments is beyond the scope of this paper,
but the same intuition applies---adding support for multiple dispatch
admits arbitrary combinations and nestings
of it and previous dispatch rules.

%\begin{exmp}
%\inputminted[firstline=6,lastline=13]{clojure}{code/demo/src/demo/hmap.clj}
%\label{example:decleaf}
%\end{exmp}
%
%The \clj{defn} macro defines a top-level function, with syntax like the typed \clj{fn}.
%The function \clj{an-exp} is verified to return an \clj{Expr}.
%
%Here \clj{defalias} defines \clj{Expr}, a type abbreviation
%that describes the structure of a recursively-defined AST as a union of HMaps.
%Keyword singleton types are quoted---\clj{':lunch}.
%A type that is a quoted map like \clj{'{:op ':if}} is a
%HMap type with a fixed number of keyword entries of the specified types
%known to be \emph{present},
%zero entries known to absolutely be \emph{absent},
%and an infinite number of \emph{unknown} entries entries.
%Since only keyword keys are allowed, they do not require quoting.

%\paragraph{HMap dispatch} The flexibility of \clj{isa?} is key to the generality of multimethods. 
%In \egref{example:incmap} we
%dispatch on the \clj{:op} key 
%of our HMap AST \clj{Expr}.
%Since keywords are functions that look themselves up in their argument, we simply
%use \clj{:op} as the dispatch function.
%
%\begin{exmp}
%\inputminted[firstline=5,lastline=18]{clojure}{code/demo/src/demo/eg5.clj}
%\label{example:incmap}
%\end{exmp}
%
%The function \clj{inc-leaf} is like \egref{example:decmap} except the nodes are incremented.
%The reasoning is similar, except we only consider one branch (the current method) by
%locally considering the current \emph{dispatch value} and reasoning about how it relates
%to the \emph{dispatch function}.
%For example, 
%in the \clj{:do} method we learn the \clj{:op} key is a \clj{:do}, which
%narrows our argument type to the \clj{:do} Expr, and similarly for the \clj{:if}
%and \clj{:const} methods.
%
%
%\subsection{Final example}
%
%\egref{example:final}
%combines everything we will cover for the rest of the paper:
%multimethod dispatch, reflection resolution via type hints, Java method
%and constructor calls, conditional and exceptional flow reasoning,
%and HMaps. 
%
%
%\begin{figure}
%\begin{exmp}
%\inputminted[firstline=6,lastline=23]{clojure}{code/demo/src/demo/eg7.clj}
%\label{example:multidispatch}
%\end{exmp}
%\begin{exmp}
%\inputminted[firstline=6,lastline=20]{clojure}{code/demo/src/demo/eg8.clj}
%\label{example:final}
%\end{exmp}
%\caption{Multimethod Examples}
%\end{figure}
%
%We dispatch on \clj{:p} to distinguish the two cases of \clj{FSM}---for example on \clj{:F}
%we know the \clj{:file} is a file.
%The body of the first method uses type hints to resolve reflection
%and conditional control flow to prove null-pointer exceptions are impossible.
%The second method is similar except it uses exceptional control flow.

\section{A Formal Model of \lambdatc{}}

\label{sec:formal}

After demonstrating the core features of Typed Clojure, 
we link them together in a formal model called
\lambdatc{}.
%
Building on occurrence typing,
we incrementally add each
novel feature of Typed Clojure to the formalism,
interleaving presentation of syntax, typing rules, operational semantics,
and subtyping.

\subsection{Core type system}
\label{sec:coretypesystem}

We start with a review of
occurrence typing~\cite{TF10}, the foundation of \lambdatc{}.
%We build up the occurrence typing calculus for illustrative purposes, 
%and present the full syntax at the end of the section.

\paragraph{Expressions} Syntax is given in \figref{main:figure:termsyntax}. Expressions \e{} 
include variables \x{}, values \v{},
applications, abstractions, conditionals, and let expressions.
All binding forms introduce fresh variables---a subtle but important point since our type environments
are not simply dictionaries.
Values include booleans \bool{}, \nil{}, class literals {\class{}}, keywords \k{},
integers {\nat{}},
constants {\const{}}, and strings \str{}. Lexical closures {\closure {\openv{}} {\abs {\x{}} {\t{}} {\e{}}}}
close value environments \openv{}---which map bindings to values---over functions.

\paragraph{Types} Types \s{} or \t{} 
include the top type \Top,
\emph{untagged} unions {\Unionsplice {\overrightarrow{\t{}}}}, 
singletons ${\Value \singletonmeta{}}$,
and class instances \class{}.
We abbreviate the classes
\Booleanlong{} to \Boolean{}, 
\Keywordlong{} to \Keyword{},
\NumberFull{}  to \Number{},
\StringFull{}  to \String{}, and 
\FileFull{}  to \File{}.
We also abbreviate the types
\EmptyUnion{}     to \Bot{}, 
{\ValueNil}       to \Nil{}, 
{\ValueTrue}      to \True, and
{\ValueFalse} to {\False}.
%
The difference between the types
\Value{\class{}} and \class{} is subtle.
The former is inhabited by class literals like \Keyword{} and the result of 
\appexp{\classconst{}}{\makekw{a}}---the latter by \emph{instances} of classes,
like a keyword literal \makekw{a}, an instance of the type \Keyword{}.
%
Function types 
$
{\ArrowOne {\x{}} {\s{}}
             {\t{}}
             {\filterset {\prop{}} {\prop{}}}
             {\object{}}}
$
contain \emph{latent} (terminology from~\cite{Lucassen88polymorphiceffect}) propositions \prop{}, object \object{}, and return type
\t{},
which may refer to the function argument \x{}.
%Latent means they are relevant when the function is applied rather than evaluated.
They are instantiated with the
actual object of the argument in applications. % before they are used in the proposition environment.

\paragraph{Objects}
%As we saw in \secref{sec:overview},
%occurrence typing is capable of reasoning
%about deeply nested expressions.
Each expression is associated with 
a symbolic representation
called an \emph{object}.
%with respect to the current lexical environment. 
For example,
  variable \makelocal{m} has object \makelocal{m};
  $\appexpone{\ccclass{\appexp{\makekw{lunch}}{\makelocal{m}}}}$ has object ${\path{\classpe{}}{\path{\keype{\makekw{lunch}}}{\makelocal{m}}}}$; and $42$ has the \emph{empty} object \emptyobject{} since it is unimportant in our system.
%
\figref{main:figure:termsyntax} gives the syntax for objects \object{}---non-empty objects 
\path{\pathelem{}}{\x{}} combine of a root variable \x{} and a \emph{path} \pathelem{},
which consists of
a possibly-empty sequence of \emph{path elements} (\pesyntax{}) applied right-to-left from the root variable.
We use two path elements---{\classpe{}} and {\keype{k}}---representing the results
of calling \classconst{} and looking up a keyword $k$, respectively.

\paragraph{Propositions with a logical system}
In standard type systems, association lists often
track the types of variables, like in LC-Let and LC-Local.
\begin{mathpar}
\infer [LC-Let]
{ \judgementtwo {{\propenv{}}}
                {\e{1}} {\s{}}
  \\
  \judgementtwo {{\propenv{}},\x{} \mapsto {\s{}}}
                {\e{2}} {\t{}}
}
{ 
  \judgementtwo {\propenv{}} 
            {\letexp{\x{}}{\e{1}}{\e{2}}} {\t{}}
           }

\infer [LC-Local]
{ {\propenv{}}(\x{}) = {\t{}}
}
{ \judgementtwo {\propenv{}} 
            {\x{}} {\t{}}
           }
\end{mathpar}

Occurrence typing instead pairs \emph{logical formulas},
that can reason about arbitrary non-empty objects,
with a \emph{proof system}.
The logical statement {\isprop{\s{}}{\x{}}} says
variable $x$ is of type \s{}. 
%A \emph{logical system}
%must now \emph{prove}
%a variable's type.
\begin{mathpar}
\infer [T0-Let]
{ \judgementtwo {{\propenv{}}}
                {\e{1}} {\s{}}
  \\
  \judgementtwo {{\propenv{}},\isprop{\s{}}{\x{}}}
                {\e{2}} {\t{}}
}
{ 
  \judgementtwo {\propenv{}} 
            {\letexp{\x{}}{\e{1}}{\e{2}}}
            {\t{}}
           }
%\judgementtwo{\isprop{\Number{}}{\x{}}}{\appexp{\inc{}}{\x{}}}{\Number}

\infer [T0-Local]
{ \inpropenv {\propenv{}} {\isprop {\t{}} {\x{}}}}
{ \judgementtwo {\propenv{}} 
            {\x{}} {\t{}}
           }
\end{mathpar}
In T0-Local, 
$
{ \inpropenv {\propenv{}} {\isprop {\t{}}{\x{}}}}
$
appeals to the proof system to solve for \t{}.
%says under logical assumptions {\propenv{}}, object {\path{\pathelem{}}{\x{}}} is of type \t{}.
%We later define the more general T-Local.

\begin{figure}[t!]
  \footnotesize
$$
\begin{array}{lrll}
  \e{} &::=& \x{}
                      \alt \v{} 
                      \alt {\comb {\e{}} {\e{}}} 
                      \alt {\abs {\x{}} {\t{}} {\e{}}} 
                      \alt {\ifexp {\e{}} {\e{}} {\e{}}}
                      %\alt {\trdiff{\doexp {\e{}} {\e{}}}}
                      \alt {\letexp {\x{}} {\e{}} {\e{}}}
                      %\alt {\errorvalv{}}
                      &\mbox{Expressions} \\
  \v{} &::=&          \singletonmeta{}
                      \alt {\nat{}}
                      \alt {\const{}}
                      \alt {\str{}}
                      \alt {\closure {\openv{}} {\abs {\x{}} {\t{}} {\e{}}}}
                &\mbox{Values} \\
                \constantssyntax{}\\
  \s{}, \t{}    &::=& \Top 
                      \alt {\Unionsplice {\overrightarrow{\t{}}}}
                      \alt
                      {\ArrowOne {\x{}} {\t{}}
                                   {\t{}}
                                   {\filterset {\prop{}} {\prop{}}}
                                   {\object{}}}
                      \alt {\Value \singletonmeta{}} 
                      \alt \trdiff{\class{}}
                &\mbox{Types} \\
  \singletonallsyntax{}
                \\ \\
  \occurrencetypingsyntax{}\\
  \propenvsyntax{}\\
  \openvsyntax{}
  %\\
  %\classliteralallsyntax{}
\end{array}
$$
\caption{Syntax of Terms, Types, Propositions and Objects}
\label{main:figure:termsyntax}
\end{figure}

We further extend logical statements to \emph{propositional logic}.
\figref{main:figure:termsyntax} describes the syntax
for propositions \prop{},
consisting of positive and negative \emph{type propositions} 
about non-empty objects---{\isprop {\t{}} {\path {\pathelem{}} {\x{}}}}
and {\notprop {\t{}} {\path {\pathelem{}} {\x{}}}}
respectively---the latter pronounced ``the object {\path {\pathelem{}} {\x{}}} is \emph{not} of type \t{}''.
The other propositions are standard logical connectives: implications, conjunctions,
disjunctions, and the trivial (\topprop{}) and impossible (\botprop{}) propositions.
%
The full proof system judgement
$
{ \inpropenv {\propenv{}} {\prop{}} }
$
says \emph{proposition environment} {\propenv{}} proves proposition \prop{}.

Each expression is associated with two propositions---when expression
\e{1} is in test position like
\ifexp{\e{1}}{\e{2}}{\e{3}},
the type system extracts \e{1}'s `then' and `else' proposition to check
\e{2} and \e{3} respectively.
For example, in \ifexp{\makelocal{o}}{\e{2}}{\e{3}}
we learn variable {\makelocal{o}} is true in \e{2} via {\makelocal{o}}'s `then' proposition $\notprop{\falsy{}}{\makelocal{o}}$, and 
that {\makelocal{o}} is false in \e{3} via {\makelocal{o}}'s `else' proposition $\isprop{\falsy{}}{\makelocal{o}}$.

To illustrate, recall \egref{example:desserts-on-meal}.
The parameter \makelocal{o} is of type $\Order$,
%by the annotation on $desserts$
written
{\isprop{\Order}{\makelocal{o}}}
as a proposition.
%
In the ${\makekw{combo}}$ method, we know
${\appexp{\makekw{Meal}}{\makelocal{o}}}$ is ${\makekw{combo}}$,
based on multimethod dispatch rules. This is written
  {\isprop{\Value{\makekw{combo}}}{\path{\keype{\makekw{Meal}}}{\makelocal{o}}}},
pronounced ``the ${\makekw{Meal}}$ path of variable \makelocal{o} is of type
{\Value{\makekw{combo}}}''.

%\paragraph{Logical system in action} 
To attain the type of \makelocal{o}, 
we must solve for \t{} in
$
{ \inpropenv 
  {\propenv{}}
  {\isprop {\t{}} {\makelocal{o}}}}
$,
under proposition environment
$
\propenv{} = {{\isprop{\Order}{\makelocal{o}}},
    {\isprop{\Value{\makekw{combo}}}{\path{\keype{\makekw{Meal}}}{\makelocal{o}}}}}
$
which deduces \t{} to be a {\makekw{combo}} meal.
The logical
system \emph{combines} pieces of type information to deduce more accurate types for lexical
bindings---this is explained in \secref{formalmodel:proofsystem}.

%The first insight about occurrence typing is that
%logical formulas
%can be used to represent type information about our programs
%by relating parts of the runtime environment to types
%via propositional logic.
%\emph{Type propositions}  make assertions like ``variable \x{} is of type \NumberFull{}'' or
%``variable \x{} is not \nil{}''---in our logical system we write these as
%{\isprop{\NumberFull}{\x{}}}
%and {\notprop{\Nil{}}{\x{}}}. 
%
%The second insight is that we can replace the traditional 
%representation of a
%type environment (eg., a map from variables to types)
%with a set of propositions, written \propenv{}. 
%Instead of mapping \x{} to
%the type \NumberFull{}, we use the proposition {\isprop{\NumberFull}{\x{}}}.





\begin{figure*}[t]
\footnotesize
    %{\TDo}
    %{\TClass}
    %{\TIf}
    %{\TAbs}
    %\begin{array}{c}
    %  {\TSubsume}\\\\
    %  {\TNum}
    %\end{array}
  \begin{mathpar}
        {\TLocal}
        {\TAbs}
        {\TIf}
    \\
    \begin{array}{c}
    {\TKw}\\
      {\TNum}\\
    \end{array}
    \begin{array}{c}
      {\TNil}\\
      {\TFalse}\\
    {\TConst}
    \end{array}
    \begin{array}{c}
    {\TStr}\\
    {\TClass}\\
    {\TTrue}
  \end{array}
        \\

    {\TLet}
    \\

    {\TApp}\ \ 
    {\TSubsume}
    \\
  \end{mathpar}
    %\begin{array}{c}
    %  {\TSubsume}\\\\
    %  {\TStr}\\\\
    %  {\TNil}\\\\
    %  {\TFalse}
    %\end{array}
  \caption{Core typing rules}
  \label{main:figure:othertypingrules}
\end{figure*}

\begin{figure}%[t!]
  \footnotesize
  \begin{mathpar}

\SUnionSuper{}\ \ \ 
\SUnionSub{}\ \ \ 
\SFunMono{}\ \ \ 
\begin{array}{l}
\SObject{}\\
\SClass{}\\
\SSBool{}
\end{array}

\SFun{}
\begin{array}{l}
    \SRefl{}\ \ \ 
    \STop{}\\
\SSKw{}
\end{array}


  \end{mathpar}
  \caption{Core subtyping rules}
  \label{main:figure:subtyping}
\end{figure}

\begin{figure}
\begin{mathpar}
    \BIfTrue{}

    \BIfFalse{}
\end{mathpar}
  \caption{Select core semantics}
\label{main:figure:coresemantics}
\end{figure}

\paragraph{Typing judgment}

We formalize our system following Tobin-Hochstadt and Felleisen \cite{TF10}.
%(with differences highlighted in $\trdiff{\text{blue}}$)
The typing judgment 
$
{\judgementrewrite   {\propenv}
              {\e{}} {\t{}}
  {\filterset {\thenprop {\prop{}}}
              {\elseprop {\prop{}}}}
  {\object{}}
  {\ep{}}}
$
says expression \e{} rewrites to \ep{}, which
is of type \t{} in the 
proposition environment $\propenv{}$, with 
`then' proposition {\thenprop {\prop{}}}, `else' proposition {\elseprop {\prop{}}}
and object \object{}. 

We write 
{\judgementtworewrite{\propenv}{\e{}} {\t{}}{\ep{}} 
to mean 
{\judgementrewrite   {\propenv}
              {\e{}} {\t{}}
  {\filterset {\thenprop {\propp{}}}
              {\elseprop {\propp{}}}}
  {\objectp{}}
  {\ep{}}}
for some {\thenprop {\propp{}}}, {\elseprop {\propp{}}}
and {\objectp{}},
and
  abbreviate self rewriting judgements
{\judgementrewrite   {\propenv}
              {\e{}} {\t{}}
  {\filterset {\thenprop {\prop{}}}
              {\elseprop {\prop{}}}}
  {\object{}}
  {\e{}}}
to
{\judgementselfrewrite   {\propenv}
              {\e{}} {\t{}}
  {\filterset {\thenprop {\prop{}}}
              {\elseprop {\prop{}}}}
  {\object{}}}.


\paragraph{Typing rules}

The core typing rules are
given as \figref{main:figure:othertypingrules}. We introduce
the interesting rules with the complement number predicate
as a running example.
\begin{equation}
\abs{\makelocal{d}}{\Top}{\ifexp{\appexp{\numberhuh{}}{\makelocal{d}}}{\false{}}{\true{}}}
\end{equation}
%, including
%a subsumption rule T-Subsume and rules for the false values---T-Nil and T-False---encoded 
%as impossible (\botprop{}) `then' propositions.

The lambda rule T-Abs introduces \isprop{\s{}}{\x{}}} = \isprop{\Top}{\makelocal{d}}
to check the body.
With \propenv{} = \isprop{\Top}{\makelocal{d}},
T-If first checks the test \e{1} = {\appexp{\numberhuh{}}{\makelocal{d}}}
via the T-App rule, with three steps.

First, in T-App the operator \e{} = \numberhuh{} is checked with T-Const, which
uses 
\constanttypeliteral{} (\figref{main:figure:constanttyping}, dynamic semantics in the supplemental material)
to type constants.
\numberhuh{} is a predicate over numbers, and
\classconst{} returns its argument's class.

Resuming {\appexp{\numberhuh{}}{\makelocal{d}}},
in T-App the operand \ep{} = \makelocal{d} is checked with
T-Local as
\begin{equation}
\judgementselfrewrite{\propenv{}}
                     {\makelocal{d}}
                     {\Top}
                     {\filterset{\notprop{\falsy}{\makelocal{d}}}
                                {\isprop{\falsy}{\makelocal{d}}}}
                     {\makelocal{d}}
\end{equation}
which encodes the type, proposition, and object information
about variables. The proposition {\notprop{\falsy}{\makelocal{d}}}
says ``it is not the case that variable {\makelocal{d}} is of type {\falsy}'';
{\isprop{\falsy}{\makelocal{d}}} says ``{\makelocal{d}} is of type {\falsy}''.

Finally, the T-App rule substitutes the operand's object \objectp{}
for the parameter \x{} in the latent type, propositions, and object. The proposition
{\isprop{\Number{}}{\makelocal{d}}} says ``{\makelocal{d}} is of type {\Number{}}'';
{\notprop{\Number{}}{\makelocal{d}}} says ``it is not the case that {\makelocal{d}}
is of type {\Number{}}''. The object {\makelocal{d}} is the symbolic representation
of what the expression {\makelocal{d}} evaluates to.
\begin{equation}
\judgementselfrewrite{\propenv{}}
  {\appexp{\numberhuh{}}{\makelocal{d}}}
  {\Boolean{}}
  {\filterset{\isprop{\Number{}}{\makelocal{d}}}
             {\notprop{\Number{}}{\makelocal{d}}}}
  {\emptyobject{}}
\end{equation}
To demonstrate, the `then' proposition---in T-App {\replacefor {\thenprop{\prop{}}} {\objectp{}} {\x{}}}---substitutes
the latent `then' proposition of \constanttype{\numberhuh{}} with 
\makelocal{d}, giving
{\replacefor {\isprop{\Number{}}{\x{}}} {\makelocal{d}} {\x{}}} =
{\isprop{\Number{}}{\makelocal{d}}}.

To check the branches of {\ifexp{\appexp{\numberhuh{}}{\makelocal{d}}}{\false{}}{\true{}}},
T-If
introduces \thenprop{\prop{1}} = \isprop{\Number{}}{\makelocal{d}}
to check \e{2} = {\false{}},
and \elseprop{\prop{1}} = \notprop{\Number{}}{\makelocal{d}}
to check 
\e{3} = \true{}.
%
The branches are first checked with T-False and T-True respectively,
the T-Subsume premises
\inpropenv {\propenv{}, {\thenprop {\prop{}}}} {\thenprop {\propp{}}}
and
\inpropenv {\propenv{}, {\elseprop {\prop{}}}} {\elseprop {\propp{}}}
allow us to pick compatible propositions for both branches.
%$$
%\judgementselfrewrite{\propenv{},{\isprop{\Number{}}{\makelocal{d}}}}
%  {\false{}}
%  {\False{}}
%  {\filterset{\botprop{}}
%             {\topprop{}}}
%  {\emptyobject{}}
%$$
$$
\begin{array}{c}
\judgementselfrewrite{\propenv{},{\isprop{\Number{}}{\makelocal{d}}}}
  {\false{}}
  {\Boolean{}}
  {\filterset{\notprop{\Number{}}{\makelocal{d}}}
             {\isprop{\Number{}}{\makelocal{d}}}}
  {\emptyobject{}}
  \\
\judgementselfrewrite{\propenv{},{\notprop{\Number{}}{\makelocal{d}}}}
  {\true{}}
  {\Boolean{}}
  {\filterset{\notprop{\Number{}}{\makelocal{d}}}
             {\isprop{\Number{}}{\makelocal{d}}}}
  {\emptyobject{}}
\end{array}
$$
%to suit the T-If outputs \t{} = \Boolean{}, \thenprop{\prop{}}
%= {\notprop{\Number{}}{\makelocal{d}}}, \elseprop{\prop{}} = {\isprop{\Number{}}{\makelocal{d}}},
%and \object{} = {\emptyobject{}}.
%
%In T-Subsume, we can upcast \t{} = \False{} to \tp{} = \Boolean{} via the premise 
%\issubtypein{}{\t{}}{\tp{}}.
%and 
%\inpropenv {\propenv{}, {\elseprop {\prop{}}}} {\elseprop {\propp{}}}
%from 
%{\elseprop {\prop{}}} = \topprop{} to {\elseprop {\propp{}}} = {\isprop{\Number{}}{\makelocal{d}}}.

Finally T-Abs assigns a type to the overall function:
$$
{\judgementselfrewrite{}{\abs{\makelocal{d}}{\Top}{\ifexp{\appexp{\numberhuh{}}{\makelocal{d}}}{\false{}}{\true{}}}}
                    {\ArrowOne {\makelocal{d}} {\Top{}}
                                      {\Boolean{}}
                                      {\filterset {\notprop{\Number{}}{\makelocal{d}}}
                                                  {\isprop{\Number{}}{\makelocal{d}}}}
                                      {\emptyobject{}}}
                    {\filterset {\topprop{}}
                                {\botprop{}}}
                    {\emptyobject{}}}
$$

%The object \x{} over latent propositions \thenprop{\prop{}} and
%\elseprop{\prop{}}, latent object \object{}, and latent type \t{}. The
%actual argument object---\objectp{}---is substituted in at application by T-App.



%The T-App rule instantiates parameters---like \x{} in T-Abs---with their actual object.
%The expression {\appexp{\numberhuh{}}{\makelocal{d}}} replaces its parameter object with \
%\judgementselfrewrite{\isprop{\Top}{\makelocal{d}}}{\appexp{\numberhuh{}}{\makelocal{d}}}
%  {\Boolean{}}
%  {\filterset{\isprop{\Number{}}{\makelocal{d}}}
%             {\notprop{\Number{}}{\makelocal{d}}}}
%  {\emptyobject{}}

%The T-If refines each branch based on the test---a
%{\appexp{\numberhuh{}}{\makelocal{d}}} test introduces \isprop{\Number{}}{\makelocal{d}}
%for checking the `then' branch and \notprop{\Number{}}{\makelocal{d}}
%for checking the `else' branch.
%%Information from each branch is combined via subsumption for the overall type, propositions, and object.
%The T-Subsume rule 

%For example
%$
%\judgementselfrewrite{\isprop{\Top}{\makelocal{d}}}
%  {\ifexp{\appexp{\numberhuh{}}{\makelocal{d}}}{\false{}}{\true{}}}
%  {\Boolean{}}
%  {\filterset{\notprop{\Number{}}{\makelocal{d}}}
%             {\isprop{\Number{}}{\makelocal{d}}}}
%  {\emptyobject{}}
%$
%type checks because T-Subsume allows us to check both branches as
%$
%\judgementselfrewrite{\isprop{\Number{}}{\makelocal{d}}}
%  {\false{}}
%  {\Boolean{}}
%  {\filterset{\notprop{\Number{}}{\makelocal{d}}}
%             {\isprop{\Number{}}{\makelocal{d}}}}
%  {\emptyobject{}}
%$
%and
%$
%\judgementselfrewrite{\notprop{\Number{}}{\makelocal{d}}}
%  {\true{}}
%  {\Boolean{}}
%  {\filterset{\notprop{\Number{}}{\makelocal{d}}}
%             {\isprop{\Number{}}{\makelocal{d}}}}
%  {\emptyobject{}}
%$
%respectively.
%
%For example, if \e{2} is true when variable {\makelocal{y}} is a \File{}
%and \e{3} is true when variable {\makelocal{z}} is a \Number{}, then
%we use T-Subsume on both branches
%to introduce a logical disjunction
%since at least one must be true if the entire expression is true.



%Given a set of propositions, we can use logical reasoning to derive
%new information about our programs
%with the judgment \inpropenv{\propenv{}}{\prop{}}.
%In addition to the standard rules for the logical connectives, the key
%rule is L-Update, which combines multiple propositions about the same variable,
%allowing us to refine its type.
%$$
%  {\LUpdate}
%$$
%For example, with L-Update we can use the knowledge of
%\inpropenv{\propenv{}}{\isprop{\UnionNilNum}{\x{}}}
%and 
%\inpropenv{\propenv{}}{\notprop{\Nil{}}{\x{}}}
%to derive \inpropenv{\propenv{}}{\isprop{\Number}{\x{}}}.
%(The metavariable \propisnotmeta{} ranges over \t{} and \nottype{\t{}} (without variables).)
%We cover L-Update in more detail in \secref{sec:formalpaths}.
%
%Finally, this approach allows the type system to track
%programming idioms from 
%dynamic languages
%using implicit type-based reasoning based on the result of
%conditional tests.
%For instance,
%\egref{example:parent-if}
%only utilizes \clj{f} once
%the programmer is convinced it is safe to do so based whether
%\clj{f}
%is
%true or false. 
%To express this in the type system, every expression 
%is described by two propositions: a `then' proposition
%for when it reduces to a true value, and an `else' proposition
%when it reduces to a false value---for \clj{f}
%the then proposition is {\notprop{\falsy}{f}} and 
%the else proposition is {\isprop{\falsy}{f}}.
%\ref{main:figure:typingrules}




%\figref{main:figure:typingrules} contains the core typing rules.
%The key rule for reasoning about conditional control flow is
%T-If. 
%
%\begin{mathpar}
%  {\TIf}
%\end{mathpar}

%The propositions of the test expression \e{1}, \thenprop{\prop{1}} and \elseprop{\prop{1}}, are 
%used as assumptions in the then and else branch respectively.

%The let rule T-Let links inferred information about
%\x{} to the expression used to instantiate \x{}, \ep{1}, via logical implications.
%
%The T-Local rule connects the type system to the proof system over type propositions
%via \inpropenv {\propenv{}} {\isprop {\t{}} {\x{}}}
%to derive a type for a variable.
%Using this rule, the type system can then appeal to L-Update to refine the type
%assigned to \x{}.
%
%We are now equipped to type check
%\egref{example:parent-if}:
%$$
%\clj{(if f (.getParent f) nil)}
%$$
%
%With {\propenv{}} = {\isprop{\UnionNilFile{}}{f}},
%$$
%\judgement{\propenv{}}{f}{\UnionNilFile{}}{\localfilterset{f}}{f}
%$$
%via T-Local.
%
%Checking the then branch involves extending
%the proposition environment with {\notprop{\falsy}{f}}
%$$
%\judgement{{\propenv{}},{\isprop{\Number}{\x{}}}}{\x{}}{\Number{}}{\filterset{\notprop{\falsy{}}{\x{}}}{\isprop{\falsy{}}{\x{}}}}{\emptyobject{}}
%$$
%because we can now satisfy the premise of T-Local:
%$$
%\inpropenv{{\propenv{}},\isprop{\Number}{\x{}}}{\isprop{\Number}{\x{}}}.
%$$
%\judgement{{\propenv{}},\isprop{\Number}{\x{}}}{\hastype{\appexp{\inc{}}{\x{}}}{\Number{}}}{\filterset{\topprop{}}{\botprop{}}}{\emptyobject{}}
%$$
%$$
%\judgement{{\propenv{}},\notprop{\Number}{\x{}}}{\hastype{\zeroliteral{}}{\Number}}{\filterset{\topprop{}}{\botprop{}}}{\emptyobject{}}
%$$

%\inc{} has type
%$$
%{\ArrowOne{\x{}}{\Number}{\Number}
%        {\filterset{\topprop{}}{\topprop{}}}{\emptyobject{}}}
%$$
%We can now check the conditional with T-If.
%$$
%\judgement{\isprop{\Number}{\x{}}}{\hastype{\ifexp{\appexp{\numberhuh{}}{\x{}}}{\appexp{\inc{}}{\x{}}}{\zeroliteral{}}}{\Number}}{\filterset{\orprop{\isprop{\Number}{\x{}}}{\topprop{}}}{\orprop{\notprop{\Number}{\x{}}}{\topprop{}}}}{\emptyobject{}}
%$$
%Finally the function can be checked with T-Abs
%$$
%\judgement{}{\hastype{\abs{\x{}}{\UnionNilNum}{\ ...}}
%                                             {\ArrowOne{\x{}}{\UnionNilNum}{\Number}
%        {\filterset{\orprop{\isprop{\Number}{\x{}}}{\topprop{}}}{\orprop{\notprop{\Number}{\x{}}}{\topprop{}}}}{\emptyobject{}}}}
%  {\filterset{\topprop{}}{\botprop{}}}{\emptyobject{}}
%$$

\paragraph{Subtyping}
\figref{main:figure:subtyping} presents subtyping
as a reflexive and transitive relation with top type \Top. 
Singleton types are instances of their respective classes---boolean singleton types
are of type \Boolean{}, class literals are instances of \Class{} and keywords are
instances of \Keyword{}.
Instances of classes \class{} are subtypes of \Object{}. Function types 
are subtypes of \IFn{}. All types except for \Nil{} are subtypes of \Object{},
so \Top{} is similar to {\Union{\Nil}{\Object}}.
Function subtyping is contravariant left of the arrow---latent propositions, object
and result type are covariant.
Subtyping for untagged unions is standard.

\paragraph{Operational semantics} We define the dynamic semantics for \lambdatc{}
in a big-step style using an environment, following~\cite{TF10}.
We include both errors and a \wrong{} value, which is provably ruled out by the
type system.
The main judgment is \opsem{\openv{}}{\e{}}{\definedreduction{}}
which states that \e{} evaluates to answer \definedreduction{} in environment
\openv{}. We chose to omit the core rules (included in supplemental material)
however a notable difference is \nil{} is a false value, which affects the
semantics of \ifliteral{} (\figref{main:figure:coresemantics}).

%The definition of \updateliteral{} supports various idioms relating to \classpe{}
%which we introduce in \secref{sec:isaformal}.

%\begin{figure*}
%  \footnotesize
%%%   \judgbox{
%%%{\judgementrewrite   {\propenv}
%%%              {\e{}} {\t{}}
%%%  {\filterset {\thenprop {\prop{}}}
%%%              {\elseprop {\prop{}}}}
%%%  {\object{}}{\ep{}}}}
%%%           {Under proposition environment $\propenv{}$, 
%%%             expression \e{} is of type \t{}
%%%             with  \\
%%%
%%%`then' proposition {\thenprop {\prop{}}}, `else' proposition {\elseprop {\prop{}}}
%%%and object \object{} and rewrites to \ep{}.}
%  \begin{mathpar}
%    %{\TDo}
%    %{\TClass}
%    %{\TIf}
%    %{\TAbs}
%    %\begin{array}{c}
%    %  {\TSubsume}\\\\
%    %  {\TNum}
%    %\end{array}
%    \begin{array}{c}
%      {\TNum}\\\\
%      {\TConst}\\\\
%      {\TKw}\\\\
%      {\TClass}\\\\
%      {\TTrue}\\\\
%    \end{array}
%    \begin{array}{c}
%      {\TSubsume}\\\\
%      {\TNil}\\\\
%      {\TFalse}
%    \end{array}
%
%    %{\TLet}
%    %{\TLocal}
%
%    %{\TApp}
%    %{\TError}
%
%  \end{mathpar}
%  \caption{Typing rules}
%  \label{main:figure:typingrules}
%\end{figure*}

%\begin{figure}
%  \footnotesize
%  \begin{mathpar}
%    {\BLocal}
%
%    %{\BDo}
%
%    {\BLet}
%
%    \BVal{}
%
    %\BIfTrue{}

%    \BIfFalse{}
%
%    \BAbs{}
%
%    \BBetaClosure{}
%
%    \BDelta{}
%  \end{mathpar}
%  \caption{Operational Semantics}
%  \label{main:figure:standardopsem}
%\end{figure}

%\subsection{Reasoning about Exceptional Control Flow}
%\label{sec:doformal}
%
%Along with conditional control flow,
%Clojure programmers rely on \emph{exceptions}
%to assert type-related invariants.
%
%\begin{exmp}
%\inputminted[firstline=13,lastline=15]{clojure}{code/demo/src/demo/do.clj}
%\label{example:doexception}
%\end{exmp}
%
%The fully expanded increment function in~\egref{example:doexception}
%guards its final call with a number check, preventing
%a possible null-pointer exception.
%Without this check, the type system would reject the program.
%
%To check this example,
%occurrence typing 
%automatically
%assumes
%\clj{x} is a number when checking the second \clj{do} subexpression
%based on the first subexpression.
%\footnote{See \url{https://github.com/typedclojure/examples}
%  for full examples.}
%We model this formally %(section~\ref{sec:doformal}) 
%and prove
%null-pointer exceptions are impossible in typed code (section~\ref{sec:metatheory}).
%
%
%We extend our model with sequencing expressions and errors, where {\errorvalv{}}
%models the result of calling Clojure's \clj{throw} special form
%with some \clj{Throwable}.
%
%\smallskip
%$
%\begin{altgrammar}
%  \e{} &::=& \ldots \alt {\errorvalv{}} \alt {\doexp {\e{}} {\e{}}} &\mbox{Expressions} 
%\end{altgrammar}
%$
%
%\smallskip
%%
%%B-Do simply evaluates its arguments sequentially and returns the right argument.
%%Since errors are not values, we define error propagation semantics
%%like BE-Do1 (figure~\ref{appendix:figure:errorstuck} for the full rules).
%%
%%\begin{mathpar}
%%    {\BDo}
%%
%%\infer [BE-Error]
%%{}
%%{ \opsem {\openv{}} 
%%         {\errorvalv{}}
%%         {\errorvalv{}}}
%%
%%\infer [BE-Do1]
%%{ \opsem {\openv{}} {\e{1}} {\errorvalv{}} }
%%{ \opsem {\openv{}} {\doexp{\e{1}}{\e{}}} {\errorvalv{}}}
%%\end{mathpar}
%
%Our main insight is as follows: 
%if the first subexpression in a sequence reduces to a value, then it is either true or false.
%If we learn some proposition in both cases then we can use that proposition as an assumption to check the second subexpression.
%T-Do formalizes this intuition.
%
%\begin{mathpar}
%    {\TDo}  
%\end{mathpar}
%
%The introduction of errors, 
%which do not evaluate to either
%a true or false value,
%makes our insight interesting.
%
%\begin{mathpar}
%    {\TError}
%\end{mathpar}
%
%Recall \egref{example:doexception}.
%\begin{minted}{clojure}
%...
%  (do (if (number? x) nil (throw (new Exception)))
%      (inc x)) 
%...
%\end{minted}
%
%As before, checking \appexp{\numberhuh{}}{\x{}} allows us to use the proposition \isprop{\Number}{\x{}}
%when checking the then branch.
%
%By T-Nil and subsumption we can propagate this  information to both propositions.
%$$
%\judgement{\isprop{\Number}{\x{}}}{\nil{}}{\Nil{}}{\filterset{\isprop{\Number}{\x{}}}{\isprop{\Number}{\x{}}}}{\emptyobject{}}
%$$
%Furthermore, using T-Error
%and subsumption we can conclude anything in the else branch.
%$$
%\judgement{\notprop{\Number}{\x{}}}{\errorvalv{}}{\Bot}{\filterset{\isprop{\Number}{\x{}}}{\isprop{\Number}{\x{}}}}{\emptyobject{}}
%$$
%Using the above as premises to T-If we conclude that if the first
%expression in the \doliteral{} evaluates successfully, \isprop{\Number}{\x{}} must be true.
%$$
%\judgement{\isprop{\UnionNilNum}{\x{}}}
%          {\ifexp{\appexp{\numberhuh{}}{\x{}}}{\nil{}}{\errorvalv{}}}{\Boolean}
%          {\filterset{\isprop{\Number}{\x{}}}{\isprop{\Number}{\x{}}}}{\emptyobject{}}
%$$
%We can now use \isprop{\Number}{\x{}} in the environment to check the second subexpression
%{\appexp{\inc{}}{\x{}}}, completing the example.

\subsection{Java Interoperability}

\begin{figure}
  \footnotesize
  $$
  \begin{altgrammar}
    \e{} &::=& \ldots   {\fieldexp {\fld{}} {\e{}}} \alt {\methodexp {\mth{}} {\e{}} {\overrightarrow{\e{}}}}
                      \alt {\newexp {\class{}} {\overrightarrow{\e{}}}}
                      &\mbox{Expressions}\\
     &\alt& \nonreflectiveexpsyntax{} &\mbox{Non-reflective Expressions}\\

    \v{} &::=& \ldots \alt {\classvalue{\classhint{}} {\overrightarrow {\classfieldpair{\fld{}} {\v{}}}}}
    &\mbox{Values} \\

    \classtableallsyntax{}
  \end{altgrammar}
  $$
  \begin{mathpar}
    {\TNew}

    {\TMethod}

    {\TField}

    %{\TInstance}
  \end{mathpar}
 %\classtablelookupsyntax{}
 \begin{mathpar}
  \begin{altgrammar}
    \convertjavatypenil{}
  \end{altgrammar}
  \begin{altgrammar}
    \convertjavatypenonnil{}
  \end{altgrammar}
  \begin{altgrammar}
    \converttctype{}
  \end{altgrammar}
\end{mathpar}
  \begin{mathpar}
    \BField{}\ \ \ 
%
    \BNew{}

    \BMethod{}
  \end{mathpar}
  \caption{Java Interoperability Syntax, Typing and Operational Semantics}
  \label{main:figure:javatyping}
\end{figure}

\begin{figure}
  $$
\constanttypefigure{}
  $$
  \caption{Constant typing}
  \label{main:figure:constanttyping}
\end{figure}

We present Java interoperability in a restricted setting without class inheritance,
overloading or Java Generics.
%
We extend the syntax in \figref{main:figure:javatyping} with Java field lookups and calls to
methods and constructors. 
To prevent ambiguity between zero-argument methods and fields,
we use Clojure's primitive ``dot'' syntax:
field accesses are written \fieldexp{\fld{}}{\e{}}
and method calls $\methodexp{\mth{}}{\e{}}{\overrightarrow{e}}$.
%and \clj{(new class es*)} is $\newexp{\class{}}{\overrightarrow{es}}$.

In \egref{example:getparent-direct-constructor}, \clj{(*interop .getParent (*interop new File "a/b" interop*) interop*)}
translates to
\begin{equation}  \label{eq:unresolved}
  \qquad {\methodexp {\getparent{}} {\newexp {\File{}} {\makestr{a/b}}} {}}
\end{equation}

But both the constructor and method are unresolved.
We introduce \emph{non-reflective} expressions for specifying exact Java overloads.
\begin{equation} \label{eq:resolved}
\qquad {\methodstaticexp {\File} {} {\String} {\getparent{}} {\newstaticexp {\String} {\File{}} {\File{}} {\makestr{a/b}}} {}}
\end{equation}
From the left, the one-argument constructor for \File takes a \String, and the 
\getparent{} method of
\File{} takes zero arguments
and
returns a \String.

We now walk through this conversion.% from unresolved expression~\ref{eq:unresolved} to 
%resolved expression~\ref{eq:resolved}.

\paragraph{Constructors} First we check and convert {\newexp {\File{}} {\makestr{a/b}}} to {\newstaticexp {\String} {\File{}} {\File{}} {\makestr{a/b}}}.
The T-New typing rule checks and rewrites constructors.
%$$
%    {\TNew}
%$$
To check
{\newexp {\File{}} {\makestr{a/b}}}
we first resolve the constructor overload in the class table---there is at most one
to simplify presentation.
With \classhint{1} = \String,
we convert to a nilable type the argument with \t{1} = \Union{\Nil}{\String}
and type check {\makestr{a/b}} against \t{1}.
Typed Clojure defaults to allowing non-nilable arguments, but this
can be overridden, so we model the more general case.
% which erases nil
The return Java type \File is converted to a non-nil
Typed Clojure type \t{} = \File for the return type,
and the propositions say constructors can never be false---constructors
can never produce the internal boolean value that Clojure uses for \false{}, or \nil{}.
Finally, the constructor rewrites to {\newstaticexp {\String} {\File{}} {\File{}} {\makestr{a/b}}}.

\paragraph{Methods} Next we convert {\methodexp {\getparent{}} {\newstaticexp {\String} {\File{}} {\File{}} {\makestr{a/b}}} {}}
to the non-reflective expression
{\methodstaticexp {\File} {} {\String} {\getparent{}} {\newstaticexp {\String} {\File{}} {\File{}} {\makestr{a/b}}} {}}.
%The T-Method rule checks unresolved methods.
%$$
%    {\TMethod}
%$$
The T-Method rule for unresolved methods
checks {\methodexp {\getparent{}} {\newstaticexp {\String} {\File{}} {\File{}} {\makestr{a/b}}} {}}.
We verify the target type \s{} = \File is non-nil by T-New.
The overload is chosen from the class table based on \classhint{1} = \File---there is at most one.
The nilable return type \t{} = \Union{\Nil}{\String} is given, and the 
entire expression rewrites to expression \ref{eq:resolved}.
%
%We allow arguments to constructors and methods to be nilable, but not method
%and field targets.

The T-Field rule (\figref{main:figure:javatyping}) is like T-Method, but without arguments.

The evaluation rules B-Field, B-New and B-Method (\figref{main:figure:javatyping}) simply evaluate their
arguments and call the relevant JVM operation, which we do not model---\secref{sec:metatheory}
states our exact assumptions.
There are no evaluation rules for reflective Java interoperability, since there are no typing
rules that rewrite to reflective calls.


%\subsection{Paths}
%\label{sec:formalpaths}
%
%Recall the first insight of occurrence typing---we can reason
%about specific \emph{parts} of the runtime environment
%using propositions.
%We refer to parts of the runtime environment via 
%a \emph{path} that consists of a series of
%\emph{path elements} applied right-to-left to a variable
%written \path{\pathelem{}}{\x{}}.
%\cite{TF10} introduce the path elements \carpe{} and \cdrpe{}
%to reason about selector operations on cons cells.
%We instead want to reason about HMap lookups and calls to \classconst{}.
%
%\paragraph{Key path element} We introduce our first path element
%{\keype{\k{}}}, which represents the operation of looking up
%a key \k{}.
%We directly relate this to our typing rule T-GetHMap
%(\figref{main:figure:hmapsyntax}) by
%checking the then branch of the first conditional test is checked in 
%an equivalent version of \egref{example:decleaf}.
%\begin{minted}{clojure}
%  (fn [m :- Expr]
%    (if (= (get m :op) :if)
%      {:op :if, ...}
%      (if ...)))
%\end{minted}
%
%We do not specifically support \equivliteral{} in our calculus, 
%but on keyword arguments it works identically to \clj{isa?} which we model
%in \secref{sec:isaformal}.
%Intuitively, if {\judgement{\propenv{}}{\e{}}{\t{}}{\filterset{\thenprop{\prop{}}}{\elseprop{\prop{}}}}{\object{}}}
%then \equivapp{\e{}}{\makekw{if}} has the true and false propositions
%$$
%{\replacefor{\filterset{\isprop{\Value{\makekw{if}}}{\x{}}}{\notprop{\Value{\makekw{if}}}{\x{}}}}{\object{}}{\x{}}}
%$$
%where substitution reduces to \topprop{} if \object{} = \emptyobject{}.
%
%We start with proposition environment \propenv{} = {\isprop{\Expr{}}{m}}.
%Since {\Expr{}} is a union of HMaps, each with the entry \makekw{op}, we can use T-GetHMap.
%$$
%\judgement{\propenv{}}{\getexp{m}{\makekw{op}}}{\Keyword}{\filterset{\topprop{}}{\topprop{}}}{\path{\keype{\makekw{op}}}{m}}
%$$
%Using our intuitive definition of \equivliteral{} above, we know
%$$
%\judgement{\propenv{}}{\equivapp{\getexp{m}{\makekw{op}}}{\makekw{if}}}{\Boolean}{\filterset{\isprop{\Value{\makekw{if}}}{\path{\keype{\makekw{op}}}{m}}}{\notprop{\Value{\makekw{if}}}{\path{\keype{\makekw{op}}}{m}}}}{\emptyobject{}}
%$$
%Going down the then branch gives us the extended environment
%\propenvp{} = {\isprop{\Expr{}}{m}},{\isprop{\Value{\makekw{if}}}{\path{\keype{\makekw{op}}}{m}}}.
%Using L-Update we can combine what we know about object $m$ and object
%{\path{\keype{\makekw{op}}}{m}}
%to derive
%$$
%\inpropenv{\propenvp{}}{\isprop{\HMapp{\mandatoryset{{\mandatoryentrykwnoarrow{op}{\makekw{if}}}, {\mandatoryentrykwnoarrow{test}{\Expr{}}},
%                                       {\mandatoryentrykwnoarrow{then}{\Expr{}}},   {\mandatoryentrykwnoarrow{else}{\Expr{}}}}}
%                                   {\emptyabsent{}}}{m}}
%$$
%
%The full definition of \updateliteral{} is given in \figref{main:figure:update}
%which considers both keys a path elements as well as the \classconst{}
%path element described below.
%In the absence of paths, update simply performs set-theoretic operations
%on types; see \figref{main:figure:restrictremove} for details.
%
%\paragraph{Class path element} Our second path element \classpe{} is used in the latent
%object of the constant \classconst{} function. Like Clojure's \clj{class}
%function \classconst{} returns the argument's class or \nil{}
%if passed \nil{}.
%$$
%\begin{array}{lrlr}
%  \pesyntax{}   &::=& \ldots \alt {\classpe{}}
%                &\mbox{Path Elements}
%\end{array}
%$$
%\begin{mathpar}
%\constanttypefigure{}
%\end{mathpar}
%The dynamic semantics are given in \figref{main:figure:primitivesem}.
%The definition of \updateliteral{} supports various idioms relating to \classpe{}
%which we introduce in \secref{sec:isaformal}.

\subsection{Multimethod preliminaries: \isaliteral}

\label{sec:isaformal}

We now consider the \isaliteral{} operation, a core part of the multimethod dispatch mechanism. 
Recalling the examples in \secref{sec:multioverview},
\isaliteral{} is
a subclassing test for classes, but otherwise is an equality test.
%---we do not model the semantics for vectors
%
The T-IsA rule uses \isacompareliteral{}
(\figref{main:figure:mmsyntax}), a metafunction which produces the propositions for
\isaliteral{} expressions.
%\begin{mathpar}
%  \TIsA{}
%\end{mathpar}

To demonstrate the first \isacompareliteral{} case,
the expression
\isaapp{\appexp{\classconst{}}{\x{}}}{\Keyword}
is true if \x{} is a keyword, otherwise false.
When checked with T-IsA,
the object of the left subexpression \object{} = {\path{\classpe{}}{\x{}}}
(which starts with the {\classpe{}} path element)
and the type of the right subexpression \t{} = {\Value{\Keyword}} (a singleton class type)
together trigger the first \isacompareliteral{} case
\isacompare{\s{}}{\path{\classpe{}}{\x{}}}{\Value{\Keyword}}{\filterset{\isprop{\Keyword}{\x{}}}{\notprop{\Keyword}{\x{}}}},
giving propositions that correspond to our informal description {\filterset{\thenprop{\prop{}}}{\elseprop{\prop{}}}} = {\filterset{\isprop{\Keyword}{\x{}}}{\notprop{\Keyword}{\x{}}}}.

The second \isacompareliteral{} case captures the simple equality mode for non-class singleton types.
For example,
the expression
\isaapp{\x{}}{\makekw{en}} produces true 
when \x{} evaluates to {\makekw{en}}, otherwise it produces false.
Using T-IsA,
it has the propositions {\filterset{\thenprop{\prop{}}}{\elseprop{\prop{}}}} = 
\isacompare{}{\x{}}{\Value{\makekw{en}}}{\filterset {\isprop {\Value{\makekw{en}}}{\x{}}}{\notprop{\Value{\makekw{en}}}{\x{}}}}
since \object{} = {\x{}} and \t{} = {\Value{\makekw{en}}}.
%
The side condition on the second \isacompareliteral{} case ensures we are in equality mode---if \x{} can possibly be a class in 
\isaapp{\x{}}{\Object{}}, \isacompareliteral{} uses its conservative default case,
since if \x{} is a class literal, subclassing mode could be triggered.
%
Capture-avoiding substitution of objects {\replacefor {} {\object{}} {\x{}}} used in this case erases propositions
that would otherwise have \emptyobject{} substituted in for their objects---it
is defined in the appendix.

The operational behavior of \isaliteral{} is given by B-IsA (\figref{main:figure:mmsyntax}). \isaopsemliteral{} explicitly handles classes in the second case.

%The definition of \isacompareliteral{} (figure~\ref{main:figure:mmsyntax}) is deliberately conservative.
%The first line considers the case where the object of the left argument
%is a non-empty path ending in \classpe{} and the type of the right argument is a singleton class.

%\constantsemfigure{main}

\subsection{Multimethods}

\begin{figure}
  \footnotesize
$$
\begin{altgrammar}
  \e{} &::=& \ldots \alt {\createmultiexp {\t{}} {\e{}}} \alt
             {\extendmultiexp {\e{}} {\e{}} {\e{}}}
             \alt {\isaapp {\e{}} {\e{}}} &\mbox{Expressions} \\
  \v{} &::=& \ldots \alt {\multi {\v{}} {\disptable{}}}
                &\mbox{Values} \\
 \disptablesyntax{} \\
  \s{}, \t{} &::=& \ldots \alt {\MultiFntype{\t{}}{\t{}}}
                &\mbox{Types}
\end{altgrammar}
$$
  \begin{mathpar}
    \TDefMulti{}

    \TDefMethod{}

    \TIsA{}
  \end{mathpar}
  \begin{mathpar}
    \isapropsfigure{}
  \end{mathpar}
  \begin{mathpar}
    \Multisubtyping{}

    \BDefMulti{}
  \end{mathpar}
  \begin{mathpar}
    \BDefMethod{}
    %\BBetaMulti{}
  \end{mathpar}
  \getmethodfigure{}
\begin{mathpar}
  {\BIsA{}}
  {\isaopsemfigure{}}
  \\
\BBetaMulti{}
\end{mathpar}
\caption{Multimethod Syntax, Typing and Operational Semantics}
\label{main:figure:mmsyntax}
\end{figure}


\figref{main:figure:mmsyntax} presents \emph{immutable} multimethods without default methods to ease presentation.
%Syntax and semantics are given in \figref{main:figure:mmsyntax}. 
%Multimethods can error if no matching method is chosen (rules in the supplemental material).
\figref{main:figure:mmexample} translates the mutable \egref{example:hi-multimethod} to \lambdatc{}.
%\begin{minted}{clojure}
%(ann hi [Kw -> Str])
%(defmulti hi identity)
%(defmethod hi :en [_] "hello")
%(defmethod hi :fr [_] "bonjour")
%(hi :en) ;=> "hello"
%\end{minted}

\begin{figure}
$\letexp{hi_0} {\createmultiexp {\ArrowOne{\x{}}{\Keyword}{\String}{\filterset{\topprop{}}{\topprop{}}}{\emptyobject{}}} {\abs{\x{}}{\Keyword}{\x{}}}}
  {\\\text{\quad}
    \letexp{hi_1} {\extendmultiexp {hi_0} {\makekw{en}} {\abs {\x{}} {\Keyword} {\makestr{hello}}}}
      {\\\text{\quad\quad}
        \letexp{hi_2} {\extendmultiexp {hi_1} {\makekw{fr}} {\abs {\x{}} {\Keyword} {\makestr{bonjour}}}}
        {\\\text{\quad\quad\quad
          \appexp{hi_2}{\makekw{en}}}}}}
$
\caption{Multimethod example}
\label{main:figure:mmexample}
\end{figure}
%
%For convenience, examples in this section are flattened when they are really nested
%let bindings. We also elide trivial latent propositions and objects.
%The following is an abbreviation of the previous expression.
%\begin{lstlisting}
%${\createmultiexp {\ArrowTwo{\x{}}{\Keyword}{\String}} {\abs{\x{}}{\Keyword}{\x{}}}}$
%${\extendmultiexp {hi} {\makekw{en}} {\abs {\x{}} {\Keyword} {\makestr{hello}}}}$
%${\extendmultiexp {hi} {\makekw{fr}} {\abs {\x{}} {\Keyword} {\makestr{bonjour}}}}$
%$\appexp{hi}{\makekw{en}}$
%\end{lstlisting}
%
%\defmethodliteral{} returns a new extended multimethod
%without changing the original multimethod. 
%
%\begin{minted}{clojure}
%(let [hi (defmulti [Kw -> Str] identity)]
%  (let [hi (defmethod hi :en [_] "hello")]
%    (let [hi (defmethod hi :fr [_] "bonjour")]
%      (hi :en))) ;=> "hello"
%\end{minted}
%
%\paragraph{How to check}
To check 
{\createmultiexp {\ArrowTwo{\x{}}{\Keyword}{\String}} {\abs{\x{}}{\Keyword}{\x{}}}},
%
we note
{\createmultiexp {\s{}} {\e{}}} creates a multimethod with \emph{interface type} \s{}, and dispatch function \e{}
of type \sp{},
producing a value of type
{\MultiFntype {\s{}} {\sp{}}}. % with interface type {\s{}} and dispatch function type {\sp{}}.
The T-DefMulti typing rule checks the dispatch function, and
verifies both the interface and dispatch type's domain agree.
%$$
%    \TDefMulti{}
%$$
Our example checks with \t{} = \Keyword, interface type \s{} = {\ArrowTwo{\x{}}{\Keyword}{\String}},
dispatch function type \sp{} = {\ArrowOne{\x{}}{\Keyword}{\Keyword}{\filterset{\topprop{}}{\topprop{}}}{\x{}}}, and overall type
$
{\MultiFntype {\ArrowTwo{\x{}}{\Keyword}{\String}}
              {\ArrowOne{\x{}}{\Keyword}{\Keyword}{\filterset{\topprop{}}{\topprop{}}}{\x{}}}}
$.

Next, we show how to check
$
{\extendmultiexp {hi_0} {\makekw{en}} {\abs {\x{}} {\Keyword} {\makestr{hello}}}}
$.
%
The expression 
{\extendmultiexp {\e{m}} {\e{v}} {\e{f}}} creates a new multimethod
that extends multimethod \e{m}'s dispatch table, mapping dispatch value
\e{v} to method \e{f}. The T-DefMulti typing rule
checks \e{m} is a multimethod with dispatch function type \t{d},
then calculates the extra information we know based on the current
dispatch value {\thenprop{\proppp{}}}, which is assumed when checking the method
body.
%$$
%    \TDefMethod{}
%$$
Our example checks with \e{m} being of type
$
{\MultiFntype {\ArrowTwo{\x{}}{\Keyword}{\String}}
              {\ArrowOne{\x{}}{\Keyword}{\Keyword}{\filterset{\topprop{}}{\topprop{}}}{\x{}}}}
$
with \objectp{} = {\x{}} (from below the arrow on the right argument of the previous type) and \t{v} = \Value{\makekw{en}}. 
Then {\thenprop{\proppp{}}} = 
{\isprop {\Value{\makekw{en}}}{\x{}}}
from
$
\isacompare{}{\x{}}{\Value{\makekw{en}}}
{\filterset {\isprop {\Value{\makekw{en}}}{\x{}}}{\notprop{\Value{\makekw{en}}}{\x{}}}}
$
(see \secref{sec:isaformal}).
Since \t{} = \Keyword{}, we check the method body with
$
\judgement{{\isprop{\Keyword}{\x{}}},{\isprop {\Value{\makekw{en}}}{\x{}}}}
  {\makestr{hello}}
  {\String}{\filterset{\topprop{}}{\topprop{}}}{\emptyobject{}}
$.
Finally from the interface type \t{m}, we know \thenprop{\prop{}} = \elseprop{\prop{}} = \topprop{},
and \object{} = \emptyobject{}, which also agrees with the method body, above.
Notice the overall type of a \defmethodliteral{} is the same as its first subexpression \e{m}.

It is worth noting the lack of special typing rules for overlapping methods---each
method is checked independently based on local type information.

%The expression {\createmultiexp {\s{}} {\e{}}} 
%defines a multimethod
%with interface type \s{} and dispatch function \e{}.
%The expression {\extendmultiexp {\e{m}} {\e{v}}{\e{f}}}
%extends multimethod \e{m} and to map
%dispatch value {\e{v}} to {\e{f}} in an extended dispatch table.
%The value {\multi {\v{}} {\disptable{}}} is the runtime value of a multimethod
%with dispatch function {\v{}} and dispatch table {\disptable{}}.
%
%The T-DefMulti rule ensures that the type of the dispatch function
%has at least as permissive a parameter type
%as the interface type.
%
%For example, we can check the definition from our translation above of \egref{example:rep}
%using T-DefMulti.
%$$
%\judgement{}%{\propenv{}}
%{\createmultiexp 
%      {\s{}}
%      {\classconst{}}}
%  {\MultiFntype {\s{}}{\sp{}}}{\filterset{\topprop{}}{\botprop{}}}{\emptyobject{}}
%$$
%where \s{}  = {\ArrowOne {\x{}} {\Top{}} {\t{}} {\filterset {\topprop{}} {\topprop{}}} {\emptyobject{}}}
%  and \sp{} = {\ArrowOne {\x{}} {\Top{}} {\Union{\Nil}{\Class}} {\filterset {\topprop{}} {\topprop{}}} {\path{\classpe{}}{\x{}}}}.
%  Since the parameter types agree, this is well-typed.
%
%The T-DefMethod rule requires a syntactic lambda expression as the method.
%This way we can manually check the body of the lambda under an extended
%environment as sketched in \egref{example:incmap}.
%We use \isacompareliteral{} to compute the proposition for this method,
%since \isaliteral{} is used at runtime in multimethod dispatch.
%
%We continue with the next line of the translation of \egref{example:rep}.
%From the previous line we have \propenv{} = {\isprop{\MultiFntype {\s{}}{\sp{}}}{path}},
%so
%$$
%\judgement{\propenv{}}
%  {\extendmultiexp {prop} {\String}
%                   {\abs {\x{}} {\Top{}} {\x{}}}}
%  {\MultiFntype {\s{}}{\sp{}}}{\filterset{\topprop{}}{\botprop{}}}{\emptyobject{}}
%$$
%We know \emph{prop} is a multimethod by \propenv{}, so now we check the body
%of this method.
%$$
%\judgement{\propenv{},{\isprop{\Top}{\x{}}},{\isprop{\String}{\x{}}}}
%  {\x{}}
%  {\String}{\filterset{\topprop{}}{\botprop{}}}{\emptyobject{}}
%$$
%%This is checked by T-Local since {\inpropenv{\propenv{},{\isprop{\Top}{\x{}}},{\isprop{\String}{\x{}}}}{\isprop{\String}{\x{}}}}.
%The new proposition {\isprop{\String}{\x{}}} is derived by 
%$$
%  \isacompare{\Top{}}{\path{\classpe{}}{\x{}}}{\Value{\File{}}}
%             {\filterset{\isprop{\String}{\x{}}}
%                        {\notprop{\String}{\x{}}}}.
%$$
%%
%The body of the \clj{let} is checked by T-App because
%{\MultiFntype {\s{}}{\sp{}}} is a subtype of its interface type {\s{}}.

\paragraph{Subtyping}
Multimethods are functions, via S-PMultiFn,
%$$
%\SPMultiFn{}
%$$
which says a multimethod can be upcast to its interface type. 
Multimethod call sites are then handled by T-App via T-Subsume. Other rules are given
in \figref{main:figure:mmsyntax}. 

\paragraph{Semantics}
Multimethod definition semantics are also given 
in \figref{main:figure:mmsyntax}. 
B-DefMulti creates a multimethod with the given dispatch function and an empty dispatch table.
B-DefMethod produces a new multimethod with an extended dispatch table.

The overall dispatch mechanism is summarised by B-BetaMulti.
First the dispatch function \v{d} is applied to the argument \vp{} to obtain
the dispatch value \v{e}.
Based on \v{e},
the \getmethodliteral{} metafunction (\figref{main:figure:mmsyntax})
extracts a method \v{f} from the method table {\disptable{}}
and applies it to the original argument for the final result.

\subsection{Precise Types for Heterogeneous maps}
\label{sec:hmapformal}

\begin{figure}
  \footnotesize
  $$
  \begin{altgrammar}
    \e{} &::=& \ldots \alt \hmapexpressionsyntax{}
    &\mbox{Expressions} \\
    \v{} &::=& \ldots \alt {\emptymap{}}
    &\mbox{Values} \\
    \t{} &::=& \ldots \alt {\HMapgeneric {\mandatory{}} {\absent{}}}
    &\mbox{Types} \\
    \auxhmapsyntax{}\\
%    \pesyntax{}   &::=& \ldots \alt {\keype{\k{}}}
%                  &\mbox{Path Elements}
  \end{altgrammar}
  $$
  \begin{mathpar}
    {\TAssoc}

    {\TGetHMap}

    {\TGetAbsent}

    {\TGetHMapPartialDefault}
    \ \ \ 
  {\SHMapMono}
  \end{mathpar}
  \begin{mathpar}
  {\SHMapP}\ \ 
  {\SHMap}

  \end{mathpar}
  \begin{mathpar}
    {\BAssoc}\ \ 
    {\BGet}\ \ 
    {\BGetMissing}
  \end{mathpar}
  \caption{HMap Syntax, Typing and Operational Semantics}
  \label{main:figure:hmapsyntax}
\end{figure}


\begin{figure}
  \footnotesize
$$
\begin{array}{lr}
  \begin{array}{llll}
    \restrictfigure{}
  \end{array}
  \ \ 
  \begin{array}{llll}
    \removefigure{}
  \end{array}
\end{array}
$$
\caption{Restrict and remove}
\label{main:figure:restrictremove}
\end{figure}

\figref{main:figure:hmapsyntax}
presents
heterogeneous map types.
The type \HMapgeneric{\mandatory{}}{\absent{}}
contains {\mandatory{}}, a map of \emph{present} entries (mapping keywords to types),
\absent{}, a set of keyword keys that are known to be \emph{absent}
and
tag \completenessmeta{} which is either {\complete{}} (``complete'') if the map is fully specified by \mandatory{},
and {\partial{}} (``partial'') if there are \emph{unknown} entries.
%
The partially specified map of
\clj{lunch} in \egref{example:lunchpartial}
is written
\HMapp{\mandatoryset{\mandatoryentrynoarrow{\Valkw{en}}{\String}, {\mandatoryentrynoarrow{\Valkw{fr}}{\String}}}}{\emptyabsent{}}
(abbreviated \Lunch).
%
The type of the fully specified map
\clj{breakfast} in \egref{example:breakfastcomplete} elides the absent entries,
written
\HMapc{\mandatoryset{\mandatoryentrynoarrow{\Valkw{en}}{\String}, {\mandatoryentrynoarrow{\Valkw{fr}}{\String}}}}
(abbreviated \Breakfast).
To ease presentation, 
if an HMap has completeness tag \complete{} then \absent{} is elided and implicitly contains all keywords not in the domain of 
\mandatory{}---dissociating keys is not modelled, so the set of absent entries otherwise
never grows.
Keys cannot be both present and absent.
%\HMapcwithabsent{\mandatory{}}{\absent{}} is abbreviated to \HMapc{\mandatory{}}. 

The metavariable \mapval{}
ranges over the runtime value of maps {\curlymapvaloverright{\k{}}{\v{}}},
usually written {\curlymapvaloverrightnoarrow{\k{}}{\v{}}}.
We %do not model keywords as functions,
only provide syntax for the empty map literal,
however when convenient we abbreviate non-empty map literals
to be a series of \assocliteral{} operations on the empty map.
We restrict lookup and extension to keyword keys. 

\paragraph{How to check}
A mandatory lookup is checked by T-GetHMap.
$$
\abs{\makelocal{b}}{\Breakfast}{\getexp{\makelocal{b}}{\makekw{en}}}
$$
The result type is \String, and the return object is \path{\keype{\makekw{en}}}{\makelocal{b}}.
The object {\replacefor {\path {\keype{k}} {\x{}}} {\object{}} {\x{}}}
is a symbolic representation for a keyword lookup of $k$ in \object{}.
The substitution for {\x{}} handles the case where \object{} is empty.
\begin{mathpar}
\begin{array}{rcl}
{\replacefor {\path {\keype{k}} {\x{}}} {\y{}} {\x{}}} &=& {\path {\keype{k}} {\y{}}} \\
\end{array}
\ \ \ \ \ \ \ 
\begin{array}{rcl}
{\replacefor {\path {\keype{k}} {\x{}}} {\emptyobject{}} {\x{}}} &=& \emptyobject{}
\end{array}
\end{mathpar}

An absent lookup is checked by T-GetHMapAbsent.
$$
\abs{\makelocal{b}}{\Breakfast}{\getexp{\makelocal{b}}{\makekw{bocce}}}
$$
The result type is \Nil---since \Breakfast is fully specified---with return object \path{\keype{\makekw{bocce}}}{\makelocal{b}}.

A lookup that is not present or absent is checked by
T-GetHMapPartialDefault.
$$
\abs{\makelocal{u}}{\Lunch}{\getexp{\makelocal{u}}{\makekw{bocce}}}
$$
The result type is \Top---since {\Lunch} has an unknown \makekw{bocce} entry---with return object \path{\keype{\makekw{bocce}}}{\makelocal{u}}.
Notice propositions are erased once they enter a HMap type.

For presentational reasons, lookups on unions of HMaps are only supported in T-GetHMap
and each element of the union must contain the relevant key.
$$
\abs{\makelocal{u}}{\Unionsplice{\Breakfast \Lunch}}{\getexp{\makelocal{u}}{\makekw{en}}}
$$
The result type is \String, and the return object is \path{\keype{\makekw{en}}}{\makelocal{u}}.
However, lookups of \makekw{bocce} on {\Unionsplice{\Breakfast \Lunch}} maps are unsupported.
This restriction still allows us to check many of the examples in \secref{sec:overview}---in
particular we can check 
\egref{example:desserts-on-meal}, as \makekw{Meal} is in common with both HMaps,
but cannot check \egref{example:desserts-on-class}
because a \makekw{combo} meal lacks a \makekw{desserts} entry.
Adding a rule to handle \egref{example:desserts-on-class} is otherwise straightforward.

Extending a map with T-AssocHMap preserves its completeness.
$$
\abs{\makelocal{b}}{\Breakfast}{\assocexp{\makelocal{b}}{\makekw{au}}{\makestr{beans}}}
$$
The result type is
$
\HMapc{\mandatoryset{\mandatoryentrynoarrow{\Valkw{en}}{\String}, {\mandatoryentrynoarrow{\Valkw{fr}}{\String}}
        ,{\mandatoryentrynoarrow{\Valkw{au}}{\String}}}}
$,
a complete map.
T-AssocHMap also enforces ${\k{}} \not\in {\absent{}}$ to prevent badly formed types.

%for cases like \egref{example:desserts-on-meal}
%where every element in the union
%contains the key we are looking up.

\paragraph{Subtyping}
Subtyping for HMaps
designate \MapLiteral{} as a common supertype for all HMaps.
S-HMap says that HMaps are subtypes if they agree
on \completenessmeta{}, agree on mandatory entries with subtyping
and at least cover the absent keys of the supertype.
Complete maps are subtypes of partial maps
as long as they agree on the mandatory entries of the partial map via subtyping (S-HMapP).

%The typing rules for \getliteral{} consider three possible cases. T-GetHMap models a lookup
%that will certainly succeed, T-GetHMapAbsent a lookup that will certainly fail
%and T-GetHMapPartialDefault a lookup with unknown results.

%The objects in the T-Get rules are more complicated than those in T-Local---the 
%next section discusses this in detail.
%Finally T-AssocHMap extends an HMap with a mandatory entry while preserving completeness
%and absent entries, and enforcing ${\k{}} \not\in {\absent{}}$ to prevent badly
%formed types.

The semantics for \getliteral{} and \assocliteral{} are straightforward.
%If the entry is missing, B-GetMissing produces \nil{}.

\begin{figure}[t]
  $$
\begin{array}{llll}
\updatefigure{}
\end{array}
$$
\caption{Type update (the metavariable \propisnotmeta{} ranges over \t{} and \nottype{\t{}} (without variables), 
  \notsubtypein{}{\Nil{}}{\nottype{\t{}}} when \issubtypein{}{\Nil{}}{\t{}}, see
\figref{main:figure:restrictremove} for \restrictliteral{} and \removeliteral{}.
  )}
\label{main:figure:update}
\end{figure}

%\begin{figure}
%  $$
%\begin{array}{llll}
%  \restrictremovefigure{}
%\end{array}
%  $$
%  \caption{Restrict and Remove}
%  \label{main:figure:restrictremove}
%\end{figure}

\subsection{Proof system}
\label{formalmodel:proofsystem}

The occurrence typing proof system uses standard propositional logic,
except for where nested information is combined. This is
handled by L-Update:
{  \footnotesize
  $$
\LUpdate{}
$$
}

It says
under \propenv{}, if object \path{\pathelemp{}}{\x{}} is of type \t{}, and 
an extension
\path{\pathelem{}}{\path{\pathelemp{}}{\x{}}}
is of possibly-negative type \propisnotmeta{}, then
{\update{\t{}}{\propisnotmeta{}}{\pathelem{}}}
is \path{\pathelemp{}}{\x{}}'s type under \propenv{}.

Recall \egref{example:desserts-on-meal}.
%, resuming from
%\secref{sec:coretypesystem}. 
Solving
$
{ \inpropenv 
  {{\isprop{\Order}{\makelocal{o}}},
    {\isprop{\Value{\makekw{combo}}}{\path{\keype{\makekw{Meal}}}{\makelocal{o}}}}}
  {\isprop {\t{}} {\makelocal{o}}}}
$
uses L-Update, where \pathelem{} = {\emptypath{}} and \pathelemp{} = [{\keype{\makekw{Meal}}}].
$$
\inpropenv{\propenv{}}{\isprop{\update{\Order}{\Value{\makekw{combo}}}{[{\keype{\makekw{Meal}}}]}}{\makelocal{o}}}
$$
Since {\Order} is a union of HMaps, we structurally recur on the first case of \updateliteral{}
(\figref{main:figure:update}),
which preserves \pathelem{}.
Each initial recursion hits the first HMap case, since there is some \t{} such that
{\inmandatory{\k{}}{\t{}}{\mandatory{}}} and 
\completenessmeta{} accepts partial maps \partial{}.

To demonstrate,
\makekw{lunch} meals are handled by the first HMap case and
update to {\HMapp {\extendmandatoryset {\mandatory{}}{\Valkw{Meal}}{\sp{}}} {\emptyabsent{}}}
where \sp{} = {\update{\Valkw{lunch}}{\Valkw{combo}}{\emptypath{}}}
and \mandatory{} = \mandatoryset{\mandatoryentry{\Valkw{Meal}}{\Valkw{lunch}},{\mandatoryentry{\Valkw{desserts}}{\Number{}}}}.
\sp{} updates to \Bot via the penultimate \updateliteral{} case,
because \restrict{\Value{\makekw{lunch}}}{\Value{\makekw{combo}}} = \Bot
by the first \restrictliteral{} case.
The same happens to \makekw{dinner} meals,
leaving just the \makekw{combo} HMap. 

In \egref{example:desserts-on-class},
$
\inpropenv{\propenv{}}{\isprop{\update{\Order}{\Long}{[{\classpe{}}, {\keype{\makekw{desserts}}}]}}{\makelocal{o}}}
$
updates the argument in the {\Long} method.
This recurs twice for each meal to handle the {\classpe{}}
path element.

We describe the other \updateliteral{} cases.
The first \classpe{} case updates
to \class{} if \classconst{} returns \Value{\class{}}.
The second \keype{\k{}} case detects contradictions in absent
keys. % not overlapping with \Nil{}.
The third \keype{\k{}} case updates unknown entries to be mapped to \t{} or absent.
The fourth \keype{\k{}} case updates unknown entries to be \emph{present}
when they do not overlap with \Nil{}.

%$
%{\update{\Number}{\Long}{[{\classpe{}}]}}}
%$
%
%$
%{\update{\Int}{\Long}{[{\classpe{}}]}}}
%$
%
%

\section{Metatheory}
\label{sec:metatheory}

We prove type soundness following Tobin-Hochstadt and Felleisen~\cite{TF10}.  Our model is extended
to include errors \errorvalv{} and a \wrong{} value, and we prove well-typed
programs do not go wrong; this is therefore a stronger theorem than
proved by Tobin-Hochstadt and Felleisen~\cite{TF10}. 
Errors behave like Java exceptions---they can be thrown and propagate ``upwards'' in the evaluation rules
(\errorvalv{} rules are deferred to the appendix).

Rather than modeling Java's dynamic semantics, a task of daunting
complexity, we instead make our assumptions about Java explicit. We
concede that method and constructor calls may diverge or error, but
assume they can never go wrong
(other assumptions given in the supplemental material).

{\javanewassumption{main}}

%For readability we define logical truth in Clojure.

%{\istruefalsedefinitions{main}}

For the purposes of our soundness proof, we require that all values
are \emph{consistent}.  Consistency (defined in the supplemental
material) states that the types of closures are well-scoped---they do
not claim propositions about variables hidden in their closures.

%{\consistentwithonlydef{main}}

We can now state our main lemma and soundness theorem.  The
metavariable \definedreduction{} ranges over \v{}, \errorvalv{} and
\wrong{}. Proofs are deferred to the supplemental material. %\ref{appendix:lemma:soundness}.

\begin{lemma}\label{main:lemma:soundness}

  {\soundnesslemmahypothesis}
\end{lemma}



{\soundnesstheoremnoproof{main}}

%{\wrongtheoremnoproof{main}}
%
%{\nilinvoketheoremnoproof{main}}

\section{Experience}
\label{sec:experience}

Typed Clojure is implemented as \coretyped{}~\cite{coretyped},
which has seen wide usage.

\subsection{Implementation}

\coretyped{} provides preliminary integration with the Clojure compilation
pipeline, primarily to resolve Java interoperability.
%, however
%most usages are entirely optionally typed.

%In contrast to Racket, Clojure does not provide extension
%points to the macroexpander. 
%To satisfy our goals of providing
%Typed Clojure as a library that works with the latest version of the Clojure
%compiler, \coretyped{} is implemented as an external static analysis pass
%that must be explicitly invoked by the programmer, and not as an
%integral part of the Clojure compilation process. 
%%Therefore, \coretyped{} is in a sense a linter.

The \coretyped{} implementation extends this paper in several key areas 
to handle checking real Clojure code, including an implementation
of Typed Racket's variable-arity polymorphism~\cite{stf-esop}, 
and support for other Clojure idioms like datatypes and protocols.
%
There is no integration with Java Generics, so only Java 1.4-style erased types are ``trusted''
by \coretyped{}.
Casts are needed to recover the discarded information, which---for collections---are 
then tracked via Clojure's universal sequence interface~\cite{CljSeqDoc}.

%Recently, steps have been taken to integrate Typed Clojure into 

%This means that type checking is  optional. 
%On the positive side, \coretyped{} is flexible to the needs of a dynamically
%typed programmer, encouraging experimentation with programs that may not
%type check.
%On the negative side, programmers must remember to type check their namespaces.
%Also, the compiler cannot depend on types, making
%type-based optimisation is impossible. 
%If this were not the case, we could dispose of type-hints
%altogether, and simply use static types to resolve reflection.

%\subsection{Let-aliasing}
%
%\begin{mathpar}
%  \footnotesize
%\infer [T-LocalAlias]
%{ \Theta[\x{}] = \object{}
%  \\
%  \inpropenv {\propenv{}} {\isprop {\t{}} {\object{}}}
%  \\\\
%  \s{} = {\falsy} }
%{ \judgement {\Theta; \propenv{}} 
%             {\hastype {\x{}} {\t{}}}
%             {\filterset {{\notprop {\s{}} {\object{}}}} {{\isprop {\s{}} {\object{}}}}}
%             {\object{}}
%                   }
%
%\infer [T-LetAlias]
%{ \judgement {\Theta; \propenv{}} {\hastype {\e{1}} {\s{}}} {\filterset {\thenprop {\prop{1}}} {\elseprop {\prop{1}}}}
%             {\object{1}}
%  \\\\
%  \object{1} \notequal \emptyobject{}
%  \\\\
%  \judgement
%       {\Theta[\x{} \mapsto \object{1}];
%         \propenv{}}
%             {\hastype {\e{}} {\t{}}} {\filterset {\thenprop {\prop{}}} {\elseprop {\prop{}}}}
%             {\object{}} 
%             }
%{ \judgement {\Theta; \propenv{}} {\hastype {\letexp {\x{}} {\e{1}} {\e{}}} {\t{}}}
%             {\filterset {\thenprop {\prop{}}} {\elseprop {\prop{}}}}
%             {\object{}} 
%             }
%\end{mathpar}

%\subsection{Further Extensions}
%
%In addition to the key features we present in this paper,
%\coretyped{} supports other extensions to handle additional Clojure
%features. 
%
%\smallsection{Datatypes, Records and Protocols}
%Clojure features datatypes and protocols. Datatypes are Java classes
%declared final with public final fields. They can implement Java interfaces
%or protocols, which are similar to interfaces but already-defined classes
%and \nil{} may extend protocols.
%%
%Typed Clojure can reason about most of these features,
%including the ability to define polymorphic datatypes and protocols and
%utilising the Java type system to help check implemented interface methods.

%\smallsection{Intersection Types}
%Typed Clojure includes simple intersection types which do no sophisticated
%reasoning with the dual subtyping rules to unions.
%
%In some cases this makes types more expressive. Say we know \clj{x} has some
%universally quantified type \clj{a} and we learn \clj{x} is a \clj{Number}.
%Without intersection types, we must choose which piece of information to forget.
%In Typed Clojure, \clj{x} is simply of type \clj{(I x Number)}.
%
%\smallsection{Mutation and Polymorphism}
%Clojure supports mutable references with software-transactional-memory
%which Typed Clojure defines \emph{bivariantly}---with write and read type parameters
%as in the atomic reference \clj{(Atom2 Int Int)} which can write and read \clj{Int}.
%Typed Clojure also supports parametric polymorphism, including
%Typed Racket's variable-arity polymorphism~\cite{stf-esop}, 
%which enables us to assign a type to functions like \clj{swap!} (\figref{main:fig:swap!}),
%which takes a mutable \emph{atom},
%a function and extra arguments, and swaps into the atom the result of
%applying the function to the atom's current value and the extra arguments.
%
%\begin{figure}
%\begin{minted}{clojure}
%(ann clojure.core/swap! (All [w r b ...] 
%                          [(Atom2 w r) [r b ... b -> w] b ... b -> w]))
%(swap! (atom :- Num 1) + 2 3);=> 6 (atom contains 6)
%\end{minted}
%%\inputminted[firstline=5,lastline=5]{clojure}{code/demo/src/demo/atom.clj}
%\caption{Type annotation and example call of \clj{swap!}}
%\label{main:fig:swap!}
%\end{figure}

\subsection{Evaluation}
\label{sec:casestudy}

Throughout this paper, we have focused on three interrelated type
system features: heterogenous maps, Java interoperability, and
multimethods. Our hypothesis is that these features are widely used in
existing Clojure programs in interconnecting ways, and that handling
them as we have done is required to type check realistic Clojure
programs.



To evaluate this hypothesis, we analyzed two existing \coretyped{}
code bases, one from the open-source community, and one from a company
that uses \coretyped{} in production. For our data gathering, we
instrumented the \coretyped{} type checker to record how often
various features were used (summarized in 
\figref{experience:featuretable}). 

\begin{figure*}[t]

\begin{tabular}{lll}
      \toprule


  & feeds2imap & CircleCI \\
  \midrule
  Total number of typed namespaces & 11 (825 LOC) & 87 (19,000 LOC) \\
  Total number of \clj{def} expressions & 93  & 1834 \\
  \tabitem
  checked & 52 (56\%) & 407 (22\%) \\
  \tabitem
  unchecked & 41 (44\%) & 1427 (78\%) \\
  Total number of Java interactions & 32 & 105 \\
  \tabitem
  static methods & 5 (16\%) & 26 (25\%) \\ 
  \tabitem
  instance methods & 20 (62\%) & 36 (34\%) \\
  \tabitem
  constructors & 6 (19\%) & 38 (36\%) \\
  \tabitem
  static fields & 1 (3\%) & 5 (5\%) \\
  Methods overriden to return non-nil & 0 & 35 \\
  Methods overriden to accept nil arguments & 0 & 1 \\
  Total HMap lookups & 27  & 328  \\
  \tabitem
  resolved to mandatory key & 20 (74\%) & 208 (64\%) \\
  \tabitem
  resolved to optional key & 6 (22\%) & 70 (21\%) \\
  \tabitem
  resolved of absent key & 0 (0\%) & 20 (6\%) \\
  \tabitem
  unresolved key & 1 (4\%) & 30 (9\%) \\
  Total number of \clj{defalias} expressions & 18  & 95 \\
  \tabitem
  contained HMap or union of HMap type & 7 (39\%)  & 62 (65\%) \\
  Total number of checked \clj{defmulti} expressions & 0  & 11 \\
  Total number of checked \clj{defmethod} expressions & 0  & 89 \\


\end{tabular}
\caption{Typed Clojure Features used in Practice}
\label{experience:featuretable}
\end{figure*}


\paragraph{feeds2imap}
feeds2imap\footnote{\url{https://github.com/frenchy64/feeds2imap.clj}}
is an open source library written in Typed Clojure. 
It provides an RSS reader using the \emph{javax.mail} framework.

% static call (:check/:static-call) = 74
% - user   5
% - inlined (:check/static-call-clojure-lang-probably-inline) 69
% static field = 13
% - user 1
% - inlined (:check/static-field-clojure-lang-probably-inline) 12
% new (:check/:new) = 11
% - user 6
% - inlined (:check/new-clojure-lang-probably-inline) 5
% instance call = 53
% - body  20
% - inlined (:check/instance-call-clojure-lang-probably-inline) 33
% total 151
% - user  32
Of 11 typed namespaces containing 825 lines of code, there are 32 Java interactions.
The majority are method calls, consisting of 20 (62\%) instance methods and 5 (16\%) static methods. 
The rest consists of 1 (3\%) static field access, and 6 (19\%) constructor calls---there are no instance field accesses.

%  from :check/find-val-type-with-hmap* numbers
There are 27 lookup operations on HMap types, of which 20 (74\%) resolve to mandatory entries, 6 (22\%) to optional entries, and 1 (4\%) is an unresolved lookup. 
No lookups involved fully specified maps.

% :collect/:def     93
% :check/checked-def 52
From 93 \clj{def} expressions in typed code, 52 (56\%) are checked, with a rate of 1 Java interaction for 1.6 checked top-level definitions, and 1 HMap lookup to 1.9 checked top-level definitions.
That leaves 41 (44\%) unchecked vars, mainly due to partially complete porting to Typed Clojure, but in some cases due to unannotated third-party libraries.

No typed multimethods are defined or used. 
% :collect/defalias-is-HMap      7
% :invoke-special-collect/(quote clojure.core.typed/def-alias*)     18
Of 18 total type aliases, 7 (39\%) contained one HMap type, and none contained unions of HMaps---on further inspection there was no HMap entry used to dictate control flow, often handled by multimethods.
This is unusual in our experience, and is perhaps explained by feeds2imap mainly wrapping existing \emph{javax.mail} functionality.

\paragraph{CircleCI}
CircleCI~\cite{CircleCI}
provides continuous integration services built with a mixture of open-
and closed-source software.
Typed Clojure was used at CircleCI in production systems for two years \cite{CircleCIUsesTC},
maintaining
87 namespaces and 19,000 lines of code,
an experience we summarise in \secref{sec:limitations}.
%
%CircleCI provided the first author access to the main closed-source backend system written in Clojure
%and Typed Clojure.
%We conducted a study of the effectiveness of Typed Clojure in practice.
%There is no clear metric for quantifying typed Clojure code, since untyped code
%can be freely mixed and some seemingly typed namespaces are not checked
%regularly. 
%We manually type checked all namespaces that depend on \clj{clojure.core.typed}
%and considered those with type errors as untyped.
%We then searched the remaining typed code for unsafe Typed Clojure operations like
%var annotations with \clj{:no-check} and the \clj{tc-ignore} macro,
%which instruct Typed Clojure to ignore the specified code,
%and also considered those untyped.
%Furthermore, we manually collected and inspected all top-level annotations and
%classified them.
%
%We determined that

%% Out of 588 top-level var annotations, 270 (46\%) were checked annotations of
%% functions defined in typed code,
%% 129 (22\%) annotations assigned types to external libraries 
%% and the remaining 189 (32\%) annotated `unchecked' user code.
%Some of the type-annotated definitions were so annotated by the first
%author and contributed back to CircleCI.
%HMaps were a valuable feature, with 38 (59\%) out of 64 total type aliases
%featuring them; see \egref{example:circleci} for an instance.
%
%Because of various shortcomings of \coretyped{}, all 57 \clj{defmethod}
%expressions in typed namespaces were unchecked.
%
%811 top-level var annotations
%
%% Due to a lack of checked multimethods,
%% the first author ported 11 previously-untyped multimethods to Typed Clojure, also checking 
%% 89 methods.

The CircleCI code base contains 11 checked multimethods.
 All 11 dispatch functions
are on a HMap key containing a keyword, in a similar style to
\egref{example:desserts-on-meal}.
Correspondingly, all 89 methods are associated with a keyword dispatch value.
The argument type was in all cases a single HMap type, however,
rather than a union type.
In our experience from porting other libraries, this is unusual.

% 87 typed namespaces
% :check/gen-analysis     87

% :check/find-val-type-with-hmap    328
% :check/find-val-type-with-hmap-present    208
% :check/find-val-type-with-hmap-with-optional     70
% :check/find-val-type-with-hmap-fall-through     30
% :check/find-val-type-with-hmap-absent     20
% :check/find-val-has-complete      2
% :merge/complete-used-on-right      5

Of 328 lookup operations on HMaps,
208 (64\%) resolve to mandatory keys,
70 (21\%) to optional keys,
20 (6\%) to absent keys, and
30 (9\%) lookups are unresolved.
%
% :collect/defalias-is-HMap     62
% :invoke-special-collect/(quote clojure.core.typed/def-alias*)     95 
Of 95 total type aliases defined with \clj{defalias},
62 (65\%) involved one or more HMap types.
%
%% :new-special/(quote clojure.lang.MultiFn)     11
%
%% :check/:static-call    525
%% :check/static-call-clojure-lang-probably-inline    499
%% = 26 user
%
%% :check/:instance-call    510
%% :check/instance-call-clojure-lang-probably-inline    474
%% = 36 user
%
%% :check/:new    159
%% :check/new-clojure-lang-probably-inline    121
%% = 38
%
%% :check/:static-field     92
%% :check/static-field-clojure-lang-probably-inline     87
%% = 5
%
%% 26 + 36 + 38 + 5 = 105
%
%% :invoke-special-collect/(quote clojure.core.typed/non-nil-return*) 35
%% :invoke-special-collect/(quote clojure.core.typed/nilable-param*)  1
%
Out of 105 Java interactions, 26 (25\%) are static methods, 36 (34\%)
are instance methods, 38 (36\%) are constructors, and 5 (5\%) are static
fields. 35 methods are overriden to return non-nil, and 1 method 
overridden to accept nil---suggesting that
\coretyped{} disallowing \clj{nil} as a method argument by default
is justified.

% :check/checked-def  407
% :check/checked-MultiFn-addMethod 57
% :instance-method-special/(quote clojure.lang.MultiFn/addMethod)     89
% = 464 checked definitions
Of 464 checked top-level definitions (which consists of
57 \clj{defmethod} calls and 407 \clj{def} expressions),
1 HMap lookup occurs per 1.4 top-level definitions,
and 1 Java interaction occurs every 4.4 top-level definitions.

% :check/def-not-checking-definition   1352
% :check/checked-def  407
% = 1759
% :collect/:def   1834
% = 1427 unchecked
From 1834 \clj{def} expressions in typed code,
%87 typed namespaces,
only 407 (22\%) were checked.
That leaves 1427 (78\%) which have unchecked definitions, either by an explicit \clj{:no-check} annotation
or \clj{tc-ignore} to suppress type checking,
or the \clj{warn-on-unannotated-vars} option, which skips \clj{def} expressions
that lack expected types via \clj{ann}.
From a brief investigation,
reasons include unannotated third-party libraries,
work-in-progress conversions to Typed Clojure,
unsupported Clojure idioms, 
and hard-to-check code.

\paragraph{Lessons}
Based on our empirical survey, HMaps and Java interoperability support
are vital features used on average more than once per typed
function. 
%
Multimethods are less common
in our case studies. The CircleCI code base contains only 26 multimethods total
in 55,000 lines of mixed untyped-typed Clojure code,
a low number in our experience.

%The
%data therefore validates our choice of a type system that supports
%expressive multimethod definition and acknowledges the relationship
%between these seemingly-distinct features. 

%
%The other lesson from our case studies and from other interactions
%with Typed Clojure users, it is clear the main barrier to entry to
%Typed Clojure for large systems is the requirement to annotate
%functions outside the borders of typed code.  We hope that this
%can be addressed by making annotations available for popular
%libraries.

\subsection{Further challenges}
\label{sec:limitations}

After a 2 year trial, the second case study decided to disabled type checking~\cite{CircleCIBlog}.
They were supportive of the fundamental ideas presented in this paper, but primarily
cited issues with the checker implementation in practice and would reconsider
type checking if they were resolved. This is also supported by \figref{experience:featuretable},
where 78\% of \clj{def} expressions are unchecked.

\smallsection{Performance}
Rechecking files with transitive dependencies is expensive since all dependencies must be rechecked.
We conjecture caching type state will significantly
improve re-checking performance,
though preserving static soundness in the context of arbitrary code reloading is a largely unexplored area.

\smallsection{Library annotations}
Annotations for external code are rarely available, so a large part of the
untyped-typed porting process is reverse engineering libraries.

\smallsection{Unsupported idioms}
While the current set of features is vital to checking Clojure code,
there is still much work to do.
For example, common Clojure functions are often too polymorphic for the current implementation
or theory to account for. The post-mortem~\cite{CircleCIBlog} contains more details.

%\smallsection{Java Arrays}
%Java arrays are known to be statically unsound.
%\cite{Bra98} summarises the approach taken to regain runtime soundness, which involves
%checking array writes at runtime.
%
%Typed Clojure implements an experimental partial solution, making arrays \emph{bivariant},
%separating the write and read types into contravariant and covariant parameters.
%If the array originates from typed code, then we may track the write and read
%parameters statically. Currently arrays from foreign sources
%have their write parameter set to to \Bot{}, protecting typed code from writing
%something of incorrect type. However there are currently no casting mechanisms to 
%convince Typed Clojure the foreign array is writeable.

%\smallsection{Array-backed sequences}
%Typed Clojure assumes sequences are immutable. This is almost always true, however
%for performance reasons,
%sequences created from Java arrays (and Iterables) reflect future writes to the array 
%in the `immutable' sequence. While disturbing and a clear unsoundness in Typed Clojure,
%this has not yet been an issue in practice and is strongly discouraged as undefined behavior:
%``Robust programs should not mutate arrays or Iterables that have seqs on them.''~\cite{CljSeqDoc}.
%
%\smallsection{Typed-untyped interoperation}
%Currently, interactions between typed and untyped Clojure code are unchecked
%which can violate the expectations of Typed Clojure.
%Gradual typing~\cite{thf06,siek06:_gradual} ensures sound interoperability between typed and untyped code by enforcing
%invariants of the type system via run-time contracts.
% We hope to add support
%for gradual typing in the future.



%OLD

%\subsection{Using negative filters}
%
%Occurrence typing plays an important role in Typed Racket and Typed Clojure.
%By maintaining a \emph{proposition environment} of propositions relating types to
%bindings, we can update bindings with more accurate types as programs progress.
%It follows that there is some correspondence between propositions and types,
%characterised by the \emph{update} function, which takes a type and a proposition
%and returns a type which updates the input type using the proposition.
%
%There is a straightforward relationship between ``positive'' propositions and types.
%For example 
%{\tt (update Number (is Integer 0))}
%updates Number by Integer, which is Integer, because Integer <: Number.
%
%The relationship between ``negative'' propositions and types is not always obvious.
%A common proposition in Typed Clojure is (! (U nil false) a): the proposition that
%local binding ``a'' is \emph{not} of type (U nil false).
%This problem is most visible in expressions like {\tt (filter identity coll)}, where
%``identity'' has a ``then'' proposition that has negative information: (! (U nil false) 0),
%which reads, the 0th argument of identity does not contain (U nil false).
%
%\subsubsection{Arrays}
%\label{sec:arrays}
%
%Supporting statically sound interactions with Java arrays is a goal
%of Typed Clojure. This is complicated by Java's decision to make
%arrays covariant in their argument, a well documented source of static
%unsoundness. Bracha~\cite{Bra98} summarises Java's approach to maintaining
%soundness at runtime, which involves all array writes being checked by
%runtime assertions.
%
%This approach fits Java's type system, but we can do better in a more powerful
%type system like Typed Clojure. Our goal is to catch all type-incorrect array
%writes at compile time so the type system can do more to help Clojure programmers
%use arrays, especially those being passed from foreign Java code.
%
%Our basic approach is to make our array types \emph{bivariant}. Array types
%look like {\ArrayTwo {\t{w}} {\t{r}}} and
%are reminiscent of functions or pipes: having a contravariant parameter for input (writing)
%and a covariant parameter for output (reading).
%This type can write type {\t{w}} and read type {\t{r}}.
%
%Most commonly, an array type is invariant in its parameter; it can
%write and read input of the same type.
%We can get the same effect by setting our input and output
%parameters to the same type. For example, {\ArrayTwo {\Number} {\Number}}
%(or equivalently, {\Array {\Number}})
%in Typed Clojure is similar to invariant array types of \Number in languages like Scala.
%
%The biggest gain in using a separate input parameter is the ability
%to specify \emph{read-only} arrays. Crucially, our type system features an
%explicit bottom type \lstinline|Nothing|, enabling a read-only \lstinline|Number| array
%to be of type \lstinline|(Array2 Nothing Number)|.
%
%To realise why defining read-only arrays are useful, we need to examine
%what makes array covariance unsound in Java.
%\begin{verbatim}
%FIXME
%Array covariance about the type of an array so the consumer
%of an array cannot tell the actual type of the array when examining a type
%signature.
%\end{verbatim}
%
%\begin{lstlisting}
%...
%public static Number[] getNumberArray() {
%  Number[] n = new Integer[10];
%  return n;
%}
%...
%\end{lstlisting}
%
%To the casual consumer \emph{getNumberArray} returns an array that can both
%read and write \lstinline|Number|s. However it is clear from the implementation
%that attempting to write say a \lstinline|Double| to this array will result
%in a runtime error.
%
%\begin{verbatim}
%...
%Number[] myArray = getNumberArray();
%myArray[0] = 1.1;
%/* Exception in thread "main" 
%   java.lang.ArrayStoreException: 
%   java.lang.Double */
%...
%\end{verbatim}
%
%Notice that this is a runtime error, and Java's type system has not helped
%statically prevent it.
%This could cause a similar issue for other statically-typed languages offering
%interoperability with Java. 
%
%To prevent these sorts of runtime exceptions in Typed Clojure, we declare
%all arrays from unknown sources to be \emph{read-only}. Put differently,
%the only way to define a writeable array is to create it in type-checked Clojure
%code.
%
%\begin{lstlisting}
%(let [n (CovariantArray/getNumberArray)]
%  (aset n 0 1.1))
%
%; Polymorphic static method clojure.lang.RT/aset could not be 
%; applied to arguments:
%; Domains: 
%;         (Array2 i o) clojure.core.typed/AnyInteger i
%; 
%; Arguments:
%;         (Array2 Nothing java.lang.Number) int (Value 1.1)
%; 
%; with expected type:
%;         Any
%\end{lstlisting}
%
%The type inferred for the local \lstinline|n| is \lstinline|(Array2 Nothing Number)|
%which tells the type system: it is never safe to write to this array, but
%it is safe to assume \lstinline|Number|s can be read from this array.
%
%To emphasise, Typed Clojure throws a static type error. Errors like this help Clojure programmers
%use foreign Java libraries more correctly.
%
%\begin{verbatim}
%Note that Java libraries are often large 
%and complex and programmers will probably
%enjoy the extra help from the type system.
%\end{verbatim}

\section{Related Work}

% Cite a few of the early papers here.
%http://www.cs.washington.edu/research/projects/cecil/www/pubs/
\paragraph{Multimethods} 
\cite{MS02} and collaborators present a sequence of
systems~\cite{Chambers:1992:OMC,Chambers:1994:TMM,MS02} with statically-typed multimethods
and modular type checking.  In contrast to Typed Clojure, in these
system methods declare the types of arguments that they expect which
corresponds to exclusively using \clj{class} as the dispatch function
in Typed Clojure. However, Typed Clojure does not attempt to rule out
failed dispatches.

% one sentence
% TC based on TR, already covered

%\paragraph{Occurrence Typing} 
%Occurrence typing~\cite{TF08,TF10} extends the type 
%system with a \emph{proposition environment} that represents 
%the information on the types of bindings down conditional branches.
%These propositions are then used to update the types associated
%with bindings in the \emph{type environment} down branches
%so binding occurrences are given different types 
%depending on the branches they appear in, and the conditionals
%that lead to that branch.

% What's diff about TC from the related work
% small summary for deisel....
% - diesel supports x
%- - calculus supports some subset of x
% we support y, which covers most of x but also foo

% eg. multiple dispatch
%     nominal vs structural

% eg. run abritrary metaprogramming over dispatch in CLOS
%  more expressive

% type systems for mm or rows
% rows vs HMap
% - no poly in HMap
% - based on subtyping
% - rows based on polymorphism

\paragraph{Record Types} Row polymorphism~\cite{Wand89typeinference,CM91,HP91}, used
in systems such as the OCaml object system, provides many of the
features of HMap types, but defined using universally-quantified row
variables. HMaps in Typed Clojure are instead designed to be used with
subtyping, but nonetheless provide similar expressiveness, including
the ability to require presence and absence of certain keys. 

Dependent JavaScript~\cite{Chugh:2012:DTJ} can track similar
invariants as HMaps with types for JS objects. They must deal with
mutable objects, they feature refinement types and strong updates to
the heap to track changes to objects.

TeJaS~\cite{TeJaS}, another type system for JavaScript,
also supports similar HMaps, with the ability to
record the presence and absence of entries, but lacks a compositional
flow-checking approach like occurrence typing.

Typed Lua~\cite{Maidl:2014:TLO} has \emph{table types} which track
entries in a mutable Lua table.  Typed Lua changes the dynamic
semantics of Lua to accommodate mutability: Typed Lua raises a runtime
error for lookups on missing keys---HMaps consider lookups on missing
keys normal.

\paragraph{Java Interoperability in Statically Typed Languages}
Scala~\cite{OCD+} has nullable references for compatibility with Java.
Programmers must manually check for
\java{null} as in Java to avoid null-pointer exceptions. 


\paragraph{Other optional and gradual type systems}
%In addition to Typed Racket, 
Several other gradual type
systems have been developed for existing
dynamically-typed languages.  Reticulated Python~\cite{Vitousek14} is
an experimental gradually typed system for Python, implemented as a
source-to-source translation that inserts dynamic checks at language
boundaries and supporting Python's first-class object system. 
Clojure's nominal classes avoids the need to support
first-class object system in Typed Clojure, however HMaps offer an alternative to
the structural objects offered by Reticulated. Similarly,
Gradualtalk~\cite{gradualtalk} offers gradual typing for Smalltalk,
with nominal classes.

Optional types
%, requiring less implementation effort and avoiding
%runtime cost, 
have been  adopted in industry, including Hack~\cite{hack}, and Flow~\cite{flow} and
TypeScript~\cite{typescript}, two extensions of JavaScript. These
systems  support  limited forms of occurrence typing,
and do not include the other features we
present.

%  \item GradualTalk
%  \item Flow
%\end{itemize}



\section{Conclusion}
\label{sec:conclusion}

Optional type systems must be designed with close attention to the
language that they are intended to work for.
We have therefore designed Typed Clojure, an optionally-typed version of
Clojure, with a type system that works with a wide variety of distinctive
Clojure idioms and features. Although based on the foundation of Typed
Racket's occurrence typing approach, Typed Clojure both extends the
fundamental control-flow based reasoning as well as applying it to
handle seemingly unrelated features such as multi-methods. In
addition, Typed Clojure supports crucial features such as
heterogeneous maps and Java interoperability while integrating these
features into the core type system. Not only are each of these
features important in isolation to Clojure and Typed Clojure
programmers, but they must fit together smoothly to ensure that
existing untyped programs are easy to convert to Typed Clojure.

The result is a sound, expressive, and useful type system which, as
implemented in \coretyped with appropriate extensions, is suitable for
typechecking a significant amount of existing Clojure programs.
%
As a result, Typed Clojure is already successful: it is used in
the Clojure community among both enthusiasts and professional
programmers.% and receives contributions from many developers.

Our empirical analysis of existing Typed Clojure programs bears out
our design choices. Multimethods, Java interoperation, and
heterogeneous maps are indeed common in both Clojure and Typed Clojure,
meaning that our type system must accommodate them. Furthermore, they
are commonly used together, and the features of each are mutually
reinforcing. Additionally, the choice to make Java's \clj{null}
explicit in the type system is validated by the many Typed Clojure
programs that  specify non-nullable types.

% Delete the following paragraphs if space is needed.

%However, there is much more that Typed Clojure can provide. Most
%significantly, Typed Clojure currently does not provide \emph{gradual
%  typing}---interaction between typed and untyped code is unchecked and
%thus unsound. We hope to explore the possibilities of using existing
%mechanisms for contracts and proxies in Java and
%Clojure to enable sound gradual typing for Clojure.
%
%Additionally, the Clojure compiler is unable to use Typed Clojure's
%wealth of static information to optimize programs. Addressing this
%requires not only  enabling sound gradual typing, but also
%integrating Typed Clojure into the Clojure tool so
%that its information can be communicated to the compiler. 

%Finally, our case study, evaluation, and broader experience indicate that Clojure
%programmers still find themselves unable to use Typed Clojure on some
%of their programs for lack of expressiveness. This requires continued
%effort to analyze and understand the features and idioms and
%develop new type checking approaches.

\counterwithin{figure}{section}
\counterwithin{assumption}{section}
\counterwithin{theorem}{section}
\counterwithin{lemma}{section}
\counterwithin{definition}{section}

\part*{Appendices}

\appendix

\singlespacing
\chapter{Full rules for \lambdatc{}}

\begin{figure}[h]
$$
\begin{altgrammar}
  \expd{}, \e{} &::=& \x{}
                      \alt \v{} 
                      \alt {\comb {\e{}} {\e{}}} 
                      \alt {\abs {\x{}} {\t{}} {\e{}}}
                      \alt {\ifexp {\e{}} {\e{}} {\e{}}}
                      \alt {\doexp {\e{}} {\e{}}}
                      \\
                      &\alt& {\letexp {\x{}} {\e{}} {\e{}}}
                      \alt {\wrongorerror{}}
                      \alt {\ReflectiveExp{}}
                      \alt {\NonReflectiveExp{}}
                      \alt {\MultimethodExp{}}
                      %\alt {\HintedExp{}}
                      \alt {\HashMapExp{}}
                &\mbox{Expressions} \\
  \v{} &::=&          \singletonmeta{}
                      \alt \classvaluemeta{}
                      \alt {\emptymap{}}
                      \alt {\const{}}
                      \alt {\num{}}
                      \alt {\str{}}
                      \alt \mapval{}
                      \alt {\closure {\openv{}} {\abs {\x{}} {\t{}} {\e{}}}}
                      \alt {\multi {\v{}} {\disptable{}}}
                &\mbox{Values} \\
  \mapval{} &::=&  {\curlymapvaloverright{\v{}}{\v{}}}
                &\mbox{Map Values} \\
                \constantssyntax{}\\
%  \HintedExp{}             &::=& \typehintedexpsyntax{}
%                &\mbox{Type Hinted Expressions} \\
  \HashMapExp{}                &::=& \hmapexpressionsyntax{}
                &\mbox{Hash Maps} \\
  \NonReflectiveExp{}     &::=& \nonreflectiveexpsyntax{}
                &\mbox{Non-Reflective %Java
                       Interop} \\
  \ReflectiveExp{}     &::=& \reflectiveexpsyntax{}
                &\mbox{Reflective %Java
                       Interop} \\
  \MultimethodExp{}     &::=& \multimethodexpsyntax{}
                &\mbox{%Immutable First-Class 
                Multimethods}
                      \\%\\ save space...
  \s{}, \t{}    &::=& \Top 
                      \alt \class{}
                      \alt {\Value \singletonmeta{}} 
                      \alt {\Unionsplice {\overrightarrow{\t{}}}}
                      \alt
                      {\ArrowOne {\x{}} {\t{}}
                                   {\t{}}
                                   {\filterset {\prop{}} {\prop{}}}
                                   {\object{}}}
                      \\
                      &\alt& {\HMapgeneric {\mandatory{}} {\absent{}}}
                      \alt {\MultiFntype{\t{}}{\t{}}}
                      
                &\mbox{Types} \\
                \auxhmapsyntax{}\\
  \singletonallsyntax{}
                \\% \\ save space...
                \openvsyntax{}\\\\

                %\tatypesyntax{}\\\\

  \occurrencetypingsyntax{}
  %\pathelemsyntax{}\\
  \propenvsyntax{}
  \\\\

 \disptablesyntax{} \\
 %\typehintenvsyntax{} \\
\classtableallsyntax{} \\
               \classliteralallsyntax{}\\

               \classvaluesyntaxentry{}\\
                      \\
  \wrongorerror{} &::=& \wrong{} \alt \errorvalv{}
                &\mbox{Wrong or error}
                      \\
  \definedreduction{} &::=& \v{} \alt \wrongorerror{}
                 &\mbox{Defined reductions}
                 \\
  \polaritymeta{} &::=& \pluspolarityliteral \alt \minuspolarityliteral
                 &\mbox{Substitution Polarity}
\end{altgrammar}
$$
\caption{Syntax of Terms, Types, Propositions, and Objects}
\end{figure}

\begin{figure}
$$
\begin{array}{lllr}
  \Nil &\equiv& {\ValueNil}\\
  \True &\equiv& {\ValueTrue}\\
  \False &\equiv& {\ValueFalse}\\
\end{array}
$$
\caption{Type abbreviations}
\end{figure}

\begin{figure}
$$
\begin{array}{lllr}
  \judgementtwo{\propenv{}}{\e{}}{\t{}} &\equiv& 
  \judgement{\propenv{}}{\e{}}{\t{}}{\filterset{\thenprop{\prop{}}}{\elseprop{\prop{}}}}{\object{}}
  & \text{for some}\ {\thenprop{\prop{}}}, {\elseprop{\prop{}}} \text{and}\ {\object{}}

  \\
  {\replacefor{\t{}}{\object{}}{\x{}}} &\equiv& {\pluspolarity{\replacefor{\t{}}{\object{}}{\x{}}}}
  \\
  {\replacefor{\prop{}}{\object{}}{\x{}}} &\equiv&  {\pluspolarity{\replacefor{\prop{}}{\object{}}{\x{}}}}
  \\
  {\replacefor{\filterset{\prop{}}{\prop{}}}{\object{}}{\x{}}} &\equiv&  {\pluspolarity{\replacefor{\filterset{\prop{}}{\prop{}}}{\object{}}{\x{}}}}
  \\
  {\replacefor{\object{}}{\object{}}{\x{}}} &\equiv& {\pluspolarity{\replacefor{\object{}}{\object{}}{\x{}}}}

\end{array}
$$
\caption{Judgment abbreviations}
\end{figure}


\begin{figure*}
\begin{mathpar}

  {\TLocal}

{\TConst}

{\TTrue}

{\TFalse}

{\TNil}

{\TNum}

{\TDo}

{\TIf}

{\TLet}
                 
{\TApp}

{\TAbs}

\infer [T-Clos]
{ \exists {\propenv{}}. \satisfies{\openv{}}{\propenv{}}
  \ \text{and}\ 
\judgementrewrite {\propenv{}} {\abs {\x{}} {\t{}} {\e{}}} {\s{}}
                 {\filterset {\thenprop {\prop{}}}
                             {\elseprop {\prop{}}}}
                 {\object{}}
                 {\abs {\x{}} {\t{}} {\ep{}}}
              }
{ \judgementrewrite {}
            {\closure {\openv{}} {\abs {\x{}} {\t{}} {\e{}}}} 
                      {\s{}}
             {\filterset {\thenprop {\prop{}}}
                         {\elseprop {\prop{}}}}
             {\object{}}
            {\closure {\openv{}} {\abs {\x{}} {\t{}} {\ep{}}}}
          }

           {\TError}

         {\TSubsume}
\end{mathpar}
\caption{Standard Typing Rules}
\end{figure*}


\begin{figure*}
\begin{mathpar}

{\TNew}

{\TNewStatic}

{\TField}

{\TFieldStatic}

{\TMethod}

{\TMethodStatic}

{\TClass}

{\TInstance}
\end{mathpar}
\caption{Java Interop Typing Rules}
\end{figure*}

\begin{figure*}
\begin{mathpar}

  \TDefMulti{}

  \TDefMethod{}

\TIsA{}

\infer [T-Multi]
{ \judgementtworewrite {} {\v{}} {\t{}} {\vp{}}
  \\
  \overrightarrow{\judgementtworewrite{}{\v{k}}{\Top}{\vp{k}}}
  \\
  \overrightarrow{\judgementtworewrite{}{\v{v}}{\s{}}{\vp{v}}}
}
{ \judgementrewrite {}
  {\multi {\v{}} {\curlymapvaloverright{\v{k}}{\v{v}}}}
                      {\MultiFntype {\s{}} {\t{}}}
             {\filterset {\topprop{}} {\botprop{}}}
           {\emptyobject{}}
  {\multi {\vp{}} {\curlymapvaloverright{\vp{k}}{\vp{v}}}}
}

\end{mathpar}
\caption{Multimethod Typing Rules}
\end{figure*}

\begin{figure*}
\begin{mathpar}

%\infer [T-Get]
%{ \judgementtwo {\propenv{}} {\hastype {\e{m}} {\Map {\t{k}}{\t{v}}}}
%  \\
%  \judgementtwo {\propenv{}} {\hastype {\e{k}} {\Top}}}
%{ \judgement {\propenv{}} {\hastype {\getexp {\e{m}} {\e{k}}} {\Union{\t{v}}{\nil{}}}}
%             {\filterset {\topprop{}} {\topprop{}}}
%           {\emptyobject{}}}
%
%\infer [T-Assoc]
%{ 
%  \judgementtwo {\propenv{}} {\hastype {\e{m}} {\Map{\t{k}}{\t{v}}}}
%  \\
%  \judgementtwo {\propenv{}} {\hastype {\e{k}} {\t{k}}}
%  \\
%  \judgementtwo {\propenv{}} {\hastype {\e{v}} {\t{v}}}
%}
%{ \judgement {\propenv{}} 
%             {\hastype {\assocexp {\e{m}} {\e{k}} {\e{v}}} {\Map {\t{k}}{\t{v}}}}
%             {\filterset {\topprop{}} {\botprop{}}}
%             {\emptyobject{}}
%}

\infer [T-HMap]
{ \overrightarrow{\judgementtworewrite {} {\v{k}}{\Value \k{}}{\vp{k}}}\\
  \overrightarrow{\judgementtworewrite {} {\v{v}}{\t{v}}{\vp{v}}}\\
  \mandatory{} = \mandatorysetoverright{\k{}}{\t{v}}
}
{ \judgementrewrite {}
             {\curlymapvaloverright{\v{k}}{\v{v}}}
                       {\HMapc {\mandatory{}}}
             {\filterset {\topprop{}} {\botprop{}}}
             {\emptyobject{}}
             {\curlymapvaloverright{\vp{k}}{\vp{v}}}
           }

    {\TKw}

    {\TGetHMap}

    {\TGetAbsent}

    {\TGetHMapPartialDefault}

    {\TAssoc}

\end{mathpar}
\caption{Map Typing Rules}
\end{figure*}


%\input{tools-analyzer-figure}

\begin{figure*}
\begin{mathpar}
\objectsub{}

\standardsubtyping{}
\SPMultiFn{}
\Multisubtyping{}

\HMapsubtyping{}
\end{mathpar}
\caption{Subtyping rules}
\end{figure*}


%$$
%\begin{tdisplay}{Evaluation Contexts}
%  \begin{altgrammar}
%    \E{} &::=& [ ] % application rules
%              \alt (\c{}\ \overrightarrow{\v{}}\ \E{}\ \overrightarrow{\exp{}}) % eval arguments left-to-right
%              % map rules
%              \alt \{\overrightarrow{\v{}\ \v{}}\ \E{}\ \exp{}\ \overrightarrow{\exp{}\ \exp{}} \} % key first
%              \alt \{\overrightarrow{\v{}\ \v{}}\ \v{}\ \E{}\ \overrightarrow{\exp{}\ \exp{}} \}   % value next
%              &\mbox{Evaluation Contexts}
%  \end{altgrammar}
%\end{tdisplay}
%$$ 

\input{esop-convert-types-figure}
\begin{figure}
\begin{mathpar}
\constanttypefigure{}
\end{mathpar}
\caption{Constant Typing}
\end{figure}

\constantsemfigure{appendix}


\begin{figure*}
\isapropsfigure{}

\isaopsemfigure{}
\caption{Definition of isa?}
\end{figure*}

%\begin{figure*}
%$$
%\begin{array}{llrr}
%  \isacompare{\HVec{\overrightarrow{{\t{}};{\prop{}};{\object{}}}^i}}
%             {\object{}}
%             {\HVec{\overrightarrow{{\t{}};{\prop{}};{\object{}}}^j}}
%             {\replacefor
%              {\filtersetparen
%                {\isprop {\HVec{\overrightarrow{\isacomparethree{\t{i}}{\object{i}}{\t{j}}}}}{\x{}}}
%                {\notprop{\HVec{\overrightarrow{\isacomparethree{\t{i}}{\object{i}}{\t{j}}}}}{\x{}}}}
%              {\object{}}
%              {\x{}}}
%              & i = j
%\end{array}
%$$
%$$
%\begin{array}{lclr}
%  \isaopsem{\rtvector{\overrightarrow{\x{}}^i}}{\rtvector{\overrightarrow{\x{}}^j}} &=& {\true{}}
%                                                                                    & i = j, \overrightarrow{\isaopsem{\x{i}}{\x{j}} = {\true{}}}^{i,j}
%  \\
%\end{array}
%$$
%\caption{isa? Vector Extensions}
%\end{figure*}

\begin{figure*}
  \getmethodfigure{}
\caption{Definition of get-method}
\end{figure*}


%\clearpage

\begin{figure*}
\begin{mathpar}

\BLocal{}

\BDo{}

\BLet{}

\BVal{}

\BIfTrue{}

\BIfFalse{}

\BAbs{}

\BBetaClosure{}

\BDelta{}

\BBetaMulti{}

\BField{}

\BMethod{}

\BNew{}

       \BDefMulti{}

       \BDefMethod{}

       \BIsA{}

       {\BAssoc}

       {\BGet}

       {\BGetMissing}
\end{mathpar}
\caption{Operational Semantics}
\label{appendix:figure:opsem}
\end{figure*}

\begin{figure*}
\begin{mathpar}

\infer [BS-MethodRefl]
{}
{\opsem {\openv{}} {\methodexp {mth} {\e{}} {\overrightarrow{\e{}}}}
        {\wrong{}}}

\infer [BS-FieldRefl]
{}
{\opsem {\openv{}} {\fieldexp {\fld{}} {\e{}}}
        {\wrong{}}}

\infer [BS-NewRefl]
{}
{\opsem {\openv{}} {\fieldexp {\fld{}} {\e{}}}
        {\wrong{}}}


\infer [BS-Beta]
{ \opsem {\openv{}}
         {\e{f}}
         {\v{}}
         \\\\
  {\v{}} \not= {\const{}}
  \\
  {\v{}} \not= {\multi {\v{d}} {\disptable{}}}
  \\\\
  {\v{}} \not= {\closure {\openv{c}} {\abs {\x{}} {\t{}} {\e{b}}}}
       }
{ \opsem {\openv{}}
         {\appexp {\e{f}} {\e{a}}}
         {\wrong{}}
       }

\infer [BS-BetaMulti]
{ \opsem {\openv{}}
         {\e{f}}
         {\multi {\v{}} {\disptable{}}}
         \\\\
  {\v{}} \not= {\const{}}
  \\
  {\v{}} \not= {\multi {\v{d}} {\disptable{}}}
  \\\\
  {\v{}} \not= {\closure {\openv{c}} {\abs {\x{}} {\t{}} {\e{b}}}}
       }
{ \opsem {\openv{}}
         {\appexp {\e{f}} {\e{a}}}
         {\wrong{}}
       }

\infer [BS-FieldTarget]
{ \opsem {\openv{}}
         {\e{}} 
       {\v{1}}
         \\\\
         {\v{}} \not= {\classvalue{\classhint{1}} {\overrightarrow {\classfieldpair{\fld{i}} {\v{i}}}}}
       }
{ \opsem {\openv{}}
         {\fieldstaticexp {\classhint{1}} {\classhint{2}} {\fld{}} {\e{}}}
         {\wrong{}}
   }

\infer [BS-FieldMissing]
{ \opsem {\openv{}}
         {\e{}} 
       {\classvalue{\classhint{1}} {\overrightarrow {\classfieldpair{\fld{i}} {\v{i}}}}}
       \\
       \fld{} \not\in \{\overrightarrow{\fld{i}}\}
       }
{ \opsem {\openv{}}
         {\fieldstaticexp {\classhint{1}} {\classhint{2}} {\fld{}} {\e{}}}
         {\wrong{}}
   }


\infer [BS-MethodTarget]
{ \opsem {\openv{}}
         {\e{m}}
         {\v{}}
  \\
         {\v{}} \not= {\classvalue{\classhint{1}} {\overrightarrow {\classfieldpair{\fld{i}} {\v{i}}}}}
}
{\opsem {\openv{}}
        {\methodstaticexp {\classhint{1}} {\overrightarrow{\classhint{a}}} {\classhint{2}} {mth} {\e{m}} {\overrightarrow{\e{a}}}}
        {\wrong{}}
      }

\infer [BS-MethodArity]
{ i \not= a
}
{\opsem {\openv{}}
        {\methodstaticexp {\classhint{1}} {\overrightarrow{\classhint{i}}} {\classhint{2}} {mth} {\e{m}} {\overrightarrow{\e{a}}}}
        {\wrong{}}
      }

\infer [BS-MethodArg]
{ \opsem {\openv{}}
         {\e{m}}
         {\v{m}}
  \\
  \overrightarrow{
  \opsem {\openv{}}
         {\e{a}}
         {\v{a}}
       }
       \\\\
  \exists a.\ 
    \v{a} \not=\ {\classvalue{\classhint{a}} {\overrightarrow {\classfieldpair{\fld{i}} {\v{i}}}}}\ or\ \v{a} \not= \nil{}
}
{\opsem {\openv{}}
        {\methodstaticexp {\classhint{1}} {\overrightarrow{\classhint{a}}} {\classhint{2}} {mth} {\e{m}} {\overrightarrow{\e{a}}}}
        {\wrong{}}
      }

\infer [BS-NewArg]
{ \overrightarrow{
  \opsem {\openv{}}
         {\e{i}}
         {\v{i}}
     }
       \\\\
  \exists i.\ 
    \v{i} \not=\ {\classvalue{\classhint{i}} {\overrightarrow {\classfieldpair{\fld{i}} {\v{i}}}}}\ or\ \v{i} \not= \nil{}
}
{\opsem {\openv{}}
        {\newstaticexp {\overrightarrow{\classhint{i}}} {\classhint{1}} 
                       {\class{}} {\overrightarrow{\e{i}}}}
        {\wrong{}}
      }

\infer [BS-NewArity]
{ i \not= a
}
{\opsem {\openv{}}
        {\newstaticexp {\overrightarrow{\classhint{i}}} {\classhint{1}} 
                       {\class{}} {\overrightarrow{\e{a}}}}
        {\wrong{}}
      }

\infer [BS-AssocMap]
{\opsem {\openv{}}
        {\e{m}} {\v{}}
        \\
        \v{} \not= {\curlymap{\overrightarrow{({\v{a}}\ {\v{b}})}}}
}
{
 \opsem {\openv{}}
        {\assocexp {\e{m}} {\e{k}} {\e{v}}} 
        {\wrong{}}
                }

\infer [BS-AssocKey]
{\opsem {\openv{}}
        {\e{m}} {\curlymap{\overrightarrow{({\v{a}}\ {\v{b}})}}}
        \\
 \opsem {\openv{}} {\e{k}} {\v{k}}
 \\\\
 {\v{k}} \not= \k{}
}
{
 \opsem {\openv{}}
        {\assocexp {\e{m}} {\e{k}} {\e{v}}} 
        {\wrong{}}
                }

\infer [BS-GetMap]
{ \opsem {\openv{}}
         {\e{m}} {\v{}}
        \\
        \v{} \not= {\curlymap{\overrightarrow{({\v{a}}\ {\v{b}})}}}
}
{\opsem {\openv{}}
        {\getexp {\e{m}} {\e{k}}}
        {\wrong{}}
}

\infer [BS-GetKey]
{ \opsem {\openv{}}
         {\e{m}} {\v{}}
        \\
 \opsem {\openv{}}
        {\e{k}} {\v{k}}
        \\\\
      \v{} \not= {\k{}}
}
{\opsem {\openv{}}
        {\getexp {\e{m}} {\e{k}}}
        {\wrong{}}
}

\infer [BS-Local]
{ \notinopenv {\openv{}} {\x{}}}
{ \opsem {\openv{}} {\x{}} {\wrong{}} }

\infer [BS-DefMethod]
{ \opsem {\openv{}}
         {\e{m}}
         {\v{m}}
         \\
         \v{m} \not= {\multi {\v{d}} {\disptable{}}}
}
{\opsem {\openv{}}
        {\extendmultiexp {\e{m}} {\e{v}} {\e{f}}}
        {\wrong{}}
      }

\end{mathpar}
\caption{Stuck programs}
\end{figure*}

\begin{figure*}
\begin{mathpar}
\infer [BE-ErrorWrong]
{}
{ \opsem {\openv{}} 
         {\wrongorerror{}}
         {\wrongorerror{}}}

\infer [BE-Let]
{ \opsem {\openv{}} {\e{a}} {\wrongorerror{}}
 }
{ \opsem {\openv{}} 
         {\letexp {\x{}} {\e{a}} {\e{}}}
       {\wrongorerror{}}}

\infer [BE-Do1]
{ \opsem {\openv{}} {\e{1}} {\wrongorerror{}} }
{ \opsem {\openv{}} {\doexp{\e{1}}{\e{}}} {\wrongorerror{}}}

\infer [BE-Do2]
{ \opsem {\openv{}} {\e{1}} {\v{1}} 
  \\\\
  \opsem {\openv{}} {\e{}}  {\wrongorerror{}}
}
{ \opsem {\openv{}} {\doexp{\e{1}}{\e{}}} {\wrongorerror{}} }

\infer [BE-If]
{  \opsem {\openv{}} {\e{1}} {\wrongorerror{}}
}
{ \opsem {\openv{}}
         {\ifexp {\e1} {\e2} {\e3}}
         {\wrongorerror{}}
       }

\infer [BE-IfTrue]
{ \opsem {\openv{}} {\e{1}} {\v{1}}
  \\\\
  {\v{1}} \not= {\false{}}
  \\
  {\v{1}} \not= {\nil{}}
  \\\\
  \opsem {\openv{}} {\e{2}} {\wrongorerror{}}
}
{ \opsem {\openv{}}
         {\ifexp {\e1} {\e2} {\e3}}
         {\wrongorerror{}}
       }

\infer [BE-IfFalse]
{  \opsem {\openv{}} {\e{1}} {\false{}}
  \ \ \text{or}\ \ 
  \opsem {\openv{}} {\e{1}} {\nil{}}
  \\\\
  \opsem {\openv{}} {\e{3}} {\wrongorerror{}}
}
{ \opsem {\openv{}}
         {\ifexp {\e1} {\e2} {\e3}}
         {\wrongorerror{}}
       }

\infer [BE-Beta1]
{ \opsem {\openv{}}
         {\e{f}}
         {\wrongorerror{}}
       }
{ \opsem {\openv{}}
         {\appexp {\e{f}} {\e{a}}}
         {\wrongorerror{}}
       }

\infer [BE-Beta2]
{ \opsem {\openv{}}
         {\e{f}}
         {\v{f}}
         \\\\
  \opsem {\openv{}}
         {\e{a}}
         {\wrongorerror{}}
       }
{ \opsem {\openv{}}
         {\appexp {\e{f}} {\e{a}}}
         {\wrongorerror{}}
       }

\infer [BE-BetaClosure]
{ \opsem {\openv{}}
         {\e{f}}
         {\closure {\openv{c}} {\abs {\x{}} {\t{}} {\e{b}}}}
         \\\\
  \opsem {\openv{}}
         {\e{a}}
         {\v{a}}
         \\\\
  \opsem {\extendopenv {\openv{c}} {\x{}} {\v{a}}}
         {\e{b}}
         {\wrongorerror{}}
       }
{ \opsem {\openv{}}
         {\appexp {\e{f}} {\e{a}}}
         {\wrongorerror{}}
       }

\infer [BE-BetaMulti1]
{ \opsem {\openv{}}
         {\e{f}}
         {\multi {\v{d}} {m}}
         \\\\
  \opsem {\openv{}}
         {\e{a}}
         {\v{a}}
         \\\\
  \opsem {\openv{}}
         {\appexp {\v{d}} {\v{a}}}
         {\wrongorerror{}}
       }
{ \opsem {\openv{}}
         {\appexp {\e{f}} {\e{a}}}
         {\wrongorerror{}}
       }

\infer [BE-BetaMulti2]
{ \opsem {\openv{}}
         {\e{f}}
         {\multi {\v{d}} {m}}
         \\\\
  \opsem {\openv{}}
         {\e{a}}
         {\v{a}}
         \\\\
  \opsem {\openv{}}
         {\appexp {\v{d}} {\v{a}}}
         {\v{e}}
         \\\\
  \getmethoderr {\disptable{}}
             {\v{e}}
             {\errorvalv{}}
       }
{ \opsem {\openv{}}
         {\appexp {\e{f}} {\e{a}}}
         {\errorvalv{}}
       }

\infer [BE-Delta]
{ \opsem {\openv{}} {\e{}} {\const{}}
  \\\\
  \opsem {\openv{}} {\ep{}} {\v{}}
  \\\\
  \constantopsem{\const{}}{\v{}} = \wrongorerror{}
}
{ \opsem {\openv{}}
         {\appexp {\e{}} {\ep{}}}
         {\wrongorerror{}}
       }

\infer [BE-Field]
{ \opsem {\openv{}}
         {\e{}} 
         {\wrongorerror{}}
       }
{ \opsem {\openv{}}
         {\fieldstaticexp {\classhint{1}} {\classhint{2}} {\fld{}} {\e{}}}
         {\wrongorerror{}}
   }

\infer [BE-Method1]
{ \opsem {\openv{}}
         {\e{m}}
         {\wrongorerror{}}
}
{\opsem {\openv{}}
        {\methodstaticexp {\classhint{1}} {\overrightarrow{\classhint{a}}} {\classhint{2}} {mth} {\e{m}} {\overrightarrow{\e{}}}}
        {\wrongorerror{}}
      }

\infer [BE-Method2]
{ \opsem {\openv{}}
         {\e{m}}
         {\v{m}}
  \\\\
  \overrightarrow{
  \opsem {\openv{}}
         {\e{n-1}}
         {\v{n-1}}
       }
         \\\\
  \opsem {\openv{}}
         {\e{n}}
         {\wrongorerror{}}
}
{\opsem {\openv{}}
        {\methodstaticexp {\classhint{1}} {\overrightarrow{\classhint{a}}} {\classhint{2}} {mth} {\e{m}} {\overrightarrow{\e{}}}}
        {\wrongorerror{}}
      }

\infer [BE-Method3]
{ \opsem {\openv{}}
         {\e{m}}
         {\v{m}}
  \\
  \overrightarrow{
  \opsem {\openv{}}
         {\e{a}}
         {\v{a}}
       }
  \\\\
  \invokejavamethod {\classhint{1}} {\v{m}} {mth}
                    {\overrightarrow{\classhint{a}}} {\overrightarrow{\v{a}}}
                    {\classhint{2}}
                    {\errorvalv{}}
}
{\opsem {\openv{}}
        {\methodstaticexp {\classhint{1}} {\overrightarrow{\classhint{a}}} {\classhint{2}} {mth} {\e{m}} {\overrightarrow{\e{a}}}}
        {\errorvalv{}}
      }

\infer [BE-New1]
{ \overrightarrow{
  \opsem {\openv{}}
         {\e{n-1}}
         {\v{n-1}}
       }
       \\\\
  \opsem {\openv{}}
         {\e{n}}
         {\wrongorerror{}}
       }
{ \opsem {\openv{}}
         {\newstaticexp {\overrightarrow{\classhint{i}}} {\classhint{1}} 
                        {\class{}} {\overrightarrow{\e{}}}}
         {\wrongorerror{}}
       }

\infer [BE-New2]
{ 
  \overrightarrow{
  \opsem {\openv{}}
         {\e{i}}
         {\v{i}}
       }
         \\\\
         \newjava {\classhint{1}}
                  {\overrightarrow{\classhint{i}}}
                  {\overrightarrow{\v{i}}}
                  {\errorvalv{}}
       }
{ \opsem {\openv{}}
         {\newstaticexp {\overrightarrow{\classhint{i}}} {\classhint{1}} 
                        {\class{}} {\overrightarrow{\e{i}}}}
         {\errorvalv{}}}

\infer [BE-DefMulti]
{ \opsem {\openv{}} {\e{d}} {\wrongorerror{}}
}
{\opsem {\openv{}}
        {\createmultiexp {\t{}}
                         {\e{d}}}
        {\wrongorerror{}}
}

\infer [BE-DefMethod1]
{ \opsem {\openv{}}
         {\e{m}}
         {\wrongorerror{}}
}
{\opsem {\openv{}}
        {\extendmultiexp {\e{m}} {\e{v}} {\e{f}}}
        {\wrongorerror{}}
      }

\infer [BE-DefMethod2]
{ \opsem {\openv{}}
         {\e{m}}
         {\multi {\v{d}} {\disptable{}}}
         \\\\
  \opsem {\openv{}}
         {\e{v}}
         {\wrongorerror{}}
}
{\opsem {\openv{}}
        {\extendmultiexp {\e{m}} {\e{v}} {\e{f}}}
        {\wrongorerror{}}
      }

\infer [BE-DefMethod3]
{ \opsem {\openv{}}
         {\e{m}}
         {\multi {\v{d}} {\disptable{}}}
         \\\\
  \opsem {\openv{}}
         {\e{v}}
         {\v{v}}
         \\\\
  \opsem {\openv{}}
         {\e{f}}
         {\wrongorerror{}}
}
{\opsem {\openv{}}
        {\extendmultiexp {\e{m}} {\e{v}} {\e{f}}}
         {\wrongorerror{}}
      }

\infer [BE-IsA1]
{ \opsem {\openv{}} {\e{1}} {\wrongorerror{}}
}
{\opsem {\openv{}} {\isaapp {\e{1}} {\e{2}}} {\wrongorerror{}}}

\infer [BE-IsA2]
{ \opsem {\openv{}} {\e{1}} {\v{1}}
  \\\\
  \opsem {\openv{}} {\e{2}} {\wrongorerror{}}
}
{\opsem {\openv{}} {\isaapp {\e{1}} {\e{2}}} {\wrongorerror{}}}

\infer [BE-Assoc1]
{\opsem {\openv{}}
        {\e{m}}{\wrongorerror{}} 
}
{
 \opsem {\openv{}}
        {\assocexp {\e{m}} {\e{k}} {\e{v}}} 
        {\wrongorerror{}}
                }

\infer [BE-Assoc2]
{\opsem {\openv{}}
        {\e{m}} {\curlymap{\overrightarrow{({\v{a}}\ {\v{b}})}}}
        \\
 \opsem {\openv{}}
        {\e{k}}{\wrongorerror{}}
}
{
 \opsem {\openv{}}
        {\assocexp {\e{m}} {\e{k}} {\e{v}}} 
        {\wrongorerror{}}
                }

\infer [BE-Assoc3]
{\opsem {\openv{}}
        {\e{m}} {\curlymap{\overrightarrow{({\v{a}}\ {\v{b}})}}}
        \\
 \opsem {\openv{}}
        {\e{k}} {\v{k}}
        \\
 \opsem {\openv{}}
        {\e{v}} {\wrongorerror{}}
}
{
 \opsem {\openv{}}
        {\assocexp {\e{m}} {\e{k}} {\e{v}}} 
        {\wrongorerror{}}
                }

\infer [BE-Get1]
{\opsem {\openv{}}
        {\e{m}} {\wrongorerror{}}
}
{
 \opsem {\openv{}}
        {\getexp {\e{m}} {\e{k}}}
        {\wrongorerror{}}
}

\infer [BE-Get2]
{\opsem {\openv{}}
        {\e{m}} {\curlymap{\overrightarrow{({\v{a}}\ {\v{b}})}}}
        \\
 \opsem {\openv{}}
        {\e{k}} {\wrongorerror{}}
}
{
 \opsem {\openv{}}
        {\getexp {\e{m}} {\e{k}}}
        {\wrongorerror{}}
}
\end{mathpar}
\caption{Error and stuck propagation}
\label{appendix:figure:errorstuck}
\end{figure*}





\begin{figure*}
\begin{mathpar}

\begin{array}{lllll}
  \inopenvalign{\openv{}}{\x{}}{\v{} & {\roundpair{\x{}}{\v{}}} \in \openv{}}\\
  \inopenvalign{\openv{}}{\path {\keype{k}} {\object{}}}{\getexp {{\openv{}}(\object{})}{\k{}}}\\
  \inopenvalign{\openv{}}{\path {\classpe{}} {\object{}}}{\appexp {\classconst{}} {{\openv{}}(\object{})}}

\end{array}

\end{mathpar}
\caption{Path translation}
\end{figure*}

\begin{figure*}
\begin{mathpar}

\begin{array}{lllll}
\updatefigure{}
\end{array}

\begin{array}{lllll}
\restrictremovefigure{}
\end{array}

\end{mathpar}
\caption{Type Update}
\label{appendix:updaterestrictremove}
\end{figure*}


\begin{figure*}
\begin{mathpar}
\infer [M-Or]
{ \satisfies{\openv{}}{\prop{1}}\ \text{or}\  \satisfies{\openv{}}{\prop{2}}}
{ \satisfies{\openv{}}{\orprop{\prop{1}}{\prop{2}}}
                   }

\infer [M-Imp]
{ \satisfies{\openv{}}{\prop{1}}\ \text{implies}\ \satisfies{\openv{}}{\prop{2}}}
{ \satisfies{\openv{}}{\impprop{\prop{1}}{\prop{2}}}
                   }

\infer [M-And]
{ \satisfies{\openv{}}{\prop{1}}
\\ \satisfies{\openv{}}{\prop{2}}}
{ \satisfies{\openv{}}{\andprop{\prop{1}}{\prop{2}}}
                   }


\infer [M-Top]
{}
{ \satisfies{\openv{}}{\topprop{}}
                   }

                   \\

\infer [M-Type]
{ \judgement {} {\openv{}({\path{\pathelem{}}{\x{}}})} {\t{}}{\filterset{\thenprop{\prop{}}}{\elseprop{\prop{}}}}{\object{}}}
{ \satisfies{\openv{}}{\isprop{\t{}}{\path{\pathelem{}}{\x{}}}}
                   }

\infer [M-NotType]
{ \judgement {} {\openv{}({\path{\pathelem{}}{\x{}}})} {\s{}}{\filterset{\thenprop{\prop{}}}{\elseprop{\prop{}}}}{\object{}}
\\\\
\text{there is no}\ \v{}\ \text{such that}\ \judgement{}{\v{}}{\t{}}{\filterset{\thenprop{\prop{1}}}{\elseprop{\prop{1}}}}{\object{1}}
\ \text{and}\ \judgement{}{\v{}}{\s{}}{\filterset{\thenprop{\prop{2}}}{\elseprop{\prop{2}}}}{\object{2}}
}
{ \satisfies{\openv{}}{\notprop{\t{}}{\path{\pathelem{}}{\x{}}}}
                   }
\end{mathpar}
\caption{Satisfaction Relation}
\end{figure*}

\input{esop-proof-system}
\begin{figure*}
$$
\begin{array}{lclr}

{\withpolarity
  {\replacefor
    {\filterset {\thenprop {\prop{}}}{\elseprop {\prop{}}}}
    {\object{}}
    {\x{}}}
  {\polaritymeta{}}}
  &=&
{\filterset 
  {\withpolarity
    {\replacefor
      {\thenprop {\prop{}}}
      {\object{}}
      {\x{}}}
    {\polaritymeta{}}}
  {\withpolarity
    {\replacefor
      {\elseprop {\prop{}}}
      {\object{}}
      {\x{}}}
    {\polaritymeta{}}}}
\\\\
{\withpolarity
  {\replacefor
    {\isprop {\propisnotmeta{}} {\path {\pathelem{}} {\x{}}}}
    {\path {\pathelemp{}} {\y{}}}
    {\x{}}}
  {\polaritymeta{}}}
&=&
  {\isprop {({\withpolarity
              {\replacefor
               {\propisnotmeta{}}
               {\path {\pathelemp{}} {\y{}}}
               {\x{}}}
              {\polaritymeta{}}})}
           {{\pathelem{}}({\path {\pathelemp{}} {\y{}}})}}
           \\

{\pluspolarity
{\replacefor
  {\isprop {\propisnotmeta{}} {\path {\pathelem{}} {\x{}}}}
  {\emptyobject{}}
  {\x{}}}
}
&=&
{\topprop{}}
\\
{\minuspolarity
{\replacefor
  {\isprop {\propisnotmeta{}} {\path {\pathelem{}} {\x{}}}}
  {\emptyobject{}}
  {\x{}}}
}
&=&
{\botprop{}}

\\
{\withpolarity
{\replacefor
  {\isprop {\propisnotmeta{}} {\path {\pathelem{}} {\x{}}}}
  {\object{}}
  {\z{}}}
{\polaritymeta{}}}
&=&
  {\isprop {\propisnotmeta{}} {\path {\pathelem{}} {\x{}}}}
  & \x{} \not= \z{}\ \text{and}\ \z{} \not\in {\fv {\propisnotmeta{}}}

\\
{\pluspolarity
{\replacefor
  {\isprop {\propisnotmeta{}} {\path {\pathelem{}} {\x{}}}}
  {\object{}}
  {\z{}}}
}
&=&
{\topprop{}}
  & \x{} \not= \z{}\ \text{and}\ \z{} \in {\fv {\propisnotmeta{}}}
\\
{\minuspolarity
{\replacefor
  {\isprop {\propisnotmeta{}} {\path {\pathelem{}} {\x{}}}}
  {\object{}}
  {\z{}}}
}
&=&
{\botprop{}}
  & \x{} \not= \z{}\ \text{and}\ \z{} \in {\fv {\propisnotmeta{}}}

\\
{\withpolarity
{\replacefor
  {\topprop{}}
  {\object{}}
  {\x{}}}
{\polaritymeta{}}}
&=&
  {\topprop{}}

\\
{\withpolarity
{\replacefor
  {\botprop{}}
  {\object{}}
  {\x{}}}
{\polaritymeta{}}}
&=&
  {\botprop{}}

\\
{\pluspolarity
{\replacefor
  {({\impprop {\prop{1}} {\prop{2}}})}
  {\object{}}
  {\x{}}}
}
&=&
{\impprop 
  {\minuspolarity {\replacefor {\prop{1}} {\object{}} {\x{}}}}
  {\pluspolarity {\replacefor {\prop{2}} {\object{}} {\x{}}}}}
\\
{\minuspolarity
{\replacefor
  {({\impprop {\prop{1}} {\prop{2}}})}
  {\object{}}
  {\x{}}}
}
&=&
{\impprop 
  {\pluspolarity {\replacefor {\prop{1}} {\object{}} {\x{}}}}
  {\minuspolarity {\replacefor {\prop{2}} {\object{}} {\x{}}}}}
\\
{\withpolarity
{\replacefor
  {({\orprop {\prop{1}} {\prop{2}}})}
  {\object{}}
  {\x{}}}
{\polaritymeta{}}}
&=&
{\orprop 
  {\withpolarity
    {\replacefor {\prop{1}} {\object{}} {\x{}}}
    {\polaritymeta{}}}
  {\withpolarity
    {\replacefor {\prop{2}} {\object{}} {\x{}}}
    {\polaritymeta{}}}}
\\
{\withpolarity
{\replacefor
  {({\andprop {\prop{1}} {\prop{2}}})}
  {\object{}}
  {\x{}}}
{\polaritymeta{}}}
&=&
{\andprop 
{\withpolarity
  {\replacefor {\prop{1}} {\object{}} {\x{}}}
  {\polaritymeta{}}}
{\withpolarity
  {\replacefor {\prop{2}} {\object{}} {\x{}}}
  {\polaritymeta{}}}}

    \\\\

{\withpolarity
{\replacefor
  {\path {\pathelem{}} {\x{}}}
  {\path {\pathelemp{}} {\y{}}}
  {\x{}}}
{\polaritymeta{}}}
           &=&
{\path{\pathelem{}}{\path {\pathelemp{}} {\y{}}}}

    \\

{\withpolarity
{\replacefor
  {\path {\pathelem{}} {\x{}}}
  {\emptyobject{}}
  {\x{}}}
{\polaritymeta{}}}
           &=&
{\emptyobject{}}

    \\

{\withpolarity
{\replacefor
  {\path {\pathelem{}} {\x{}}}
  {\object{}}
  {\z{}}}
{\polaritymeta{}}}
           &=&
{\path {\pathelem{}} {\x{}}}

& \x{} \not= \z{}
    \\

{\withpolarity
{\replacefor
  {\emptyobject{}}
  {\object{}}
  {\x{}}}
{\polaritymeta{}}}
           &=&
{\emptyobject{}}

\end{array}
$$
\center{\text{Substitution on types is capture-avoiding structural recursion.}}
\caption{Substitution}
\end{figure*}


\section{Soundness for Typed Clojure}

{\javaassumptionsall{appendix}}

\begin{lemma} \label{appendix:lemma:envagree}
  If \openv{} and \openvp{} agree on \fv{\prop{}}
  and \satisfies{\openv{}}{\prop{}}
  then \satisfies{\openvp{}}{\prop{}}.
\begin{proof}
  Since the relevant parts of \openv{} and \openvp{} agree, the proof follows trivially.
\end{proof}
\end{lemma}

\begin{lemma} \label{appendix:lemma:substfilter}
  If 
  \begin{itemize}
    \item \prop{1} = {\replacefor {\prop{2}} {\object{}} {\x{}}},
    \item
  {\satisfies{\openv{2}}{\prop{2}}},
    \item
  $\forall v \in \fv{\prop{2}} - \x{}$.
                              {\inopenvnoeq{\openv{1}}{v}} = {\inopenvnoeq {\openv{2}}{v}},
    \item
  and {\inopenvnoeq{\openv{2}}{\x{}}} = {\inopenvnoeq{\openv{1}}{\object{}}}
  \end{itemize}
  then \satisfies{\openv{1}}{\prop{1}}.

  \begin{proof}
    By induction on the derivation of the model judgement.
  \end{proof}
\end{lemma}

\begin{lemma} \label{appendix:lemma:satisfies}
  If \satisfies{\openv{}}{\propenv{}} and \inpropenv{\propenv{}}{\prop{}} then \satisfies{\openv{}}{\prop{}}.

  \begin{proof}
    By structural induction on \inpropenv{\propenv{}}{\prop{}}.
%    \begin{itemize}
%      \item[]
%        \begin{case}[L-True]
%
%          Holds by M-Top.
%        \end{case}
%      \item[]
%        \begin{case}[L-False]
%          {\inpropenv{\propenv{}}{\botprop{}}}
%
%          ??? TODO
%        \end{case}
%      \item[]
%        \begin{case}[L-AndI]
%          \inpropenv{\propenv{}}{\andprop{\prop{1}}{\prop{2}}}, \satisfies{\openv{}}{\propenv{}}
%
%          By inversion on the proof system we know \inpropenv{\propenv{}}{\prop{1}}
%          and
%          \inpropenv{\propenv{}}{\prop{2}}.
%
%          By the induction hypothesis we know \satisfies{\openv{}}{\prop{1}}
%          and
%          \satisfies{\openv{}}{\prop{2}}.
%
%          By M-And we know \satisfies{\openv{}}{\andprop{\prop{1}}{\prop{2}}}
%          and we are done.
%        \end{case}
%      \item[]
%        \begin{case}[L-AndE]
%          \inpropenv{\propenv{},{\andprop{\prop{1}}{\prop{2}}}}{\prop{}}, \satisfies{\openv{}}{\propenv{},{\andprop{\prop{1}}{\prop{2}}}}
%
%
%          By inversion on the proof system we know  either
%          \inpropenv{\propenv{},{\prop{1}}}{\prop{}}
%          or
%          \inpropenv{\propenv{},{\prop{2}}}{\prop{}}.
%
%          %TODO
%         % By the induction hypothesis we know 
%         % either
%         % \satisfies{\openv{}}{\prop{1}}
%         % and
%         % \satisfies{\openv{}}{\prop{2}}.
%        \end{case}
%    \end{itemize}
  \end{proof}
\end{lemma}

\begin{lemma} \label{appendix:lemma:goodobjects+ve}
  If \inpropenv{\propenv{}}{\isprop{\t{}}{\path{\pathelem{}}{\x{}}}},
  \satisfies{\openv{}}{\propenv{}}
  and \inopenv{\openv{}}{\path{\pathelem{}}{\x{}}}{\v{}}
  then
  \judgementselfrewrite{}{\v{}}{\t{}}{\filterset{\thenprop{\propp{}}}{\elseprop{\propp{}}}}{\objectp{}}
  for some {\thenprop{\propp{}}}, {\elseprop{\propp{}}} and {\objectp{}}.
  \begin{proof}
    Corollary of lemma~\ref{appendix:lemma:satisfies}.
  \end{proof}
\end{lemma}

\begin{lemma}[Paths are independent] \label{appendix:lemma:pathindependent}
  If \inopenvnoeq{\openv{}}{\object{}} = \inopenvnoeq{\openv{1}}{\objectp{}}
  then \inopenvnoeq{\openv{}}{\path{\pathelem{}}{\object{}}} =
       \inopenvnoeq{\openv{1}}{\path{\pathelem{}}{\objectp{}}}
 \begin{proof}
   By induction on \pathelem{}.
   % FIXME
%   \begin{case}[\pathelem{} = \emptypath{}]
%     \inopenvnoeq{\openv{}}{\object{}} = {\inopenvnoeq{\openv{}}{\objectp{}}}
%
%     As 
%     \inopenvnoeq{\openv{}}{\path{\emptypath{}}{\object{}}} = \inopenvnoeq{\openv{}}{\object{}}
%     and
%     \inopenvnoeq{\openv{}}{\path{\emptypath{}}{\objectp{}}} = \inopenvnoeq{\openv{}}{\objectp{}}
%     we can conclude 
%     \inopenvnoeq{\openv{}}{\path{\emptypath{}}{\object{}}} = \inopenvnoeq{\openv{}}{\path{\emptypath{}}{\objectp{}}}.
%   \end{case}
%   \begin{case}[\pathelem{} = {\destructpath{\pesyntax{}}{\pathelem{1}}}]
%     \inopenvnoeq{\openv{}}{\object{}} = {\inopenvnoeq{\openv{}}{\objectp{}}}
%
%     By cases on \pesyntax{}.
%
%     \begin{itemize}
%       \item[]
%   \begin{subcase}[\pesyntax{} = {\keype{\k{}}}] 
%
%%     TODO
%     By the induction hypothesis on {\pathelem{1}}
%     we know {\inopenvnoeq{\openv{}}{\path{\pathelem{1}}{\object{}}}} =
%             {\inopenvnoeq{\openv{1}}{\path{\pathelem{1}}{\objectp{}}}}.
%             By the definition of path translation 
%             {\inopenvnoeq{\openv{}}{\path{\pathelem{1}}{\object{}}}} = {\getexp {{\openv{}}(\object{})}{\k{}}}
%             and 
%             {\inopenvnoeq{\openv{}}{\path{\pathelem{1}}{\objectp{}}}} = {\getexp {{\openv{}}(\objectp{})}{\k{}}}
%   \end{subcase} 
%     \end{itemize}
%%     TODO
%   \end{case}
 \end{proof}
\end{lemma}

\begin{lemma}[\classconst]\label{appendix:lemma:classconst}
  If
  {\opsem{\openv{}}{\appexp{\classconst{}}{\openv{}({\path{\pathelem{}}{\x{}}})}}{\class{}}} then
  {\satisfies{\openv{}}{\isprop{\class{}}{\path{\pathelem{}}{\x{}}}}}.

  \begin{proof}
    Induction on the definition of {\classconst{}}.
  \end{proof}
\end{lemma}

{\consistentwithdefinition{appendix}}

{\istruefalsedefinitions{appendix}}

%\begin{lemma}[Path substitution] \label{appendix:lemma:pathsubustitution}
%  If \satisfies{\openv{}}{\prop{}} and 
%  \openv(\object{}) = \openv(\objectp{})
%  then \satisfies{\openv{}}{\replacefor{\prop{}}{\object{}}{\objectp{}}}.
%  \begin{proof}
%    By straightforward induction on \prop{}.
%  \end{proof}
%\end{lemma}
%

\begin{lemma}[isa? has correct propositions] \label{appendix:lemma:isa}
  If
  \begin{itemize}
    \item
  \judgementrewrite {\propenv{}} {\v{1}} {\t{1}}
             {\filterset {\thenprop {\prop{1}}}
                         {\elseprop {\prop{1}}}}
                       {\object{1}}
                       {\v{1}},
    \item
  \judgementrewrite {\propenv{}} {\v{2}} {\t{2}}
             {\filterset {\thenprop {\prop{2}}}
                         {\elseprop {\prop{2}}}}
                       {\object{2}}
                       {\v{2}},
    \item
        \isaopsem{\v{1}}{\v{2}} = {\v{}}, 
    \item
        \satisfies{\openv{}}{\propenv{}},
    \item
  \isacompare{\t{1}}{\object{1}}{\t{2}}{\filterset {\thenprop {\propp{}}} {\elseprop {\propp{}}}},
    \item
        \inpropenv{\thenprop{\propp{}}}{\thenprop{\prop{}}}, and
    \item
        \inpropenv{\elseprop{\propp{}}}{\elseprop{\prop{}}},
    \end{itemize}
  then either
\begin{itemize}
  \item
        if
        \istrueval{\v{}}
        then {\satisfies{\openv{}}{\thenprop{\prop{}}}}, or
  \item
        if
        \isfalseval{\v{}}
        then {\satisfies{\openv{}}{\elseprop{\prop{}}}}.
\end{itemize}
\begin{proof}
        By cases on the definition of \isaopsemliteral
        and subcases on \isaopsemliteral.

        \begin{itemize} % isaopsem
          \item[]
            \begin{subcase}[\isaopsem{\v{1}}{\v{1}} = {\true{}}, \text{if} \v{1} \notequal\ {\class{}}]
              \ 

              \v{1} = \v{2}, \v{1} \notequal\ {\class{}}, \v{2} \notequal\ {\class{}}, \istrueval{\v{}}
              
              Since \istrueval{\v{}} we prove {\satisfies{\openv{}}{\thenprop{\prop{}}}}
              by cases on the definition of \isacompareliteral{}:
              \begin{itemize} % isacompare
                \item[]
                  \begin{subcase}[\isacompare{\s{}}{\path{\classpe{}}{\path{\pathelem{}}{\x{}}}}{\Value{\class{}}}
                                 {\filterset{\isprop{\class{}} {\path{\pathelem{}}{\x{}}}}
                                            {\notprop{\class{}}{\path{\pathelem{}}{\x{}}}}}]
                    \ 


                    \object{1} = {\path{\classpe{}}{\path{\pathelem{}}{\x{}}}},
                    \t{2} = {\Value{\class{}}},
                    \inpropenv{\isprop{\class{}} {\path{\pathelem{}}{\x{}}}}{\thenprop{\prop{}}}

                    Unreachable by inversion on the typing relation, since \t{2} = {\Value{\class{}}},
                    yet \v{2} \notequal\ {\class{}}.

%                    By inversion on the typing relation, since \classpe{} is the last path element of \object{1}
%                    then \opsem{\openv{}}{\appexp{\classconst{}}{\openv{}({\path{\pathelem{}}{\x{}}})}}{\v{1}}.
%
%                    Since {\v{1}} = {\v{2}} then {\t{1}} = {\t{2}}, and because {\t{2}} = {\Value{\class{}}}
%                    then {\t{1}} = {\Value{\class{}}}.
%
%                    By inversion {\v{1}} = {\class{}}, via T-Class.
%
%                    Since {\opsem{\openv{}}{\appexp{\classconst{}}{\openv{}({\path{\pathelem{}}{\x{}}})}}{\class{}}}
%                    we conclude by lemma~\ref{appendix:lemma:classconst}
%                    with {\satisfies{\openv{}}{\isprop{\class{}} {\path{\pathelem{}}{\x{}}}}}.

                  \end{subcase}
                \item[]
                  \begin{subcase}[\isacompare{\s{}}{\object{}}{\Value{\singletonmeta{}}}
                    {\replacefor
                      {\filtersetparen{\isprop{\Value{\singletonmeta{}}} {\x{}}}
                        {\notprop{\Value{\singletonmeta{}}}{\x{}}}}
                      {\object{}}
                      {\x{}}}\ 
                    \text{if}\ {\singletonmeta{}} \notequal \class{}]
                    \ 

                    \t{2} = {\Value{\singletonmeta{}}}, 
                    {\singletonmeta{}} \notequal \class{},
                    \inpropenv{\replacefor{\isprop{\Value{\singletonmeta{}}} {\x{}}}
                                           {\object{1}}
                                           {\x{}}}{\thenprop{\prop{}}}
                    %\elseprop{\prop{}} = {\replacefor{\notprop{\Value{\singletonmeta{}}} {\x{}}}
                    %                       {\object{1}}
                    %                       {\x{}}}

                    Since \t{2} = {\Value{\singletonmeta{}}} where {\singletonmeta{}} \notequal \class{},
                    by inversion on the typing judgement 
                    {\v{2}} is either \true{}, \false{}, \nil{} or \k{}
                    by T-True, T-False, T-Nil or T-Kw.

                    Since \v{1} = {\v{2}} then \t{1} = \t{2}, and since \t{2} = {\Value{\singletonmeta{}}}
                      then \t{1} = {\Value{\singletonmeta{}}}, so
                    \judgementtwo {} {\v{1}} {\Value{\singletonmeta{}}}

                    If \object{1} = \emptyobject{} then \thenprop{\prop{}} = \topprop{} and
                    we derive
                    {\satisfies{\openv{}}{\topprop{}}} with M-Top.

                    Otherwise \object{1} = \path{\pathelem{}}{\x{}} and 
                    \inpropenv{\isprop{\Value{\singletonmeta{}}}{\path{\pathelem{}}{\x{}}}}{\thenprop{\prop{}}},
                    and since
                    \judgementtwo {} {\v{1}} {\Value{\singletonmeta{}}}
                    then
                    \judgementtwo {} {{\openv{}}(\path{\pathelem{}}{\x{}})} {\Value{\singletonmeta{}}},
                    which we can use M-Type to derive
                    {\satisfies{\openv{}}{\isprop{\Value{\singletonmeta{}}}{\path{\pathelem{}}{\x{}}}}}.
                  \end{subcase}
                \item[]
                  \begin{subcase}[\isacompare{\s{}}{\object{}}{\t{}} {\filterset{\topprop{}} {\topprop{}}}]
                    \ 

                    {\thenprop{\prop{}}} = {\topprop{}}

                    {\satisfies{\openv{}}{\topprop{}}} holds by M-Top.

                  \end{subcase}
              \end{itemize}
            \end{subcase}
          \item[]
            \begin{subcase}[\isaopsem{\class{1}}{\class{2}} = {\true{}}, \text{if}\ \issubtypein{}{\class{1}}{\class{2}}]
              \ 

              \v{1} = \class{1}, \v{2} = \class{2},
              \issubtypein{}{\class{1}}{\class{2}},
              \istrueval{\v{}}
              
              Since \istrueval{\v{}} we prove {\satisfies{\openv{}}{\thenprop{\prop{}}}}
              by cases on the definition of \isacompareliteral{}:
              \begin{itemize} % isacompare
                \item[]
                  \begin{subcase}[\isacompare{\s{}}{\path{\classpe{}}{\path{\pathelem{}}{\x{}}}}{\Value{\class{}}}
                                 {\filterset{\isprop{\class{}} {\path{\pathelem{}}{\x{}}}}
                                            {\notprop{\class{}}{\path{\pathelem{}}{\x{}}}}}]
                    \ 


                    \object{1} = {\path{\classpe{}}{\path{\pathelem{}}{\x{}}}},
                    \t{2} = {\Value{\class{2}}},
                    \inpropenv{\isprop{\class{2}} {\path{\pathelem{}}{\x{}}}}{\thenprop{\prop{}}}

                    By inversion on the typing relation, since \classpe{} is the last path element of \object{1}
                    then \opsem{\openv{}}{\appexp{\classconst{}}{\openv{}({\path{\pathelem{}}{\x{}}})}}{\v{1}}.

                    Since {\opsem{\openv{}}{\appexp{\classconst{}}{\openv{}({\path{\pathelem{}}{\x{}}})}}{\class{1}}},
                    as {\v{1}} = {\class{1}},
                    we can derive from lemma~\ref{appendix:lemma:classconst}
                    {\satisfies{\openv{}}{\isprop{\class{1}} {\path{\pathelem{}}{\x{}}}}}.

                    By the induction hypothesis we can derive 
                    {\inpropenv{\propenv{}}{\isprop{\class{1}} {\path{\pathelem{}}{\x{}}}}},
                    and with the fact {\issubtypein{}{\class{1}}{\class{2}}}
                    we can use L-Sub to conclude 
                    {\inpropenv{\propenv{}}{\isprop{\class{2}} {\path{\pathelem{}}{\x{}}}}},
                    and finally by lemma~\ref{appendix:lemma:satisfies}
                    we derive
                    {\satisfies{\openv{}}{\isprop{\class{2}} {\path{\pathelem{}}{\x{}}}}}.

                  \end{subcase}
                \item[]
                  \begin{subcase}[\isacompare{\s{}}{\object{}}{\Value{\singletonmeta{}}}
                    {\replacefor
                      {\filtersetparen{\isprop{\Value{\singletonmeta{}}} {\x{}}}
                        {\notprop{\Value{\singletonmeta{}}}{\x{}}}}
                      {\object{}}
                      {\x{}}}\ 
                    \text{if}\ {\singletonmeta{}} \notequal \class{}]
                    \ 

                    \t{2} = {\Value{\singletonmeta{}}}, 
                    {\singletonmeta{}} \notequal \class{},
                    \inpropenv{\replacefor{\isprop{\Value{\singletonmeta{}}} {\x{}}}
                                           {\object{1}}
                                           {\x{}}}{\thenprop{\prop{}}}

                    Unreachable case since 
                    \t{2} = {\Value{\singletonmeta{}}} where 
                    {\singletonmeta{}} \notequal \class{},
                    but \v{2} = \class{2}.
                  \end{subcase}
                \item[]
                  \begin{subcase}[\isacompare{\s{}}{\object{}}{\t{}} {\filterset{\topprop{}} {\topprop{}}}]
                    \ 

                    {\thenprop{\prop{}}} = {\topprop{}}

                    {\satisfies{\openv{}}{\topprop{}}} holds by M-Top.

                  \end{subcase}
              \end{itemize}
            \end{subcase}
          \item[]
            \begin{subcase}[\isaopsem{\v{1}}{\v{2}} = {\false{}}, otherwise]
              \ 

              \v{1} \notequal\ \v{2},
              \isfalseval{\v{}}
              
              Since \isfalseval{\v{}} we prove {\satisfies{\openv{}}{\elseprop{\prop{}}}}
              by cases on the definition of \isacompareliteral{}:
              \begin{itemize} % isacompare
                \item[]
                  \begin{subcase}[\isacompare{\s{}}{\path{\classpe{}}{\path{\pathelem{}}{\x{}}}}{\Value{\class{}}}
                                 {\filterset{\isprop{\class{}} {\path{\pathelem{}}{\x{}}}}
                                            {\notprop{\class{}}{\path{\pathelem{}}{\x{}}}}}]
                    \ 


                    \object{1} = {\path{\classpe{}}{\path{\pathelem{}}{\x{}}}},
                    \t{2} = {\Value{\class{}}},
                    \inpropenv{\notprop{\class{}} {\path{\pathelem{}}{\x{}}}}{\elseprop{\prop{}}}

                    By inversion on the typing relation, since \classpe{} is the last path element of \object{1}
                    then \opsem{\openv{}}{\appexp{\classconst{}}{\openv{}({\path{\pathelem{}}{\x{}}})}}{\v{1}}.
                    
                    By the definition of {\classconst{}} either {\v{1}} = {\class{}} or {\v{1}} = \nil{}.

                    If {\v{1}} = \nil{}, then we know from the definition of \isaopsemliteral that 
                    {\openv{}({\path{\pathelem{}}{\x{}}})} = \nil{}.

                    Since \judgementtwo{}{\openv{}({\path{\pathelem{}}{\x{}}})}{\Nil},
                    and there is no \v{1} such that both \judgementtwo{}{\openv{}({\path{\pathelem{}}{\x{}}})}{\class} and
                    \judgementtwo{}{\openv{}({\path{\pathelem{}}{\x{}}})}{\Nil{}},
                    we use M-NotType to derive 
                    \satisfies{\openv{}}{\notprop{\class{}} {\path{\pathelem{}}{\x{}}}}.

                    Similarly if {\v{1}} = \class{1}, by the definition of \isacompareliteral
                    we know {\notsubtypein{}{\class{1}}{\class{}}} and 
                    {\openv{}({\path{\pathelem{}}{\x{}}})} = \class{1}.

                    Since \judgementtwo{}{\openv{}({\path{\pathelem{}}{\x{}}})}{\class{1}},
                    and there is no \v{1} such that both 
                    \judgementtwo{}{\v{1}}{\class{}} and
                    \judgementtwo{}{\v{1}}{\class{1}},
                    we use M-NotType to derive 
                    \satisfies{\openv{}}{\notprop{\class{}} {\path{\pathelem{}}{\x{}}}}.


                  \end{subcase}
                \item[]
                  \begin{subcase}[\isacompare{\s{}}{\object{}}{\Value{\singletonmeta{}}}
                    {\replacefor
                      {\filtersetparen{\isprop{\Value{\singletonmeta{}}} {\x{}}}
                        {\notprop{\Value{\singletonmeta{}}}{\x{}}}}
                      {\object{}}
                      {\x{}}}\ 
                    \text{if}\ {\singletonmeta{}} \notequal \class{}]
                    \ 

                    \t{2} = {\Value{\singletonmeta{}}}, 
                    {\singletonmeta{}} \notequal \class{},
                    %\thenprop{\prop{}} = {\replacefor{\isprop{\Value{\singletonmeta{}}} {\x{}}}
                    %                       {\object{1}}
                    %                       {\x{}}}
                    \inpropenv{\replacefor{\notprop{\Value{\singletonmeta{}}} {\x{}}}
                                           {\object{1}}
                                           {\x{}}}{\elseprop{\prop{}}}

                    Since \t{2} = {\Value{\singletonmeta{}}} where {\singletonmeta{}} \notequal \class{},
                    by inversion on the typing judgement 
                    {\v{2}} is either \true{}, \false{}, \nil{} or \k{}
                    by T-True, T-False, T-Nil or T-Kw.

                    If \object{1} = \emptyobject{} then \elseprop{\prop{}} = \topprop{} and
                    we derive
                    {\satisfies{\openv{}}{\topprop{}}} with M-Top.

                    Otherwise \object{1} = \path{\pathelem{}}{\x{}} and 
                    \inpropenv{\notprop{\Value{\singletonmeta{}}}{\path{\pathelem{}}{\x{}}}}{\elseprop{\prop{}}}.
                    Noting that \v{1} \notequal\ \v{2},
                    we know \judgementtwo{}{\openv{}({\path{\pathelem{}}{\x{}}})}{\s{}}
                    where \s{} \notequal\ {\Value{\singletonmeta{}}},
                    and there is no \v{1} such that both 
                    \judgementtwo{}{\v{1}}{\Value{\singletonmeta{}}} and
                    \judgementtwo{}{\v{1}}{\s{}}
                    so we can use M-NotType to derive
                    {\satisfies{\openv{}}{\notprop{\Value{\singletonmeta{}}}{\path{\pathelem{}}{\x{}}}}}.
                  \end{subcase}
                \item[]
                  \begin{subcase}[\isacompare{\s{}}{\object{}}{\t{}} {\filterset{\topprop{}} {\topprop{}}}]
                    \ 

                    {\elseprop{\prop{}}} = {\topprop{}}

                    {\satisfies{\openv{}}{\topprop{}}} holds by M-Top.

                  \end{subcase}
              \end{itemize}
            \end{subcase}
        \end{itemize}
      \end{proof}
\end{lemma}

\begin{lemma} \label{appendix:lemma:soundness}
{\soundnesslemmahypothesis}
\begin{proof}
By induction and cases on the derivation of \opsem {\openv{}} {\e{}} {\a{}},
and subcases on the penultimate rule of the derivation of
\judgementrewrite{\propenv{}}{\ep{}}{\t{}}{\filterset{\thenprop{\prop{}}}{\elseprop{\prop{}}}}{\object{}}{\e{}}
followed by T-Subsume as the final rule.

% induction on the derivation of the evaluation semantics because we want to apply
% the induction hypothesis to subderivations of the eval sem. If eg. used the typing
% judgement, we couldn't use the induction hypothesis on applications of higher-order
% functions, since the subderivation of T-Abs wouldn't be present.

\begin{case}[B-Val]

  \begin{itemize}
    \item[] 
      \begin{subcase}[T-True]
        \v{} = \true{},
  \ep{} = \true{},
  \e{} = \true{}, \issubtypein{}{\True}{\t{}}, \inpropenv{\topprop{}}{\thenprop{\prop{}}}, 
  \inpropenv{\botprop{}}{\elseprop{\prop{}}}, \issubtypein{}{\emptyobject{}}{\object{}}

        Proving part 1 is trivial: \object{} is a superobject of \emptyobject{}, which can only be \emptyobject{}.

        To prove part 2, we note that \v{} = \true{}
        and \inpropenv{\topprop{}}{\thenprop{\prop{}}},
        so \satisfies{\openv{}}{\thenprop{\prop{}}} by M-Top.

        Part 3 holds as \e{} can only be reduced to itself via B-Val.

        Part 4 holds vacuously.
      \end{subcase}
    \item[]
      \begin{subcase}[T-HMap] \v{} = {\curlymapvaloverright{\v{k}}{\v{v}}},
  \ep{} = {\curlymapvaloverright{\v{k}}{\v{v}}},
  \e{} = {\curlymapvaloverright{\v{k}}{\v{v}}},
  \issubtypein{}{\HMapc {\mandatory{}}}{\t{}},
  \inpropenv{\topprop{}}{\thenprop{\prop{}}},
  \inpropenv{\botprop{}}{\elseprop{\prop{}}},
  \issubtypein{}{\emptyobject{}}{\object{}},
  $\overrightarrow{\judgementtwo {} {\v{k}}{\Value \k{}}}$,
  $\overrightarrow{\judgementtwo {} {\v{v}}{\t{v}}}$,
  \mandatory{} = \mandatorysetoverright{\k{}}{\t{v}}

        Similar to T-True.

        Part 4 holds by the induction hypothese on {\overr{\v{k}}} and {\overr{\v{v}}}.
      \end{subcase}
    \item[]
      \begin{subcase}[T-Kw] \v{} = {\k{}},
  \ep{} = {\k{}},
  \e{} = {\k{}},
  \issubtypein{}{\Value{\k{}}}{\t{}},
  \inpropenv{\topprop{}}{\thenprop{\prop{}}},
  \inpropenv{\botprop{}}{\elseprop{\prop{}}},
  \issubtypein{}{\emptyobject{}}{\object{}}

        Similar to T-True.
      \end{subcase}
      \begin{subcase}[T-Str]
        Similar to T-Kw.
      \end{subcase}
  \item[] 
    \begin{subcase}[T-False]
      \v{} = \false{},
\ep{} = \false, 
\e{} = \false, 
\issubtypein{}{\False}{\t{}},
\inpropenv{\botprop{}}{\thenprop{\prop{}}},
\inpropenv{\topprop{}}{\elseprop{\prop{}}},
\issubtypein{}{\emptyobject{}}{\object{}}

Proving part 1 is trivial: \object{} is a superobject of \emptyobject{}, which must be \emptyobject{}. 

To prove part 2, we note that \v{} = \false{}
and \inpropenv{\topprop{}}{\elseprop{\prop{}}}, so \satisfies{\openv{}}{\elseprop{\prop{}}} by M-Top. 

Part 3 holds as \e{} can only be reduced to itself via B-Val.

Part 4 holds vacuously.
\end{subcase}
    \item[]
      \begin{subcase}[T-Class] \v{} = {\class{}},
  \ep{} = {\class{}},
  \e{} = {\class{}},
  \issubtypein{}{\Value{\class{}}}{\t{}},
  \inpropenv{\topprop{}}{\thenprop{\prop{}}},
  \inpropenv{\botprop{}}{\elseprop{\prop{}}},
  \issubtypein{}{\emptyobject{}}{\object{}}

        Similar to T-True.
      \end{subcase}
    \item[]
      \begin{subcase}[T-Instance] \v{} = {\classvalue{\classhint{}} {\overrightarrow {\classfieldpair{\fld{i}} {\v{i}}}}},
  \ep{} = {\classvalue{\classhint{}} {\overrightarrow {\classfieldpair{\fld{}} {\v{}}}}},
  \e{} = {\classvalue{\classhint{}} {\overrightarrow {\classfieldpair{\fld{}} {\v{}}}}},
  \issubtypein{}{\class{}}{\t{}},
  \inpropenv{\topprop{}}{\thenprop{\prop{}}},
  \inpropenv{\botprop{}}{\elseprop{\prop{}}},
  \issubtypein{}{\emptyobject{}}{\object{}}


        Similar to T-True.

        Part 4 holds by the induction hypotheses on ${\overrightarrow{\v{i}}}$.
      \end{subcase}
  \item[] 
    \begin{subcase}[T-Nil] 
      \v{} = \nil{},
\ep{} = \nil, 
\e{} = \nil, 
\issubtypein{}{\Nil}{\t{}},
\inpropenv{\botprop{}}{\thenprop{\prop{}}},
\inpropenv{\topprop{}}{\elseprop{\prop{}}},
\issubtypein{}{\emptyobject{}}{\object{}}

      Similar to T-False.
\end{subcase}
\item[]
\begin{subcase}[T-Multi] 
  \v{} = {\multi {\v{1}} {\curlymapvaloverright{\v{k}}{\v{v}}}}
  \ep{} = {\multi {\v{1}} {\curlymapvaloverright{\v{k}}{\v{v}}}},
  \judgementtworewrite {} {\v{1}} {\t{1}}{\v{1}},
  \overr{\judgementtworewrite{}{\v{k}}{\Top{}}{\v{k}}},
  \overr{\judgementtworewrite{}{\v{v}}{\s{}}{\v{v}}},
  \e{} = {\multi {\v{1}} {\curlymapvaloverright{\v{k}}{\v{v}}}},
  \issubtypein{}{\MultiFntype {\s{}} {\t{1}}}{\t{}},
  \inpropenv{\topprop{}}{\thenprop{\prop{}}},
  \inpropenv{\botprop{}}{\elseprop{\prop{}}},
  \issubtypein{}{\emptyobject{}}{\object{}}

        Similar to T-True.
\end{subcase}
\item[]
\begin{subcase}[T-Const]
  \e{} = {\const{}},
  \issubtypein{}{\constanttype{\const{}}}{\t{}},
  \inpropenv{\topprop{}}{\thenprop{\prop{}}},
  \inpropenv{\botprop{}}{\elseprop{\prop{}}},
  \issubobjin{}{\emptyobject{}}{\object{}}

        Parts 1, 2 and 3 hold for the same reasons as T-True. 
\end{subcase}


  \end{itemize}
\end{case}



\begin{case}[B-Local]
{ \inopenv {\openv{}} {\x{}} {\v{}} },
{ \opsem {\openv{}} {\x{}} {\v{}} }

\begin{itemize}
  \item[]
\begin{subcase}[T-Local]
  \ep{} = \x{}, 
  \e{} = \x{}, 
  \inpropenv{\notprop {\falsy{}} {\x{}}}{\thenprop{\prop{}}},
  \inpropenv{\isprop {\falsy{}} {\x{}}}{\elseprop{\prop{}}},
\issubtypein{}{\x{}}{\object{}},
\inpropenv{\propenv{}}{\isprop{\t{}}{\x{}}}

Part 1 follows from \inopenv{\openv{}}{\object{}} {\v{}}, since either {\object{}} = \x{}
and \inopenv{\openv{}}{\x{}} {\v{}} is a premise of B-Local, or {\object{}} = {\emptyobject{}} which also
satisfies the goal.

Part 2 considers two cases: if \istrueval{\v{}}, then 
\satisfies{\openv{}}{\notprop{\falsy}{\x{}}} holds by M-NotType; if \isfalseval{\v{}}, then 
\satisfies{\openv{}}{\isprop{\falsy}{\x{}}} holds by M-Type.

We prove part 3 by observing
\inpropenv{\propenv{}}{\isprop{\t{}}{\x{}}},
\satisfies{\openv{}}{\propenv{}},
and
\inopenv {\openv{}} {\x{}} {\v{}}
(by B-Local)
which gives us the desired result.

Part 4 holds vacuously.
\end{subcase}
\end{itemize}

\end{case}

\begin{case}[B-Do]
  \opsem {\openv{}} {\e{1}} {\v{1}},
  \opsem {\openv{}} {\e{2}} {\v{}}

\begin{itemize}
  \item[] \begin{subcase}[T-Do]
\ep{} = {\doexp {\ep1} {\ep2}},
  \judgementrewrite {\propenv{}} 
             {\ep1} {\t1}
             {\filterset {\thenprop {\prop{1}}} {\elseprop {\prop1}}} 
             {\object{1}}
             {\e1},
\judgementrewrite {\propenv{}, {\orprop {\thenprop {\prop{1}}} {\elseprop {\prop{1}}}}}
           {\ep{}} {\t{}}
           {\filterset {\thenprop {\prop{}}} {\elseprop {\prop{}}}} 
           {\object{}}
           {\e{}},
\e{} = {\doexp {\e1} {\e2}}

For all parts we note 
    since {\e{1}} can be either a true or false value
    then
    {\satisfies{\openv{}}{{\propenv{}},{\orprop {\thenprop {\prop{1}}} {\elseprop {\prop{1}}}}}}
    by M-Or,
    which together with 
\judgement {\propenv{}, {\orprop {\thenprop {\prop{1}}} {\elseprop {\prop{1}}}}}
           {\e{2}} {\t{}}
           {\filterset {\thenprop {\prop{}}} {\elseprop {\prop{}}}} 
           {\object{}},
    and
  \opsem {\openv{}} {\e{2}} {\v{}}
    allows us to apply the induction hypothesis on \e{2}.

To prove part 1 we use the induction hypothesis on \e{2}
to show either \object{} = \emptyobject{} 
or \inopenv {\openv{}} {\object{}} {\v{}}, since \e{} always
evaluates to the result of \e{2}.

For part 2 we use the induction hypothesis on \e{2}
to show if \istrueval{\v{}} then
        {\satisfies{\openv{}}{\thenprop{\prop{}}}}
        or
  if \isfalseval{\v{}} then
        {\satisfies{\openv{}}{\elseprop{\prop{}}}}.

Parts 3 and 4 follow from the induction hypothesis on \e{2}.
    \end{subcase}
\end{itemize}
\end{case}

\begin{case}[BE-Do1]
  \opsem {\openv{}} {\e{1}} {\errorval{\v{e}}},
  \opsem {\openv{}} {\e{}} {\errorval{\v{}}}

        Trivially reduces to an error.
\end{case}

\begin{case}[BE-Do2]
  \opsem {\openv{}} {\e{1}} {\v{1}},
  \opsem {\openv{}} {\e{2}} {\errorvalv{}},
  \opsem {\openv{}} {\e{}} {\errorvalv{}}

        As above.
\end{case}

\begin{case}[B-New]
  $
  \overrightarrow{
  \opsem {\openv{}}
         {\e{i}}
         {\v{i}}
       }$,
         $\newjava {\classhint{1}}
                  {\overrightarrow{\classhint{i}}}
                  {\overrightarrow{\v{i}}}
                  {\v{}}$

\begin{itemize}
  \item[]
\begin{subcase}[T-New]
  \ep{} = {\newexp {\class{}} {\overrightarrow{\ep{i}}}},
  \inct{\ctctorentry{\overr{\class{i}}}}{\ctlookupctors{\ct{}}{\classhint{}}},
  \overr{\javatotcnil{\classhint{i}}{\t{i}}},
  \overr{
  \judgementtworewrite {\propenv{}}
                    {\ep{i}} {\t{i}}
                    {\e{i}}
                  },
  \e{} = {\newstaticexp {\overrightarrow{\classhint{i}}} {\classhint{}} 
                                                          {\class{}} {\overrightarrow{\e{i}}}},
  \issubtypein{}{\javatotcexp{\classhint{}}}{\t{}},
  \inpropenv{\topprop{}}{\thenprop{\prop{}}},
  \inpropenv{\botprop{}}{\elseprop{\prop{}}},
  \issubobjin{}{\emptyobject{}}{\object{}}

Part 1 follows \object{} = \emptyobject{}.

Part 2 requires some explanation. The two false values in Typed Clojure
cannot be constructed with \newliteral{}, so the only case is \v{} $\not=$ \false\ (or \nil)
where \thenprop{\prop{}} = \topprop{} so \satisfies{\openv{}}{\thenprop{\prop{}}}.
\Void{} also lacks a constructor.

Part 3 holds as B-New reduces to a \emph{non-nilable}
instance of \class{} via \newjavaliteral (by assumption~\ref{appendix:assumption:new}), 
and {\t{}} is a supertype of \javatotcexp{\classhint{}}.

\end{subcase}
\item[]
\begin{subcase}[T-NewStatic]
  {\ep{}} = {\newstaticexp {\overrightarrow{\classhint{i}}} {\classhint{}}
                                                          {\class{}} {\overrightarrow{\e{i}}}}

  Non-reflective constructors cannot be written directly by the user, so we can assume
  the class information attached to the syntax corresponds to an actual constructor by inversion
  from T-New.

  The rest of this case progresses like T-New.
\end{subcase}
\end{itemize}
\end{case}

\begin{case}[BE-New1] $\overrightarrow{
  \opsem {\openv{}}
         {\e{i-1}}
         {\v{i-1}}
       }$,
  \opsem {\openv{}}
         {\e{i}}
         {\errorvalv{}},
  \opsem {\openv{}} {\e{}} {\errorvalv{}}

        Trivially reduces to an error.

\end{case}

\begin{case}[BE-New2] 
  $\overrightarrow{
  \opsem {\openv{}}
         {\e{i}}
         {\v{i}}
       }$,
         \newjava {\classhint{1}}
                  {\overrightarrow{\classhint{i}}}
                  {\overrightarrow{\v{i}}}
                  {\errorvalv{}},
        \opsem {\openv{}} {\e{}} {\errorvalv{}}

        As above.

\end{case}

\begin{case}[B-Field]
  \opsem {\openv{}}
         {\e{1}} 
         {\classvalue{\classhint{1}} {\classfieldpair{\fld{}} {\v{}}}}

\begin{itemize}
  \item[]
\begin{subcase}[T-Field]
  \ep{} = {\fieldexp {\fld{}} {\ep{1}}},
  \judgementtworewrite {\propenv{}} {\ep{}} {\s{}} {\e{}},
  \issubtypein{}{\s{}}{\Object{}},
  \tctojava{\s{}}{\classhint{1}},
  \inct{\ctfldentry{\fld{}}{\classhint{2}}}{\ctlookupfields{\ct{}}{\classhint{1}}},
  \e{} = {\fieldstaticexp {\classhint{1}} {\classhint{2}} {\fld{}} {\e{1}}}
  \issubtypein{}{\javatotcnilexp{\classhint{2}}}{\t{}},
  \inpropenv{\topprop{}}{\thenprop{\prop{}}},
  \inpropenv{\topprop{}}{\elseprop{\prop{}}},
  \issubobjin{}{\emptyobject{}}{\object{}}


Part 1 is trivial as \object{} is always \emptyobject{}.

Part 2 holds trivially; \v{} can be either a true or false value
and both {\thenprop{\prop{}}} and {\elseprop{\prop{}}}
are \topprop{}.

Part 3 relies on the semantics of \getfieldliteral (assumption~\ref{appendix:assumption:field})
in B-Field, which returns a \emph{nilable} instance of \classhint{2},
and \t{} is a supertype of \javatotcnilexp{\classhint{2}}.
Notice \issubtypein{}{\s{}}{\Object{}} is required to guard from dereferencing \nil{},
as {\classhint{1}} erases occurrences of \Nil{} in \s{} via  \tctojava{\s{}}{\classhint{1}}.
\end{subcase}
  \item[]

\begin{subcase}[T-FieldStatic]
  {\ep{}} = {\fieldstaticexp {\classhint{1}} {\classhint{2}} {\fld{}} {\e{1}}}

  Non-reflective field lookups cannot be written directly by the user, so we can assume
  the class information attached to the syntax corresponds to an actual field by inversion
  from T-Field.

  The rest of this case progresses like T-Field.
\end{subcase}

\end{itemize}
\end{case}

\begin{case}[BE-Field]
  \opsem {\openv{}}
         {\e{1}} 
         {\errorvalv{}},
  \opsem {\openv{}}
         {\e{}}
         {\errorvalv{}}

         Trivially reduces to an error.

\end{case}

\begin{case}[B-Method]
  \opsem {\openv{}}
         {\e{m}}
         {\v{m}},
  $\overrightarrow{
  \opsem {\openv{}}
         {\e{a}}
         {\v{a}}}$,
  \invokejavamethod {\classhint{1}} {\v{m}} {mth}
                    {\overrightarrow{\classhint{a}}} {\overrightarrow{\v{a}}}
                    {\classhint{2}}
                    {\v{}}

\begin{itemize}
  \item[]
\begin{subcase}[T-Method]
  \judgementtworewrite {\propenv{}} {\ep{}} {\s{}} {\e{}},
             \issubtypein{}{\s{}}{\Object{}},
  \tctojava{\s{}}{\classhint{1}},
                  \inct{\ctmthentry{\mth{}}{\overrightarrow{\classhint{i}}}{\classhint{2}}}{\ctlookupmethods{\ct{}}{\classhint{1}}},
                  \overr{\javatotcnil{\classhint{i}}{\t{i}}},
             \overr{
  \judgementtworewrite {\propenv{}} {\ep{i}} {\t{i}} {\e{i}}
                  },
  \e{} = {\methodstaticexp {\classhint{1}} 
                          {\overr {\classhint{i}}} 
                          {\classhint{2}}
                          {\mth{}} {\e{m}} {\overr{\e{a}}}},
                        \issubtypein{}{\javatotcnilexp{\classhint{2}}}{\t{}},
  \inpropenv{\topprop{}}{\thenprop{\prop{}}},
  \inpropenv{\topprop{}}{\elseprop{\prop{}}},
  \issubobjin{}{\emptyobject{}}{\object{}}


Part 1 is trivial as \object{} is always \emptyobject{}.

Part 2 holds trivially, \v{} can be either a true or false value
and both {\thenprop{\prop{}}} and {\elseprop{\prop{}}}
are \topprop{}.

Part 3 relies on the semantics of \invokejavamethodliteral (assumption~\ref{appendix:assumption:method})
in B-Method, which returns a \emph{nilable} instance of \classhint{2},
and \t{} is a supertype of \javatotcnil{\classhint{2}}.
Notice \issubtypein{}{\s{}}{\Object{}} is required to guard from dereferencing \nil{},
as {\classhint{1}} erases occurrences of \Nil{} in \s{} via  \tctojava{\s{}}{\classhint{1}}.
\end{subcase}
\item[]
\begin{subcase}[T-MethodStatic]
  \ep{} = 
  {\methodstaticexp {\classhint{1}} 
        {\overrightarrow {\classhint{i}}} 
        {\classhint{2}}
        {\mth{}} {\e{1}} {\overrightarrow{\e{i}}}}

  Non-reflective method invocations cannot be written directly by the user, so we can assume
  the class information attached to the syntax corresponds to an actual method by inversion
  from T-Method.

  The rest of this case progresses like T-Method.
\end{subcase}


\end{itemize}

\end{case}

\begin{case}[BE-Method1]
  \opsem {\openv{}}
         {\e{m}}
         {\errorval{\v{}}},
  \opsem {\openv{}}
         {\e{}}
         {\errorval{\v{}}}

         Trivially reduces to an error.
\end{case}
\begin{case}[BE-Method2]
  \opsem {\openv{}}
         {\e{m}}
         {\v{m}},
 $\overrightarrow{
  \opsem {\openv{}}
         {\e{n-1}}
         {\v{n-1}}
       }$,
  \opsem {\openv{}}
         {\e{n}}
         {\errorval{\v{}}},
  \opsem {\openv{}}
         {\e{}}
         {\errorval{\v{}}}

  As above.
\end{case}
\begin{case}[BE-Method3]
  \opsem {\openv{}}
         {\e{m}}
         {\v{m}},
  $\overrightarrow{
  \opsem {\openv{}}
         {\e{a}}
         {\v{a}}
       }$,
  \invokejavamethod {\classhint{1}} {\v{m}} {mth}
                    {\overrightarrow{\classhint{a}}} {\overrightarrow{\v{a}}}
                    {\classhint{2}}
                    {\errorvalv{}},
  \opsem {\openv{}} {\e{}} {\errorvalv{}}

  As above.

\end{case}

\begin{case}[B-DefMulti]
  \v{} = {\multi {\v{d}} {\emptydisptable}},
  \opsem {\openv{}} {\e{d}} {\v{d}}



\begin{itemize}
  \item[]
\begin{subcase}[T-DefMulti]
  \ep{} = {\createmultiexp {\s{}} {\ep{d}}},
  \s{} = {\ArrowOne {\x{}} {\t{1}} {\t{2}}
                          {\filterset {\thenprop {\prop{1}}}
                                      {\elseprop {\prop{1}}}}
                          {\object{1}}},
  \t{d} = {\ArrowOne {\x{}} {\t{1}} {\t{3}}
                          {\filterset {\thenprop {\prop{2}}}
                                      {\elseprop {\prop{2}}}}
                          {\object{2}}},
\judgementtworewrite {\propenv{}} {\ep{}} {\sp{}} {\e{}},
  \e{} = {\createmultiexp {\s{}} {\e{d}}},
  \issubtypein{}{\MultiFntype {\s{}} {\t{d}}}{\t{}},
  \inpropenv{\topprop{}}{\thenprop{\prop{}}},
  \inpropenv{\botprop{}}{\elseprop{\prop{}}},
  \issubobjin{}{\emptyobject{}}{\object{}}


Part 1 and 2 hold for the same reasons as T-True.
For part 3 we show \judgementtwo{}{\multi {\v{d}} {\emptydisptable}}{\MultiFntype {\s{}} {\t{d}}}
by T-Multi, since \judgementtwo {} {\v{d}} {\t{d}} by the inductive hypothesis on {\e{d}}
and {\emptydisptable} vacuously satisfies the other premises of T-Multi, so we are done.

\end{subcase}
\end{itemize}
\end{case}

\begin{case}[BE-DefMulti] \opsem {\openv{}} {\e{d}} {\errorvalv{}},
        \opsem {\openv{}} {\e{}} {\errorvalv{}}

        Trivially reduces to an error.

\end{case}

\begin{case}[B-DefMethod]

        \ 

        \begin{enumerate}
          \item
       \v{} = {\multi {\v{d}} {\disptablep{}}},
          \item
        \opsem {\openv{}}
               {\e{m}}
               {\multi {\v{d}} {\disptable{}}},
          \item
  \opsem {\openv{}}
         {\e{v}}
         {\v{v}},
          \item
  \opsem {\openv{}}
         {\e{f}}
         {\v{f}},
          \item
         \disptablep{} = {\extenddisptable {\disptable{}} 
                                {\v{v}}
                                {\v{f}}}
        \end{enumerate}

  \begin{itemize}
    \item[]
      \begin{subcase}[T-DefMethod]
        \ 
        
        \begin{enumerate}[resume]

          \item
  \ep{} = {\extendmultiexp {\ep{m}} {\ep{v}} {\ep{f}}},
          \item
  \t{m} = {\ArrowOne {\x{}} {\t{1}} {\s{}}
                     {\filterset {\thenprop {\prop{m}}}
                                 {\elseprop {\prop{m}}}}
                     {\object{m}}},
          \item
  \t{d} = {\ArrowOne {\x{}} {\t{1}} {\sp{}}
                     {\filterset {\thenprop {\prop{d}}}
                                 {\elseprop {\prop{d}}}}
                     {\object{d}}},
          \item
\judgementtworewrite {\propenv{}}
                  {\ep{m}} {\MultiFntype {\t{m}} {\t{d}}}
                  {\e{m}}
          \item
  \isacompare{\sp{}}{\object{d}}{\t{v}}{\filterset {\thenprop {\prop{i}}} {\elseprop {\prop{i}}}},
          \item
\judgementtworewrite {\propenv{}}
           {\e{v}} {\t{v}}
           {\e{v}}
          \item
  \judgementrewrite {\propenv{}, {\isprop{\t{1}} {\x{}}}, {\thenprop {\prop{i}}}}
           {\ep{f}} {\s{}}
           {\filterset {\thenprop {\prop{m}}}
                       {\elseprop {\prop{m}}}}
           {\object{m}}
           {\e{f}}
          \item
  \e{} = {\extendmultiexp {\e{m}} {\e{v}} {\e{f}}},
          \item
  \e{f} = {\abs {\x{}} {\t{1}} {\e{b}}},
\item
  \issubtypein{}{\MultiFntype {\t{m}} {\t{d}}}{\t{}},
\item
  \inpropenv{\topprop{}}{\thenprop{\prop{}}},
\item
  \inpropenv{\botprop{}}{\elseprop{\prop{}}},
\item
  \issubobjin{}{\emptyobject{}}{\object{}}
        \end{enumerate}

                                Part 1 and 2 hold for the same reasons as T-True, noting that the propositions
                                and object agree with T-Multi.

For part 3 we show
\judgementtwo{}{\multi {\v{d}} {\extenddisptable {\disptable{}}{\v{v}}{\v{f}}}}{\MultiFntype {\t{m}} {\t{d}}}
by noting \judgementtwo {} {\v{d}} {\t{d}},
  \judgementtwo{}{\v{v}}{\Top{}}
  and
  \judgementtwo{}{\v{f}}{\t{m}}, and since \disptable{} is in the correct form by the inductive
  hypothesis on {\e{m}} we can satisfy all premises of T-Multi, so we are done.


      \end{subcase}

  \end{itemize}
\end{case}

      \begin{case}[BE-DefMethod1]
        \opsem {\openv{}}
               {\e{m}}
               {\errorval{\v{}}},
        \opsem {\openv{}}
                  {\e{}}
                {\errorval{\v{}}}

                Trivially reduces to an error.

      \end{case}
      \begin{case}[BE-DefMethod2]
        \opsem {\openv{}}
         {\e{m}}
         {\multi {\v{d}} {\disptable{}}},
  \opsem {\openv{}}
         {\e{v}}
         {\errorval{\v{}}},
        \opsem {\openv{}}
                  {\e{}}
                {\errorval{\v{}}}

                Trivially reduces to an error.
      \end{case}
      \begin{case}[BE-DefMethod3]
        \opsem {\openv{}}
         {\e{m}}
         {\multi {\v{d}} {\disptable{}}},
  \opsem {\openv{}}
         {\e{v}}
         {\v{v}},
  \opsem {\openv{}}
         {\e{f}}
         {\errorval{\v{}}},
        \opsem {\openv{}}
                  {\e{}}
                {\errorval{\v{}}}

                Trivially reduces to an error.

      \end{case}

\begin{case}[B-BetaClosure]
  \ 

  \begin{itemize}
    \item
  \opsem{\openv{}}{\e{}}{\v{}},
    \item
  \opsem {\openv{}}
         {\e{1}}
         {\closure {\openv{c}} {\abs {\x{}} {\s{}} {\e{b}}}},
    \item
  \opsem {\openv{}}
         {\e{2}}
         {\v{2}},
    \item
  \opsem {\extendopenv {\openv{c}} {\x{}} {\v{2}}}
         {\e{b}}
         {\v{}}
     \end{itemize}


\begin{itemize}
  \item[]
\begin{subcase}[T-App]
  \ 

  \begin{itemize}
    \item
  \ep{} = {\appexp {\ep{1}} {\ep{2}}},
    \item
  \judgementrewrite {\propenv{}} {\ep{1}} {\ArrowOne {\x{}} {\s{}}
                                                       {\t{f}}
                                                       {\filterset {\thenprop {\prop{f}}}
                                                                   {\elseprop {\prop{f}}}}
                                                       {\object{f}}}
                {\filterset {\thenprop {\prop{1}}}
                            {\elseprop {\prop{1}}}}
                {\object{1}}
                {\e{1}},
    \item
  \judgementrewrite {\propenv{}}
                 {\ep{2}} {\s{}}
                 {\filterset {\thenprop {\prop{2}}}
                             {\elseprop {\prop{2}}}}
                 {\object{2}}
                 {\e{2}},
    \item
  \e{} = {\appexp {\e{1}} {\e{2}}},
    \item
      \issubtypein{}  {\replacefor {\t{f}} {\object{2}} {\x{}}}{\t{}},
    \item
      \inpropenv{\replacefor {\thenprop {\prop{f}}} {\object{2}} {\x{}}} {\thenprop {\prop{}}},
    \item
      \inpropenv{\replacefor {\elseprop {\prop{f}}} {\object{2}} {\x{}}} {\elseprop {\prop{}}},
    \item
      \issubobjin{}{\replacefor {\object{f}} {\object{2}} {\x{}}} {\object{}}
  \end{itemize}

         By inversion on \e{1} from T-Clos
         there is some environment {\propenvc{}} such that
         \begin{itemize}
           \item
              \satisfies{\openv{c}}{\propenvc{}} and
            \item
  \judgement {\propenvc{}} {\abs {\x{}} {\s{}} {\e{b}}} {\ArrowOne {\x{}} {\s{}}
                                                       {\t{f}}
                                                       {\filterset {\thenprop {\prop{f}}}
                                                                   {\elseprop {\prop{f}}}}
                                                       {\object{f}}}
                {\filterset {\thenprop {\prop{1}}}
                            {\elseprop {\prop{1}}}}
                {\object{1}},
         \end{itemize}
         and also by inversion on \e{1} from T-Abs
         \begin{itemize}
           \item
  { \judgementrewrite {\propenvc{}, {\isprop {\s{}} {\x{}}}}
              {\ep{b}} {\t{f}}
               {\filterset {\thenprop {\prop{f}}}
                           {\elseprop {\prop{f}}}}
               {\object{f}}
               {\e{b}}}.
         \end{itemize}

          From 
          \begin{itemize}
            \item
              \satisfies{\openv{c}}{\propenvc{}},
            \item
  \judgementrewrite {\propenv{}}
                 {\ep{2}} {\s{}}
                 {\filterset {\thenprop {\prop{2}}}
                             {\elseprop {\prop{2}}}}
                 {\object{2}}
                 {\e{2}} and 
            \item
  \opsem {\openv{}}
         {\e{2}}
         {\v{2}},
     \end{itemize}
              we know (by substitution)
              \satisfies{\extendopenv {\openv{c}} {\x{}} {\v{2}}}{\propenvc{},{\isprop{\s{}}{\x{}}}}.

              We want to prove
        \judgementrewrite {\propenvc{}}
                          {\replacefor{\ep{b}}{\v{2}}{\x{}}}
                          {\replacefor{\t{f}}{\object{2}}{\x{}}}
               {\replacefor{\filterset {\thenprop {\prop{f}}}
                                       {\elseprop {\prop{f}}}}{\object{2}}{\x{}}}
                          {\replacefor{\object{f}}{\object{2}}{\x{}}}
                          {\replacefor{\e{b}}{\v{2}}{\x{}}}, 
                          which can be justified by noting 
          \begin{itemize}
            \item
  \judgementtworewrite {\propenvc{},{\isprop{\s{}}{\x{}}}}{\ep{b}}{\t{f}}{\e{b}},
            \item
  \judgementrewrite {\propenv{}}
                 {\ep{2}} {\s{}}
                 {\filterset {\thenprop {\prop{2}}}
                             {\elseprop {\prop{2}}}}
                 {\object{2}}
                 {\e{2}} and 
            \item
  \opsem {\openv{}}
         {\e{2}}
         {\v{2}}.
     \end{itemize}

     From the previous fact and \satisfies{\openv{c}}{\propenvc{}},
              we know
  \opsem {\openv{c}}
         {\replacefor{\e{b}}{\v{2}}{\x{}}}
         {\v{}}.

                    Noting that 
      \issubtypein{}  {\replacefor {\t{f}} {\object{2}} {\x{}}}{\t{}},
      \inpropenv{\replacefor {\thenprop {\prop{f}}} {\object{2}} {\x{}}} {\thenprop {\prop{}}},
      \inpropenv{\replacefor {\elseprop {\prop{f}}} {\object{2}} {\x{}}} {\elseprop {\prop{}}}
      and
      \issubobjin{}{\replacefor {\object{f}} {\object{2}} {\x{}}} {\object{}},
                    we can use
         \begin{itemize}
           \item
        \judgementrewrite {\propenvc{}}
                          {\replacefor{\ep{b}}{\v{2}}{\x{}}}
                          {\replacefor{\t{f}}{\object{2}}{\x{}}}
               {\replacefor{\filterset {\thenprop {\prop{f}}}
                                       {\elseprop {\prop{f}}}}{\object{2}}{\x{}}}
                          {\replacefor{\object{f}}{\object{2}}{\x{}}}
                          {\replacefor{\e{b}}{\v{2}}{\x{}}}, 
           \item
              \satisfies{\openv{c}}{\propenvc{}},
           \item
\isconsistent{\openv{c}} (via induction hypothesis on {\ep{1}}), and
           \item 
  \opsem {\openv{c}}
         {\replacefor{\e{b}}{\v{2}}{\x{}}}
         {\v{}}.
         \end{itemize}
         to apply the induction hypothesis on {\replacefor{\ep{b}}{\v{2}}{\x{}}} and satisfy
         all conditions.

\end{subcase}
\end{itemize}
\end{case}

\begin{case}[B-Delta]
  \opsem {\openv{}} {\e{1}} {\const{}},
  \opsem {\openv{}} {\e{2}} {\v{2}},
  \constantopsem{\const{}}{\v{2}} = \v{}

\begin{itemize}
  \item[]
\begin{subcase}[T-App]
  \ 

  \begin{itemize}
    \item
  \ep{} = {\appexp {\ep{1}} {\ep{2}}},
    \item
  \judgementrewrite {\propenv{}} {\ep{1}} {\ArrowOne {\x{}} {\s{}}
                                                       {\t{f}}
                                                       {\filterset {\thenprop {\prop{f}}}
                                                                   {\elseprop {\prop{f}}}}
                                                       {\object{f}}}
                {\filterset {\thenprop {\prop{1}}}
                            {\elseprop {\prop{1}}}}
                {\object{1}}
                {\e{1}},
    \item
  \judgementrewrite {\propenv{}}
                 {\ep{2}} {\s{}}
                 {\filterset {\thenprop {\prop{2}}}
                             {\elseprop {\prop{2}}}}
                 {\object{2}}
                 {\e{2}},
    \item
  \e{} = {\appexp {\e{1}} {\e{2}}},
    \item
      \issubtypein{}  {\replacefor {\t{f}} {\object{2}} {\x{}}}{\t{}},
    \item
      \inpropenv{\replacefor {\thenprop {\prop{f}}} {\object{2}} {\x{}}} {\thenprop {\prop{}}},
    \item
      \inpropenv{\replacefor {\elseprop {\prop{f}}} {\object{2}} {\x{}}} {\elseprop {\prop{}}},
    \item
      \issubobjin{}{\replacefor {\object{f}} {\object{2}} {\x{}}} {\object{}}
  \end{itemize}

  % TODO do I need to prove anything about the argument in the definition
  % of the constant being under \s{}?

  Prove by cases on \const{}.
  \begin{itemize}
    \item[] \begin{subcase}[\const{} = \classconst]
        \issubtypein{}
  {\ArrowOne {\x{}} {\Top{}}
                                      {\Union{\Nil}{\Class}}
                                      {\filterset {\topprop{}}
                                                  {\topprop{}}}
                                      {\path {\classpe{}} {\x{}}}}
    {\ArrowOne {\x{}} {\s{}}
                                                       {\t{f}}
                                                       {\filterset {\thenprop {\prop{f}}}
                                                                   {\elseprop {\prop{f}}}}
                                                       {\object{f}}}

    Prove by cases on \v{2}.

        \begin{itemize}
          \item[] \begin{subcase}[\v{2} = \classvalue{\class{}} {\overrightarrow {\classfieldpair{\fld{i}} {\v{i}}}}]
                    \v{} = \class{}

                    To prove part 1,
                    note
                    \issubobjin{}{\replacefor {\object{f}} {\object{2}} {\x{}}} {\object{}},
                    and \issubobjin{}{\path {\classpe{}} {\x{}}}{\object{f}}.
                    Then either \object{} = \emptyobject{} and we are done,
                    or \object{} = {\path {\classpe{}}{\object{2}}} and
                    by the induction hypothesis on \e{2} we know \inopenv {\openv{}} {\object{2}} {\v{2}}
                    and by the definition of path translation we know
                    {\openv{}}({\path {\classpe{}} {\object{2}}}) = {\appexp {\classconst{}} {{\openv{}}(\object{2})}},
                    which evaluates to \v{}.

                    Part 2 is trivial since both propositions can only be \topprop{}.
                    
                    Part 3 holds because 
                    \v{} = \class{},
                    \issubtypein{}{\Union{\Nil}{\Class}}{\replacefor {\t{f}} {\object{2}} {\x{}}}
                    and
                    \issubtypein{}{\replacefor {\t{f}} {\object{2}} {\x{}}}{\t{}},
                    so
                    {\judgementtwo{}{\v{}}{\t{}}}
                    since
                    {\judgementtwo{}{\class{}}{\Union{\Nil}{\Class}}}.
                  \end{subcase}
          \item[] \begin{subcase}[\v{2} = \class{}] \v{} = \Class{}

              As above.
                  \end{subcase}
          \item[] \begin{subcase}[\v{2} = \true{}] \v{} = \Boolean{}

              As above.
                  \end{subcase}
          \item[] \begin{subcase}[\v{2} = \false{}] \v{} = \Boolean{}


              As above.
                  \end{subcase}
          \item[] \begin{subcase}[\v{2} = {\closure {\openv{}} {\abs {\x{}} {\t{}} {\e{}}}}] \v{} = \IFn{}


              As above.
                  \end{subcase}
          \item[] \begin{subcase}[\v{2} = {\multi {\v{d}} {\disptable{}}}] \v{} = \HMapInstance{}


              As above.
                  \end{subcase}
          \item[] \begin{subcase}[\v{2} = {\curlymapvaloverright{\v{1}}{\v{2}}}] \v{} = \Keyword{}


              As above.
                  \end{subcase}
          \item[] \begin{subcase}[\v{2} = {\nil{}}] \v{} = \nil{}

             Parts 1 and 2 as above.
                    Part 3 holds because \v{} = \nil{}
                    and {\judgementtwo{}{\nil{}}{\Union{\Nil}{\Class}}}.
                  \end{subcase}
        \end{itemize}
      \end{subcase}
    %\item[]
    %  \begin{subcase}[\const{} = \throwconst]
    %    {\ArrowOne {\x{}} {\s{}}
    %                                                   {\t{f}}
    %                                                   {\filterset {\thenprop {\prop{f}}}
    %                                                               {\elseprop {\prop{f}}}}
    %                                                   {\object{f}}}
    %                                                   =
    %    {\ArrowOne {\x{}} {\Top{}}
    %                                  {\Bot{}}
    %                                  {\filterset {\botprop{}}
    %                                              {\botprop{}}}
    %                                  {\emptyobject{}}}

    %                                  Part 1 is trivial since \object{} = \emptyobject{} after substitution.
    %                                  Part 2 holds vacuously as both propositions are \botprop{} after substitution.
    %                                  Finally part 3 holds since {\judgementtwo{}{\hastype{\errorval{\v{2}}}{\Bot{}}}}.

    %  \end{subcase}
  \end{itemize}

\end{subcase}
\end{itemize}
\end{case}

\begin{case}[B-BetaMulti]
  \ 

  \begin{itemize}
    \item
  \opsem {\openv{}}
         {\e{1}}
         {\multi {\v{d}} {\disptable{}}},
    \item
  \opsem {\openv{}}
         {\e{2}}
         {\v{2}},
    \item
  \opsem {\openv{}}
         {\appexp {\v{d}} {\v{2}}}
         {\v{e}},
    \item
  \getmethod {\disptable{}}
             {\v{e}}
             {\v{l}}
             {\v{g}},
    \item
  \opsem {\openv{}}
         {\appexp {\v{g}} {\v{2}}}
         {\v{}},
       \item {\disptable{}} = {\curlymapvaloverright{\v{k}}{\v{v}}}
     \end{itemize}
     \begin{itemize}
       \item[]
\begin{subcase}[T-App]
  \ 

  \begin{itemize}
    \item
  \ep{} = {\appexp {\ep{1}} {\ep{2}}},
    \item
  \judgementrewrite {\propenv{}} {\ep{1}} {\ArrowOne {\x{}} {\s{}}
                                                       {\t{f}}
                                                       {\filterset {\thenprop {\prop{f}}}
                                                                   {\elseprop {\prop{f}}}}
                                                       {\object{f}}}
                {\filterset {\thenprop {\prop{1}}}
                            {\elseprop {\prop{1}}}}
                {\object{1}}
                {\e{1}},
    \item
  \judgementrewrite {\propenv{}}
                 {\ep{2}} {\s{}}
                 {\filterset {\thenprop {\prop{2}}}
                             {\elseprop {\prop{2}}}}
                 {\object{2}}
                 {\e{2}},
    \item
  \e{} = {\appexp {\e{1}} {\e{2}}},
    \item
      \issubtypein{}  {\replacefor {\t{f}} {\object{2}} {\x{}}}{\t{}},
    \item
      \inpropenv{\replacefor {\thenprop {\prop{f}}} {\object{2}} {\x{}}} {\thenprop {\prop{}}},
    \item
      \inpropenv{\replacefor {\elseprop {\prop{f}}} {\object{2}} {\x{}}} {\elseprop {\prop{}}},
    \item
      \issubobjin{}{\replacefor {\object{f}} {\object{2}} {\x{}}} {\object{}},
  \end{itemize}

     By inversion on \e{1} via T-Multi we know 
     \begin{itemize}
       \item
         \judgementrewrite{\propenv{}}{\ep{1}}{\MultiFntype{\s{t}}{\s{d}}}
                {\filterset {\thenprop {\prop{1}}}
                            {\elseprop {\prop{1}}}}
                {\object{1}}{\e{1}},
           
         \item \s{t} = {\ArrowOne {\x{}} {\s{}}
                                                       {\t{f}}
                                                       {\filterset {\thenprop {\prop{f}}}
                                                                   {\elseprop {\prop{f}}}}
                                                       {\object{f}}},
         \item \s{d} = {\ArrowOne {\x{}} {\s{}}
                                                       {\t{d}}
                                                       {\filterset {\thenprop {\prop{d}}}
                                                                   {\elseprop {\prop{d}}}}
                                                       {\object{d}}},
       \item
         \judgementtwo{}{\v{d}}{\s{d}}
              \item
  $\overrightarrow{\judgementtwo{}{\v{k}}{\Top{}}}$, and 
\item
  $\overrightarrow{\judgementtwo{}{\v{v}}{\s{t}}}$.
  \end{itemize}

  % FIXME do we really know this? seems obvious but the IH says something subtly different, might need
  % a lemma to bridge this. Same problem in T-IsA case.
  By the inductive hypothesis on 
  \opsem {\openv{}}
         {\e{2}}
         {\v{2}}
  we know 
  \judgementrewrite {\propenv{}} {\v{2}} {\s{}}
             {\filterset {\thenprop {\prop{2}}}
                         {\elseprop {\prop{2}}}}
                       {\object{2}}
                       {\v{2}}.

We then consider applying the evaluated argument to the dispatch function:
  \opsem {\openv{}}
         {\appexp {\v{d}} {\v{2}}}
         {\v{e}}.

         Since we can satisfy T-App with
       \begin{itemize}
         \item
         \judgementtwo{}{\v{d}}{\ArrowOne {\x{}} {\s{}}
                                                       {\t{d}}
                                                       {\filterset {\thenprop {\prop{d}}}
                                                                   {\elseprop {\prop{d}}}}
                                                       {\object{d}}}, and
         \item
  \judgementrewrite {\propenv{}} {\v{2}} {\s{}}
             {\filterset {\thenprop {\prop{2}}}
                         {\elseprop {\prop{2}}}}
                       {\object{2}}
                       {\v{2}}.
       \end{itemize}
       we can then apply the inductive hypothesis
       to derive
  \judgementrewrite {\propenv{}} {\v{e}} 
  {\replacefor{\t{d}}
              {\object{2}}
              {\x{}}}
             {\replacefor{\filterset {\thenprop {\prop{d}}}
                                     {\elseprop {\prop{d}}}}
                         {\object{2}}
                         {\x{}}}
                       {\replacefor
                         {\object{d}}
                         {\object{2}}
                         {\x{}}}
                       {\v{e}}.

 Now we consider how we choose which method to dispatch to.

 As 
  \getmethod {\disptable{}}
             {\v{e}}
             {\v{l}}
             {\v{g}}, by inversion on \getmethodliteral
             we know
   there exists exactly one \v{k} such that 
   \entryinmap{\mapvalentry{\v{k}}{\v{g}}}{\disptable{}} and \isaopsem{\v{e}}{\v{k}} = {\true{}}.

   By inversion we know T-DefMethod must have extended \disptable{} 
   with the well-typed dispatch value \v{k},
   thus {\judgementtwo{}{\v{k}}{\t{k}}}, and
   the well-typed method \v{g},
   so {\judgementtwo{}{\v{g}}{\s{t}}}.

  We can also prove that given
        \begin{itemize}
          \item
  \judgementrewrite {\propenv{}} {\v{e}} 
  {\replacefor{\t{d}}
              {\object{2}}
              {\x{}}}
             {\replacefor{\filterset {\thenprop {\prop{d}}}
                                     {\elseprop {\prop{d}}}}
                         {\object{2}}
                         {\x{}}}
                       {\replacefor
                         {\object{d}}
                         {\object{2}}
                         {\x{}}}
                       {\v{e}}.
    \item
  \judgementtwo {\propenv{}} {\v{k}} {\t{k}},
                     \item
        \isaopsem{\v{e}}{\v{k}} = {\true{}}, 
      \item
        \satisfies{\openv{}}{\propenv{}},
    \item
  \isacompare{\t{d}}
                       {\replacefor
                         {\object{d}}
                         {\object{2}}
                         {\x{}}}
  {\t{k}}{\filterset {\thenprop {\propp{}}} {\elseprop {\propp{}}}},
          \item
        \inpropenv{\thenprop{\propp{}}}{\thenprop{\propp{}}}, and
    \item
        \inpropenv{\elseprop{\propp{}}}{\elseprop{\propp{}}}.
        \end{itemize}
  we can apply \lemref{appendix:lemma:isa} to derive
  then {\satisfies{\openv{}}{\thenprop{\propp{}}}}.

   Now we consider applying the evaluated argument to the chosen method:
  \opsem {\openv{}}
         {\appexp {\v{g}} {\v{2}}}
         {\v{}}.

  By inversion via B-DefMethod we can assume {\v{g}} = {\abs{\x{}}{\s{}}{\e{b}}}, 
  ie. that we have chosen a method to dispatch to that is a closure.

  Because 
  \opsem {\openv{}}
         {\appexp {\v{g}} {\v{2}}}
         {\v{}}
         and
    \judgementtwo{\propenv{}}{\v{2}}{\s{}},
  by inversion via B-BetaClosure we know {\v{}} = {\replacefor{\e{b}}{\v{2}}{\x{}}}.

  With the following premises:
\begin{itemize}
  \item
{\judgementrewrite{{\propenv{}},{\thenprop{\propp{}}}}
                  {\replacefor{\ep{b}}{\v{2}}{\x{}}}
                  {\replacefor{\t{f}} {\object{2}}{\x{}}}
                  {\replacefor
                   {\filterset {\thenprop {\prop{f}}}
                               {\elseprop {\prop{f}}}}
                             {\object{2}}
                             {\x{}}}
                  {\replacefor
                          {\object{f}}
                             {\object{2}}
                             {\x{}}}
                  {\replacefor{\e{b}}{\v{2}}{\x{}}}
                        },
    \begin{itemize}
      \item From
{\judgementrewrite{{\propenv{}},{\isprop{\s{}}{\x{}}}}
                  {\e{b}}
                  {\t{f}}
               {\filterset {\thenprop {\prop{f}}}
                                       {\elseprop {\prop{f}}}}
                          {\object{f}}
                          {\e{b}}}
          via the inductive hypothesis on 
  \opsem {\openv{}}
         {\appexp {\abs{\x{}}{\s{}}{\e{b}}} {\v{2}}}
         {\v{}},
      \item then we can derive
{\judgementrewrite{{\propenv{}}}
                  {\replacefor{\ep{b}}{\v{2}}{\x{}}}
                  {\replacefor{\t{f}} {\object{2}}{\x{}}}
                  {\replacefor
                   {\filterset {\thenprop {\prop{f}}}
                               {\elseprop {\prop{f}}}}
                             {\object{2}}
                             {\x{}}}
                  {\replacefor
                          {\object{f}}
                             {\object{2}}
                             {\x{}}}
                  {\replacefor{\e{b}}{\v{2}}{\x{}}}
                        } via substitution and the fact that {\x{}} is fresh 
                        therefore \x{} $\not\in$ \fv{\propenv{}} so we do not need to substitution for \x{} in \propenv{}. %TODO lemma for this
                        
      \item 
        \satisfies{\openv{}}{\propenv{}, {\thenprop{\propp{}}}}
        because
        \satisfies{\openv{}}{\propenv{}} and {\satisfies{\openv{}}{\thenprop{\propp{}}}} via M-And.
    \end{itemize}
  \item
              \satisfies{\openv{}}{{\propenv{}},{\thenprop{\propp{}}}},
    \begin{itemize}
      \item From \satisfies{\openv{}}{\propenv{}} and \item {\satisfies{\openv{}}{\thenprop{\propp{}}}}  via M-And.
    \end{itemize}
           \item
\isconsistent{\openv{}}, and
           \item 
  \opsem {\openv{}}
         {\replacefor{\e{b}}{\v{2}}{\x{}}}
         {\v{}}.
\end{itemize}

we can apply the inductive hypothesis to satisfy our overall goal for this subcase.
\end{subcase}
     \end{itemize}
\end{case}

\begin{case}[BE-Beta1]
  \ 

  Reduces to an error.
\end{case}
\begin{case}[BE-Beta2]
  \ 

  Reduces to an error.
\end{case}
\begin{case}[BE-BetaClosure]
  \ 

  Reduces to an error.
\end{case}
\begin{case}[BE-BetaMulti1]
  \ 

  Reduces to an error.
\end{case}
\begin{case}[BE-BetaMulti2]
  \ 

  Reduces to an error.
\end{case}
\begin{case}[BE-Delta]
  \ 

  Reduces to an error.
\end{case}

\begin{case}[B-IsA]
        \opsem {\openv{}} {\e{1}} {\v{1}},
        \opsem {\openv{}} {\e{2}} {\v{2}},
        \isaopsem{\v{1}}{\v{2}} = {\v{}}


  \begin{itemize}
    \item[]
      \begin{subcase}[T-IsA]
  \ep{} = {\isaapp {\ep{1}} {\ep{2}}},
  \judgementrewrite {\propenv{}} {\ep{1}} {\t{1}}
             {\filterset {\thenprop {\prop{1}}}
                         {\elseprop {\prop{1}}}}
                       {\object{1}}
                       {\e{1}},
  \judgementrewrite {\propenv{}} {\ep{2}} {\t{2}}
             {\filterset {\thenprop {\prop{2}}}
                         {\elseprop {\prop{2}}}}
                       {\object{2}}
                       {\e{2}},
  \e{} = {\isaapp {\e{1}} {\e{2}}},
  \issubtypein{}{\Boolean}{\t{}},
  \isacompare{\t{1}}{\object{1}}{\t{2}}{\filterset {\thenprop {\propp{}}} {\elseprop {\propp{}}}},
  \inpropenv{\thenprop {\propp{}}}{\thenprop {\prop{}}},
  \inpropenv{\elseprop {\propp{}}}{\elseprop {\prop{}}},
  \issubobjin{}{\emptyobject{}}{\object{}}

        Part 1 holds trivially with \object{} = \emptyobject{}.

        For part 2, by the induction hypothesis on \e{1} and \e{2}
        we know
  \judgementrewrite {\propenv{}} {\v{1}} {\t{1}}
             {\filterset {\thenprop {\prop{1}}}
                         {\elseprop {\prop{1}}}}
                       {\object{1}}
                       {\v{1}} and
  \judgementrewrite {\propenv{}} {\v{2}} {\t{2}}
             {\filterset {\thenprop {\prop{2}}}
                         {\elseprop {\prop{2}}}}
                       {\object{2}}
                       {\v{2}},
                       so we can then apply
        \lemref{appendix:lemma:isa}
        to reach our goal.

        Part 3 holds because by the definition of \isaopsemliteral
        \v{} can only be \true\ or \false, 
        and since \judgementtwo{\propenv{}}{\true}{\t{}}
        and
        \judgementtwo{\propenv{}}{\false}{\t{}}
        we are done.
      \end{subcase}
  \end{itemize}
\end{case}

      \begin{case}[BE-IsA1]
        \opsem {\openv{}} {\e{1}} {\errorvalv{}}

        Trivially reduces to an error.
      \end{case}
      \begin{case}[BE-IsA2]
       \opsem {\openv{}} {\e{1}} {\v{1}},
       \opsem {\openv{}} {\e{2}} {\errorvalv{}}

        Trivially reduces to an error.
      \end{case}

\begin{case}[B-Get]
      $\opsem {\openv{}} {\e{m}}{\v{m}}$,
        $\v{m} = {\curlymap{\overrightarrow{({\v{a}}\ {\v{b}})}}}$,
         \opsem {\openv{}} {\e{k}} {\k{}},
         $\keyinmap{\k{}}{\curlymap{\overrightarrow{({\v{a}}\ {\v{b}})}}}$,
         \getmap{\curlymap{\overrightarrow{({\v{a}}\ {\v{b}})}}} {\k{}} = {\v{}}

  \begin{itemize}
    \item[]
      \begin{subcase}[T-GetHMap]
  \ep{} = {\getexp {\ep{m}} {\ep{k}}},
  \judgementrewrite {\propenv{}} {\ep{m}} {\Unionsplice {\overrightarrow {\HMapgeneric {\mandatory{}} {\absent{}}}}}
           {\filterset {\thenprop {\prop{m}}} {\elseprop {\prop{m}}}}
           {\object{m}}
           {\e{m}},
  \judgementtworewrite {\propenv{}} {\ep{k}} {\Value {k}}{\e{k}},
  \overr{\inmandatory{\k{}}{\t{i}}{\mandatory{}},}
  \e{} = {\getexp {\e{m}} {\e{k}}},
  \issubtypein{}{\Unionsplice {\overrightarrow {\t{i}}}}{\t{}} ,
  \thenprop{\prop{}} = {\topprop{}},
  \elseprop{\prop{}} = {\topprop{}},
  \issubobjin{}{\replacefor {\path {\keype{k}} {\x{}}}
                          {\object{m}}
                          {\x{}}}
                        {\object{}}


         To prove part 1 we consider two cases on the form of \object{m}: 
         \begin{itemize}
           \item
         if {\object{m}} = \emptyobject{}
         then \object{} = \emptyobject{} by substitution, which gives the desired result;
           \item
         if \object{m} = {\path {\pathelem{m}} {\x{m}}}
         then \issubobjin{}{\path {\keype{k}} {\object{m}}}{\object{}} by substitution.
         We note by the definition of path translation
         {\openv{}}({\path {\keype{k}} {\object{m}}}) =
         {\getexp {{\openv{}}(\object{m})}{\k{}}}
         and by the induction hypothesis on \e{m}
         {{\openv{}}(\object{m})} = {\curlymap{\overrightarrow{({\v{a}}\ {\v{b}})}}},
         which together imply 
         \inopenv {\openv{}} {\object{}} {\getexp {\curlymap{\overrightarrow{({\v{a}}\ {\v{b}})}}} {\k{}}}.
         Since this is the same form as B-Get, we can apply the premise
         \getmap{\curlymap{\overrightarrow{({\v{a}}\ {\v{b}})}}} {\k{}} = {\v{}}
         to derive \inopenv {\openv{}} {\object{}} {\v{}}.
         \end{itemize}
         
         Part 2 holds trivially as \thenprop{\prop{}} = {\topprop{}}
         and \elseprop{\prop{}} = {\topprop{}}.

         To prove part 3 we note that (by the induction hypothesis on \e{m})
         $\judgementtwo{}{\v{m}}{\Unionsplice{\overrightarrow {\HMapgeneric {\mandatory{}} {\absent{}}}}}$,
         where $\overrightarrow{\inmandatory{\k{}}{\t{i}}{\mandatory{}}}$, and 
         both
         $\keyinmap{\k{}}{\curlymap{\overrightarrow{({\v{a}}\ {\v{b}})}}}$
         and
         \getmap{\curlymap{\overrightarrow{({\v{a}}\ {\v{b}})}}} {\k{}} = {\v{}}
         imply \judgementtwo{}{\v{}}{\Unionsplice {\overrightarrow {\t{i}}}}.

      \end{subcase}
    \item[]
      \begin{subcase}[T-GetHMapAbsent]
  \ep{} = {\getexp {\ep{m}} {\ep{k}}},
  \judgementtworewrite {\propenv{}} {\ep{k}} {\Value {k}} {\e{k}},
  \judgementrewrite {\propenv{}} {\ep{m}} {\HMapgeneric {\mandatory{}} {\absent}}
           {\filterset {\thenprop {\prop{m}}} {\elseprop {\prop{m}}}}
           {\object{m}}
           {\e{m}},
  {\inabsent{\k{}}{\absent{}}},
  \e{} = {\getexp {\e{m}} {\e{k}}},
  \issubtypein{}{\Nil}{\t{}},
  \thenprop{\prop{}} = {\topprop{}},
  \elseprop{\prop{}} = {\topprop{}},
  \issubobjin{}{\replacefor
               {\path {\keype{k}} {\x{}}}
                          {\object{m}}
                          {\x{}}}
                        {\object{}}

       Unreachable subcase because 
         $\keyinmap{\k{}}{\curlymap{\overrightarrow{({\v{a}}\ {\v{b}})}}}$,
         contradicts
                {\inabsent{\k{}}{\absent{}}}.
      \end{subcase}
    \item[]
      \begin{subcase}[T-GetHMapPartialDefault]
  \ep{} = {\getexp {\ep{m}} {\ep{k}}},
  \judgementtworewrite {\propenv{}} {\ep{k}} {\Value {k}}{\e{k}},
 \judgementrewrite {\propenv{}} {\ep{m}} {\HMapp {\mandatory{}} {\absent}}
           {\filterset {\thenprop {\prop{m}}} {\elseprop {\prop{m}}}}
           {\object{m}}
           {\e{m}},
             {\notinmandatory{\k{}}{\t{}}{\mandatory{}}},
             {\notinabsent{\k{}}{\absent{}}},
  \e{} = {\getexp {\e{m}} {\e{k}}},
  \t{} = \Top,
  \thenprop{\prop{}} = {\topprop{}},
  \elseprop{\prop{}} = {\topprop{}},
  \issubobjin{}{\replacefor
               {\path {\keype{k}} {\x{}}}
                          {\object{m}}
                          {\x{}}}{\object{}}

         Parts 1 and 2 are the same as the B-Get subcase.
         Part 3 is trivial as \t{} = \Top.
      \end{subcase}
  \end{itemize}
\end{case}

\begin{case}[B-GetMissing]
        \v{} = \nil,
        $\opsem {\openv{}}
        {\e{m}} {\curlymap{\overrightarrow{({\v{a}}\ {\v{b}})}}}$,
       \opsem {\openv{}} {\e{k}} {\k{}},
       \keynotinmap{\k{}}{\curlymap{\overrightarrow{({\v{a}}\ {\v{b}})}}}

  \begin{itemize}
    \item[]
      \begin{subcase}[T-GetHMap]
  \ep{} = {\getexp {\ep{m}} {\ep{k}}},
  \judgementrewrite {\propenv{}} {\ep{m}} {\Unionsplice {\overrightarrow {\HMapgeneric {\mandatory{}} {\absent{}}}}}
           {\filterset {\thenprop {\prop{m}}} {\elseprop {\prop{m}}}}
           {\object{m}}
           {\e{m}},
  \judgementtworewrite {\propenv{}} {\ep{k}} {\Value {k}}{\e{k}},
  \overr{\inmandatory{\k{}}{\t{i}}{\mandatory{}},}
  \e{} = {\getexp {\e{m}} {\e{k}}},
  \issubtypein{}{\Unionsplice {\overrightarrow {\t{i}}}}{\t{}},
  \thenprop{\prop{}} = {\topprop{}},
  \elseprop{\prop{}} = {\topprop{}},
  \issubobjin{}{\replacefor {\path {\keype{k}} {\x{}}}
                          {\object{m}}
                          {\x{}}}{\object{}}

       Unreachable subcase because 
       \keynotinmap{\k{}}{\curlymap{\overrightarrow{({\v{a}}\ {\v{b}})}}}
       contradicts ${\inmandatory{\k{}}{\t{}}{\mandatory{}}}$.
      \end{subcase}
    \item[]
      \begin{subcase}[T-GetHMapAbsent]
  \ep{} = {\getexp {\ep{m}} {\ep{k}}},
  \judgementtworewrite {\propenv{}} {\ep{k}} {\Value {k}} {\e{k}},
  \judgementrewrite {\propenv{}} {\ep{m}} {\HMapgeneric {\mandatory{}} {\absent}}
           {\filterset {\thenprop {\prop{m}}} {\elseprop {\prop{m}}}}
           {\object{m}}
           {\e{m}},
  {\inabsent{\k{}}{\absent{}}},
  \e{} = {\getexp {\e{m}} {\e{k}}},
  \issubtypein{}{\Nil}{\t{}},
  \thenprop{\prop{}} = {\topprop{}},
  \elseprop{\prop{}} = {\topprop{}},
  \issubobjin{}{\replacefor
               {\path {\keype{k}} {\x{}}}
                          {\object{m}}
                          {\x{}}}{\object{}}

         To prove part 1 we consider two cases on the form of \object{m}: 
         \begin{itemize}
           \item
         if {\object{m}} = \emptyobject{}
         then \object{} = \emptyobject{} by substitution, which gives the desired result;
           \item
         if \object{m} = {\path {\pathelem{m}} {\x{m}}}
         then \issubobjin{}{\path {\keype{k}} {\object{m}}}{\object{}} by substitution.
         We note by the definition of path translation
         {\openv{}}({\path {\keype{k}} {\object{m}}}) =
         {\getexp {{\openv{}}(\object{m})}{\k{}}}
         and by the induction hypothesis on \e{m}
         {{\openv{}}(\object{m})} = {\curlymap{\overrightarrow{({\v{a}}\ {\v{b}})}}},
         which together imply 
         \inopenv {\openv{}} {\object{}} {\getexp {\curlymap{\overrightarrow{({\v{a}}\ {\v{b}})}}} {\k{}}}.
         Since this is the same form as B-GetMissing, we can apply the premise
        \v{} = \nil\ 
         to derive \inopenv {\openv{}} {\object{}} {\v{}}.
         \end{itemize}
         
         Part 2 holds trivially as \thenprop{\prop{}} = {\topprop{}}
         and \elseprop{\prop{}} = {\topprop{}}.

         To prove part 3 we note that \e{m} has type {\HMapgeneric {\mandatory{}} {\absent{}}}
         where {\inabsent{\k{}}{\absent{}}}, and
         the premises of B-GetMissing
         \keynotinmap{\k{}}{\curlymap{\overrightarrow{({\v{a}}\ {\v{b}})}}}
         and
          \v{} = \nil\ 
         tell us {\v{}} must be of type {\t{}}.
      \end{subcase}
    \item[]
      \begin{subcase}[T-GetHMapPartialDefault]
  \ep{} = {\getexp {\ep{m}} {\ep{k}}},
  \judgementtworewrite {\propenv{}} {\ep{k}} {\Value {k}}{\e{k}},
 \judgementrewrite {\propenv{}} {\ep{m}} {\HMapp {\mandatory{}} {\absent}}
           {\filterset {\thenprop {\prop{m}}} {\elseprop {\prop{m}}}}
           {\object{m}}
           {\e{m}},
             {\notinmandatory{\k{}}{\t{}}{\mandatory{}}},
             {\notinabsent{\k{}}{\absent{}}},
  \e{} = {\getexp {\e{m}} {\e{k}}},
  \t{} = \Top,
  \thenprop{\prop{}} = {\topprop{}},
  \elseprop{\prop{}} = {\topprop{}},
  \issubobjin{}{\replacefor
               {\path {\keype{k}} {\x{}}}
                          {\object{m}}
                          {\x{}}}{\object{}}

         Parts 1 and 2 are the same as the B-GetMissing subcase of T-GetHMapAbsent.
         Part 3 is trivial as \t{} = \Top.
      \end{subcase}
  \end{itemize}
\end{case}

\begin{case}[BE-Get1]
  \ 

  Reduces to an error.
\end{case}

\begin{case}[BE-Get2]
  \ 

  Reduces to an error.
\end{case}

\begin{case}[B-Assoc]
        \v{} = 
        {\extendmap{\curlymap{\overrightarrow{({\v{a}}\ {\v{b}})}}}
                {\k{}}{\v{v}}},
        \opsem {\openv{}}
        {\e{m}} {\curlymap{\overrightarrow{({\v{a}}\ {\v{b}})}}},
        \opsem {\openv{}} {\e{k}} {\k{}},
        \opsem {\openv{}} {\e{v}} {\v{v}}

  \begin{itemize}
    \item[]
      \begin{subcase}[T-AssocHMap]
  \judgementtworewrite {\propenv{}} {\ep{m}} {\HMapgeneric {\mandatory{}} {\absent}} {\e{m}},
  \judgementtworewrite {\propenv{}} {\ep{k}} {\Value{\k{}}}{\e{k}},
  \judgementtworewrite {\propenv{}} {\ep{v}} {\t{}}{\e{v}},
  {\k{}} $\not\in$ {\absent{}},
  \ep{} = {\assocexp {\ep{m}} {\ep{k}} {\ep{v}}},
  \e{} = {\assocexp {\e{m}} {\e{k}} {\e{v}}},
  \issubtypein{}{\HMapgeneric {\extendmandatoryset {\mandatory{}}{\k{}}{\t{}}} {\absent}}{\t{}},
  \thenprop{\prop{}} = {\topprop{}},
  \elseprop{\prop{}} = {\botprop{}},
  \object{} = \emptyobject{}

        Parts 1 and 2 hold for the same reasons as T-True.
        %TODO part 3
      \end{subcase}
  \end{itemize}
\end{case}

\begin{case}[BE-Assoc1]
  \ 

  Reduces to an error.
\end{case}

\begin{case}[BE-Assoc2]
  \ 

  Reduces to an error.
\end{case}

\begin{case}[BE-Assoc3]
  \ 

  Reduces to an error.
\end{case}

\begin{case}[B-IfFalse]
        \opsem {\openv{}} {\e{1}} {\false}
        \ \ \text{or}\ \ 
        \opsem {\openv{}} {\e{1}} {\nil},
        \opsem {\openv{}} {\e{3}} {\v{}}

  \begin{itemize}
    \item[]
      \begin{subcase}[T-If]
  \ep{} = {\ifexp {\ep1} {\ep2} {\ep3}},
  \judgementrewrite {\propenv{}} {\ep1} {\t{1}} {\filterset {\thenprop {\prop{1}}} {\elseprop {\prop{1}}}}
                 {\object{1}}
                 {\e1},
  \judgementrewrite {\propenv{}, {\thenprop {\prop{1}}}}
                 {\ep2} {\t{}} {\filterset {\thenprop {\prop{2}}} {\elseprop {\prop{2}}}}
                 {\object{}}
                 {\e2},
  \judgementrewrite {\propenv{}, {\elseprop {\prop{1}}}}
                 {\ep3} {\t{}} {\filterset {\thenprop {\prop{3}}} {\elseprop {\prop{3}}}}
                 {\object{}}
                 {\e3},
  \e{} = {\ifexp {\e1} {\e2} {\e3}},
  \inpropenv{\orprop {\thenprop {\prop{2}}} {\thenprop {\prop{3}}}}{\thenprop{\prop{}}},
  \inpropenv{\orprop {\elseprop {\prop{2}}} {\elseprop {\prop{3}}}}{\elseprop{\prop{}}}

              For part 1, either \object{} = \emptyobject{}, or \e{} evaluates to the
              result of \e{3}.

              To prove part 2, we consider two cases:
              \begin{itemize}
                \item if \isfalseval{\v{}}
                  then \e{3} evaluates to a false value so {\satisfies{\openv{}}{\elseprop {\prop{3}}}}, and thus
                  {\satisfies{\openv{}}{\orprop {\elseprop {\prop{2}}} {\elseprop {\prop{3}}}}} by M-Or, 
                \item otherwise
                  \istrueval{\v{}},
                  so \e{3} evaluates to a true value so {\satisfies{\openv{}}{\thenprop {\prop{3}}}}, and thus
                  {\satisfies{\openv{}}{\orprop {\thenprop {\prop{2}}} {\thenprop {\prop{3}}}}} by M-Or.
              \end{itemize}

              Part 3 is trivial as
              \opsem {\openv{}} {\e{3}} {\v{}}
              and {\judgementtwo{}{\v{}}{\t{}}} by the induction hypothesis on {\e{3}}.
      \end{subcase}
  \end{itemize}
\end{case}

\begin{case}[B-IfTrue]
        \opsem {\openv{}} {\e{1}} {\v{1}},
              ${\v{1}} \not= {\false}$,
              ${\v{1}} \not= {\nil}$,
              \opsem {\openv{}} {\e{2}} {\v{}}

  \begin{itemize}
    \item[]
      \begin{subcase}[T-If]
  \ep{} = {\ifexp {\ep1} {\ep2} {\ep3}},
  \judgementrewrite {\propenv{}} {\ep1} {\t{1}} {\filterset {\thenprop {\prop{1}}} {\elseprop {\prop{1}}}}
                 {\object{1}}
                 {\e1},
  \judgementrewrite {\propenv{}, {\thenprop {\prop{1}}}}
                 {\ep2} {\t{}} {\filterset {\thenprop {\prop{2}}} {\elseprop {\prop{2}}}}
                 {\object{}}
                 {\e2},
  \judgementrewrite {\propenv{}, {\elseprop {\prop{1}}}}
                 {\ep3} {\t{}} {\filterset {\thenprop {\prop{3}}} {\elseprop {\prop{3}}}}
                 {\object{}}
                 {\e3},
  \e{} = {\ifexp {\e1} {\e2} {\e3}},
  \inpropenv{\orprop {\thenprop {\prop{2}}} {\thenprop {\prop{3}}}}{\thenprop{\prop{}}},
  \inpropenv{\orprop {\elseprop {\prop{2}}} {\elseprop {\prop{3}}}}{\elseprop{\prop{}}}

              For part 1, either \object{} = \emptyobject{}, or \e{} evaluates to the
              result of \e{2}.

              To prove part 2, we consider two cases:
              \begin{itemize}
                \item if \isfalseval{\v{}}
                  then \e{2} evaluates to a false value so {\satisfies{\openv{}}{\elseprop {\prop{2}}}}, and thus
                  {\satisfies{\openv{}}{\orprop {\elseprop {\prop{2}}} {\elseprop {\prop{3}}}}} by M-Or, 
                \item otherwise
                  \istrueval{\v{}},
                  so \e{2} evaluates to a true value so {\satisfies{\openv{}}{\thenprop {\prop{2}}}}, and thus
                  {\satisfies{\openv{}}{\orprop {\thenprop {\prop{2}}} {\thenprop {\prop{3}}}}} by M-Or.
              \end{itemize}

              Part 3 is trivial as
              \opsem {\openv{}} {\e{2}} {\v{}}
              and {\judgementtwo{}{\v{}}{\t{}}} by the induction hypothesis on {\e{2}}.

      \end{subcase}
  \end{itemize}
\end{case}

\begin{case}[BE-If]
  \ 

  Reduces to an error.
\end{case}

\begin{case}[BE-IfFalse]
  \ 

  Reduces to an error.
\end{case}

\begin{case}[BE-IfTrue]
  \ 

  Reduces to an error.
\end{case}

\begin{case}[B-Let]
  \e{} = {\letexp {\x{}} {\e{1}} {\e{2}}},
        \opsem {\openv{}} {\e{1}} {\v{1}},
        \opsem {\extendopenv{\openv{}}{\x{}}{\v{1}}} {\e{2}} {\v{}}


  \begin{itemize}
    \item[]
      \begin{subcase}[T-Let]
  \ep{} = {\letexp {\x{}} {\ep{1}} {\ep{2}}},
  \judgementrewrite {\propenv{}} {\ep{1}} {\s{}} {\filterset {\thenprop {\prop{1}}} {\elseprop {\prop{1}}}}
             {\object{1}}
             {\ep{1}},
             \propp{} = {\impprop {\notprop {\falsy{}} {\x{}}} {\thenprop {\prop{1}}}},
             \proppp{} = {\impprop {\isprop {\falsy{}} {\x{}}} {\elseprop {\prop{1}}}},
  \judgementrewrite
       {\propenv{}, {\isprop {\s{}} {\x{}}},
         {\propp{}},
         {\proppp{}}}
             {\ep{2}} {\t{}} {\filterset {\thenprop {\prop{}}} {\elseprop {\prop{}}}}
             {\object{}} 
             {\e{2}}

        For all the following cases (with a reminder that \x{} is fresh)
        we apply the induction hypothesis on \e{2}. We justify this by noting
        that occurrences of \x{} inside \e{2} have the same type as \e{1} and 
        simulate the propositions of \e{1}
        because 
        \begin{itemize}
          \item
  \judgementrewrite
       {\propenv{}, {\isprop {\s{}} {\x{}}},
         {\propp{}},
         {\proppp{}}}
             {\ep{2}} {\t{}} {\filterset {\thenprop {\prop{}}} {\elseprop {\prop{}}}}
             {\object{}} 
             {\e{2}},
           \item
        \satisfies{\extendopenv{\openv{}}{\x{}}{\v{1}}}{\propenv{}, {\isprop {\s{}} {\x{}}}, \propp{}, \proppp{}},
           \item
        {\isconsistent{\extendopenv{\openv{}}{\x{}}{\v{1}}}},
        and
           \item
        \opsem {\extendopenv{\openv{}}{\x{}}{\v{1}}} {\e{2}} {\v{}}.
    \end{itemize}

        We prove parts 1, 2 and 3 by directly using the induction hypothesis on \e{2}.
      \end{subcase}
  \end{itemize}
\end{case}

\begin{case}[BE-Let]
  \ 
  
  Reduces to an error.
\end{case}

\begin{case}[B-Abs] 
        \v{} = {\closure {\openv{}} {\abs {\x{}} {\s{}} {\e{1}}}}

  \begin{itemize}
    \item[]
      \begin{subcase}[T-Clos]
  \ep{} = {\closure {\openv{}} {\abs {\x{}} {\s{}} {\e{1}}}},
  $\exists {\propenvp{}}. \satisfies{\openv{}}{\propenvp{}}$
  \ \text{and}\ 
\judgementrewrite {\propenvp{}} {\abs {\x{}} {\s{}} {\e{1}}} {\t{}}
                 {\filterset {\thenprop {\prop{f}}}
                             {\elseprop {\prop{f}}}}
                 {\object{f}}
                 {\abs {\x{}} {\s{}} {\e{1}}},
  \e{} = {\closure {\openv{}} {\abs {\x{}} {\s{}} {\e{1}}}},
                 {\thenprop{\prop{}}} = \topprop{},
                 {\elseprop{\prop{}}} = \botprop{},
                 {\object{}} = \emptyobject{}

        We assume some \propenvp{}, such that
        \begin{itemize}
          \item \satisfies{\openv{}}{\propenvp{}}
          \item \judgement {\propenvp{}} {\abs {\x{}} {\s{}} {\e{1}}} {\t{}}
                           {\filterset {\thenprop {\prop{}}}
                                       {\elseprop {\prop{}}}}
                           {\object{}}.
       \end{itemize}
       Note the last rule in the derivation of
          \judgement {\propenvp{}} {\abs {\x{}} {\s{}} {\e{1}}} {\t{}}
                           {\filterset {\thenprop {\prop{}}}
                                       {\elseprop {\prop{}}}}
                           {\object{}}
                           must be T-Abs, so 
                           {\thenprop {\prop{}}} = {\topprop{}},
                           {\elseprop {\prop{}}} = {\botprop{}}
                           and {\object{}} = {\emptyobject{}}.
         Thus parts 1 and 2 hold for the same reasons as T-True.
         Part 3 holds as \v{} has the same type as {\abs {\x{}} {\s{}} {\e{1}}}
         under \propenvp{}.

      \end{subcase} 
  \end{itemize}
\end{case}

\begin{case}[B-Abs]
        \v{} = ${\closure {\openv{}} {\abs {\x{}} {\s{}} {\e{1}}}}$,
          { \opsem {\openv{}}
                   {\abs {\x{}} {\t{}} {\e{1}}}
                   {\closure {\openv{}} {\abs {\x{}} {\s{}} {\e{1}}}}}

  \begin{itemize}
    \item[]
      %TODO
      \begin{subcase}[T-Abs]
  \ep{} = {\abs {\x{}} {\s{}} {\ep{1}}},
{ \judgementrewrite {\propenv{}, {\isprop {\s{}} {\x{}}}}
            {\ep{1}} {\t{}}
             {\filterset {\thenprop {\prop{1}}}
                         {\elseprop {\prop{1}}}}
             {\object{1}}
             {\e{1}}},
           \issubtypein{}
           {\ArrowOne {\x{}} {\s{}}
                      {\t{1}}
                      {\filterset {\thenprop {\prop{1}}}
                                  {\elseprop {\prop{1}}}}
                      {\object{1}}}
          {\t{}},
          \inpropenv{\topprop{}}{\thenprop{\prop{}}},
          \inpropenv{\botprop{}}{\elseprop{\prop{}}},
          {\object{}} = {\emptyobject{}}

        Parts 1 and 2 hold for the same reasons as T-True.
        Part 3 holds directly via T-Clos, since \v{} must be a closure.
      \end{subcase}
  \end{itemize}
\end{case}

\begin{case}[BE-Error]
        \opsem {\openv{}} {\e{}} {\errorval{\v{1}}}


  \begin{itemize}
    \item[]
      \begin{subcase}[T-Error] 
  \ep{} = \errorval{\v{1}},
  \e{} = \errorval{\v{1}},
  \t{} = \Bot,
  \thenprop{\prop{}} = \botprop{}, \elseprop{\prop{}} = \botprop{}, \object{} = \emptyobject{}

        Trivially reduces to an error.
      \end{subcase}
  \end{itemize}
\end{case}

\end{proof}

\end{lemma}

{\wrongtheorem{appendix}}

{\soundnesstheorem{appendix}}



%\Dchapter{Introduction}
\label{infer:chapter:intro}

%\emph{It is better to have 100 functions operate on one data structure than 10 functions on 10 data structures.}---Alan Perlis

Optional type systems~\cite{bracha2004pluggable} extend existing untyped languages
with type checking.
For example,
TypeScript~\cite{typescript} and Flow~\cite{flow} extend
JavaScript,
Hack~\cite{hack} extends PHP,
and
mypy~\cite{mypy} extends Python.
They typically support existing syntax and idioms of their target languages.

Transitioning to an optional type system
requires adding type annotations to existing code,
a significant manual burden.
This overhead has sparked interest in 
tooling to help create~\cite{saftoiu2010jstrace,pyannotate,typette18,An10dynamicinference,pytype} and
evolve~\cite{kristensen2017inference}
these annotations.

Clojure~\cite{Hic08} is an untyped language that compiles to the Java
Virtual Machine. Compared to languages already mentioned, it
strongly encourages programming with plain data structures, and is
a good example of implementing Perlis' advise in our opening quote.
Clojure provides
many functions and idioms around persistent, immutable hash-maps,
including literal map syntax \clj{\{k v ...\}}, interned \emph{keywords}
suitable both for map keys (e.g., \clj{:a}, \clj{:b}) and
functions that look themselves up in a map (e.g., \clj{(:a \{:a 1\}) => 1}),
and a suite of functions to deeply transform, manipulate, and validate
maps, with multimethods providing open extension.
Typed Clojure~\cite{bonnaire2016practical} is an optional type system for Clojure
designed to recognize these idioms given sufficient type annotations---which our
tool assists the programmer in writing.

\begin{figure}
\begin{cljlisting}
(defn nodes
  "Returns the number of nodes in a binary tree."
  [t] (case (:op t)
        :leaf 1
        :node (+ 1 (nodes (:left t))
                   (nodes (:right t)))))
(assert (= 3 (nodes
               {:op :node,
                :left {:op :leaf, :val 2},
                :right {:op :leaf, :val 3}})))
\end{cljlisting}
\caption{A typical use of maps in Clojure
to represent records,
that requires type annotations
to check with Typed Clojure.
Our tool %\texttt{core.typed.annotator}
will automatically annotate this program
with a useful recursive type (\figref{fig:infer:nodestype}).
}
\label{fig:infer:nodes}
\end{figure}

Maps in Clojure often replace records or objects,
demonstrated in \figref{fig:infer:nodes}:
instead of representing a binary tree with \texttt{Node} and \texttt{Leaf} classes,
they are encoded in maps
with an explicit keyword \emph{dispatch entry} (e.g., \clj{:op}) to distinguish
cases---\clj{\{:op :leaf, :val ...\}} for instances of \texttt{Leaf}, and
\clj{\{:op :node, :left ..., :right ...\}} for \texttt{Node}.

This emphasis on maps has far-reaching
implications for
Typed Clojure. %~\cite{bonnaire2016practical}, an optional type system for Clojure.
Types for maps (written
\clj{'\{:op ':leaf, :val ...\}} and
\clj{'\{:op ':node, :left ..., :right ...\}}
for our examples) combine with
ad-hoc union types and
equirecursive type aliases
in the type:

\begin{cljlisting}
(defalias Tree
  (U '{:op ':node, :left Tree, :right Tree}
     '{:op ':leaf, :val Int}))
\end{cljlisting}

Porting \figref{fig:infer:nodes} to Typed Clojure involves writing a type definition
\clj{Tree}, and annotating \clj{nodes} as
\begin{cljlisting}
(ann nodes [Tree -> Int])
\end{cljlisting}
Existing annotation tools fail to
generate these types since
only classes are given
recursive types,
%are inferred via class invariants,
%like a \texttt{Tree} class with fields of type \texttt{Tree},
%however these approaches depend on classes to infer
%recursive types
and so cannot infer recursive types for plain data,
such as JSON or heterogeneous dictionaries.
%
To demonstrate the current state-of-the-art in automatic annotation tools for optional type systems,
we transliterate \figref{fig:infer:nodes} to JavaScript using plain objects. 
We use TypeWiz~\cite{typewiz} to generate TypeScript annotations via dynamic analysis,
as it is well maintained and generates comparable annotations
to similar tools~\cite{saftoiu2010jstrace,pyannotate,typette18,An10dynamicinference,pytype,kristensen2017inference}.

\begin{figure}
\begin{lstlisting}[language=JavaScript]
function nodes(t: {left: {op: string, val: number},
                   op: string,
                   right: {op: string, val: number}}
                | {op: string, val: number}) ...
\end{lstlisting}
%\begin{lstlisting}[language=JavaScript]
%function nodes(t: @{left: {op: string, val: number},
%                   op: string,
%                   right: {op: string, val: number}}
%                | {op: string, val: number}@) {
%  switch t.op {
%    case "node": 
%      return 1 + nodes(t.left) + nodes(t.right);
%    case "leaf": return 1;
%    default: throw t.op;
%  }
%}
%nodes({op: "node",
%       left: {op: "leaf", val: 1},
%       right: {op: "leaf", val: 2}}); //test
%\end{lstlisting}
\caption{TypeWiz's TypeScript annotations for \figref{fig:infer:nodes}.}
\label{fig:infer:typewiz}
\end{figure}

\figref{fig:infer:typewiz} shows the actual output of TypeWiz for a JavaScript translation of \figref{fig:infer:nodes}.
Unfortunately, the annotation is too specific: it only accepts trees of height 1 or 2.
For fair comparison, even a class-based translation to JavaScript with
a common \js{nodes} method yields a shallow annotation:
\begin{lstlisting}[language=JavaScript]
class Node {              class Leaf { 
  public left: Leaf;        public data: number;
  public right: Leaf;
  ...                       ...
}                         }
\end{lstlisting}
%\begin{lstlisting}[language=JavaScript]
%class Node {                class Leaf { 
%  public left: @Leaf@;          public data: @number@;
%  public right: @Leaf@;
%  //constructors omitted
%  nodes() {                   nodes() {
%    return 1 +                  return 1;
%     this.left.nodes() +      }
%     this.right.nodes();    }
%  }
%}
%new Node(new Leaf(1), new Leaf(2)).nodes(); //test
%\end{lstlisting}

Since TypeWiz uses dynamic analysis, it is faithfully
providing the exact types that are observed at runtime.
Unfortunately, incrementally better test coverage does not quickly converge
to useful annotations.
For example, if we add an example of a nested \js{Node}
on just the left branch
%\begin{lstlisting}[language=JavaScript]
%new Node(new Node(...), new Leaf(...)).nodes();
%\end{lstlisting}
the annotation for the plain objects version is still not recursive (alarmingly, it instead
grows linearly in the number of nodes), and the class-based
version helpfully updates the type of \js{left} to \js{Leaf|Node},
but \js{right} still remains \js{Leaf}.
%\begin{lstlisting}[language=JavaScript]
%class Node {               class Leaf { 
%  public left: Leaf@|Node@;    public data: number;
%  public right: Leaf;        ...
%  ...                      }
%}
%\end{lstlisting}

%In Python, the PyType tool for generating mypy generates an
%even less interesting annotation.
%\begin{lstlisting}[language=python]
%def nodes(t) -> int: ...
%\end{lstlisting}
%In contrast, encoding \clj{nodes} as a method gives PyType
%\begin{lstlisting}
%class Leaf:
%    val = ...  # type: Any
%    def __init__(self, val) -> None: ...
%    def nodes(self) -> int: ...
%class Node:
%    left = ...  # type: Any
%    right = ...  # type: Any
%    def __init__(self, left, right) -> None: ...
%    def nodes(self) -> Any: ...
%\end{lstlisting}

%
%Existing automated annotation approaches cannot derive
%any further structure to the input of \clj{nodes}, other
%than its explicit runtime representation.
%In Typed Clojure~\cite{bonnaire2016practical} notation,
%they infer:
%
%\begin{cljlisting}
%(ann nodes ['{:op Keyword,
%              :left '{:op Keyword :val Int},
%              :right '{:op Keyword :val Int}}
%            -> Int])
%\end{cljlisting}

On the other hand,
our approach recognizes \figref{fig:infer:nodes} as traversing recursively defined
data with two cases, distinguished by the \clj{:op}
entry (\figref{fig:infer:nodestype}).

\begin{figure}
\begin{cljlisting}
(defalias Op 
  (U '{:op ':node, :left Op, :right Op}
     '{:op ':leaf, :val Int}))
(ann nodes [Op -> Int])
\end{cljlisting}
  \caption{Our tool's Typed Clojure annotation of \figref{fig:infer:nodes}.
  \clj{defalias} introduces an equirecursive type alias,
  \clj{U} is a set-theoretic union type constructor,
  and \clj{':node} is a singleton type containing just the keyword value
  \clj{:node}.
}
\label{fig:infer:nodestype}
\end{figure}

Our approach is sensitive enough to
to compute optional keys for each constructor.
Adding a unit test that includes a \clj{:val}
entry in a \clj{:node} yields the annotation
%For example, the following call uses a \clj{:val}
%entry for in a \clj{:node} map:
%\begin{cljlisting}
%(nodes {:op :node,
%        :left {:op :node, :val 42, ...},
%        :right {:op :leaf, ...}})
%\end{cljlisting}
\begin{cljlisting}
(defalias Op 
  (U (HMap :mandatory
           {:op ':node, :left Op, :right Op}
           :optional {:val Int})
     '{:op ':leaf, :val Int}))
\end{cljlisting}

Furthermore, we aggressively combine recursive data used
in the same file.
Given a 3-way node example with the key \clj{:node3},
used as input to a distinct function \clj{nodes'},
our tool generates the combined type shown in \figref{fig:infer:node3}.
%Say \clj{nodes'}
%is similar to \clj{nodes},
%but
%whose unit tests also use a 3-way node \clj{:node3}.
%%\begin{cljlisting}
%%(nodes' {:op :node3,
%%         :left {:op :leaf, :val 1}
%%         :mid {:op :leaf, :val 2}
%%         :right {:op :leaf, :val 3}})
%%\end{cljlisting}
%The information derived from observing the executions
%of both functions are combined based on
%the common dispatch key (\figref{fig:infer:node3}).

\begin{figure}
\begin{cljlisting}
(defalias Op 
  (U (HMap :mandatory
           {:op ':node, :left Op, :right Op}
           :optional {:val Int})
     '{:op ':node3, :left Op, :mid Op, :right Op}
     '{:op ':leaf, :val Int}))
(ann nodes [Op -> Int])
(ann nodes' [Op -> Int])
\end{cljlisting}
\caption{Optional entries and combined information across functions.
  \clj{HMap} is the expanded heterogeneous map type constructor, specifying both
  mandatory and optional entries.
  }
\label{fig:infer:node3}
\end{figure}

%, where plain
%heterogeneous hash-maps are idiomatic over predefined classes or records.
%Typed Clojure, in turn, has extensive support for specifying and checking
%heterogeneous maps.
%However, the high annotation burden has put off industrial users in practice.
%

% - set the scene for inferring types
%   - Typed Clojure
%   - optional/gradual typing requires annotations

% Typed Clojure has interesting idioms.
% Similarly we have to take into account specific idioms,
% Must deal with same realities as ESOP
% - need HMap's because we know.
% What does TypeScript auto ann do with `nodes`?
% - not class based
% - try python systems
% 
% - need to guess nominal names
% - need to guess grouping
% - nominal types: don't need to think about recursion
%  - just give a name, indirection is directed in a big table of names
%  - humans are used to referring things by name

% other 

% a class invariant is a recursive property "for free"
% - check locally, get recursion for free

% nothing for free in our setting

% we have names but not nominal types

%This paper starts to address a major usability flaw
%for gradually and optionally typed languages:
%writing type annotations is a manual process.
%
%Take \texttt{nodes} (\figref{infer:fig:nodes}),
%written in Clojure.
%As is good style, it comes with a unit test.
%Our goal is to \textit{generate} Typed Clojure~\cite{bonnaire2016practical}
%annotations
%for this function, relieving most of the annotation
%burden.

%\begin{figure}
%\begin{cljlisting}
%(defn nodes [t]
%  (case (:op t)
%    :leaf 1
%    :node (+ 1 (nodes (:left t))
%               (nodes (:right t)))))
%(assert (= 3 (nodes
%               {:op :node 
%                :left {:op :leaf :val 2}
%                :right {:op :leaf :val 3}})))
%\end{cljlisting}
%\caption{A Clojure function counting tree nodes, with test.}
%\label{infer:fig:nodes}
%\end{figure}

%\begin{figure}
%\begin{cljlisting}
%(defalias Op (U '{:op ':node ':left Op ':right Op}
%                '{:op ':leaf ':val Int}))
%(ann nodes [Op -> Int])
%\end{cljlisting}
%\caption{Automatically inferred Typed Clojure annotations.}
%\label{infer:fig:nodestype}
%\end{figure}

%Our approach features several stages.
%First, we \textit{instrument} top-level functions
%(Section \ref{instrument-TODO}),
%then run the unit tests and \textit{track}
%how they are used at runtime
%(Section \ref{track-TODO}).
%At this point, we have a preliminary
%annotation:
%
%\begin{cljlisting}
%(ann vertices ['{:op ':node,
%                 :left '{:op ':leaf :val Int},
%                 :right '{:op ':leaf :val Int}}
%               -> Int])
%\end{cljlisting}
%
%This type is too specific---trees are recursively
%defined---we \textit{squash} types to be
%recursive from example unrollings (Section \ref{recursive-TODO}):
%
%\begin{cljlisting}
%(defalias Op 
%  (U '{:op ':node, :left Op, :right Op}
%     '{:op ':leaf, :val Int}))
%(ann vertices [Op -> Int])
%\end{cljlisting}

%\begin{Verbatim}
%(declare Node Leaf)
%(defalias Op (U Node Leaf))
%(defalias Node 
%  '{:op ':node :left Op :right Op})
%(defalias Leaf '{:op ':leaf :val int})
%(ann verbatim [Op -> Int]})
%\end{Verbatim}
%
%If \texttt{Op} is used in multiple positions
%in the program, local recursive types are redundant.
%In this paper, we name and \textit{merge} recursive
%types, reusing them in annotations.
% 
%\begin{cljlisting}
%(ann vertices [Op -> Int])
%(ann sum-tree [Op Op -> Op])
%\end{cljlisting}
% 
%If minor variants of the recursive types occur
%across a program,
%we use \textit{optional} entries%~\cite{typed-clojure}
%to reduce redundancy (Section \ref{optional-merge-TODO}).
% 
%\begin{cljlisting}
%(defalias Op 
%  (U '{:op ':node, :left Op, :right Op}
%     (HMap :mandatory {:op ':leaf :val Int}
%           :optional {:label Str})))
%\end{cljlisting}
%
%After inserting these annotations, we can run the
%type checker over them to check their usefulness.
%We found annotations to be readable and minimize
%redundancy compared to hand-written annotations
%(Section \ref{experiment1}).
%Minimal changes were needed to successfully type check
%functions with the generated annotations,
%mostly consisting of local function and loop annotations,
%and renaming of type aliases
%(Section \ref{experiment2}).
%Generating and running \textit{tests} improved the quality
%of type annotations by exercising more paths through the
%program (Section \ref{experiment3}).

%Several open questions remain.
%Automatically
%drawing the typed-untyped boundary in gradual typing
%would mean less manual casts are needed.
%(Section \ref{boundaries}).

%The Clojure programming language has several verification
%systems that require annotating your programs.
%Typed Clojure is a type system that supports many Clojure
%idioms. Here, we must provide type annotations for
%top level variables, local functions, and invoked libraries.
%Clojure.spec is a pseudo contract system
%that can also generate tests.
%Similarly, specifications (``specs'') must be provided
%for all top level variables.
%
%These annotations are useful for learning about our programs,
%but they can be burdensome to write and maintain.
%Currently, one must reverse engineer annotations
%by visual analysis of the source code.
%
%In this paper, we present a tool that automatically
%generates annotations, based on the tests already present
%in idiomatic Clojure programs.
%These annotations are readable, compact, feature good
%names, and recover recursively defined records.
%There is no guarantee the generated annotations will
%immediately type check, however.
%
%Our goal is to minimize the difference needed
%to type check programs from the generated annotations.
%We envision programmers running our tool, generating
%a few dozen lines of annotations, and only a fraction
%of them should need manual changing to actually type
%check a program.

% - give introductory example
%   - generate types + specs
%   - show delta needed to typecheck
% - enumerate our contributions
% - signpost the rest of the paper


\Dsection*{Contributions}
\begin{itemize}
\item We outline a generalized approach to automatically
  generating type annotations (\secref{infer:sec:overview}).
%\item
%  Our main contribution is a robust, easy to use, open source tool that 
%  Clojure programmers can use to help learn about and specify 
%  their programs.
\item
  We describe a novel approach to reconstructing recursively
  defined structural records from fully unrolled examples
  in a formal model of our inference algorithm (\secref{infer:sec:formalism}).
\item
  We show how to extend our approach with space-efficient and lazy
  runtime tracking (\secref{infer:sec:extensions}).
\item
  We report our experience using this algorithm to generate
  types, tests, and contracts on several
  Clojure libraries and programs (\secref{infer:chap:evaluation}).
\end{itemize}


Consider the exercise of counting binary tree nodes using JavaScript.
With a class-based tree representation, we naturally add a method
to each kind of node like so.

\begin{lstlisting}[language=JavaScript]
class Node { nodes() { return 1 + this.left.nodes() + this.right.nodes(); } }
class Leaf { nodes() { return 1; } }
new Node(new Leaf(1), new Leaf(2)).nodes(); //=> 3 (constructors implicit)
\end{lstlisting}

An alternative ``ad-hoc'' representation uses plain JavaScript Objects
with explicit tags.
Then, the method becomes a recursive function that explicitly takes a tree as input.
We trade the extensiblility and (presumably) speed of a method for a simple, reversible serialization to JSON.

\begin{lstlisting}[language=JavaScript]
function nodes(t) { switch t.op { 
                      case "node": return 1 + nodes(t.left) + nodes(t.right);
                      case "leaf": return 1; } }
nodes({op: "node", left:{op: "leaf", val: 1}, right:{op: "leaf", val: 2}})//=>3
\end{lstlisting}

Now, consider the problem of inferring type annotations for these programs.
The class-based representation is idiomatic to popular dynamic languages
like JavaScript and Python, and so many existing solutions support it.
%~\infercitep{saftoiu2010jstrace,pyannotate,typette18,An10dynamicinference,pytype,kristensen2017inference}
For example, TypeWiz~\infercitep{typewiz} uses dynamic analysis to generate
the following TypeScript annotations from the above example execution of \js{nodes}.

\begin{lstlisting}[language=JavaScript]
class Node { public left: @Leaf@; public right: @Leaf@; ... }
class Leaf { public val: @number@; ... }
\end{lstlisting}

The intuition behind inferring such a type is straightforward.
For example, an instance of \js{Leaf} was observed in \js{Node}'s \js{left} field,
and so the nominal type \js{Leaf} is used for its annotation.

The second ``ad-hoc'' style of programming seems peculiar in JavaScript, Python, and, indeed,
object-oriented style in general.
Correspondingly, existing state-of-the-art automatic annotation tools are not designed
to support them.
There are several ways to trivially handle such cases.
Some enumerate the tree representation ``verbatim'' in a union, like TypeWiz~\infercitep{typewiz}.

\begin{lstlisting}[language=JavaScript]
function nodes(t: {left: {op: string, val: number}, op: string,
                   right: {op: string, val: number}}
                | {op: string, val: number}) ...
\end{lstlisting}

Others ``discard'' most (or all) structure, like Typette~\infercitep{typette18} 
and PyType~\infercitep{pytype} for Python.

\begin{lstlisting}[language=Python]
def nodes(t: Dict[(Sequence, object)]) -> int: ... # Typette
def nodes(t) -> int: ...                           # PyType
\end{lstlisting}

Each annotation is clearly insufficient to meaningfully check both the function definition
and valid usages. To show a desirable annotation for the ``ad-hoc'' program,
we port it to Clojure~\infercitep{Hic08}, where it
enjoys full support from the built-in runtime verification library
clojure.spec and primary optional type system Typed Clojure~\infercitep{bonnaire2016practical}.

\begin{cljlisting}
(defn nodes [t] (case (:op t)
                  :node (+ 1 (nodes (:left t)) (nodes (:right t)))
                  :leaf 1))
(nodes {:op :node, :left {:op :leaf, :val 1}, :right {:op :leaf, :val 2}}) ;=>3
\end{cljlisting}

Making this style viable requires a harmony of language features, in particular to
support programming with functions and immutable values, but
none of which comes at the expense of object-orientation. Clojure is hosted on the Java Virtual Machine
and has full interoperability with Java objects and classes---even Clojure's core design embraces
object-orientation by exposing a collection of Java interfaces to create new kinds of data structures.
The \clj{\{k v ...\}} syntax creates a persistent and immutable Hash Array Mapped Trie~\cite{bagwell2001ideal},
which can be efficiently manipulated by dozens of built-in functions.
The leading colon syntax like \clj{:op} creates an interned \emph{keyword}, which are ideal for map keys
for their fast equality checks, and also look themselves up in maps when used as functions
(e.g., \clj{(:op t)} is like JavaScript's \js{t.op}).
\emph{Multimethods} regain the extensibility we lost when abandoning methods, like the following.

\begin{cljlisting}
(defmulti nodes-mm :op)
(defmethod nodes-mm :node [t] (+ 1 (nodes-mm (:left t)) (nodes-mm (:right t))))
(defmethod nodes-mm :leaf [t] 1)
\end{cljlisting}

On the type system side, Typed Clojure supports a variety of heterogeneous types,
in particular for maps, along with occurrence typing \cite{TF10} to follow local control flow.
Many key features come together to represent our ``ad-hoc'' binary tree as the following type.

\begin{cljlisting}
(defalias Tree
  (U '{:op ':node, :left Tree, :right Tree}
     '{:op ':leaf, :val Int}))
\end{cljlisting}

The \clj{defalias} form introduces an equi-recursive type alias \clj{Tree},
\clj{U} a union type, \clj{'\{:kw Type ...\}} for heterogeneous keyword map types,
and \clj{':node} for keyword singleton types.
With the following function annotation, Typed Clojure can intelligently type check
the definition and usages of \clj{nodes}.

\begin{cljlisting}
(ann nodes [Tree -> Int])
\end{cljlisting}

This (manually written) Typed Clojure annotation involving \clj{Tree}
is significantly different from TypeWiz's ``verbatim'' annotation for \js{nodes}.
First, it is recursive, and so supports trees of arbitrary depth (TypeWiz's annotation
supports trees of height $<3$).
Second, it uses singleton types \clj{':leaf} and \clj{':node} to distinguish each case
(TypeWiz upcasts \js{"leaf"} and \js{"node"} to \js{string}).
Third, the tree type is factored out under a name to enhance readability and reusability.
On the other end of the spectrum, the ``discarding'' annotations of Typette and PyType
are too imprecise to use meaningfully (they include trees of arbitrary depth, but
also many other values).

The challenge we overcome in this research is to automatically generate
annotations like Typed Clojure's \clj{Tree}, in such a way that the ease of manual amendment is
only mildly reduced by unresolvable ambiguities and incomplete data collection.

%This research presents an approach based on dynamic analysis to automatically inferring 
%recursive, structural, and compact annotations
%like the Typed Clojure annotation for \clj{nodes} along with \clj{Tree}.
%We show how our approach tolerates unresolvable ambiguities and incomplete data collection
%to generate close-enough annotations that are often straightforward to manually amend.

\begin{figure*}
  $$
  \begin{altgrammar}
    \val{} &::=& \num
       \alt {\kw{}}
       \alt [\lambda \xvar{}. \exp{}, \rho]
       \alt \{\overrightarrow{\kw{}\ \val{}}\}
       &\mbox{Values} \\
   \exp{} &::=& \xvar{}
       \alt \val
       \alt \trackE{\exp}{\pi{}}
       \alt \lambda \xvar{}. \exp{}
       \alt \{\overrightarrow{\exp\ \exp}\}
       \alt (\exp{}\ \exp{})
       &\mbox{Expressions} \\
    \rho &::=& \{\overrightarrow{x \mapsto \val}\}
       &\mbox{Runtime environments} \\
   \pth{}
      &::=& x 
       \alt \dompe{}
       \alt \rngpe{}
       \alt \inferkeype{\overrightarrow{\kw{}}}{\kw{}}
       &\mbox{Path Elements} \\
     \pi &::=& \overrightarrow{\pth{}}
       &\mbox{Paths} \\
       \res{}
      &::=& \{\overrightarrow{\path{} : \tau}\}
      &\mbox{Inference results} \\
    \ty{}, \sigma
      &::=& \IntT{}
       \alt [\tau \rightarrow \tau]
       %\alt [\overrightarrow{\tau} \tau * \rightarrow \tau]
       \alt \{\overrightarrow{\kw{}\ \tau}\}
       \\
       &\alt& \HMaptwo{\{\overrightarrow{\kw{}\ \tau}\}}{\{\overrightarrow{\kw{}\ \tau}\}}
       \alt \tau \cup \tau
       \alt \alias{} % type alias
       \alt \UnknownT{}
      &\mbox{Types} \\
    \tenv{} &::=& \{\overrightarrow{\xvar{} : \tau}\}
      &\mbox{Type environments} \\
    \aenv{} &::=& \{\overrightarrow{\alias{} \mapsto \tau}\}
      &\mbox{Type alias environments} \\
    \atenv{} &::=& (\aenv{}, \tenv{})
      &\mbox{Combined environments} \\
  \end{altgrammar}
  $$
\end{figure*}

\begin{figure*}
\begin{mathpar}
%\infer [B-Var]
%{ track(\rho(x), [x]) = v\ ; res }
%{ \rho \vdash x \Downarrow v\ ; res}

% track is inserted manually
\infer [B-Var]
{}
{\bigstep{\rho}{\xvar{}}{\rho(\xvar{})}{\{\}}}

\infer [B-Track]
{ \bigstep{\rho}{\exp{}}{\val{}}{\res{}} \\
  \trackmeta{}(\val{}, \inferpath{}) = \val{}'\ ; \res{}' }
{ \bigstep{\rho}{\trackE{\exp{}}{\inferpath{}}}{\val{}'}{\res{} \sqcup \res{}'} }

\infer [B-App]
{ \bigstep{\rho}{\exp1}{[\lambda \xvar{}. \exp{}, \rho']}{\res1} \\\\
  \bigstep{\rho}{\exp2}{\val{}}{\res2} \\\\
  \bigstep{\rho', \xvar{} \mapsto \val{}}{\exp{}}{\val{}'}{\res3} \\
}
{ \bigstep{\rho}{(\exp1\ \exp2)}{\val{}'}{\res1 \sqcup \res2 \sqcup \res3}}

\infer [B-Clos]
{}
{ \bigstep{\rho}{\lambda \xvar{}. \exp{}}{[\lambda \overrightarrow{\xvar{}} . \exp{}, \rho ]}{\{\}}}

\infer [B-Val]
{}
{ \bigstep{\rho}{\val{}}{\val{}}{\{\}} }

%\infer [B-Map]
%{ \overrightarrow{\rho \vdash e_1 \Downarrow v_1\ ; res_1 \ \ 
%  \rho \vdash e_2 \Downarrow v_2\ ; res_2} \\
%}
%{ \rho \vdash \{\overrightarrow{e_1\ e_2}\} \Downarrow \{\overrightarrow{v_1\ v_2}\}\ ; \overrightarrow{res_1} \sqcup \overrightarrow{res_2}}

\infer [B-Get]
{ \bigstep{\rho}{\exp1}{\{ \overrightarrow{\kw{}\ \val{}} \}}{\res1} \\
  \bigstep{\rho}{\exp2}{\kw1}{\res2} \\
}
{ \bigstep{\rho}{(get\ \exp1\ \exp2)}{\{ \overrightarrow{\kw{}\ \val{}} \}[\kw1]}{\res1 \sqcup \res2}
}

\infer [B-Assoc]
{ 
  \bigstep{\rho}{\exp1}{\{ \overrightarrow{\kw{}\ \val{}}\}}{\res1} \\
  \bigstep{\rho}{\exp2}{\kw1}{\res2} \\
  \bigstep{\rho}{\exp3}{\val{}}{\res3} \\
}
{ \bigstep{\rho}{\assocE{\exp1}{\exp2}{\exp3}}{\{ \overrightarrow{\kw{}\ \val{}} \}[\kw1 \mapsto \val{}]}{\res1 \sqcup \res2 \sqcup \res3}
}
\end{mathpar}
  \label{infer:semantics}
  \caption{Runtime instrumentation semantics for the automatic annotation tool}
\end{figure*}

\Dchapter{Overview}
\label{infer:sec:overview}

Now that we have introduced the problem,
we can flesh out our philosophy and overall
approach.
%
At a high level, there are two phases to
generating types---\textbf{collection} and
\textbf{inference}.
%
The collection phase (\secref{infer:sec:formal:collection-phase}),
gathers observations about a running program.
This is achieved by instrumenting the program and exercising
it, usually by running its unit tests.
%
The inference phase (\secref{infer:sec:formal:inference-phase})
uses these runtime observations to generate the final type annotations,
with recursive types, optional entries, and good names.

\figref{infer:fig:cljs} demonstrates sample output from our
tool. It shows the (inferred) recursively defined type
for the ClojureScript compiler's AST format.
Similar to our opening example, it uses the \clj{:op}
key to disambiguate between (16) cases, and has recursive
references (\clj{Op}).

We just present the first 4 cases.
The first case \clj{':binding} has 4 required
and 8 optional entries.
The \clj{:info} and \clj{:env} entries refer to
other \clj{HMap} type aliases generated by the tool.
Similar to \clj{:op},
the \clj{:local} entry maps to a keyword singleton
type,
however our tool wisely chose to cluster types 
based on the \clj{:op} entry since it is common to all cases.


%is split into 
%\textbf{instrumentation}, involves
%rewriting the code we wish to annotate such
%that we can record its runtime behavior.
%In this phase, we require the programmer to
%indicate which code we wish to generate types
%for, in advance.
%
%Once instrumented, we observe our running program
%via \textbf{runtime tracking}. To exercise our programs,
%we usually run their unit tests, generative tests,
%or just normally run the program (eg. to generate types for
%a game, we can simply play the game for a few minutes).
%We accumulate the results of tracking via \textbf{paths}.
%If we think of types as trees and supply a label
%for each branching path, our inference results
%specify the type down a particular path in this tree.
%Both phases are described in ,
%collectively as the \emph{collection phase}.
%
%Finally, the information collected during runtime tracking
%is combined into annotations by our \textbf{inference algorithm}.
%We first combine all inference result into a large tree of
%types. If we were to convert this tree into annotations directly,
%our annotations would be too specific---they would be too
%deep and fine-grained.
%Instead, our algorithm iterates over several passes to massage
%this tree, generating good names for the nodes, compacting similar
%types across the tree, and
%eventually converting the tree into a directed graph by reconstructing
%recursive types.
%This \emph{inference phase} is described in 
%\secref{infer:sec:formal:inference-phase}.

\begin{figure}
  % indented so line numbers can line up more tastefully
\begin{cljlistingnumbered}
  (defalias Op
    (U (HMap
        :mandatory
        {:op ':binding,
         :info (U NameShadowMap
                  FnScopeFnSelfNameNsMap),
         :local (U ':arg ':let ':fn),
         :name Sym}
        :optional
        {:arg-id Int,
         :binding-form? Boolean,
         :column Int,
         :env ColumnLineContextMap,
         :init Op,
         :line Int,
         :shadow (U nil Op),
         :tag Any})
      '{:op ':const,
        :env HMap49305,
        :form (U nil Int ':a),
        :tag Sym,
        :val (U nil Int ':a)}
      '{:op ':do,
        :body? Any,
        :children Any,
        :env HMap49305,
        :form (Coll Any),
        :ret Op,
        :statements (Vec Nothing),
        :tag Any}
      '{:op ':fn-method,
        :body Op,
        :children '[':params ':body],
        :env HMap49305,
        :fixed-arity Int,
        :form (Coll (Coll Any)),
        :name Op,
        :params '[Op],
        :recurs nil,
        :type nil,
        :variadic? false}
      ; omitted 10 cases
      ...))
\end{cljlistingnumbered}
\caption{While imperfect, this recursive type generated by our tool 
         is an invaluable starting
         point for further annotations.
         It describes the AST format for a compiler called cljs.compiler
         (\secref{infer:chap:evaluation}), and
         has 14 distinct operators, 5 inferred
         to have optional entries,
         and 22 recursive references.
}
\label{infer:fig:cljs}
%    '{:op ':host-call,
%      :args '[Op],
%      :children Any,
%      :env context-statement-tmp-HMap-alias20275,
%      :form (Coll Sym),
%      :method Sym,
%      :tag Any,
%      :target Op}
%    '{:op ':host-field,
%      :children '[':target],
%      :env context-statement-tmp-HMap-alias20275,
%      :field Sym,
%      :form (Coll Sym),
%      :tag Sym,
%      :target Op}
%    '{:op ':if,
%      :children '[':test ':then ':else],
%      :else Op,
%      :env context-statement-tmp-HMap-alias20275,
%      :form (Coll Any),
%      :tag (Set (U nil Sym)),
%      :test Op,
%      :then Op,
%      :unchecked Boolean}
%    '{:op ':invoke,
%      :args '[Op],
%      :children '[':fn ':args],
%      :env context-statement-tmp-HMap-alias20275,
%      :fn Op,
%      :form (Coll Any),
%      :tag Sym}
%    (HMap
%      :mandatory
%      {:op ':js,
%       :env context-statement-tmp-HMap-alias20275,
%       :form (Coll (U nil Str Sym)),
%       :js-op Sym,
%       :numeric nil,
%       :tag Sym}
%      :optional
%      {:args '[Op Op],
%       :children '[':args],
%       :code Str,
%       :segs (Coll Str)})
%    (HMap
%      :mandatory
%      {:op ':js-var, :name Sym, :ns Sym}
%      :optional
%      {:tag Sym})
%    '{:op ':let,
%      :bindings '[Op Op Any],
%      :body Any,
%      :children Any,
%      :env context-statement-tmp-HMap-alias20275,
%      :form Any,
%      :tag Any}
%    (HMap
%      :mandatory
%      {:op ':local,
%       :env context-statement-tmp-HMap-alias20275,
%       :form Sym,
%       :info Op,
%       :local (U ':arg ':let),
%       :name Sym}
%      :optional
%      {:arg-id Int, :init Op, :tag Sym})
%    '{:op ':map,
%      :children '[':keys ':vals],
%      :env context-statement-tmp-HMap-alias20275,
%      :form AMap,
%      :keys '[Op],
%      :tag Sym,
%      :vals '[Op]}
%    (HMap
%      :mandatory
%      {:op ':var, :name Sym, :ns Sym}
%      :optional
%      {:arglists (Coll Any),
%       :arglists-meta (Coll nil),
%       :column Int,
%       :doc Str,
%       :end-column Int,
%       :end-line Int,
%       :env context-statement-tmp-HMap-alias20275,
%       :file (U nil Str),
%       :fn-var Boolean,
%       :form Sym,
%       :info (U nil ColumnFileLineMap),
%       :line Int,
%       :max-fixed-arity Int,
%       :meta
%       (U
%         ColumnFileLineMap__0
%         FileArglistsColumnMap
%         ColumnEndColumnEndLineMap),
%       :method-params (Coll (Coll Sym)),
%       :protocol-impl nil,
%       :protocol-inline nil,
%       :ret-tag Sym,
%       :tag Sym,
%       :top-fn ArglistsArglistsMetaMaxFixedArityMap,
%       :variadic? Boolean})))
\end{figure}


An important question to answer is ``how accurate are these annotations?''.
Unlike previous work in this area~\cite{An10dynamicinference}, we do not aim for soundness guarantees
in our generated types. 
Our main contribution is a tool that Clojure programmers
can use to help learn about and specify their programs.
In that spirit, annotations should meet several criteria.

\paragraph{Good names}
Typed Clojure and clojure.spec annotations are abundant
with useful names for types. A good name often increases
readability.
Good names can sometimes be reconstructed from the program source,
like function or parameter names, and other times 
we can use the shape of a type to summarize it.

\paragraph{Compact}
Idiomatic Clojure code rarely mixes certain types in the same position,
unless the program is polymorphic. Using this knowledge---which we observed
by the annotations and specs assigned to idiomatic Clojure 
code---we can rule out certain combinations of types to compact our
resulting output, without losing information that would help us
type check our programs.

\paragraph{Recursive}
Maps in Clojure are often heterogeneous, and recursively defined.
Typed Clojure and clojure.spec supplies mechanisms for the most
common case: maps of known keyword entries.
We strategically \textbf{squash} flat types to be recursive
based on their unrolled shape.
For example, a recursively defined union of maps almost always
contains a known keyword ``tag'' mapped to a keyword.
By identifying this tag, we can reconstruct a good recursive
approximation of this type.

\Dsection{Naming}

For a type to be immediately useful to a programmer, it helps
to have a great name. We explored several avenues for
generating good names.

For types that occured as function arguments, the name of
the argument often indicated its role in the program.
Names like \textbf{config} or \textbf{env} are often used
for an environment being functionally threaded through
the program.

Similarly, types that occur as values in configuration
maps often have descriptive keys.
For example, one of our case studies, a Star Trek 
game written in Clojure,
features a configuration map with a \textbf{:stardate}
entry containing
a map that of three number entries: 
\textbf{:start},
\textbf{:current}, and
\textbf{:end}.

What if a type occurs in the return position of a function?
Sometimes these are named by \textbf{let} binding the result
of the computation.

Failing these heuristics, we fall back of several approaches
to naming.
First, if we are naming a keyword map which is part of a tagged
union, we use the tag as the name. For example, if the tagged entry
maps \textbf{:op} to \textbf{:fn}, we name this map \textbf{FnOp}.
Otherwise, for maps with less than three entries, we simply
enumerate its entries as the name.
Finally, for large keyword maps, we give an abbreviation
of its keyset as a name.

\Dsection{Algorithm}

A key hypothesis in our algorithm is that functions in the same
namespace operate on related data. Our inference is built
to be per-namespace (i.e., per-file),
and aggressively merges all apparently-related HMap types.
This often yields useful types in our benchmarks.
This approach does not work well for libraries providing polymorphic
functions, since unit tests may use sample data that are unrelated
to the details of the function, but our algorithm merges and displays
them as an ``important'' top-level type alias.
Our algorithm excels
with programs that mainly operate on one or two (possibly recursive) map-based
data representations,
which, outside of polymorphic libraries, are common in the Clojure ecosystem
in our experience.

%\begin{Verbatim}
%(defn f [x] (inc x))
%\end{Verbatim}
%
%\begin{figure}
%\begin{cljlisting}
%(defn vertices [m]
%  (case (:op m)
%    :leaf 1
%    :node (+ 1 (:left m)
%               (:right m))))
%\end{cljlisting}
%\label{code:vertices}
%\end{figure}

%%%% An old introductory example %%%%%%

%Consider the problem of inferring the type for
%the following Clojure program, a one
%argument function $f$ that increments numbers.
%
%\begin{verbatim}
%(defn f [x] (inc x))
%\end{verbatim}
%
%We might unit test this feature on the integers,
%to ensure incrementing $1$ gives $2$.
%
%\begin{verbatim}
%(deftest f-test
%  (is (= (f 1) 2)))
%\end{verbatim}
%
%We can instrument this program to observe its runtime behavior.
%
%\begin{verbatim}
%(deftest f-test
%  (is (= ((track f ['f]) 1) 2)))
%\end{verbatim}
%
%The $track$ function takes a value and a \emph{path},
%which tracks the subcomponent of the current environment
%the given value represents. For example, the path
%
%\begin{verbatim}
%  ['f :domain]
%\end{verbatim}
%
%represents the domain of $f$.
%
%After instrumentation we collect some inference results,
%associating paths with types: $path : \tau$.
%
%\begin{verbatim}
%  [['f :domain] Int]
%  [['f :range] Int]
%\end{verbatim}
%
%We combine this information into a type environment
%$\Gamma$ mapping variables to types: $\{x : \tau\}$.
%
%\begin{verbatim}
%  {x : [Int -> Int]}
%\end{verbatim}
%
%Now consider the case where we add a new unit test.
%
%\begin{verbatim}
%(deftest f-test
%  (is (= (f 1) 2))
%  (is (= (f 2.5) 3.5)))
%\end{verbatim}
%
%We now have a new set of inference results:
%
%\begin{verbatim}
%  [['f :domain] Num]
%  [['f :range] Num]
%\end{verbatim}
%
%which we want to combine with our previously inferred type environment
%
%\begin{verbatim}
%  {x : [Int -> Int]}
%\end{verbatim}
%
%How do join $[Int -> Int]$
%and $[Num -> Num]$?
%We have several options.
%
%
%\begin{verbatim}
%[(I Num Int) -> (U Num Int)]
%\end{verbatim}
%
%\begin{verbatim}
%[(U Num Int) -> (U Num Int)]
%\end{verbatim}
%
%\begin{verbatim}
%(IFn [Int -> Int]
%     [Num -> Num])
%\end{verbatim}

%\begin{figure}
%\begin{tikzpicture}[level distance=0.8cm, scale=0.9]
%\tikzstyle{every node}=[font=\small]
%\tikzset{grow'=down}
%\tikzset{every tree node/.style={align=center,anchor=north}}
%\Tree [.\node[draw](P0){\texttt{P0 = (Pairof P1 P2)}};
%        [.\node[draw](P2){\texttt{P2 = (Pairof P3 N)}};
%          [ \texttt{N} ]
%          [.\node[draw](P3){\texttt{P3 = (Pairof N N)}};
%            [ \texttt{N} ] [ \texttt{N} ]]
%]
%        [.\node[fill=gray!80](P1){\texttt{P1 = (Pairof N N)}};
%          [ \texttt{N} ] [ \texttt{N} ] ]
% ]
%\end{tikzpicture}
%%
%%\hspace{0.15cm}
%%
%\begin{tikzpicture}[level distance=0.8cm]
%\tikzstyle{every node}=[font=\small]
%\tikzset{grow'=down}
%\tikzset{every tree node/.style={align=center,anchor=north}}
%\Tree [.\node[fill=gray!80](P0){\texttt{P0 = (Pairof ($\cup$ N P0) ($\cup$ N P2))}};
%        [.\node[draw](P2){\texttt{P2 = (Pairof P3 N)}};
%          [ \texttt{N} ]
%          [.\node[draw]{\texttt{P3 = (Pairof N N)}};
%            [ \texttt{N} ] [ \texttt{N} ]]
%]
%          [ \texttt{N} ] [ \texttt{N} ] 
% ]
%%\draw[semithick,->] (P1)..controls +(west:1) and +(west:1)..(P0);
%\end{tikzpicture}
%\end{figure}
%

\begin{figure*}
\begin{mathpar}

  \begin{array}{lllll}
    \trackmeta{}(\val{}, \inferpath{}) = \val{}\ ;\ \res{}\\\\

    \trackmeta{}(\num{}, \inferpath{})
    &=&
    n\ ; \{\inferpath{} : \IntT{}\}
    \\
    \trackmeta{}([\lambda \xvar{}. \exp{}, \rho], \inferpath{})
    &=&
    [
    \lambda \yvar{}.
      \trackE{((\lambda \xvar{}. \exp{}) \trackE{\yvar{}}{\appendone{\inferpath{}}{\dompe{}}})}
             {\appendone{\inferpath{}}{\rngpe{}}}
         , \rho]
         \ ; \{\inferpath{} : [\UnknownT{} \rightarrow \UnknownT{}] \}
         \\
    &&
    \text{where}\ \yvar{} \text{ is fresh}
    \\
    \trackmeta{}(\{\overrightarrow{\val1\ \val2}\}, \inferpath{})
    &=&
    \{\overrightarrow{\val1\ \val2{}'}\}
    \ ;\ \overrightarrow{\sqcup\ \res{}}
      \sqcup
    \{\inferpath{} : \{\overrightarrow{\val1\ \UnknownT{}}\} \}
    \\
    &&
    \text{where}\ \overrightarrow{\trackmeta{}(\val2, \appendone{\inferpath{}}{\inferkeype{\overrightarrow{\val1}}{\val1}}) = \val2{}'\ ;\ \res{}}
    \\
    \\

         
    \inferupdatenoalign{\aenv{}}{\tenv{}}{\inferpath{}}{\tau}{\atenv{}}
    \\\\

    \inferupdatealign{\aenv{}}{\tenv{}}{\appendone{\inferpath{}}{\inferkeype{\overrightarrow{\kw{}'}}{\kw{}}}}{\ty{}}
            {\inferupdate{\aenv{}}{\tenv{}}{\inferpath{}}{\{\overrightarrow{\kw{}' : \UnknownT{}},\ \kw{} : \ty{} \}}}
    \\
    \inferupdatealign{\aenv{}}{\tenv{}}{\appendone{\inferpath{}}{\dompe{}}}{\ty{}}
                {\inferupdate{\aenv{}}{\tenv{}}{\inferpath{}}{\arrow{\ty{}}{\UnknownT{}}}}
    \\
    \inferupdatealign{\aenv{}}{\tenv{}}{\appendone{\inferpath{}}{\rngpe{}}}{\ty{}}
                {\inferupdate{\aenv{}}{\tenv{}}{\inferpath{}}{\arrow{\UnknownT{}}{\ty{}}}}
    \\
    \inferupdatealign{\aenv{}}{\tenv{}}{[x]}{\ty{}}{(\tenv{}[\sigma / \xvar{}] , \aenv{}')}
    \\
    && \text{where}\ \tenv{}(\xvar{}) = \ty{}', join(\aenv{}, \ty{}, \ty{}') = \sigma\ ;\ \aenv{}'
    \\
    \inferupdatealign{\aenv{}}{\tenv{}}{[\xvar{}]}{\ty{}}{(\tenv{}[\ty{} / \xvar{}], \aenv{})}
    \\
    && \text{where}\ \xvar{} \not\in dom(\tenv{})
    \\

    \\
    \joinnoalign{\aenv{}}{\tau}{\tau}{\tau}
    \\ \\

    \joinalign{\aenv{}}{(\cup\ \overrightarrow{\tau_1})}{\tau}{(\cup\ \overrightarrow{\sigma})}\\
                                                    && \text{where }
                                                    \overrightarrow{\joinnoalign{\aenv{}}{\tau_1}{\tau}{\sigma}}
    \\
    \joinalign{\aenv{}}{\tau}{(\cup\ \overrightarrow{\tau_1})}{(\cup\ \overrightarrow{\sigma})} \\
                                                    && \text{where }
                                                    \overrightarrow{\joinnoalign{\aenv{}}{\tau_1}{\tau}{\sigma}}
    \\
    \joinalign{\aenv{}}{\UnknownT{}}{\tau}{\tau}
    \\
    \joinalign{\aenv{}}{\tau}{\UnknownT{}}{\tau}
    \\
    \joinalign{\aenv{}}{[\overrightarrow{\tau_1} \rightarrow \sigma_1]}{[\overrightarrow{\tau_2} \rightarrow \sigma_2]}{[\overrightarrow{\joinexpression{\aenv{}}{\tau_1}{\tau_2}}
    \rightarrow
    \joinexpression{\aenv{}}{\sigma_1}{\sigma_2}]}
    \\
    % TODO add condition on should-join-HMap?
    %join(A, (HMap\  m_1\ o_1), (HMap\  m_2\ o_2))
    %&=& joinHMap(A, (HMap\  m_1\ o_1), (HMap\  m_2\ o_2))
    %\\
    \joinalign{\aenv{}}{\tau}{\sigma}{(\cup\ \tau\ \sigma)}

  \end{array}

\end{mathpar}
\end{figure*}

\section{The algorithm}

\subsection{Step 1}

\begin{figure*}
\begin{mathpar}

  \begin{array}{lllll}
    joinHMap(A, (HMap\  m_1\ o_1),
                (HMap\  m_2\ o_2))
      = (HMap\ m\ o) ; A
      \\
  \begin{array}{lllll}
      \text{ where }
          &m_1 = \{\overrightarrow{km_1 : tm_1}\}\\
          &m_2 = \{\overrightarrow{km_2 : tm_2}\}\\
          &o_1 = \{\overrightarrow{ko_1 : to_1}\}\\
          &o_2 = \{\overrightarrow{ko_2 : to_2}\}\\
          &allreq = km_1 \cup km_2 \\
          &commonreq = km_1 \cap km_2 \\
          &reqtoopt = allreq \setminus commonreq\\
          &newopt = ko_1 \cup ko_2 \cup reqtoopt \\
          &newreq = commonreq \setminus newopt,\\
          &m = \{k : t_m | k \in newreq, t_m = join*(A, [t | \overrightarrow{\bigvee(k, t) \in s_i}^{s = [m_1, m_2]}])\}\\
          &o = \{k : t_o | k \in newopt, t_o = join*(A, [t | \overrightarrow{\bigvee(k, t) \in s_i}^{s = [m_1, m_2, o_1, o_2]}])\}\\
  \end{array}
    \\
    \\
    containsUnknown(t) = (t = (HMap\ \{k : ?, \overrightarrow{k': t'}\}\ o) \vee t = (HMap\ m \{k' : ?, \overrightarrow{k'': t''}\})))
    \\
    unionHMaps(A, hmaps) = hmapsmerged2
    \\
  \begin{array}{lllll}
    \text{ where }
      &commonkeys = \bigcap[k_m | (HMap\ \{\overrightarrow{k_m : t}\} \ o) \in ts]\\
      &likelytags = [k | k \in commonkeys \wedge \bigwedge [\textbf{if}\ (k,k_1) \in m\ \textbf{then}\ T\ \textbf{else}\ F\ | (HMap\ m\ o) \in ts]]\\
      &hmapsbykeys = [(ks, ts) | t  \in hmaps \wedge t = (HMap\ \{\overrightarrow{ks: t}\}\ o) \wedge 
                                 ts = [t | t \in hmaps \wedge t = (HMap\ \{\overrightarrow{ks: t'}\}\ o')]
                                 ]\\
      &hmapsmerged = \textbf{if}\ (\textbf{empty?}\ likelytags)\ \textbf{then}\ hmaps\ \textbf{else}\ 
                            [t | ts \in hmapsbykeys \wedge t = \textbf{fold}\ (joinHMap\ A)\ ts]
      \\
      &isolatetag() =
\begin{cases}
    [(Map\ Any\ Any)], \text{if } \textbf{empty?}(commonkeys)\\
    newmaps, \text{if } \neg \textbf{empty?}(likelytags)\\
  \begin{array}{lllll}
    \text{where}
      &k_t = \textbf{first}(likelytags)\\
      &hmapsbytags = %[(k_m, ts) | t  \in hmapsmerged \wedge t = (HMap\ \{k_t : k_m, \overrightarrow{k': t'}\}\ o) \wedge 
                     %             ts = [t | t \in hmaps \wedge t = (HMap\ \{k_t : k_m, \overrightarrow{k'' : t''}\}\ o')]
                     %             ]
                     \textbf{group-by}(\lambda (HMap\ \{k_t : k_m, \overrightarrow{k': t'}\}\ o). k_m)
                     hmapsmerged
                                  \\
      &newmaps = [t | ts \in hmapsbytags \wedge t = \textbf{fold}\ (joinHMap\ A)\ ts]
  \end{array}\\
    hmapsmerged , \text{if} \neg\textbf{empty?}([t | t \in hmapsmerged \wedge containsUnknown(t)])\\
    \textbf{fold}\ (joinHMap\ A)\ hmapsmerged, \text{otherwise}
\end{cases}
      \\
      % check common keys
      &hmapsmerged2 = \textbf{if}\ (\textbf{length}\ hmapsmerged) > 1\ \textbf{then}\ isolatetag() \textbf{else}\ hmapsmerged
  \end{array}\\
    \\
    makeUnion(A, ts) = u
    \\
  \begin{array}{lllll}
    \text{ where }
      &hmaps = [(HMap\ m\ o) | (HMap\ m\ o) \in ts]\\
      &nonhmaps = [t | t \in ts \wedge t \not= (HMap\ m\ o)]\\
      &hmapsmerged = unionHMaps(A, hmaps)\\
      &u = nonhmaps \cup hmapsmerged
  \end{array}
      \\
      \\
  \end{array}
\end{mathpar}
\caption{Union constructor. Does not create new aliases, merges HMaps on the same "level"
using optional entries where appropriate. Omitted: merging function types, upcasting certain
combinations of types.}
\end{figure*}

Where: \emph{make-Union}

This step begins by merging HMap's 
and functions that are contained in the same union.

Algorithm: Traverse each union and merge members
of the same union with the following merge strategy.

HMaps with 100\% same keys are merged, with
entries being the union of the original HMaps.
In the special case where we can guess a common
dispatch key for all maps 
(a common entry mapped to a keyword),
we instead merge using the value of that entry.

If there is a mix of collection and non-collection
types, we upcast the entire union to Any.
From experience, these combinations are rare in
non-polymorphic functions, so instead we require
a polymorphic type variable, which is future work.

Functions with the same arities are merged, with
the new domains being the join of the domains
point-wise, and the new range being the join
of the old ranges.

Assumption: No optional keys. No recursive types.

%TODO
\subsection{Step 2.a}

\begin{figure*}
\begin{mathpar}
  \begin{array}{lllll}
      walk(A, t) =
\begin{cases}
  doalias(A, t), \text{if } t = (HMap\ m\ o)\\
  doalias(A, t), \text{if } t = (\cup\ \overrightarrow{ts}) \wedge \neg\textbf{every?}(\textbf{simple?},\ ts)\\
  t, \text{otherwise}
\end{cases}
    \\
  \end{array}

\end{mathpar}
\caption{Step 2.a}
\end{figure*}


\begin{figure*}
\begin{mathpar}
  \begin{array}{lllll}
      doalias(A, t) =   res\\
  \begin{array}{lllll}
    \text{ where } 
      &t' =
\begin{cases}
  (\cup\ \overrightarrow{\textbf{map}(\textbf{fullyResolveType}, \overrightarrow{ts})}), \text{if } t = (\cup\ \overrightarrow{ts})\\
  t, \text{otherwise}
\end{cases}\\
      &n = \textbf{generateName}(A, t')\\
      &A' = \textbf{registerAlias}(A, n, t')\\
      &res = (A', n')
  \end{array}
  \\
      walk(A, t) =
\begin{cases}
  doalias(A, t), \text{if } t = (HMap\ m\ o)\\
  doalias(A, t), \text{if } t = (\cup\ \overrightarrow{ts}) \wedge \neg\textbf{every?}(\textbf{simple?},\ ts)\\
  t, \text{otherwise}
\end{cases}
\\
    aliasHMap(A, t) = \textbf{postwalk}(A, t, walk)
    \\
  \end{array}

\end{mathpar}
\caption{Step 2.a}
\end{figure*}

Where: \emph{alias-HMap}

Relabel graph nodes to keysets.

Traverse graph up from leaves.
Label each HMap as a singleton set containing
its keyset.
Label each union as a set of keysets of its
members. Delete HMap labels of its members.

There are effectively two types of edges.
"Union edges" don't really exist, since
they're on the same level as their members.
Other edges are normal, thus we don't delete
HMaps labels they point to.

Assumption: No recursive types.

\subsection{Step 2.b}

\begin{figure*}
\begin{mathpar}
  \begin{array}{lllll}
    tryMergeAliases(A, f, t) = 
      \begin{cases}
        A2, \text{if } (\textbf{deep-keysets}(f) \cap \textbf{deep-keysets}(t)) \ne \varnothing\\
        \begin{array}{lllll}
          \text{ where }
          % FIXME pretty sure each join argument needs `substAlias` first, check fixed implementation
          &nt = join(\textbf{get}(A, f), \textbf{substAlias}(\textbf{get}(A, t), f, t))\\
          &A1 = \textbf{assoc}(A, f, t)\\
          &A2 = \textbf{assoc}(A1, t, nt)\\
        \end{array}\\
        A, \text{otherwise}
      \end{cases}
      \\
    squash(A, t) = h(A, [t], [])\\
  \begin{array}{lllll}
    \text{ where }
      h(A, w, d) = 
      \begin{cases}
        A, \text{if } \textbf{empty?}(w)\\
        h(A',w',d'), \text{otherwise}\\
  \begin{array}{lllll}
    \text{ where }
      &t = \textbf{first}(w)\\
      &as = \textbf{aliases-in}(A, \textbf{resolve-alias}(A, t))\\
      &ap = d \setminus \{t\}\\
      &A' = 
      \begin{cases}
        \textbf{fold}(\lambda (A', f). tryMergeAliases(A', f, t), A, as \cup ap), \text{if } \textbf{contains?}(d, t)\\
        A, \text{otherwise}
      \end{cases}\\
      &d' = (d \cup \{t\})
      \\
      &w' = \textbf{rest}(w) \cup (\textbf{aliases-in}(A, t) \setminus d')
      \\
  \end{array}
      \end{cases}\\
  \end{array}
    \\
    squashAll(A, t) = A'
    \\
  \begin{array}{lllll}
    \text{ where }
      &as = \textbf{aliases-in}(A, t)\\
      &A' = \textbf{fold}(squash, A, as)\\
  \end{array}
  \end{array}

\end{mathpar}
\caption{Step 2.b}
\end{figure*}

\begin{figure*}
\begin{mathpar}
  \begin{array}{lllll}
    \steptwohelper{} : \atenv{}, (\xvar{}, \ty{}) \rightarrow \atenv{}\\
    \steptwohelper{} ((\aenv{}, \Gamma), (\xvar{}, \ty{})) = (\aenv2, \tenv{}[\xvar{} \mapsto \ty2])\\
  \begin{array}{lllll}
    \text{ where }
      &(\aenv1, \ty1) = aliasHMap(\aenv{}, \ty{})\\
      &(\aenv2, \ty2) = squashAll(\aenv1, \ty1)\\
  \end{array}
  \\
  \\
  \steptwo{} : \tenv{} \rightarrow \atenv{}
  \\
  \steptwo{}(\tenv{}) = \textbf{fold}(\steptwohelper{}, ([], \tenv{}), \tenv{})\\
  \end{array}
\end{mathpar}
\caption{Step 2 summary: Create aliases for HMaps (graph nodes), then squash recursive types locally
(don't try to merge data examples from different paths). Omitted: follow-aliases call, that erases
redundant aliases.}
\end{figure*}

Where: \emph{squash-all}

Vertical recursive types.

Now recursive types are derived by
depth traversal of the types.

We start at the top of each branch and remember
the keysets of the current node.
We then traverse down depth-first and remember
the keyset of each subsequent node. 
If we find two nodes that have overlapping set of keysets,
we merge them.

The set of keysets of a HMap are every union of the keys
in its required entry set and the subsets of the
keys in its optional entry set.
The set of keysets of a type is the union of
all the keysets in its members (eg., for a union type,
take the union of the set of keysets of its members).

Here is the merge algorithm:

Given nodes m and n, where non-empty(intersection(keysets(m), keyset(n))),
and m was found before n.

1. Redirect all in/out edges on n to m.
2. Delete node n.
3. Let type(m) = join(type(m), type(n))
4. Let label(m) = union(label(m), label(n))

Question: This algorithm could run until a fixed point.
Should it? The next step does something similar that
merges types in different graph forests, but does
not handle recursive types.

Assumption: No recursive types before we start 
(?but surely as we go?)

\subsection{Step 3.a}

Where: \emph{alias-single-HMaps}

Relabel nodes to HMap keyset.

Relabel each node to its keyset if it is
a HMap, otherwise do not label.

\subsection{Step 3.b}

Where: \emph{squash-horizontally}

Horizontally merge similar types.

Given a collection of sets of labelled nodes (to HMap
keysets), in each set, choose just the types
that are not recursive (by traversing the type and
noticing if a cycle is made to the original label)
and merge these nodes into the same graph.

The merge is performed by joining the types
of all the candidate nodes to a "new" node. Then reroute all
in/out edges belonging to the candidate
node to the "new" node.

Note: This only merges non-recursive nodes.

\begin{figure*}
\begin{mathpar}
  \begin{array}{lllll}
    \stepthree{} : \atenv{} \rightarrow \atenv{}\\
    \stepthree{} = \squashhorizonally{} \circ \aliassinglehmap{}
  \\
  \end{array}
\end{mathpar}
\caption{Step 3 summary: First ensure all HMaps correspond to an alias. Then merge
aliases that point to a HMaps with identical required keysets (aliases must point to exactly one
top-level HMap, no unions).}
\end{figure*}

\subsection{Step 4}
%\begin{figure*}
%\begin{mathpar}
%  \begin{array}{lllll}
%    followTEnv'(A, t, f, s) = 
%\begin{cases}
%  t, \text{if } \textbf{contains?}(s, f)\\
%  t', \text{otherwise}\\
%  \begin{array}{lllll}
%    \text{ where }
%    &t' = 
%\begin{cases}
%  followTEnv'(A, t, r, s \cup \{f\}), \text{if } \textbf{alias?}(r)\\
%  t'', \text{otherwise } \textbf{alias?}(r)\\
%  \begin{array}{lllll}
%    \text{ where }
%    &r' = 
%\begin{cases}
%  r, \text{if } \textbf{simpleAlias?}(A, f)\\
%  f, \text{otherwise}
%\end{cases}\\
%    &t'' = t[f \mapsto r']
%  \end{array}
%
%\end{cases}\\
%  \end{array}
%\end{cases}\\
%  \begin{array}{lllll}
%    \text{ where }
%      &r = \textbf{resolve-alias}(A, f)
%  \end{array}\\
%    followTypeEnv(A, t) = t'\\
%  \begin{array}{lllll}
%    \text{ where }
%    &(A', t') = \textbf{fold}(\lambda ((A, t), f). followTEnv'(A, t, f, []), (A, t), \textbf{aliases-in}(A, t))
%  \end{array}\\
%
%    followAEnv'(A, t, f, s) = 
%\begin{cases}
%  t, \text{if } \textbf{contains?}(s, f)\\
%  t', \text{otherwise}\\
%  \begin{array}{lllll}
%    \text{ where }
%    &t' = 
%\begin{cases}
%  followAEnv'(A, t, r, s \cup \{f\}), \text{if } \textbf{alias?}(r)\\
%  t'', \text{otherwise } \textbf{alias?}(r)\\
%  \begin{array}{lllll}
%    \text{ where }
%    &r' = 
%\begin{cases}
%  r, \text{if } \textbf{simpleAlias?}(A, f)\\
%  f, \text{otherwise}
%\end{cases}\\
%    &t'' = t[f \mapsto r']
%  \end{array}
%
%\end{cases}\\
%  \end{array}
%\end{cases}\\
%  \begin{array}{lllll}
%    \text{ where }
%      &r = \textbf{resolve-alias}(A, f)
%  \end{array}\\
%
%    followAliasEnv(A, \Gamma) = \textbf{fold}(followAEnv', A, as)\\
%  \begin{array}{lllll}
%    \text{ where }
%    &A = \{\overrightarrow{as \mapsto ts}\}
%  \end{array}\\
%    followall(A, \Gamma) = res\\
%  \begin{array}{lllll}
%    \text{ where }
%      &(A1, \Gamma_1) = \textbf{fold}(followTypeEnv, (A, \Gamma), \Gamma)\\
%      &(A2, \Gamma_2) = followAliasEnv(A1, \Gamma_1)\\
%      &res = (A2, \Gamma_2)
%  \end{array}\\
%  \end{array}
%\end{mathpar}
%\caption{Follow-all: Simplify aliases in the type environment and alias environment.}
%\end{figure*}

Where: \emph{follow-all}

Remove aliases that just point to other aliases.

\begin{figure*}
\begin{mathpar}
  \begin{array}{lllll}
    % FIXME the types for 'update' don't quite work out
    \genupdate{} : \Gamma, (\inferpath{}, \tau) \rightarrow \tau\\
    \genupdate{}(\Gamma, (\inferpath{}, \tau)) = update([], \Gamma, \inferpath{}, \tau)\\
    \\
    \generatetenv{} : \res{} \rightarrow \Gamma\\
    \generatetenv{} (\res{}) = \textbf{fold}(\genupdate{}, [], \res{})\\
    \\
    \inferanns{} : \res{} \rightarrow \atenv{}\\
    \inferanns{} = \stepthree{} \circ \steptwo{} \circ \generatetenv{}\\
  \end{array}
\end{mathpar}
\caption{Algorithm summary: Generate an initial type environment from inference results.
Then generate an updated type environment paired with a type alias environment by 1: creating
a recursive type if a HMap contains another HMap whose keysets have a non-empty intersection, 2:
globally merging type aliases based on identical HMap keysets, 3: cleaning up redundant aliases.}
\end{figure*}

\Dchapter{Formalism}
\label{infer:sec:formalism}

We present \lambdatrack{}, an untyped $\lambda$-calculus
describing the essense of our approach to automatic annotations.
We split our model into two phases: the collection phase 
\collectOp{}
that runs an instrumented program and collects observations, and
an inference phase 
\inferanns{}
that derives type annotations from these observations
that can be used to automatically annotate the program.

We define the top-level driver function \annotateOp{} that connects
both pieces.
It says, given a program \e{}
and top-level variables $\ova{\x{}}$ to infer annotations for,
return an annotation environment \atenv{} with possible entries for
$\ova{\x{}}$ based on observations from evaluating
an instrumented \e{}.
%
%\begin{mathpar}
%\infer[]
%{ \collectnoalign{\e{}}{\ova{\x{}}}{\res{}}
%  \\
%  \inferannsnoalign{\res{}}{\atenv{}}
%}
%{ \annotatenoalign{\e{}}{\ova{\x{}}}{\atenv{}} }
%\end{mathpar}
\begin{mathpar}
  \begin{array}{lllll}
    \annotateOp{} : \e{}, {\ova{\x{}}} \rightarrow \atenv{}\\
    \annotateOp{} = \inferanns{} \circ \collectOp{}
  \end{array}
\end{mathpar}

To contextualize the presentation of these phases, we begin a running example:
inferring the type of a top-level function $f$, that takes a map and
returns its {\makekw{a}} entry, 
based on the following usage.
%
\begin{Verbatim}[commandchars=\\\{\}, codes={\catcode`$=3\catcode`^=7}]
  define $f$ = \uabs{m}{\getexp{m}{\makekw{a}}}
  \appexp{f}{\curlymap{\makekw{a} 42}} => 42
\end{Verbatim}
%
Plugging this example into our driver function
we get a candidate annotation for $f$:
$$
\annotatenoalign{\appexp{f}{\curlymap{\makekw{a}\ 42}}}{[f]}{\{\hastype{f}{[\{\makekw{a}\ \IntT{}\} \rightarrow \IntT{}]}\}}
$$

\Dsection{Collection phase}
\label{infer:sec:formal:collection-phase}

\begin{figure}
  $$
  \begin{altgrammar}
    \val{} &::=& \num{}
       \alt {\kw{}}
       \alt {\closure{\uabs{\x{}}{\e{}}}{\openv{}}}
       \alt {\curlymap{\ova{\kw{}\ {\val{}}}}}
       \alt {\const{}}
       &\mbox{Values} \\
   \e{} &::=& \x{}
       \alt \val
       \alt \trackE{\e{}}{\inferpath{}}
       \alt {\uabs{\x{}}{\e{}}}
       \alt {\curlymapvaloverrightnoarrow{\e{}}{\e{}}}
       \alt {\appexp{\e{}}{\ova{\e{}}}}
       &\mbox{Expressions} \\
    \openv{} &::=& \{\ova{x \mapsto \val{}}\}
       &\mbox{Runtime environments} \\
   \inferpth{}
      &::=& \x{}
       \alt \dompe{}
       \alt \rngpe{}
       \alt {\inferkeype{\HMapreq{}}{\kw{}}}
       &\mbox{Path Elements} \\
   \inferpath{} &::=& \ova{\inferpth{}}
       &\mbox{Paths} \\
       \res{}
      &::=& \restwoarrow{\inferpath{}}{\tau{}}
      &\mbox{Inference results} \\
    \t{}, \s{}
      &::=& \IntT{}
       \alt \arrow{\t{}}{\t{}}
       %\alt \HMappretty{\ova{\kw{}\ \t{}}}
       \alt \HMaptwo{\HMapreq{}}{\HMapopt{}}
       \alt \Unionsplice{\ova{\t{}}}
       \\
       &\alt& \alias{} % type alias
       \alt \kw{}
       \alt \Keyword{}
       \alt \Top{}
       \alt \IPersistentMap{\t{}}{\t{}}
       \alt \UnknownT{}
      &\mbox{Types} \\
    \tenv{} &::=& \{\ova{\hastype{\x{}}{\t{}}}\}
      &\mbox{Type environments} \\
    \HMapreq{}, \HMapopt{}
      &::=& \{ \ova{\kw{}\ {\t{}}} \}
      &\mbox{HMap entries} \\
    \aenv{} &::=& \{\ova{\alias{} \mapsto \tau}\}
      &\mbox{Type alias environments} \\
    \atenv{} &::=& (\aenv{}, \tenv{})
      &\mbox{Annotation environments} \\
  \end{altgrammar}
  $$
\caption{Syntax of Terms, Types, Inference results, and Environments for \lambdatrack{}}
\label{infer:fig:syntax}
\end{figure}

%\begin{figure*}
%  \ifdefined\PAPER
%  \footnotesize
%  \fi
%\begin{mathpar}
%  \begin{array}{llll}
%    \infer [B-Track]
%    { \opsemtrack{\openv{}}{\e{}}{\v{}}{\res{}} \\\\
%    \trackmeta{\v{}}{\inferpath{}}{\vp{}}{\resp{}}}
%    { \opsemtrack{\openv{}}{\trackE{\e{}}{\inferpath{}}}{\vp{}}{\unionres{\res{}}{\resp{}}}
%    }
%  \end{array}
%
%\infer [B-App]
%{ \opsemtrack{\openv{}}{\e{1}}{\closure{\uabs{\x{}}{\e{}}}{\openvp{}}}{\res{1}} \\\\
%  \opsemtrack{\openv{}}{\e{2}}{\v{}}{\res{2}} \\\\
%  \opsemtrack{\extendopenv{\openvp{}}{\x{}}{\v{}}}{\e{}}{\vp{}}{\res{3}} \\
%}
%{ \opsemtrack{\openv{}}{\appexp{\e{1}}{\e{2}}}{\vp{}}{\bigunionres{\ova{\res{i}}}}
%}
%
%  \begin{array}{lll}
%    \infer [B-Clos]
%    {}
%    { \opsemtrack{\openv{}}{\uabs{\x{}}{\e{}}}{\closure{\uabs{\x{}}{\e{}}}{\openv{}}}{\emptyres{}}}
%    \\\\
%    \infer [B-Val]
%    {}
%    { \opsemtrack{\openv{}}{\val{}}{\val{}}{\emptyres{}} }
%    \ \ \ \ \ \ 
%
%    \infer [B-Var]
%    {}
%    { \opsemtrack{\openv{}}{\xvar{}}{\inopenvnoeq{\openv{}}{\xvar{}}}{\emptyres{}}
%    }
%  \end{array}
%
%\infer [B-Delta]
%{ \opsemtrack{\openv{}}{\e{}}{\const{}}{\res{1}}
%  \\
%  \overrightarrow{\opsemtrack {\openv{}}{\ep{}}{\v{}}{\resp{}}}
%  \\\\
%  \inferconstantopsem{\const{}}{\ova{\v{}}}{\vp{}}{\res{2}}
%}
%{ \opsemtrack {\openv{}}
%              {\appexp {\e{}} {\overrightarrow{\ep{}}}}
%              {\vp{}}
%              {\overrightarrow{\unionres{\res{}}{\resp{}}}}
%       }
%
%\end{mathpar}
%\caption{Operational Semantics for \lambdatrack{}}
%\label{infer:fig:semantics}
%\end{figure*}

Now that we have a high-level picture of how these phases interact,
we describe the syntax and semantics of \lambdatrack{}, before
presenting the details of \collectOp{}.
%
\figref{infer:fig:syntax} presents the syntax of \lambdatrack{}.
Values \v{} consist of numbers \num{}, Clojure-style keywords {\kw{}},
closures {\closure{\uabs{\x{}}{\e{}}}{\openv{}}}, constants \const{},
and keyword keyed hash maps {\curlymapvaloverrightnoarrow{\kw{}}{\val{}}}.

Expressions \e{} consist of variables \x{}, values,
functions, maps, and function applications.
The special form
\trackE{\e{}}{\inferpath{}}
observes {\e{}} as related to path {\inferpath{}}.
Paths \inferpath{} 
record the source of a runtime value with respect
to a sequence of path elements \inferpth{}, always starting with
a variable \x{}, and are read left-to-right.
Other path elements are
a function domain \dompe{}, 
a function range \rngpe{},
and a map entry {\inferkeype{\ova{\kw{1}}}{\kw{2}}}
which represents the result of looking up {\kw{2}}
in a map with keyset ${\ova{\kw{1}}}$.

Inference results \restwoarrow{\inferpath{}}{\t{}}
are pairs of paths {\inferpath{}} and types \t{}
that say the path \inferpath{} was observed to be 
type \t{}.
Types \t{} are numbers \IntT{}, function types \arrow{\t{}}{\t{}},
ad-hoc union types \Union{\t{}}{\t{}},
type aliases \alias{},
%top type \Top{},
and unknown type \UnknownT{} that represents
a temporary lack of knowledge during the inference process.
Heterogeneous keyword map types \HMappretty{\ova{\kw{}\ \t{}}}
for now represent a series of required keyword entries---we will extend
them to have optional entries in later phases.

The big-step operational semantics
{\opsemtrack{\openv{}}{\e{}}{\v{}}{\res{}}}
(\figref{infer:fig:trackmeta})
says under runtime environment \openv{}
expression \e{} evaluates to value \v{}
with inference results \res{}.
Most rules are standard, with extensions to correctly
propagate inference results \res{}.
B-Track is the only interesting rule, which instruments
its fully-evaluated argument with the \trackmetaOp{}
metafunction.

The metafunction \trackmeta{\v{}}{\inferpath{}}{\vp{}}{\res{}} (\figref{infer:fig:trackmeta})
says if value \v{} occurs at path {\inferpath{}}, then return a possibly-instrumented
\vp{} paired with inference results {\res{}} that can be immediately derived
from the knowledge that \v{} occurs at path {\inferpath{}}.
It has a case for every kind of value.
The first three cases records the number input as type {\IntT{}}.
The fourth case, for closures, returns a wrapped value
resembling higher-order function contracts~\infercitep{findler2002contracts},
but we track the domain and range rather than verify them.
The remaining rules case, for maps, recursively tracks each map value,
and returns a map with possibly wrapped values.
Immediately accessible inference results are combined
and returned.
A specific rule for the empty map is needed because we otherwise only rely on
recursive calls to \trackEOp{} to gather inference results---in the empty case,
we have no data to recur on.

\begin{figure}
\begin{mathpar}
    \infer [B-Track]
    { \opsemtrack{\openv{}}{\e{}}{\v{}}{\res{}} \\\\
    \trackmeta{\v{}}{\inferpath{}}{\vp{}}{\resp{}}}
    { \opsemtrack{\openv{}}{\trackE{\e{}}{\inferpath{}}}{\vp{}}{\unionres{\res{}}{\resp{}}}
    }

\infer [B-App]
{ \opsemtrack{\openv{}}{\e{1}}{\closure{\uabs{\x{}}{\e{}}}{\openvp{}}}{\res{1}} \\\\
  \opsemtrack{\openv{}}{\e{2}}{\v{}}{\res{2}} \\\\
  \opsemtrack{\extendopenv{\openvp{}}{\x{}}{\v{}}}{\e{}}{\vp{}}{\res{3}} \\
}
{ \opsemtrack{\openv{}}{\appexp{\e{1}}{\e{2}}}{\vp{}}{\bigunionres{\ova{\res{i}}}}
}

    \infer [B-Clos]
    {}
    { \opsemtrack{\openv{}}{\uabs{\x{}}{\e{}}}{\closure{\uabs{\x{}}{\e{}}}{\openv{}}}{\emptyres{}}}

    \infer [B-Val]
    {}
    { \opsemtrack{\openv{}}{\val{}}{\val{}}{\emptyres{}} }


    \infer [B-Var]
    {}
    { \opsemtrack{\openv{}}{\xvar{}}{\inopenvnoeq{\openv{}}{\xvar{}}}{\emptyres{}}
    }

\infer [B-Delta]
{ \opsemtrack{\openv{}}{\e{}}{\const{}}{\res{1}}
  \\
  \overrightarrow{\opsemtrack {\openv{}}{\ep{}}{\v{}}{\resp{}}}
  \\\\
  \inferconstantopsem{\const{}}{\ova{\v{}}}{\vp{}}{\res{2}}
}
{ \opsemtrack {\openv{}}
              {\appexp {\e{}} {\overrightarrow{\ep{}}}}
              {\vp{}}
              {\overrightarrow{\unionres{\res{}}{\resp{}}}}
       }

  \arraycolsep=1.4pt
  \begin{array}{lllll}
    %\trackmeta{\v{}}{\inferpath{}}{\v{}}{\res{}}\\\\

    \trackmetaalign{\num{}}{\inferpath{}}{\num{}}{\singletonres{\inferpath{}}{\IntT{}}}
    \\
    \trackmetaalign{\kw{}}{\inferpath{}}{\kw{}}
                   {\singletonres{\inferpath{}}
                                 {\Keyword{}}}
    \\
    \trackmetaalign{\const{}}{\inferpath{}}{\const{}}{\emptyres{}}
    \\
    \trackmetalhs{\closure{\uabs{\x{}}{\e{}}}{\openv{}}}
                 {\inferpath{}}
                 &=&
                 \trackmetarhs
                   {\closure{\ep{}}{\openv{}}}
                   {\emptyres{}}
                   % let's treat functions as opaque wrt arity as they are in clojure
                   %{\singletonres{\inferpath{}}
                   %              {\arrow{\UnknownT{}}{\UnknownT{}}}}
         \\
    &&
    \begin{array}{@{}llll}
      \text{where } \y{} \text{ is fresh},\\
       \text{ }    \begin{array}{@{}llll}
                      \ep{} =
                        \uabs{\y{}}{\trackE{&\appexp{(\uabs{\x{}}{\e{}})}{\trackE{\yvar{}}{\appendone{\inferpath{}}{\dompe{}}}}}
                                           {\\&\appendone{\inferpath{}}{\rngpe{}}}}
                   \end{array}
    \end{array}
    \\
    \trackmetaalign{\{\}}
                   {\inferpath{}}
                   {\{\}}
                   {\singletonres{\inferpath{}}
                                 {\HMappretty{}}}
    \\
    \trackmetaalign{\{\ova{\kw{1}\ {\kw{2}}}\ 
                      \ova{\kw{}\ {\v{}}}
                    \}}
                   {\inferpath{}}
                   {\{\ova{\kw{1}\ {\kw{2}}}\ 
                      \ova{\kw{}\ {\vp{}}}
                    \}}
                   {\bigunionres{\res{}}}
    \\
    &&
    \text{where } \ova{\trackmeta{\v{}}
                                            {\appendone{\inferpath{}}
                                                       {\inferkeype{\{\ova{\kw{1}\ {\kw{2}}}\ 
                                                                      \ova{\kw{}\ {\UnknownT{}}}
                                                                    \}}
                                                                   {\kw{}}}}
                                            {\vp{}}
                                            {\res{}}}
    \\
  \end{array}
  
  \arraycolsep=1.4pt
  \begin{array}{lllr}
    \inferconstantopsemalign{\assocliteral{}}{\curlymap{\ova{\kw{}\ \v{}}}, \kwp{}, \vp{}}{\updatemap{\curlymap{\ova{\kw{}\ \v{}}}}{\kwp{}}{\vp{}}}
                            {\emptyres{}}\\
    \inferconstantopsemalign{\getliteral{}}{\curlymap{\kw{}\ \v{}, \ova{\kwp{}\ \vp{}}}, \kw{}}{\v{}}
                            {\emptyres{}}\\
    \inferconstantopsemalign{\dissocliteral{}}{\curlymap{\kw{}\ \v{}, \ova{\kwp{}\ \vp{}}}, \kw{}}{\curlymap{\ova{\kwp{}\ \vp{}}}}
                            {\emptyres{}}\\
  \end{array}
\end{mathpar}
\caption{Operational semantics, \trackmeta{\v{}}{\inferpath{}}{\v{}}{\res{}} and constants}
\label{infer:fig:trackmeta}
\end{figure}

Now we have sufficient pieces to describe the initial collection phase of our model.
Given an expression \e{} and variables ${\ova{\x{}}}$ to track,
\instrumentnoalign{\e{}}{\ova{\x{}}}{\ep{}}
returns an instrumented expression \ep{}
that tracked usages of $\ova{\x{}}$.
It is defined via capture-avoiding substitution:
$$
\instrumentnoalign{\e{}}{\ova{\x{}}}{\replacefor{\e{}}{\ova{\trackE{\x{}}{[\x{}]}}}{\ova{\x{}}}}
$$

Then, the overall collection phase 
\collectnoalign{\e{}}{\ova{\x{}}}{\res{}}
says, given an expression \e{}
and variables
$\ova{\x{}}$
to track,
returns inference results {\res{}}
that are the results of evaluating \e{}
with instrumented occurrences of $\ova{\x{}}$.
It is defined as:
%
$$
\collectnoalign{\e{}}{\ova{\x{}}}{\res{}}, \text{ where }
  \opsemtrack{}{\instrument{\e{}}{\ova{\x{}}}}{\v{}}{\res{}}
$$

For our running example
of collecting for the program \appexp{f}{\curlymap{\makekw{a}\ 42}},
we instrument the program by wrapping occurrences of $f$ with \trackEOp{}
with path $[f]$.
$$
\instrumentnoalign{\appexp{f}{\curlymap{\makekw{a}\ 42}}}{[f]}{\appexp{\trackE{f}{[f]}}{\curlymap{\makekw{a}\ 42}}}
$$

Then we evaluate the instrumented program and derive two inference results
(colored in red for readability):
$$
\opsemtrack{}{\appexp{\trackE{f}{[f]}}{\curlymap{\makekw{a}\ 42}}}{42}{\resflatcolor{\resentry{[f, \dompe{}, \inferkeypenokeyset{[\makekw{a}]}{\makekw{a}}]}{\IntT{}}, \resentry{[f, \rngpe{}]}{\IntT{}}}}
$$

Here is the full derivation:
\begin{Verbatim}[commandchars=\\\{\}, codes={\catcode`$=3\catcode`^=7}]
=> \appexp{\trackE{f}{[f]}}{\curlymap{\makekw{a}\ 42}}
=> \trackE{\getexp{\trackE{\curlymap{\makekw{a}\ 42}}{[f, \dompe{}]}}{\makekw{a}}}{[f, \rngpe{}]}
=> \trackE{\getexp{{\curlymap{\makekw{a}\ 42}} ; \resflatcolor{\resentry{[f, \dompe{}, \inferkeypenokeyset{[\makekw{a}]}{\makekw{a}}]}{\IntT{}}}}{\makekw{a}}}{[f, \rngpe{}]}
=> \trackE{42 ; \resflatcolor{\resentry{[f, \dompe{}, \inferkeypenokeyset{[\makekw{a}]}{\makekw{a}}]}{\IntT{}}}}{[f, \rngpe{}]}
=> $42 ; \resflatcolor{\resentry{[f, \dompe{}, \inferkeypenokeyset{[\makekw{a}]}{\makekw{a}}]}{\IntT{}}, \resentry{[f, \rngpe{}]}{\IntT{}}}$
\end{Verbatim}

Notice that intermediate values can have inference results (colored) attached to them with a semicolon,
and the final value has inference results about both $f$'s domain and range.

\Dsection{Inference phase}
\label{infer:sec:formal:inference-phase}

After the collection phase, we have a collection of inference results \res{}
which can be passed to the 
metafunction \inferanns{}(\res{}) = \atenv{} to produce an annotation environment:
\begin{mathpar}
  \begin{array}{lllll}
    \inferanns{} : \res{} \rightarrow \atenv{}\\
    \inferanns{} = \inferrecOp{} \circ \generatetenv{}\\\\
  \end{array}
\end{mathpar}
%
The first pass $\generatetenv{} (\res{}) = \tenv{}$ generates an initial type environment
from inference results \res{}.
%It is defined (in \figref{infer:fig:generatetenv})
%as a fold over \res{}, building a \tenv{} incrementally via the \inferupdateOp{}
%metafunction.
%
The second pass
$$\squashlocal{}(\tenv{}) = \atenvp{}$$
creates individual type aliases
for each HMap type in \tenv{} and then merges aliases that both occur inside the same
nested type into possibly recursive types. % (\figref{infer:fig:squashlocal}).
%
The third pass $\squashglobal{} (\atenv{}) = \atenvp{}$
merges type aliases in \atenv{} based on their similarity. % (\figref{infer:fig:squashglobal}).

\Dsubsection{Pass 1: Generating initial type environment}
\label{infer:sec:formal:inference-phase:genenv}

\begin{figure*}
\begin{mathpar}
  \begin{array}{rllll}
    \joinOp{} : \t{}, \t{} \rightarrow \t{}
    \\
    \joinalign{\Unionsplice{\ova{\s{}}}}{\t{}}{\Unionsplice{\ova{\joinexpression{\s{}}{\t{}}}}}
    \\
    \joinalign{\t{}}{\Unionsplice{\ova{\s{}}}}{\Unionsplice{\ova{\joinexpression{\s{}}{\t{}}}}}
    \\
    \joinalign{\UnknownT{}}{\t{}}{\t{}}
    \\
    \joinalign{\t{}}{\UnknownT{}}{\t{}}
    \\
    \joinalign{\arrow{\t{1}}{\s{1}}}{\arrow{\t{2}}{\s{2}}}
              {\arrow{\joinexpression{\t{1}}{\t{2}}}
                     {\joinexpression{\s{1}}{\s{2}}}}
    \\
    \joinalign{\HMaptwo{\HMapreq{1}}{\HMapopt{1}}}
              {\HMaptwo{\HMapreq{2}}{\HMapopt{2}}}
              {\joinHMapexpression{\HMaptwo{\HMapreq{1}}{\HMapopt{1}}}
                                  {\HMaptwo{\HMapreq{2}}{\HMapopt{2}}}}
                                  \text{,}
                                  \\
    &&\ova{(\kw{}, \kw{i}) \in {\HMapreq{i}}} \Rightarrow \ova{\kw{i-1} = \kw{i}}
    \\
    \joinalign{\t{}}{\s{}}{\Union{\t{}}{\s{}}} \text{, otherwise}
  \end{array}
%
  \begin{array}{lllll}
    \joinHMapnoalign{\HMaptwo{\HMapreq{1}}{\HMapopt{1}}}{\HMaptwo{\HMapreq{2}}{\HMapopt{2}}}{\HMaptwo{\HMapreq{}}{\HMapopt{}}}
    \\
    \begin{array}{lllll}
      \text{where}
          &\mathsf{req}  = \bigcup \ova{\textsf{dom}({\HMapreq{i}})} \\
          &\mathsf{opt}  = \bigcup \ova{\textsf{dom}({\HMapopt{i}})} \\
          &\ova{\kw{}^r} = \bigcap \ova{\textsf{dom}({\HMapreq{i}})} \setminus \mathsf{opt}\\
          &\ova{\kw{}^o} = \mathsf{opt} \cup (\mathsf{req} \setminus \ova{\kw{}^r})\\
          &\HMapreq{}    = \{\ova{\kw{}^r\ \joinstarexpression{\ova{\HMapreq{i}[\kw{}^r]}}} \} \\
          &\HMapopt{}    = \{\ova{\kw{}^o\ \joinstarexpression{\ova{\HMapreq{i}[\kw{}^o], \HMapopt{i}[\kw{}^o]}}} \}
    \end{array}
  \end{array}

  \begin{array}{lllll}
    \textsf{fold} : \forall \alpha, \beta. (\alpha, \beta \rightarrow \alpha), \alpha, \ova{\beta} \rightarrow \alpha\\
    \textsf{fold}(\textsf{f}, \textsf{a}_0, \ova{\textsf{b}}^n) = \textsf{a}_n\\
    \begin{array}{llll}
      \text{where } \ova{\textsf{a}_i = \textsf{f}(\textsf{a}_{i-1}, \textsf{b}_{i})}^{1 \leq i \leq n}\\
    \end{array}
    \\\\
    \generatetenv{} : \res{} \rightarrow \tenv{}\\
    \generatetenv{} (\res{}) = \textsf{fold}(\inferupdateOp{}, \{\}, \res{})\\
  \end{array}
  \begin{array}{lllll}
    \inferupdateOp : \tenv{}, \resentry{\inferpath{}}{\t{}} \rightarrow  \tenv{} 
    \\
    \inferupdatealign{\tenv{}}{\appendone{\inferpath{}}{\inferkeype{\{\ova{\kwp{}\ \s{}} \}}{\kw{}}}}{\t{}}
            {\inferupdate{\tenv{}}{\inferpath{}}{\{\ova{\kwp{}\ \s{}}\ \kw{}\ \t{} \}}}
    \\
    \inferupdatealign{\tenv{}}{\appendone{\inferpath{}}{\dompe{}}}{\t{}}
                     {\inferupdate{\tenv{}}{\inferpath{}}{\arrow{\t{}}{\UnknownT{}}}}
    \\
    \inferupdatealign{\tenv{}}{\appendone{\inferpath{}}{\rngpe{}}}{\t{}}
                {\inferupdate{\tenv{}}{\inferpath{}}{\arrow{\UnknownT{}}{\t{}}}}
    \\
    \inferupdatealign{\updatemap{\tenv{}}{\x{}}{\s{}}}{[x]}{\t{}}
                     {\updatemap{\tenv{}}
                                {\x{}}
                                {\joinexpression{\t{}}{\s{}}}
                                 }
    \\
    \inferupdatealign{\tenv{}}{[\xvar{}]}{\t{}}{\updatemap{\tenv{}}{\x{}}{\t{}}}
    \\
  \end{array}

\end{mathpar}
\caption{Definition of $\generatetenv{}(\res{}) = \tenv{}$}
\label{infer:fig:generatetenv}
\end{figure*}

The first pass is given in \figref{infer:fig:generatetenv}.
The entry point \generatetenv{} folds over inference results
to create an initial type environment via \inferupdateOp{}.
This style is inspired by occurrence typing~\infercitep{TF10},
from which we also borrow the concepts of paths into types.

We process paths right-to-left in \inferupdateOp{}, building
up types from leaves to root, before joining the fully constructed type with the existing
type environment via \joinOp{}.
The first case handles the \keypeOp{} path element.
The extra map of type information preserves both keyset
information and any entries that might represent tags
(populated by the final case of \trackEOp{}, \figref{infer:fig:trackmeta}).
This information helps us avoid prematurely collapsing tagged maps,
by the side condition of the HMap \joinOp{} case.
The \joinHMapOp{} metafunction aggressively combines two HMaps---required
keys in both maps are joined and stay required, otherwise keys
become optional.

The second and third \inferupdateOp{} cases update the domain and range of a function type,
respectively.
The \joinOp{} case for function types joins covariantly on the domain to yield more useful
annotations. For example, if a function accepts \IntT{} and \Keyword{},
it will have type
\joinnoalign{\arrow{\IntT{}}{\UnknownT{}}}{\arrow{\Keyword{}}{\UnknownT{}}}
{\arrow{\Union{\IntT{}}{\Keyword{}}}{\UnknownT{}}}.

Returning to our running example, we now want to convert our inference results
$$
\res{} = \{\resentry{[f, \dompe{}, \inferkeypenokeyset{[\makekw{a}]}{\makekw{a}}]}{\IntT{}}, \resentry{[f, \rngpe{}]}{\IntT{}}\}.
$$
into a type environment. Via $\generatetenv{}(\res{})$, we start to trace
\inferupdate{\{\}}{[f, \dompe{}, \inferkeypenokeyset{[\makekw{a}]}{\makekw{a}}]}{\IntT{}}

%\begin{figure*}
%\begin{mathpar}
%  \begin{array}{lllll}
%    \inferanns{} : \res{} \rightarrow \atenv{}\\
%    \inferanns{} = \squashglobal{} \circ \squashlocal{} \circ \generatetenv{}\\
%  \end{array}
%\end{mathpar}
%\caption{Algorithm summary: Generate an initial type environment from inference results.
%Then generate an updated type environment paired with a type alias environment by 1: creating
%a recursive type if a HMap contains another HMap whose keysets have a non-empty intersection, 2:
%globally merging type aliases based on identical HMap keysets, 3: cleaning up redundant aliases.}
%\label{infer:fig:inferanns}
%\end{figure*}

\Dsubsection{Pass 2: Squash locally}
\label{infer:sec:formal:inference-phase:squash-local}

\begin{figure*}
\begin{mathpar}
  \begin{array}{lllll}
    \aliashmap{} : \atenv{}, \t{} \rightarrow (\atenv{}, \t{})\\
    \aliashmap{}(\atenv{}, \t{}) = \textsf{postwalk}(\atenv{}, \t{}, \textsf{f})\\
    \begin{array}{lllll}
      \text{where} %&\textsf{f} : \atenv{}, \t{} \rightarrow (\atenv{}, \t{})\\
                   &\textsf{f}(\atenv{}, {\HMaptwo{\HMapreq{1}}{\HMapreq{2}}}) = \register{}(\atenv{},\HMaptwo{\HMapreq{1}}{\HMapreq{2}})\\
                   &\textsf{f}(\atenv{}, \Unionsplice{\ova{\t{}}}) = \register{}(\atenv{},\Unionsplice{\ova{\fullyresolve{}(\t{})}}) \text{,}\\
                   %&\text{ if } \exists\t{}(\ova{\s{}}^m) \in\ova{\t{}} .\ m > 0\\
                   &\text{ if } \alias{} \in\ova{\t{}}\\
                   &\textsf{f}(\atenv{}, \t{}) = (\atenv{}, \t{}), \text{otherwise}
    \end{array}
  \end{array}
  \begin{array}{lllll}
    \register{} : \atenv{}, \t{} \rightarrow (\atenv{}, \t{})\\
    \register{}(\atenv{}, \t{}) = (\updatemap{\atenv{}}{\alias{}}{\t{}}, \alias{}), \text{ where } \alias{} \text{ is fresh}
    \\\\
    \fullyresolve{} : \atenv{}, \t{} \rightarrow \t{}\\
    \fullyresolve{}(\atenv{}, \alias{}) = \fullyresolve{}(\atenv{}[\alias{}])\\
    \fullyresolve{}(\atenv{}, \t{}) = \t{}  \text{, otherwise}
  \end{array}

  \begin{array}{lllll}
    \aliasesin{} : \t{} \rightarrow \ova{\alias{}}\\
    \aliasesin{}(\alias{}) = [\alias{}]\\
    \aliasesin{}(\t{}(\ova{\s{}})) = \bigcup{\ova{\aliasesin{}(\s{})}}
    \\\\
    \textsf{postwalk} : \atenv{}, \t{}, (\atenv{}, \t{} \rightarrow (\atenv{}, \t{})) \rightarrow (\atenv{}, \t{})\\
    \textsf{postwalk}(\atenv{0}, \t{}(\ova{\s{}}^n), \textsf{w}) = \textsf{w}(\atenv{n}, \t{}(\ova{\sp{}}))\\
    \begin{array}{lllll}
      \text{where}
        &\ova{(\atenv{i}, \sp{i}) = \textsf{postwalk}(\atenv{i-1}, \s{i}, \textsf{w})}\\
    \end{array}
    \\\\
    \mergealiases{} : \atenv{}, \ova{\alias{}} \rightarrow \atenv{}\\
    \mergealiases{}(\atenv{}, []) = \atenv{}\\
    \mergealiases{}(\atenv{}, [\alias{1} ... \alias{n}]) =
        \updatemap{\updatemapmulti{\atenv{}}{\alias{i}}{\alias{1}}}{\alias{1}}{\s{}}\\\
    \begin{array}{llll}
      \text{where } &\s{} = \joinstarexpression{\ova{\replacefor{\textsf{f}(\fullyresolve{}(\atenv{},\alias{i}))}{\alias{1}}{\alias{i}}}}
                    \\
                    &\textsf{f}(\aliasp{}) = \EmptyUnion{}, \text{ if } \aliasp{} \in \ova{\alias{}}\\
                    &\textsf{f}(\Unionsplice{\ova{\t{}}}) = \Unionsplice{\ova{\textsf{f}(\t{})}}\\
                    &\textsf{f}(\t{}) = \t{} \text{, otherwise}
    \end{array}
    \\\\
    %\trymergealias{} : \atenv{}, \alias{}, \alias{} \rightarrow \atenv{}
    %\\
    %\trymergealias{}(\atenv{}, \alias{1}, \alias{2}) =\\
    %\begin{array}{llll}
    %  \textsf{if } \neg \shouldmergeOp{}(\ova{\fullyresolve{}(\atenv{}, \alias{})}) \textsf{ then } \atenv{}\\
    %  \textsf{else } \updatemap{\updatemap{\atenv{}}{\alias{2}}{\alias{1}}}
    %                           {\alias{1}}
    %                           % this fixes the "too much garbage" issue
    %                           {\joinstarexpression{\ova{\replacefor{\atenv{}[\alias{i}]}{\alias{1}}{\alias{2}}}}}\\
    %\end{array}\\
      %fold-based version
    \squashlocal{} : \tenv{} \rightarrow \atenv{}\\
    \squashlocal{}(\tenv{}) = \textbf{fold}(\steptwohelper{}, \emptyatenv{}, \tenv{})\\
    \begin{array}{lllll}
      \text{where} &\steptwohelper{} (\atenv{}, \hastype{\x{}}{\t{}}) = \updatemap{\atenv{2}}{\x{}}{\t{2}}\\
                   &\begin{array}{lllll}
                      \text{where}
                        &(\atenv{1}, \t{1}) = \aliashmap{}(\atenv{}, \t{})\\
                        &(\atenv{2}, \t{2}) = \squashall{}(\atenv{1}, \t{1})\\
                    \end{array}
    \end{array}
      %nonfold-based version
    %\squashlocal{} : \atenv{} \rightarrow \atenv{}\\
    %\squashlocal{}(\atenv{0}) = \atenv{n}\\
    %\begin{array}{lllll}
    %  \text{where}\\
    %  \begin{array}{lllll}
    %    \steptwohelper{} (\atenv{}, \x{}, \t{}) = \replacefor{\atenv{2}}{\t{2}}{\x{}}\\
    %    \begin{array}{lllll}
    %      \text{where}
    %      &(\atenv{1}, \t{1}) = \aliashmap{}(\atenv{}, \t{})\\
    %      &(\atenv{2}, \t{2}) = \squashall{}(\atenv{1}, \t{1})\\
    %    \end{array}
    %    \\
    %    \ova{\hastype{\x{}}{\t{}}}^n = \atenv{0}[\tenv{}]\\
    %    \ova{\atenv{i} = \steptwohelper{}(\atenv{i-1},{\x{i}},{\t{i}})} \\
    %  \end{array}
    %\end{array}
  \end{array}
  \begin{array}{llll}
    \squashall{} : \atenv{}, \t{} \rightarrow \atenv{}\\
    \squashall{}(\atenv{0}, \t{}) = \atenv{n} \\
    \begin{array}{llll}
      \text{where }
      &\ova{\alias{}}^n = \aliasesin{}(\t{})\\
      &\ova{\atenv{i} = \squash{}(\atenv{i-1}, [\alias{i}], [])}\\
    \end{array}
    \\\\
    \squash : \atenv{}, \ova{\alias{}}, \ova{\alias{}} \rightarrow \atenv{}\\
    \squash(\atenv{}, [], \textsf{d}) = \atenv{}\\
    \squash(\atenv{}, \alias{1} :: \textsf{w}, \textsf{d}) = 
      \squash(\atenvp{}, \textsf{w} \cup \textsf{as}, \textsf{d} \cup \{\alias{1}\})\\
    \begin{array}{lllll}
      \begin{array}{@{}llll}
        \text{where }
        &\textsf{as} = \aliasesin{}(\atenv{}[\alias{1}]) \setminus \textsf{d}\\
        &\textsf{ap} = \textsf{d} \setminus \{\alias{1}\}\\
        &\begin{array}{@{}llll}
           \textsf{f}(\atenv{}, \alias{2}) = 
             &\textsf{if } \neg \shouldmergeOp{}(\ova{\fullyresolve{}(\atenv{}, \alias{})}),\\
             &\textsf{then } \atenv{}\\
             &\textsf{else } \mergealiases{}(\atenv{}, \ova{\alias{i}})\\
        \end{array}\\
        &\begin{array}{@{}llll}
          \atenvp{} = &\textsf{if } \alias{} \in \textsf{d} \textsf{, then } \atenv{} \text{,}\\
          &\textsf{else } \textsf{fold}(\textsf{f}, \atenv{}, \textsf{ap} \cup \textsf{as})
        \end{array}
      \end{array}
    \end{array}
    \\\\
    \shouldmergeOp{} : \ova{\t{}} \rightarrow \textbf{Bool}\\
    \shouldmergeOp{}(\ova{\HMaptwo{\HMapreq{i}}{\HMapopt{i}}}) = \exists\kw{}. \ova{(\kw{}, \kw{i}) \in \HMapreq{i}}\\
    \shouldmergeOp{}(\ova{\t{}}) = \textbf{F}, \text{ otherwise}
  \end{array}

\end{mathpar}
\caption{Definition of $\squashlocal{}(\tenv{}) = \atenv{}$
%\aliasesin{}(\t{}) returns the set of aliases that syntactically occur in \t{}.
%  Step 2 summary: Create aliases for HMaps (graph nodes), then squash recursive types locally
%(don't try to merge data examples from different paths). Omitted: follow-aliases call, that erases
%redundant aliases.
  }
  \label{infer:fig:squashlocal}
\end{figure*}

\begin{figure*}
\begin{mathpar}
  \begin{array}{lllll}
    \textsf{req}: \atenv{}, \alias{} \rightarrow \HMapreq{}\\
    \textsf{req}(\atenv{}, \alias{}) = \textsf{req}(\atenv{}, \atenv{}[\alias{}])\\
    \textsf{req}(\atenv{}, \HMaptwo{\HMapreq{}}{\HMapopt{}}) = \HMapreq{}\\
    \\\\
    %\textsf{req}: \atenv{}, \alias{} \rightarrow \ova{\HMapreq{}}\\
    %\textsf{req}(\atenv{}, \alias{}) = \textsf{req}(\atenv{}, \alias{})\\
    %\textsf{req}(\atenv{}, \HMaptwo{\HMapreq{}}{\HMapopt{}}) = [\HMapreq{}]\\
    %\textsf{req}(\atenv{}, \Unionsplice{\ova{\t{}}}) = \bigcup\ova{\textsf{req}(\atenv{},\t{})}\\
    %\textsf{req}(\atenv{}, \t{}) = [] \text{, otherwise}
    \squashhorizonally{} : \atenv{} \rightarrow \atenv{}\\
    \squashhorizonally{}(\atenv{}) = \\
    \ \ \ \ 
      \textsf{fold}(\mergealiases{}, \atenv{}, \textsf{groupSimilarReq}(\atenv{}))\\\\
    \squashglobal{} : \atenv{} \rightarrow \atenv{}\\
    \squashglobal{} = \\
    \ \ \ \ \squashhorizonally{} \circ \aliassinglehmap{}
  \end{array}
  \begin{array}{lllll}
    \aliassinglehmap{} : \atenv{} \rightarrow \atenv{}\\
    \aliassinglehmap{}(\atenv{}) = \textsf{fold}(\textsf{f}, \atenvp{}, \atenvp{}[\aenv{}])\\
    \begin{array}{llll}
      \text{where } &\atenvp{} = \textsf{fold}(\singlehmap{}, \atenv{}, \atenv{}[\tenv{}])\\
                    &\textsf{f}(\atenv{0}, \t{}(\ova{\s{}}^n)) = (\atenv{n}, \t{}(\sp{})) \text{, if } \t{} = \HMaptwo{\HMapreq{}}{\HMapopt{}}\\
                    &\begin{array}{lllll}
                      \text{where } \ova{(\atenv{i}, \s{i}) = \singlehmap{}(\atenv{i-1}, \s{i})}
                     \end{array}
                    \\
                    &\textsf{f}(\atenv{}, \t{}) = \singlehmap{}(\atenv{}, \t{}) \text{, otherwise}
    \end{array}
    \\\\
    \singlehmap{} : \forall \alpha. \atenv{}, (\alpha, \t{}) \rightarrow \atenv{}\\
    \singlehmap{}(\atenv{}, (\textsf{x}, \t{})) = \updatemap{\atenv{}}{\textsf{x}}{\s{}}\\
    \begin{array}{lllll}
      \text{where} &(\atenvp{}, \s{}) = \textsf{postwalk}(\atenv{}, \t{}, \textsf{f})\\
                   &\textsf{f}(\atenv{}, {\HMaptwo{\HMapreq{1}}{\HMapreq{2}}}) = \register{}(\atenv{},\HMaptwo{\HMapreq{1}}{\HMapreq{2}})\\
                   &\textsf{f}(\atenv{}, \t{}) = (\atenv{}, \t{}), \text{otherwise}
    \end{array}
    %\\\\
    %\inferrecOp{} : \atenv{} \rightarrow \atenv{}\\
    %\inferrecOp{} = \squashglobal{} \circ \squashlocal{}
  \end{array}

  \begin{array}{lllll}
    \textsf{groupSimilarReq} : \atenv{} \rightarrow \ova{\ova{\alias{}}}\\
    \textsf{groupSimilarReq}(\atenv{}) = 
                   [\ova{\alias{}} | \ova{\kw{}} \in \textsf{dom}(\textsf{r}),
                                     \ova{\alias{}} = \textsf{remDiffTag}(\textsf{similarReq}(\ova{\kw{}}))
                                     ]\\
    \begin{array}{lllll}
      \text{where} &\textsf{r} = \{(\ova{\kw{}}, \ova{\alias{}}) | \HMaptwo{\{\ova{\kw{}\ \t{}}\}}{\HMapopt{}} \in \textsf{rng}(\atenv{}[\aenv{}]),
                                                                              \ova{\alias{}} = \textsf{matchingReq}(\ova{\kw{}})\}\\
                   &\textsf{matchingReq}(\ova{\kw{}}) = [\alias{} | (\alias{}, \HMaptwo{\HMapreq{}}{\HMapopt{}}) \in \atenv{}]\\
                   &\textsf{similarReq}(\ova{\kw{}}) = [\alias{} | \ova{\kwp{}}^n \subseteq \ova{\kw{}}^m,
                                                                   m-n \leq \textsf{thres}(m),
                                                                   \alias{} \in \textsf{r}[\ova{\kwp{}}]]\\
                   &\textsf{remDiffTag}(\ova{\alias{}}) = [\aliasp{} | \aliasp{} \in \ova{\alias{}},
                                                                       \text{ if } (\kw{}, \kwp{}) \in \textsf{req}(\atenv{}, \aliasp{})
                                                                       \text{ and } 
                                                                       \bigvee\ova{(\kw{}, \kwpp{}) \in \textsf{req}(\atenv{}, \alias{})}
                                                                       \text{ then }
                                                                       \ova{\kwp{} = \kwpp{}}
                                                                      ]
    \end{array}
  \end{array}
\end{mathpar}
\caption{
    Definition of $\squashglobal{}(\atenv{}) = \atenvp{}$ %\\
%  Step 3 summary: First ensure all HMaps correspond to an alias. Then merge
%aliases that point to a HMaps with identical required keysets (aliases must point to exactly one
%top-level HMap, no unions).
  }
  \label{infer:fig:squashglobal}
\end{figure*}


We now describe the algorithm for generating recursive type aliases.
The first step \squashlocal{} creates recursive types from directly nested types.
It folds over each type in the type environment, first
creating aliases with \aliashmap{}, and then
attempting to merge these aliases by \squashall{}.

A type is aliased by \aliashmap{} either if it is a union containing a HMap,
or a HMap that is not a member of a union.
While we will use the structure of HMaps to determine when to create a recursive
type, keeping surrounding type information close to HMaps helps create more
compact and readable recursive types.
The implementation uses a post-order traversal via \textsf{postwalk},
which also threads an annotation environment as it applies
the provided function.

Then, \squashall{} follows each alias \alias{i} reachable from the type environment
and attempts to merge it with any alias reachable from \alias{i}.
The \squash{} function maintains 
a set of already visited aliases to avoid infinite loops.

The logic for merging aliases is contained in \mergealiases{}.
Merging \alias{2} into \alias{1} involves mapping \alias{2}
to \alias{1} and \alias{1} to the join of both definitions.
Crucially, before joining, we rename occurrences of 
\alias{2} to \alias{1}. This avoids a linear increase in the
width of union types, proportional to the number of merged aliases.
%For example,
%without this, if we further merge \alias{3} into \alias{1},
%like \Unionsplice{{\alias{1}}\ {\alias{2}}\ {\alias{3}}} occur
%instead of simply \alias{1}.
The running time of our algorithm is proportional to the
width of union types (due to the quadratic combination of
unions in the join function) and this optimization greatly
helped the running time of several benchmarks.
To avoid introducing infinite types,
top-level references to other aliases we are merging with
are erased with the helper \textsf{f}.

The \shouldmergeOp{} function determines whether two types are related enough
to warrant being merged. We present our current implementation, which is simplistic,
but is fast and effective in practice, but many variations are possible.
Aliases are merged if they are all HMaps (not contained in unions), that
contain a keyword key in common, with possibly disjoint mapped values.
For example, our opening example has the \clj{:op} key mapped to either
\clj{:leaf} or \clj{:node}, and so aliases for each map would be merged.
Notice again, however, the join operator does not collapse differently-tagged
maps, so they will occur recursively in the resulting alias, but separated by union.

Even though this implementation of \shouldmergeOp{} does not directly utilize the aliased
union types carefully created by \aliashmap{}, they still affect the final types.
For example, squashing \clj{T} in
\begin{cljlisting}
(defalias T 
  (U nil '{:op :node :left '{:op :leaf ...} ...}))
\end{cljlisting}
results in 
\begin{cljlisting}
(defalias T 
  (U nil '{:op :node :left T ...} '{:op :leaf ...}))
\end{cljlisting}
rather than
\begin{cljlisting}
(defalias T2 (U '{:op :node :left T ...}
                '{:op :leaf ...}))
(defalias T (U nil T2))
\end{cljlisting}
An alternative implementation of \shouldmergeOp{} we experimented with included computing sets of keysets
for each alias, and merging if the keysets overlapped. This, and many of our early experimentations,
required expensive computations of keyset combinations and traversals over them that could be emulated
with cruder heuristics like the current implementation.

\Dsubsection{Pass 3: Squash globally}
\label{infer:sec:formal:inference-phase:squash-global}

The final step combines aliases
without restriction on whether they occur ``together''.
This step combines type information between different positions
(such as in different arguments or functions) so that any deficiencies
in unit testing coverage are massaged away.

The \squashglobal{} function is the entry point in this pass,
and is similar in structure to the previous pass.
It first creates aliases for each HMap via \aliassinglehmap{}.
Then, HMap aliases are grouped and merged in \squashhorizonally{}.

The \aliassinglehmap{} function first traverses the type environment
to create HMap aliases via \singlehmap{}, and binds the resulting
envionment as \atenvp{}.
Then, alias environment entries are updated with \textsf{f}, whose first
case prevents re-aliasing a top-level HMap, before we call \singlehmap{}
(\singlehmap{}'s second argument accepts both \x{} and \alias{}).
The \t{}(\ova{\s{}}) syntax represents a type \t{} whose constructor
takes types \ova{\s{}}.

After that, \squashhorizonally{} creates groups of related
aliases with \textsf{groupSimilarReq}.
Each group contains HMap aliases whose required keysets are similar,
but are never differently-tagged.
The code creates a map \textsf{r} from keysets to groups of HMap
aliases with that (required) keyset.
Then, for every keyset \ova{\kw{}}, \textsf{similarReq} adds aliases to the group
whose keysets are a subset of \ova{\kw{}}. The number of missing
keys permitted is determined by \textsf{thres}, for which we do not provide a
definition.
Finally, \textsf{remDiffTag} removes differently-tagged HMaps from each group,
and the groups are merged via \mergealiases{} as before.

\Dsubsection{Implementation}

Further passes are used in the implementation.
In particular, we trim unreachable aliases and remove aliases
that simply point to another alias (like \alias{2} in \mergealiases{})
between each pass.

% Things we could prove:
% - update is commutative

\Dchapter{Evaluation}
\label{infer:chap:evaluation}

We performed a quantitative evaluation of our workflow on several open source programs
in three experiments.
We ported five programs to Typed Clojure with our workflow,
and merely generated types for one larger program we deemed too difficult to port,
but features interesting data types.

Experiment 1 involves a manual inspection of the types from our automatic algorithm.
We detail our experience in generating types for part of an industrial-grade compiler which
we ultimately decided not to manually port to Typed Clojure.
This was because it uses many programming idioms beyond Typed Clojure's capabilities
(those detailed as ``Further Challenges'' by \infercitet{bonnaire2016practical}),
and so the final part of the workflow mostly involves working around its shortcomings.

Experiment 2 studies the kinds of the manual changes needed to port our five programs
to Typed Clojure, starting from the automatically generated annotations.
Experiment 3 enforces the initially generated annotations for these programs at runtime
to check they are meaningfully underprecise.

%\paragraph{cljs.compiler}
%ClojureScript (CLJS) is a Clojure variant that runs on JavaScript
%virtual machines. We infer types for its compiler (written in Clojure)
%which emits JavaScript from
%a recursively defined map-based abstract syntax tree format.

\Dsection{Experiment 1: Manual inspection}
\label{infer:sec:experiment1}

For the first experiment, we manually inspect the types automatically generated by our tool.
We judge our tool's ability to
use recognizable names,
favor compact annotations, and
not overspecify types.

\begin{figure}
  % indented so line numbers can line up more tastefully
\begin{cljlistingnumbered}
  (defalias Op(*@\label{infer:listing:cljs:Op}@*) ; omitted some entries and 11 cases
    (U (HMap :mandatory(*@\label{infer:listing:cljs:Op:op:bindingStart}@*)
             {:op ':binding,(*@\label{infer:listing:cljs:Op:op:binding}@*) :info (U NameShadowMap(*@\label{infer:listing:cljs:Op:op:binding:NameShadowMap}@*) FnScopeFnSelfNameNsMap(*@\label{infer:listing:cljs:Op:op:binding:FnScopeFnSelfNameNsMap}@*)), ...}
             :optional(*@\label{infer:listing:cljs:Op:optional}@*)
             {:env ColumnLineContextMap, :init Op,(*@\label{infer:listing:cljs:Op:optional:init:Op}@*) :shadow (U nil Op),(*@\label{infer:listing:cljs:Op:optional:shadow:Op}@*) ...})(*@\label{infer:listing:cljs:Op:optionalEnd}@*)(*@\label{infer:listing:cljs:Op:op:bindingEnd}@*)
      '{:op ':const,(*@\label{infer:listing:cljs:Op:op:const}@*) :env HMap49305,(*@\label{infer:listing:cljs:Op:op:const:HMap49305}@*) ...}
      '{:op ':do,(*@\label{infer:listing:cljs:Op:op:do}@*) :env HMap49305,(*@\label{infer:listing:cljs:Op:op:do:HMap49305}@*) :ret Op,(*@\label{infer:listing:cljs:Op:op:do:Op}@*) :statements (Vec Nothing)(*@\label{infer:listing:cljs:Op:op:do:statements}@*), ...}
      ...))(*@\label{infer:listing:cljs:Op-End}@*)
  (defalias ColumnLineContextMap(*@\label{infer:listing:cljs:ColumnLineContextMap}@*)
    (HMap :mandatory {:column Int, :line Int} :optional {:context ':expr}(*@\label{infer:listing:cljs:ColumnLineContextMap:optional}@*)))(*@\label{infer:listing:cljs:ColumnLineContextMapEnd}@*)
  (defalias HMap49305 ; omitted some extries(*@\label{infer:listing:cljs:HMap49305}@*)
    (U nil
       '{:context ':statement, :column Int, ...}
       '{:context ':return, :column Int, ...}
       (HMap :mandatory {:context ':expr, :column Int, ...} :optional {...})))(*@\label{infer:listing:cljs:HMap49305End}@*)
  (ann emit [Op -> nil])(*@\label{infer:listing:cljs:emit}@*)
  (ann emit-dot [Op -> nil])(*@\label{infer:listing:cljs:emit-dot}@*)
\end{cljlistingnumbered}
\caption{Sample generated types for cljs.compiler.
}
\label{infer:fig:cljs}
  %(ann emit-let [Op Any -> Any])(*@\label{infer:listing:cljs:emit-let}@*)
%    '{:op ':fn-method,
%      :body Op,
%      :children '[':params ':body],
%      :env HMap49305,
%      :fixed-arity Int,
%      :form (Coll (Coll Any)),
%      :name Op,
%      :params '[Op],
%      :recurs nil,
%      :type nil,
%      :variadic? false}
%    '{:op ':host-call,
%      :args '[Op],
%      :children Any,
%      :env context-statement-tmp-HMap-alias20275,
%      :form (Coll Sym),
%      :method Sym,
%      :tag Any,
%      :target Op}
%    '{:op ':host-field,
%      :children '[':target],
%      :env context-statement-tmp-HMap-alias20275,
%      :field Sym,
%      :form (Coll Sym),
%      :tag Sym,
%      :target Op}
%    '{:op ':if,
%      :children '[':test ':then ':else],
%      :else Op,
%      :env context-statement-tmp-HMap-alias20275,
%      :form (Coll Any),
%      :tag (Set (U nil Sym)),
%      :test Op,
%      :then Op,
%      :unchecked Boolean}
%    '{:op ':invoke,
%      :args '[Op],
%      :children '[':fn ':args],
%      :env context-statement-tmp-HMap-alias20275,
%      :fn Op,
%      :form (Coll Any),
%      :tag Sym}
%    (HMap
%      :mandatory
%      {:op ':js,
%       :env context-statement-tmp-HMap-alias20275,
%       :form (Coll (U nil Str Sym)),
%       :js-op Sym,
%       :numeric nil,
%       :tag Sym}
%      :optional
%      {:args '[Op Op],
%       :children '[':args],
%       :code Str,
%       :segs (Coll Str)})
%    (HMap
%      :mandatory
%      {:op ':js-var, :name Sym, :ns Sym}
%      :optional
%      {:tag Sym})
%    '{:op ':let,
%      :bindings '[Op Op Any],
%      :body Any,
%      :children Any,
%      :env context-statement-tmp-HMap-alias20275,
%      :form Any,
%      :tag Any}
%    (HMap
%      :mandatory
%      {:op ':local,
%       :env context-statement-tmp-HMap-alias20275,
%       :form Sym,
%       :info Op,
%       :local (U ':arg ':let),
%       :name Sym}
%      :optional
%      {:arg-id Int, :init Op, :tag Sym})
%    '{:op ':map,
%      :children '[':keys ':vals],
%      :env context-statement-tmp-HMap-alias20275,
%      :form AMap,
%      :keys '[Op],
%      :tag Sym,
%      :vals '[Op]}
%    (HMap
%      :mandatory
%      {:op ':var, :name Sym, :ns Sym}
%      :optional
%      {:arglists (Coll Any),
%       :arglists-meta (Coll nil),
%       :column Int,
%       :doc Str,
%       :end-column Int,
%       :end-line Int,
%       :env context-statement-tmp-HMap-alias20275,
%       :file (U nil Str),
%       :fn-var Boolean,
%       :form Sym,
%       :info (U nil ColumnFileLineMap),
%       :line Int,
%       :max-fixed-arity Int,
%       :meta
%       (U
%         ColumnFileLineMap__0
%         FileArglistsColumnMap
%         ColumnEndColumnEndLineMap),
%       :method-params (Coll (Coll Sym)),
%       :protocol-impl nil,
%       :protocol-inline nil,
%       :ret-tag Sym,
%       :tag Sym,
%       :top-fn ArglistsArglistsMetaMaxFixedArityMap,
%       :variadic? Boolean})))
\end{figure}

We take this opportunity to juxtapose some strengths and weaknessess
of our tool by discussing a somewhat problematic benchmark,
a namespace from the ClojureScript compiler called cljs.compiler
(the code generation phase).
We generate 448 lines of type annotations
for the 1,776 line file, and present a sample
of our tool's output as \figref{infer:fig:cljs}.
We were unable to fully complete the porting to Typed Clojure due to
type system limitations, but the annotations yielded by this benchmark
are interesting nonetheless.

The compiler's AST format is inferred as \clj{Op} (lines \ref{infer:listing:cljs:Op}-\ref{infer:listing:cljs:Op-End})
with 22 recursive references
(like lines \ref{infer:listing:cljs:Op:optional:init:Op}, \ref{infer:listing:cljs:Op:optional:shadow:Op}, \ref{infer:listing:cljs:Op:op:do:Op})
and 14 cases distinguished by \clj{:op} (like lines \ref{infer:listing:cljs:Op:op:binding},
\ref{infer:listing:cljs:Op:op:const}, \ref{infer:listing:cljs:Op:op:do}),
5 of which have optional entries (like lines \ref{infer:listing:cljs:Op:optional}-\ref{infer:listing:cljs:Op:optionalEnd}).
To improve inference time,
only the code emission unit tests were exercised (299 lines containing 39 assertions)
which normally take 40 seconds to run, from which we
generated 448 lines of types and 517 lines of specs
in 2.5 minutes on a 2011 MacBook Pro (16GB RAM, 2.4GHz i5),
in part because of key optimizations discussed in \Dchapref{infer:sec:extensions}.

The main function of the code generation phase is \clj{emit}, which
effectfully converts a map-based AST
to JavaScript.
The AST is created by functions in cljs.analyzer,
a significantly larger 4,366 line Clojure file.
Without inspecting cljs.analyzer,
our tool annotates \clj{emit} on line \ref{infer:listing:cljs:emit}
with a recursive AST type \clj{Op} (lines \ref{infer:listing:cljs:Op}-\ref{infer:listing:cljs:Op-End}).

Similar to our opening example \clj{nodes}, it uses the \clj{:op}
key to disambiguate between (16) cases, and has recursive
references (\clj{Op}).
We just present the first 4 cases.
The first case \clj{':binding} has 4 required
and 8 optional entries, whose
\clj{:info} and \clj{:env} entries refer to
other \clj{HMap} type aliases generated by the tool.
%%deleted this code
%Similar to \clj{:op},
%the \clj{:local} entry maps to a keyword singleton
%type,
%however our tool wisely chose to cluster types 
%based on the \clj{:op} entry since it is common to all cases.

%\Dsection{Philosophy}

An important question to address is ``how accurate are these annotations?''.
Unlike previous work in this area~\infercitep{An10dynamicinference}, we do not aim for soundness guarantees
in our generated types. 
A significant contribution of our work is a tool that Clojure programmers
can use to help learn about and specify their programs.
In that spirit, we strive to generate annotations meeting more qualitative criteria.
Each guideline by itself helps generate more useful annotations, and
they combine in interesting ways help to make up for shortcomings.
%in generated annotations.
%which we outline along with a commentary
%judging \figref{infer:fig:cljs} along these lines.

\paragraph{Choose recognizable names}
%Typed Clojure and clojure.spec annotations are abundant
%with useful names for types.
Assigning a good name for a type increases
readability by succinctly conveying its purpose.
Along those lines, a good name for the AST representation
on lines \ref{infer:listing:cljs:Op}-\ref{infer:listing:cljs:Op-End}
might be \clj{AST} or \clj{Expr}.
However, these kinds of names can be very misleading when incorrect, so
instead of guessing them,
our tool takes a more consistent approach and generates \emph{easily recognizable}
names based on the type the name points to.
Then, those with a passing familiarity with the data flowing through the program
can quickly identify and rename them.
For example,
\begin{itemize}
  \item
    \clj{Op} (lines \ref{infer:listing:cljs:Op}-\ref{infer:listing:cljs:Op-End})
    is chosen because \clj{:op} is
    clearly the dispatch key (the \clj{:op} entry is also helpfully placed
    as the first entry in each case to aid discoverability),
  \item
    \clj{ColumnLineContextMap} (lines \ref{infer:listing:cljs:ColumnLineContextMap}-\ref{infer:listing:cljs:ColumnLineContextMapEnd})
    enumerates the keys of the map type it points to,
  \item
    \clj{NameShadowMap} and \clj{FnScopeFnSelfNameNsMap} (%referenced on
    line
    \ref{infer:listing:cljs:Op:op:binding:NameShadowMap}% and \ref{infer:listing:cljs:Op:op:binding:FnScopeFnSelfNameNsMap}
    )
    similarly, and
  \item
    \clj{HMap49305} (lines \ref{infer:listing:cljs:HMap49305}-\ref{infer:listing:cljs:HMap49305End})
    shows how our tool fails to give names to certain combinations
    of types (we now discuss the severity of this particular situation).
\end{itemize}

A failure of cljs.compiler's
generated types was \clj{HMap49305}.
It clearly fails to be a recognizable name.
However, all is not lost:
the compactness and recognizable names of other adjacent annotations
makes it plausible for a programmer with some
knowledge of the AST representation to 
recover.
In particular 13/14 cases in \clj{Op}
have entries from \clj{:env} to \clj{HMap49305}, 
(like lines \ref{infer:listing:cljs:Op:op:const:HMap49305} and \ref{infer:listing:cljs:Op:op:do:HMap49305}),
and the only exception (line \ref{infer:listing:cljs:Op:optional:init:Op})
maps to \clj{ColumnLineContextMap}. From this information the user can
decide to combine these aliases.



%Good names can sometimes be reconstructed from the program source,
%like function or parameter names, and other times 
%we can use the shape of a type to summarize it.

\paragraph{Favor compact annotations}
Literally translating runtime observations into
annotations without compacting them
leads to unmaintainable and impractical types resembling
TypeWiz's ``verbatim'' annotation for \clj{nodes}.
To avoid this, we
  use optional keys where possible, like line \ref{infer:listing:cljs:ColumnLineContextMap:optional},
  infer recursive types like \clj{Op}, and
  reuse type aliases in function annotations, like
    \clj{emit} and \clj{emit-dot} (lines \ref{infer:listing:cljs:emit}, \ref{infer:listing:cljs:emit-dot}).

One remarkable success in the generated types
was the automatic inference \clj{Op} (lines \ref{infer:listing:cljs:Op}-\ref{infer:listing:cljs:Op-End})
with 14 distinct cases, and other features described in \figref{infer:fig:cljs}.
Further investigation reveals that
the compiler actually features 36 distinct AST nodes---unsurprisingly, 39 assertions was not sufficient
test coverage to discover them all.
However, because of the recognizable name and organization of
\clj{Op}, it's clear where to add the missing nodes
if no further tests are available.

These processes of compacting annotations often makes them more general,
which leads into our next goal.

%Idiomatic Clojure code rarely mixes certain types in the same position,
%unless the program is polymorphic. Using this knowledge---which we observed
%by the annotations and specs assigned to idiomatic Clojure 
%code---we can rule out certain combinations of types to compact our
%resulting output, without losing information that would help us
%type check our programs.

\paragraph{Don't overspecify types}
Poor test coverage can easily skew the results of dynamic analysis tools,
so we choose to err on the side of generalizing types
where possible.
Our opening example \clj{nodes}
is a good example of this---our inferred type
is recursive, despite \clj{nodes} only being tested with a tree of height 2.
This has several benefits.
\begin{itemize}
  \item We avoid exhausting the pool of easily recognizable names
    by generalizing types to communicate the general role
    of an argument or return position.
    For example, \clj{emit-dot} (line \ref{infer:listing:cljs:emit-dot})
    is annotated to take \clj{Op}, but in reality accepts only a subset
    of \clj{Op}.
    Programmers can combine the recognizability of \clj{Op} with the
    suggestive name of \clj{emit-dot} (the dot operator in Clojure handles host interoperability) to decide whether, for instance,
    to split \clj{Op} into smaller type aliases
    or add type casts in the definition of \clj{emit-dot} to please 
    the type checker
    (some libraries require more casts than others to type check, as discussed in \secref{infer:sec:experiment2}).
  \item Generated Clojure spec annotations (an extension discussed in \secref{infer:sec:spec-extension})
        are more likely to accept valid input with specs enabled, even with incomplete unit tests
        (we enable generated specs on several libraries in \secref{infer:sec:experiment3}).
  \item Our approach becomes more amenable to extensions improving the running time
        of runtime observation without significantly deteriorating annotation quality,
        like lazy tracking (\secref{infer:sec:lazy-tracking}).
\end{itemize}

Several instances of overspecification are evident,
such as the \clj{:statements} entry of a \clj{:do} AST node being inferred as an always-empty vector
(line \ref{infer:listing:cljs:Op:op:do:statements}).
In some ways, this is useful information, showing that
test coverage for \clj{:do} nodes could be improved.
To fix the annotation, we could rerun the tool with better tests.
If no such test exists, we would have to fall back
to reverse-engineering code to identify the correct
type of \clj{:statements}, which is \clj{(Vec Op)}.

Finally, 19 functions in cljs.compiler are annotated to 
take or return \clj{Op} (like lines \ref{infer:listing:cljs:emit}, \ref{infer:listing:cljs:emit-dot}).
This kind of alias reuse enables annotations
to be relatively compact (only 16 type aliases are used by the
49 functions that were exercised).

%
%We rate the quality of generated annotations
%on several axes.
%
%\paragraph{Compactness} Type annotations should be succinct,
%        but without sacrificing too much accuracy.
%        Are our type aliases intelligently combined
%        with good choices for optional keys?
%
%  \paragraph{Accuracy} Would executing a program with these
%      type annotations cause an error?
%      Have we too eagerly erased information in favor
%      of compactness?
%
%  \paragraph{Organization} Have we chosen good recursive types?
%      Do they have good names?
%
%
%\figref{infer:fig:gentype} shows our results.
%Our first program is an implementation of a
%1971 Star Trek game.
%It comes with minimal tests, so to complete this experiment,
%we instead played the game for 30 seconds.

%\begin{figure*}
%  \footnotesize
%\begin{tabular}
%{|         l   || l   | l  | l   || l  | l | l | l | l | l | l | l | l | l | l | l | l | l |}
%  Lib           & LOC  & GT  & LA & MD      & C  & I & P & L & S & O & U & N & V & R & K & F & H \\ 
%  \hline
%  \hline
%  sc            & 166  & 133 & 3  & 70/41   & 5  & 0 & 0 & 2 & 13& 1 & 5 & 1 & 1 & 2 & 0 & 0 & 0 \\
%  mc            & 923  & 395 & 147& 124/120 & 23 & 1 & 11& 19& 2 & 5 & 0 & 9 & 3 & 2 & 4 & 1 & 3 \\
%  fs            & 588  & 157 & 1  & 119/86  & 50 & 0 & 0 & 2 & 3 & 4 & 4 & 11& 2 & 9 & 0 & 0 & 0 \\
%  dj            & 528  & 168 & 9  & 94/125 \\
%  mo            & 530  & 49  & 1  & 46/26%\\
% %data.xml      &      & \\
% % cc            & 1776 & 448 & 4  & N/A 
%  %\\
%\end{tabular}
%  \caption{\emph{The number of type annotations generated for each program}:
%  Lib = Abbreviated library names in the order we introduce them on page \pageref{infer:chap:evaluation},
%  LOC = Number of lines of code we generate types for,
%  GT = Total number of lines of generated types after running our tool,
%  LA = The number of local annotations generated by our tools.
%  \emph{Number of manual changes needed to type check, and why they were needed}:
%  MD = Lines added/removed diff from git comparing initial generated types to
%       the manual amendments needed to
%       type check with Typed Clojure (unless it was too difficult to port),
%  C = Casts,
%  I = Instantiation,
%  P = Polymorphic annotation,
%  L = Local annotation,
%  S = Work around type system Shortcoming,
%  O = Overprecise argument type,
%  U = Uncalled function due to bad test coverage,
%  N = Add No-check annotation to skip checking function,
%  V = Add Variable arity argument type,
%  R = Overprecise return type,
%  K = Add Keyword argument types,
%  F = Added filter annotation,
%  H = Erase/upcast HVec annotation.
%  }
%\end{figure*}

\Dsection{Experiment 2: Changes needed to type check}
\label{infer:sec:experiment2}
% TODO examples for all kinds of things
% TODO bucket how many changes are needed for each kind of thing
%      - eg. varargs, polymorphism
% TODO how many lines of code were skipped

We used our workflow to port the following open source Clojure programs to Typed Clojure.

\paragraph{startrek-clojure}
A reimplementation of a Star Trek text adventure game,
created as a way to learn Clojure.

\paragraph{math.combinatorics}
The core library for common combinatorial functions
on collections,
with implementations based on Knuth's Art of Computer
Programming, Volume 4.

\paragraph{fs}
A Clojure wrapper library over common file-system operations.

\paragraph{data.json}
A library for working with JSON.

%\paragraph{data.xml} A library for manipulating and outputting XML in Clojure.

\paragraph{mini.occ}
A model of occurrence typing by an author of the
current paper. It utilizes three mutually recursive
ad-hoc structures to represent expressions, types,
and propositions.

In this experiment, we first generated types with our algorithm
by running the tests, then amended the program so that it
type checks.
\figref{infer:fig:gentype} summarizes our results.
After the lines of code we generate types for, the next two columns show how many lines of
types were generated and the lines manually changed, respectively.
The latter is a git line diff between commits of the initial
generated types and the final manually amended annotations.
While an objectively fair measurement,
it is not a good indication of the effort needed to port annotations
(a 1 character changes on a line is represented by 1 line addition and 1 line deletion)
The rest of the table enumerates the different kinds of changes needed 
and their frequency.

\begin{figure*}
\begin{tabular}{|r||c|c|c||c|c|c|c|c|c|c|c|c|c|c|c|c|c|c|}
  Library       & \rotatebox{270}{Lines of code}
                & \rotatebox{270}{Lines of Generated Global/Local Types}
                & \rotatebox{270}{Lines manually added/removed}
                & \rotatebox{270}{Casts/Instantiations}
                & \rotatebox{270}{Polymorphic annotation}
                & \rotatebox{270}{Local annotation}
                & \rotatebox{270}{Type System Workaround/no-check}
                & \rotatebox{270}{Overprecise argument/return type}
                & \rotatebox{270}{Uncalled function (bad test coverage)}
                & \rotatebox{270}{Variable-arity/keyword arg type}
                & \rotatebox{270}{Add occurrence typing annotation}
                & \rotatebox{270}{Erase or upcast HVec annotation}
                & \rotatebox{270}{Add missing case in defalias}
                \\ 
  \hline
  \hline
  startrek   & 166  & 133/3   & 70/41    & 5  / 0 & 0 & 2 & 13/1 & 1 /2 & 5 &  1 /  0 & 0 & 0 & 0\\
  math.comb  & 923  & 395/147 & 124/120  & 23 / 1 & 11& 19& 2 /9 & 5 /2 & 0 &  3 /  4 & 1 & 3 & 0\\
  fs         & 588  & 157/1   & 119/86   & 50 / 0 & 0 & 2 & 3 /11& 4 /9 & 4 &  2 /  0 & 0 & 0 & 0\\
  data.json  & 528  & 168/9   & 94/125   & 6  / 0 & 0 & 2 & 4 /5 & 11/7 & 5 &  0 /  20& 0 & 0 & 0\\
  mini.occ   & 530  & 49/1    & 46/26    & 7  / 0 & 0 & 2 & 5 /2 & 4 /2 & 6 &  0 /  0 & 0 & 1 & 5\\
 % cc            & 1776 & 448 & 4  & N/A 
  %\\
\end{tabular}
  \caption{Lines of generated annotations, git line diff for total manual changes to type check the program,
  and the kinds of manual changes.
  }
  \label{infer:fig:gentype}
\end{figure*}

\paragraph{Uncalled functions}
A function without tests receives a broad type annotation that
must be amended.
%
For example, the startrek-clojure game has several exit
conditions, one of which is running out of time.
Since the tests do not specifically call this function,
nor play the game long enough to invoke this condition,
no useful type is inferred.

\begin{cljlisting}
(ann game-over-out-of-time AnyFunction)
\end{cljlisting}

In this case, minimal effort is needed to amend this
type signature: the appropriate type alias
already exists:

\begin{cljlisting}
(defalias CurrentKlingonsCurrentSectorEnterpriseMap
  (HMap :mandatory
    {:current-klingons (Vec EnergySectorMap),
     :current-sector (Vec Int), ...}
    :optional {:lrs-history (Vec Str)}))
\end{cljlisting}
%\begin{cljlisting}
%(defalias CurrentKlingonsCurrentSectorEnterpriseMap
%  (HMap :mandatory
%    {:current-klingons (Vec EnergySectorMap),
%     :current-sector (Vec Int), 
%     :enterprise EnergyIsDockedQuadrantMap,
%     :quads (Vec BasesKlingonsQuadrantMap), 
%     :stardate CurrentEndStartMap,
%     :starting-klingons Int}
%    :optional {:lrs-history (Vec Str)}))
%\end{cljlisting}

So we amend the signature as

\begin{cljlisting}
(ann game-over-out-of-time
  [(Atom1 CurrentKlingonsCurrentSectorEnterpriseMap) 
   -> Boolean])
\end{cljlisting}


\paragraph{Over-precision}
Function types are often too restrictive due to
insufficient unit tests.

There are several instances of this in math.combinatorics.
The \clj{all-different?} function
takes a collection and returns true only if the collection
contains distinct elements.
As evidenced in the generated type, the tests exercise
this functions with collections of integers, atoms,
keywords, and characters.

\begin{cljlisting}
(ann all-different?
  [(Coll (U Int (Atom1 Int) ':a ':b Character)) 
   -> Boolean])
\end{cljlisting}

In our experience, the union is very rarely a good candidate
for a Typed Clojure type signature, so a useful heuristic to improve
the generated types would be to upcast such unions to a more permissive
type, like \clj{Any}.
When we performed that case study, we did not yet add that heuristic
to our tool,
so in this case, we manually amend the signature as

\begin{cljlisting}
(ann all-different? [(Coll Any) -> Boolean])
\end{cljlisting}

Another example of overprecision is the generated type
of \clj{initial-perm-numbers} a helper function
taking a \emph{frequency map}---a hash map from values
to the number of times they occur---which is the shape
of the return value of the core \clj{frequencies}
function.

The generated type shows only a frequency map where
the values are integers are exercised.
%
\begin{cljlisting}
(ann initial-perm-numbers
  [(Map Int Int) -> (Coll Int)])
\end{cljlisting}
%
A more appropriate type instead takes \clj{(Map Any Int)}.
%
%\begin{cljlisting}
%(ann initial-perm-numbers
%  [(Map Any Int) -> (Coll Int)])
%\end{cljlisting}
%
In many examples of overprecision, while the generated
type might not be immediately useful to check programs,
they serve as valuable starting points and also provide
an interesting summary of test coverage.

\paragraph{Missing polymorphism}

We do not attempt to infer polymorphic function types, 
so these amendments are expected. However, it is useful
to compare the optimal types with our generated ones.

For example, the \clj{remove-nth} function in \clj{math.combinatorics}
returns a functional delete operation on its argument.
Here we can see the tests only exercise this function with
collections of integers.

\begin{cljlisting}
(ann remove-nth [(Coll Int) Int -> (Vec Int)])
\end{cljlisting}

However, the overall shape of the function is intact,
and the manually amended type only requires a few 
keystrokes.

\begin{cljlisting}
(ann remove-nth
  (All [a] [(Coll a) Int -> (Vec a)]))
\end{cljlisting}

Similarly, \clj{iter-perm} could be polymorphic, 
but its type is generated as

\begin{cljlisting}
(ann iter-perm [(Vec Int) -> (U nil (Vec Int))])
\end{cljlisting}

We decided this function actually works over any number,
and bounded polymorphism was more appropriate, encoding
the fact that the elements of the output collection
are from the input collection.

\begin{cljlisting}
(ann iter-perm
  (All [a]
    [(Vec (I a Num)) -> (U nil (Vec (I a Num)))]))
\end{cljlisting}
%
%\paragraph{Missing return}
%Sometimes a function never returns, because of infinite loops
%or exceptions.

\paragraph{Missing argument counts}
Often, variable argument functions are given very precise types.
Our algorithm does not apply any heuristics to approximate
variable arguments --- instead we emit types that reflect
only the arities that were called during the unit tests.

The \clj{math.combinatorics} experiment contains
a good example of this phemonenon in the type inferred
for the \clj{plus} helper function.
From the generated type, we can see the tests exercise this function with 2, 6,
and 7 arguments.

\begin{cljlisting}
(ann plus (IFn [Int Int Int Int Int Int Int -> Int]
               [Int Int Int Int Int Int -> Int]
               [Int Int -> Int]))
\end{cljlisting}

Instead, \clj{plus} is actually variadic and works over any number of arguments.
It is better annotated as the following, which is easy to guess based on
both the annotated type and manually viewing the function implementation.

\begin{cljlisting}
(ann plus [Int * -> Int])
\end{cljlisting}

A similar issue occurs with \clj{mult}.

\begin{cljlisting}
(ann mult [Int Int -> Int]) ;; generated
(ann mult [Int * -> Int])   ;; amended
\end{cljlisting}

A similar issue is inferring keyword arguments. Clojure implements
keyword arguments with normal variadic arguments. Notice
the generated type for \clj{lex-partitions-H},
which takes a fixed argument, followed by some optional integer keyword
arguments. 

\begin{cljlisting}
(ann lex-partitions-H
  (IFn [Int -> (Coll (Coll (Vec Int)))]
       [Int ':min Int ':max Int 
        -> (Coll (Coll (Coll Int)))]))
\end{cljlisting}

While the arity of the generated type is too specific,
we can conceivably use the type to help us write a better one.

\begin{cljlisting}
(ann lex-partitions-H
  [Int & :optional {:min Int :max Int}
   -> (Coll (Coll (Coll Int)))])
\end{cljlisting}

\paragraph{Weaknesses in Typed Clojure}

We encountered several known weaknesses in Typed Clojure's type system
that we worked around.
%
The most invasive change needed was in startrek-clojure, which
strongly updated the global mutable configuration map on initial
play. We instead initialized the map with a dummy
value when it is first created.

\paragraph{Missing \clj{defalias} cases}

With insufficient test coverage, our tool can miss cases in a recursively defined
type.
In particular, mini.occ features three recursive types---for the representation
of types \clj{T}, propositions \clj{P}, and expressions \clj{E}.
For \clj{T}, three cases were missing, along with having to upcast the \clj{:params}
entry from the singleton vector \clj{'[NameTypeMap]}.
Two cases were missing from \clj{E}.
The manual changes are highlighted (\clj{P} required no changes with five cases).

\begin{minipage}[t]{0.54\linewidth}
\begin{cljlisting}
(defalias T
  (U (*@\colorbox{pink}{'\{:T ':not, :type T\}}@*)
     (*@\colorbox{pink}{'\{:T ':refine, :name t/Sym, :prop P\}}@*)
     (*@\colorbox{pink}{'\{:T ':union, :types (t/Set T)\}}@*)
     '{:T ':false}
     '{:T ':fun,
       :params (*@\colorbox{pink}{(t/Vec}@*) NameTypeMap(*@\colorbox{pink}{)}@*),
       :return T}
     '{:T ':intersection, :types (Set T)}
     '{:T ':num}))
\end{cljlisting}
\end{minipage}
%
\begin{minipage}[t]{0.4\linewidth}
\begin{cljlisting}
(defalias E
  (U (*@\colorbox{pink}{'\{:E ':add1\}}@*)
     (*@\colorbox{pink}{'\{:E ':n?\}}@*)
     '{:E ':app, :args (Vec E),
       :fun E}
     '{:E ':false}
     '{:E ':if, :else E,
       :test E, :then E}
     '{:E ':lambda, :arg Sym,
       :arg-type T, :body E}
     '{:E ':var, :name Sym}))
\end{cljlisting}
\end{minipage}


%cljs.compiler uses many polymorphic idioms that Typed Clojure is
%poor at checking, so we deemed it too difficult to attempt to
%type check. In particular, there are many of usages of the
%core functions
%\clj{get-in} and \clj{update-in} (functions that deeply lookup
%and manipulate maps) which are not even assigned types
%in Typed Clojure.
%Many function definitions would need to be ignored by the type
%checker to work around this.
%Furthermore, many manual instantiations
%would be needed to check transducers and polymorphic functions
%passed to other polymorphic functions.

%\begin{verbatim}
%  - get/get-in
%  - apply + kw args
%  - strong updates
%\end{verbatim}

%\paragraph{Possible errors in programs}


\Dsection{Experiment 3: Specs pass unit tests}
\label{infer:sec:experiment3}

Our final experiment uses our tool to
generate specs (\secref{infer:sec:spec-extension})
instead of types.
Specs are checked at runtime,
so to verify the utility of generated specs,
we enable spec checking while
rerunning the unit tests that were used
in the process of creating them.

\begin{figure*}
\begin{tabular}
{|         l   || l   | l || l  | l  | l || l |}
  Library       & LOC  &  Lines of specs  & Recursive & Instance & Het. Map & Passed Tests?\\ 
  \hline
  \hline
  startrek      & 166  &  25  & 0  & 10   & 0  & Yes\\
  math.comb     & 923  &  601 & 0  & 320  & 0  & Yes\\
  fs            & 588  &  543 & 0  & 215  & 0  & Yes\\
  data.json     & 528  &  401 & 0  & 174  & 0  & No (1/79 failed)\\ % pprinting related test
  mini.occ      & 530  &  131 & 3  & 25   & 15 & Yes\\
 %data.xml      &      & \\
 % cc            & 1776 & 448 & 4  & N/A 
  %\\
\end{tabular}
  \caption{Summary of the quantity and kinds of generated specs and whether they passed
  unit tests when enabled.
  The one failing test was related to pretty-printing JSON, and seems to be an artifact
  of our testing environment, as it still fails with all specs removed.
  }
\label{infer:fig:genspec}
\end{figure*}


At first this might seem like a trivial property, but it serves as
a valuable test of our inference algorithm.
The aggressive merging strategies to minimize aliases and
maximize recognizability, while unsound transformations,
are based on hypotheses about Clojure idioms and how
Clojure programs are constructed.
If, hypothetically, we generated singleton specs for numbers
like we do for keywords and did not eventually upcast
them to \clj{number?}, the specs might be too strict
to pass its unit tests.
Some function specs also perform generative testing based on
the argument and return types provided.
If we collapse a spec too much and include it in such
a spec, it might feed a function invalid input.

Thankfully, we avoid such pitfalls, and so
our generated specs pass their tests for the benchmarks
we tried.
\figref{infer:fig:genspec} shows
our preliminary results. All inferred specs pass the unit
tests when enforced, which tells us they are at least well formed.
We had some seemingly unrelated difficulty with a test in data.json which we explain
in the caption.
Since hundreds of invariants are checked---mostly ``instance'' checks that a value is of a particular class or interface---we can also be more confident
that the specs are useful.


%\Dsubsection{Experiment 3: Generating generative tests}

% We should generate the card playing specs in this guide:
% http://clojure.org/guides/spec

% # How evaluate
% ## qualitative
% Does it make sense??
% 
% 1. Don't run, gen type, manual inspection
%   - done on something small but real
%   - star trek game?
% 
% - Try different eval methods on different programs
%   - try different projects on different methods
%
% 2. Generate types, try type checking programs
%   - record what changes needed to get it to
%     type check 
%   - (on a different program than 1.) 
% 
% 3. Generate spec, insert the spec, run the test
%    with the spec on, also generate tests
%   - does spec ignore the input??
%     or just generate tests
%   - best situation:
%     - spec all passes
%     - then types check with minimal changes
%   - Q: can we use spec's tests to improve
%        types, iteratively?
%        (could throw away exceptions, throw
%         away bad input etc., different options
%         here)
% (optional)
% 4. Generate types, use gradual typing

\Dchapter{Related work\either{ to Automatic Annotations}{}}

%The field of dynamic analysis has a rich history.
%Ball~\cite{ball1999concept}
%introduces frequency spectrum analysis,
%an approach that observes a running program
%that is similar to our instrumentation approach.
%Mock~\cite{mock2003dynamic}
%makes the case for efficient profiling of programs
%to better facilitate usage of instrumentation.
%
%Value Profiling is another related area which characterises
%programs based on their running entities.
%
%Daikon~\cite{ernst2001dynamically}
%uses dynamic analysis to recover likely program invariants
%in C programs.
%
%Dynamic type inference has been attempted in many different
%areas.
%Rubydust~\cite{An10dynamicinference}
%infers static types for Ruby. They prove the generated types
%sound, which we do not. 
%\begin{verbatim}
%Conversely, they do not generate
%recursive types, but recursive types in ruby are probably
%nominal, so how different are we?
%\end{verbatim}
%
%In the context of JavaScript, several usages of this technique
%can be found.
%JSTrace~\cite{saftoiu2010jstrace}
%generates types for (?).
%Separately, work has been done to generate JSDoc-like annotations~\cite{odgaard2014}.
%TypeDevil~\cite{pradel2015typedevil}
%uses dynamic analysis to warn JavaScript programmers of possible inconsistencies
%in their programs.
%
%Work in recovering context-free grammars is most related to our algorithm
%to recover recursive types.
%% TODO Shamir~\cite{shamir1962remark} notes that it is impossible
%% TODO \cite{knobe1976method}
%
%In the context of machine learning, 
%this area is called grammar induction or language learning.  % according to vcrepinvsek2005inferring
%% TODO Wang\cite{wang1998grammar} summarises 
%{\v{C}}repin{\v{s}}ek et. al~\cite{vcrepinvsek2005inferring}
%use genetic programming to infer context-free grammars
%for domain-specific languages.
%Most work in this area assume both positive and negative
%examples. We cannot distinguish between these two in
%our system, so we assume all examples are positive.
%
%Chen~\cite{chen1995bayesian} uses Bayesian inference to converge
%on a suitable grammar, given examples.
%
%There has been recent interest in approximate type inference.
%
%Pluquet et. al~\cite{marot2009fast} investigate heuristics
%to quickly infer types in dynamic programs.
%So does Milojkovi{\'c}
%\cite{milojkovic2016exploring}.
%Spasojevi{\'c} et. al~\cite{spasojevic2014mining}
%compare types across a cross section of projects to improve
%inference.
%
%Adamsen et. al~\cite{adamsen2016analyzing} verify test suite completeness using a hybrid approach of lightweight dependency analysis, static type checking and dynamic instrumentation.
%
%% Inference and Evolution of TypeScript Declaration Files
%% - they submit pull requests from their tool's output
%% https://cs.au.dk/~amoeller/papers/tstools/paper.pdf
%
%% Automatic TS annotations from JSON (including recursive types)
%% https://github.com/shakyShane/json-ts

\paragraph{Automatic annotations}
There are two common implementation strategies for automatic annotation tools. The first
strategy, ``ruling-out'' (for invariant detection), assumes all invariants are true 
and then use runtime analysis results to rule out
impossible invariants. The second ``building-up'' strategy (for dynamic type inference)
assumes nothing and uses runtime analysis results to build up invariant/type knowledge.

Examples of invariant detection tools include Daikon~\infercitep{ernst2001dynamically},
DIDUCE~\infercitep{hangal2002tracking}, and Carrot~\infercitep{pytlik2003automated}, and
typically enhance statically typed languages with more expressive types or contracts.
Examples of dynamic type inference include our tool, Rubydust \infercitep{An10dynamicinference},
JSTrace~\infercitep{saftoiu2010jstrace}, and TypeDevil~\infercitep{pradel2015typedevil},
and typically target untyped languages.

Both strategies have different space behavior with respect to representing
the set of known invariants.
The ruling-out strategy typically uses a lot of memory at the beginning,
but then can free memory as it rules out invariants. For example, if
\texttt{odd(x)} and \texttt{even(x)} are assumed, observing \texttt{x = 1}
means we can delete and free the memory recording \texttt{even(x)}.
Alternatively, the building-up strategy uses the least memory storing
known invariants/types at the beginning, but increases memory usage
as more the more samples are collected. For example, if we know
\texttt{x : Bottom}, and we observe \texttt{x = "a"} and \texttt{x = 1}
at different points in the program, we must use more memory to
store the union \texttt{x : String $\cup$ Integer} in our set of known invariants.

\paragraph{Daikon}
Daikon can reason about very expressive relationships between variables
using properties like ordering ($x < y$), linear relationships ($y = ax + b$),
and containment ($x \in y$). It also supports reasoning with ``derived variables''
like fields ($x.f$), and array accesses ($a[i]$).
%
Typed Clojure's dynamic inference can record heterogeneous data structures
like vectors and hash-maps, but otherwise cannot express relationships
between variables.

There are several reasons for this. The most prominent is that Daikon
primarily targets Java-like languages, so inferring simple type information
would be redundant with the explicit typing disciplines of these languages.
On the other hand, the process of moving from Clojure to Typed Clojure
mostly involves writing simple type signatures without dependencies
between variables. Typed Clojure recovers relevant dependent information
via occurrence typing~\infercitep{TF10}, and gives the option to manually annotate necessary
dependencies in function signatures when needed.

\paragraph{Reverse Engineering Programs with Static Analysis}
Rigi~\infercitep{muller1992reverse} analyzes
the structure of large software systems,
combining static analysis 
with a user-facing graphical environment to allow users to view and manipulate
the in-progress reverse engineering results.
We instead use a static type system as a feedback mechanism,
which forces more aggressive compacting of generated annotations.

Lackwit~\infercitep{o1997lackwit} uses static analysis to identify abstract 
data types in C programs. Like our work, they share representations between
values, except they use type inference with representations encoded as types.
Recursive representations are inferred via Felice and Coppos's
work on type inference with recursive types~\infercitep{cardone1991type},
where we rely on our imprecise ``squashing'' algorithms over incomplete runtime samples.

Soft Typing~\infercitep{CF91} uses static analysis to insert runtime checks into untyped
programs for invariants that cannot be proved statically. Our approach is instead to let
the user check the generated annotations with a static type system, with static type errors
guiding the user to manually add casts when needed.

\paragraph{Schema Inference}
\infercitet{baazizi2017schema}
infer structural properties of JSON data using a custom JSON schema format.
Their schema inference algorithm proceeds in two stages:
schema inference and schema fusion.
This resembles our collection and naive type environment construction phases.
There are slight differences between schema fusion and our approach.
Schema fusion upcasts heterogeneous array types to be homogeneous, where
we maintain heterogeneous vector types until a differently-sized
vector type is found in the same position.
We also support function types, which JSON lacks.
While they support nested data, they do not attempt to factor out common types as names
or create recursive types like our squashing algorithms.

\infercitet{discala2016automatic}
present a machine learning algorithm to translate denormalized
and nested data that is commonly found in NoSQL databases to traditional
relational formats used by standard RDBMS.
A key component is a schema generation algorithm which arranges related
data into tables via a matching algorithm which discovers related attributes.
Phases 1 and 2 of their algorithm are similar to our local and global
squashing algorithms, respectively, in that first locally accessible information
is combined, and then global information.
%TODO do they infer recursive schemas?
They identify groups of attributes that have (possibly cyclic) relationships.
%They choose a loose method of relating entities (soft functional dependencies) to compensate
%for data inconsistencies and to help users learn about their data via higher-level abstractions.
Where our squashing algorithms for map types are based on (sets of) keysets---on the 
assumption that related entities use similar keysets---they also join attributes
based on their similar values.
This enables more effective entity matching via equivalent attributes
with different names (e.g., ``Email'' vs ``UserEmail'').
Our approach instead assumes programs are somewhat internally consistent, and instead
optimizes to handle missing samples from incomplete dynamic analysis.

%TODO Inferring XML Schema Definitions from XML Data - Bex, Neven, Vansummeren

% Inference and Evolution of TypeScript Declaration Files
% - they submit pull requests from their tool's output
% https://cs.au.dk/~amoeller/papers/tstools/paper.pdf
\paragraph{Other Annotation Tools}
Static analyzers
for JavaScript
(TSInfer~\infercitep{kristensen2017inference}) and for Python (Typpete~\infercitep{typette18}
and PyType~\infercitep{pytype})
automatically annotate code with types.
PyType and Typpete inferred \texttt{nodes}
as \texttt{(? -> int)}
and \texttt{Dict[(Sequence, object)] -> int}, respectively---our tool 
infers it as \clj{[Op -> Int]} by also generating a compact recursive
type.
Similarly, a class-based translation of
inferred both \texttt{left} and \texttt{right}
fields
as \texttt{Any} by PyType, and as \texttt{Leaf} by Typpete---our tool
uses \clj{Op},
a compact recursive type containing \emph{both} \clj{Leaf} and \clj{Node}.
This is similar to our experience with TypeWiz in \Dchapref{infer:chapter:intro}.
(We were unable to install TSInfer.)

NoRegrets~\infercitep{noregrets2018} uses dynamic analysis to learn how a program
is used, and automatically runs the tests of downstream projects to
improve test coverage.
Their \emph{dynamic access paths} represented as
a series of \emph{actions} are analogous to our paths of path elements.

% distinguishes public/private API

% Python
% - MaxSMT-Based Type Inference for Python 3
%  - cites other python based projects
%  - https://link.springer.com/content/pdf/10.1007%2F978-3-319-96142-2_2.pdf
% - pytype
%  - static analysis to generate python annotations
%  - https://github.com/google/pytype
% - pyannotate
%   - dynamic analysis
%   - https://github.com/dropbox/pyannotate

% A Survey of Dynamic Analysis and Test Generation for JavaScript
%  - http://mp.binaervarianz.de/js_survey_2017.pdf



\section{Introduction}

To contextualize the thread of work I propose in this thesis, this section
gives a general introduction to Clojure and Typed Clojure, and
then motivates the idea of automatically generating Typed Clojure annotations.
%and finally describes clojure.spec along with some of its features.

\section{Clojure with static typing}

% current situation 

The popularity of dynamically-typed languages in software
development, combined with a recognition that types often improve
programmer productivity, software reliability, and performance, has
led to the recent development of a wide variety of optional and
gradual type systems aimed at checking existing programs written in
existing languages.  These include  TypeScript~\cite{typescript} and Flow~\cite{flow} for
JavaScript, Hack~\cite{hack} for PHP, and mypy~\cite{mypy}
for Python among the optional systems, and Typed Racket~\cite{TF08}, Reticulated
Python~\cite{Vitousek14}, and GradualTalk~\cite{gradualtalk} among gradually-typed systems.\footnote{We
  use ``gradual typing'' for systems like Typed Racket with sound
  interoperation between typed and untyped code; Typed Clojure or
 TypeScript which don't
  enforce type invariants we describe as ``optionally typed''.}

One key lesson of these systems, indeed a lesson known to early
developers of optional type systems such as StrongTalk, is that type
systems for existing languages must be designed to work with the
features and idioms of the target language. Often this takes the form
of a core language, be it of functions or classes and objects,
together with extensions to handle distinctive language features.


We synthesize these lessons to present \emph{Typed Clojure}, an
optional type system for Clojure. 
%
Clojure is a dynamically
typed language in the Lisp family---built on the Java Virtual
Machine (JVM)---which has recently gained popularity as an alternative
JVM language.  It offers the flexibility of a Lisp dialect, including
macros, emphasizes a functional style via
immutable data structures, and provides
interoperability with existing Java code, allowing programmers to use
existing Java libraries without leaving Clojure.
%
Since its initial release in 2007, Clojure has been widely adopted for
``backend'' development in places where its support for parallelism,
functional programming, and Lisp-influenced abstraction is desired on
the JVM. As a result, there is an extensive base of existing untyped
programs whose developers can benefit from Typed Clojure,
an experience we discuss in this paper.

Since Clojure is a language in the
Lisp family, we apply the lessons of Typed Racket, an existing gradual type
system for Racket, to the core of Typed Clojure, consisting of an extended
$\lambda$-calculus over a variety of base types shared between all Lisp systems.
%
Furthermore, Typed Racket's \emph{occurrence typing} has proved
necessary for type checking realistic Clojure programs.

\begin{figure*}[t!]
  \normalsize
\begin{lstlisting}
(*typed ann pname [(U File String) -> (U nil String)] typed*)
(defmulti pname class)  ; multimethod dispatching on class of argument
(defmethod pname String [s] (*invoke pname (*interop new File s interop*) invoke*)) ; String case 
(defmethod pname File [f] (*interop .getName f interop*)) ; File case, static null check
(*invoke pname "STAINS/JELLY" invoke*) ;=> "JELLY" :- (U nil Str)
\end{lstlisting}
\caption{A simple Typed Clojure program (delimiters: {\color{interop}Java interoperation (green)}, 
  {\color{types}type annotation (blue)},
  {\color{invoke}function invocation (black)}, {\color{red}collection literal (red)}, {\color{mygray}other (gray)})}
\label{fig:ex1}
\end{figure*}


However, Clojure goes beyond Racket in many ways, requiring several
new type system features which we detail in this paper.
%
Most significantly, Clojure supports, and Clojure developers use,
\textbf{multimethods} to structure their code in extensible
fashion. Furthermore, since Clojure is an untyped language, dispatch
within multimethods is determined by application of dynamic predicates
to argument values. 
%
Fortunately, the dynamic dispatch used by multimethods has surprising
symmetry with the conditional dispatch handled by occurrence
typing. Typed Clojure is therefore able to effectively handle complex
and highly dynamic dispatch as present in existing Clojure programs. 

But multimethods are not the only Clojure feature crucial to type
checking existing programs. As a language built on the Java Virtual
Machine, Clojure provides flexible and transparent access to existing
Java libraries, and \textbf{Clojure/Java interoperation} is found in almost
every significant Clojure code base. Typed Clojure therefore builds in
an understanding of the Java type system and handles interoperation
appropriately. Notably, \texttt{null} is a distinct type in Typed Clojure,
designed to automatically rule out null-pointer exceptions.

An example of these features is given in
\figref{fig:ex1}. Here, the \clj{pname} multimethod dispatches
on the \clj{class} of the argument---for \clj{String}s,
the first method implementation is called, for \clj{File}s, the
second. The \clj{String} method calls
a \clj{File} constructor, returning a non-nil \clj{File} instance---the 
\clj{getName} method 
on \clj{File} requires a non-nil target, returning a nilable
type.  
%Typed Clojure offers an opt-in mode that
%resolves JVM overloading, avoiding expensive runtime reflective calls.

Finally, flexible, high-performance immutable dictionaries
are the most common Clojure data structure.
Simply treating them as uniformly-typed
key-value mappings would be insufficient for existing
programs and programming styles. Instead, Typed Clojure provides a
flexible \textbf{heterogenous map} type, in which specific entries can be specified. 

While these features may seem disparate, they are unified in important
ways. First, they leverage the type system mechanisms
inherited from Typed Racket---multimethods when using 
dispatch via predicates, Java interoperation for handling
\texttt{null} tests, and heterogenous maps using union types and
reasoning about subcomponents of data. Second,
they are crucial features for handling Clojure code in
practice. Typed Clojure's use in real Clojure deployments would not be
possible without effective handling of these three Clojure features. 

%\subsection{Contributions}

Our main contributions are as follows:

\begin{enumerate}
  \item We motivate and describe  Typed Clojure, an optional
    type system for Clojure that understands existing Clojure idioms.
  \item We present a sound formal model for three crucial type
    system features: multi-methods, Java
    interoperability, and heterogenous maps.
  \item We evaluate the use of Typed Clojure features on existing
    Typed Clojure code, including both open source and in-house systems.
\end{enumerate}



%% \begin{figure}
%% \inputminted[firstline=5]{clojure}{code/demo/src/demo/parent2.clj}
%% \caption{A simple Typed Clojure program}
%% \label{fig:ex1}
%% \end{figure}

%% Figure~\ref{fig:ex1} presents a simple program demonstrating many
%% aspects of our system, from simple type annotations to explicit
%% handling of Java's \java{null} (written \clj{nil}) in interoperation, as well as an
%% extended form of occurrence typing and method resolution of
%% Java interoperability based on static type information.

%% The \clj{parent} function has the type 
%% $$
%% \clj{['{:file (U nil File)} -> (U nil Str)]}
%% $$
%% which means that it takes a hash table whose \clj{:file} key maps to either
%% \clj{nil} or a \clj{File}, and it produces either \clj{nil} or a
%% \clj{String}. The \clj{parent} function uses the \clj{:file} keyword
%% as an accessor to get the file, checks that it isn't \clj{nil}, and
%% then obtains the parent by making a Java method call.

\noindent
 The remainder of this paper begins with an example-driven
 presentation of the main type system features in
 \secref{sec:overview}. We then incrementally present a core calculus
 for Typed Clojure covering all of these features together in
 \secref{sec:formal} and prove type soundness
 (\secref{sec:metatheory}). We then 
 %discuss the full implementation of
 %Typed Clojure, \coretyped{}, which extends the formal model in many ways, 
 present an empirical analysis of significant code bases written
 in \coretyped{}---the full implementation of Typed Clojure---in \secref{sec:experience}. 
 Finally, we discuss related work and conclude.


\subsection{Automatic Type Annotations}

% - set the scene for inferring types
%   - Typed Clojure
%   - optional/gradual typing requires annotations

We now shift gears from introducing Typed Clojure to
addressing a major usability flaw that many
gradually and optionally typed languages share (including Typed Clojure):
writing type annotations is a \emph{manual} process.
%
Take \texttt{vertices} (below, written in Clojure)
a function that returnes the number of vertices
in a tree of tagged hash-maps.
As is good style, it comes with a unit test.
Our goal is to \textit{automatically generate} semi-accurate Typed Clojure %~\cite{bonnaire2016practical}
annotations
for this function, relieving most of the annotation
burden.

\begin{Verbatim}
(defn vertices [m]
  (case (:op m)
    :leaf 1
    :node (+ 1 (:left m) (:right m))))

(deftest test-vertices
  (is (= 3 {:op :node
            :left {:op :leaf :val 42} 
            :right {:op :leaf :val 24}})))
\end{Verbatim}

Our approach to automatic annotations features several stages.
First, we \textit{instrument} top-level functions.
Then, we \textit{exercise} the code by running its unit tests and \textit{observe}
the runtime behavior of the program.
If we pause at this point, we have collected enough data 
to generate a preliminary annotation:
%
\begin{Verbatim}
(ann vertices ['{:op ':node, :left '{:op ':leaf, :val Int}, :right '{:op ':leaf, :val Int}} -> Int]})
\end{Verbatim}
%
However, this type is too specific: trees are recursively defined
and the argument type is difficult to read and maintain.
To remedy this, we attempt to roll recursive-looking
types to be recursive from their example unrollings.
For example, below we have generalized the preliminary annotation's
depth 2 tree to the recursive \texttt{NodeLeaf}.
%
\begin{Verbatim}
(defalias NodeLeaf 
  (U '{:op ':node :left NodeLeaf :right NodeLeaf}
     '{:op ':leaf :val Int}))
(ann vertices [NodeLeaf -> Int])
\end{Verbatim}
%
%\begin{Verbatim}
%(declare Node Leaf)
%(defalias NodeLeaf (U Node Leaf))
%(defalias Node 
%  '{:op ':node :left NodeLeaf :right NodeLeaf})
%(defalias Leaf '{:op ':leaf :val int})
%(ann verbatim [NodeLeaf -> Int]})
%\end{Verbatim}
%
Now, if \texttt{NodeLeaf} is used in multiple positions
in the program, we don't want to repeat its definition multiple times.
Our type inference algorithm attempts merge recursive
types found throughout the program, reusing them in annotations.
For example,
if another function \texttt{sum-tree} accepts two
trees, we want reuse \texttt{NodeLeaf} in both annotations
like so:
%
\begin{Verbatim}
(ann vertices [NodeLeaf -> Int])
(ann sum-tree [NodeLeaf NodeLeaf -> NodeLeaf])
\end{Verbatim}

If minor variants of the recursive types occur
across a program,
we use optional \texttt{HMap} entries %~\cite{bonnaire2016practical}
to reduce redundancy.
%
\begin{Verbatim}
(defalias NodeLeaf 
  (U '{:op ':node :left NodeLeaf :right NodeLeaf}
     (HMap :mandatory {:op ':leaf :val Int}
           :optional {:label Str})))
\end{Verbatim}
%
After inserting these annotations, we can run the
type checker over them to check their usefulness.
Ideally, minimal changes will be needed to successfully type check
functions with the generated annotations,
mostly consisting of local function and loop annotations,
and renaming of type aliases.
Annotations should also be readable and minimize
redundancy, even when compared to hand-written annotations.
We will test this hypothesis with case studies
(\secref{sec:casestudy}).
%Generating and running \textit{tests} improved the quality
%of type annotations by exercising more paths through the
%program. %(Section \ref{experiment3})



%%\subsection{Overview of clojure.spec}
%
%This thesis claims that we can automatically generate annotations to other verification
%systems similar to Typed Clojure.
%We add a second usecase to strengthen our claim: generating annotations for \texttt{clojure.spec},
%a runtime verification library recently added to Clojure's core library.
%It resembles common approaches to runtime verification, such as Racket's contract
%system, but is different in several important ways.
%
%Firstly, \texttt{clojure.spec} is designed to treat most values as ``data at rest''. That is,
%at verification sites, values are eagerly traversed without waiting to see
%if or how the program actually uses them.
%When we consider that \texttt{clojure.spec} treats infinite streams
%and functions as data at rest, we begin to see the tradeoffs that have been
%made.
%
%Secondly, specifications (called ``specs'') are not enforced by default. Users must
%opt-in to enforcing specs via an explicit instrumentation phase.
%This is also different than most contract systems, many of which are enforced
%by default. There is no standard way to integrate spec enforcement into a
%test suite, so it is difficult to tell whether specific specs are primarily 
%unchecked documentation, or actually used for runtime verification.
%
%Since \texttt{clojure.spec} has a unique feature set amongst runtime verification
%libraries, it is interesting to consider how programmers use \texttt{clojure.spec}
%in practice. For example, do programmers find the semantics of treating functions
%as data at rest useful?
%
%%Unfortunately since specs are opt-in, it is difficult to
%%correlate someone writing a spec with that person \emph{using} the spec,
%%implying spec's semantics as being useful.
%%Nevertheless, in the following sections we attempt to draw conclusions about
%%spec's common usage based mostly on the frequency of spec annotations.
%
%\subsection{Function specifications in clojure.spec}
%
%We now give a brief introduction to what using clojure.spec is like,
%focussing on the semantics of the different kinds of function specifications
%it supports.
%
%(From here, we map the namespace prefix \texttt{s} to \texttt{clojure.spec.alpha},
%and \texttt{stest} to \texttt{clojure.spec.test.alpha}.)
%
%\begin{verbatim}
%(require '[clojure.spec.alpha :as s])
%(require '[clojure.spec.test.alpha :as stest])
%\end{verbatim}
%
%There are two kinds of function checking semantics in \texttt{clojure.spec}.
%We use \texttt{intmap}, a higher-order function that maps a function over 
%a collection of ints, to demonstrate both semantics.
%
%\begin{verbatim}
%(defn intmap
%  "Maps a collection of ints over a function."
%  [f c]
%  (map f c))
%\end{verbatim}
%
%If the programmer wants to write a higher-order function spec to
%verify \texttt{intmap}, they might write the following spec.
%
%\begin{verbatim}
%(s/fdef intmap
%  :args (s/cat :f (s/fspec :args (cat :x int?) :ret int?)
%               :c (s/coll-of int?))
%  :ret (s/coll-of int?))
%\end{verbatim}
%
%The \texttt{s/fdef} form signals we are annotating a top-level
%function, in this case \texttt{intmap}. Argument specs are
%provided with the \texttt{:args} keyword option
%in the form of the ``tagged'' heterogeneous collection spec
%\texttt{s/cat}---here 2 arguments are allowed, tagged as
%\texttt{:f} for the function and \texttt{:c} as the collection.
%
%The \texttt{s/fspec} spec is another kind of function spec,
%specifically for non-top-level functions (such as function arguments
%to top-level functions). It has a similar syntax to \texttt{s/fdef},
%but a function name is not provided.
%
%In a nutshell, \texttt{s/fdef} provides traditional proxy-based
%verification semantics while \texttt{s/fspec} uses eager \emph{generative testing}
%to exercise a function before letting it pass the spec boundary, bare (without a proxy).
%
%We will now demonstrate how the following call gets checked.
%
%\begin{verbatim}
%(intmap inc [1 2 3])
%;=> (2 3 4)
%\end{verbatim}
%
%First, the programmer instruments \texttt{intmap} with:
%
%\begin{verbatim}
%(stest/instrument `intmap)
%\end{verbatim}
%
%This mutates the top-level binding associated with \texttt{intmap}, wrapping a function
%proxy around the original value.
%
%Now, when checking \texttt{(intmap inc [1 2 3])}, the \texttt{inc} function is
%called several hundred times with generated values conforming to \texttt{int?},
%and checks each call returns an \texttt{int?}.
%Then, \texttt{[1 2 3]} is eagerly checked against \texttt{(s/coll-of int?)}.
%The original \texttt{intmap} function is then called with the original arguments,
%yielding a value \texttt{(2 3 4)}. Instrumentation does not check return value specs,
%so \texttt{(s/coll-of int?)} is ignored, and the original return value is passed to the calling
%context.
%
%\subsection{Automatic Annotations for clojure.spec}
%
%Having introduced \texttt{clojure.spec}, we now give an overview
%of how to repurpose our automatic annotation technology to generate specs.
%At a glance, the problems of generating annotations for Typed Clojure
%and \texttt{clojure.spec} are similar, and indeed we can reuse much
%of the machinery from our approach to generating Typed Clojure annotations.
%The differences lie mostly in the finally type reconstruction phase.
%
%There are several important components in \texttt{clojure.spec}'s
%philosophy and design that complicate its annotation story.
%First, \texttt{clojure.spec} does not support local key-type pairs
%in its heterogeneous map specification--instead you are forced
%to globally define map entries as \emph{spec aliases}.
%
%For example, the Typed Clojure type \clj{'\{:a Int\}} must be
%written \clj{(s/keys :req-un [:my-ns/a])}, along with the following
%global definition of the spec alias \clj{:my-ns/a}.
%\begin{verbatim}
%(s/def :my-ns/a int?)
%\end{verbatim}
%When \clj{(s/keys :req-un [:my-ns/a])} is checked against a value,
%it first locates the spec under \clj{:my-ns/a}, and (because
%we have declared the key to be \clj{:req-un}, that is, required
%but unqualified), we use the located \clj{int?} spec
%to check against the \clj{:a} entry of the value.
%Furthermore, if the key was declared as \clj{:req} instead of \clj{:req-un},
%the fully qualified keyword entry \clj{:my-ns/a} would be checked
%against \clj{int?}.
%This means that there is \emph{exactly one} spec for each fully
%qualified keyword entry.
%
%This raises several interesting design issues related to reusing
%specs that are very different from the simpler problem of generating
%Typed Clojure \clj{HMap} types. 
%How do we decide which namespace to register unqualified map entries?
%Should we assume similar looking keyword entries are in fact the same, and register their specs under the same namespace?
%How do we handle fully qualified keyword entries?
%
%The second component of


\chapter*{Typed Clojure in Theory and Practice}

Typed Clojure is an optional type system for the Clojure programming language
hat aims to provide a type system to check idiomatic Clojure code.

\section*{Thesis Statement}

My thesis statement is:

\begin{quote}
Typed Clojure is a sound and practical optional type system for Clojure.
%Typed Clojure is a sound and practical optional type system for Clojure and the process of porting to Typed Clojure can be partially-automated, and this automation technology can be repurposed to further reveal how Clojure is used in real projects.
%Typed Clojure is a sound and practical optional type system for Clojure, and we can relieve the annotation burden for such verification systems by automatically annotating programs based on their runtime behavior.
%Typed Clojure is a sound and practical optional type system for Clojure whose annotation burden is partially-automatable, and this automation technology can be repurposed to generate clojure.spec annotations and test its effectiveness in hundreds of projects.
% to study how Clojure is used in real projects.
%Typed Clojure is sound, practical, and its annotation burden is partially-automatable,
%and we can repurpose this annotation technology to
%answer broad questions about how Clojure is used.
%Typed Clojure is a well-founded and practical optional type system for Clojure
%whose useability can be improved by
%developing a tool to automatically generate type annotations,
%and, by repurposing this tool, we can study general Clojure idioms and practices across
%hundreds of projects by generating, running, and exercising clojure.spec runtime specifications.
\end{quote}

%I will support this thesis statement with the following:
%

\section*{Structure of this thesis}

This document progresses in several parts that support my thesis statement, presented in chronological order
in when they were developed.

Part~\ref{part:types} motivates and presents the core type system of Typed Clojure.

\begin{itemize}
  \item \emph{Typed Clojure is sound} We formalize Typed Clojure, including
    its characteristic features like hash-maps, multimethods, and Java interoperability,
    and prove the model type sound.
  \item \emph{Typed Clojure is practical} 
    \begin{itemize}
      \item We present an empirical study of real-world Typed Clojure usage
        in over 19,000 lines of code, showing its features correspond to actual usage patterns.
    \end{itemize}
    % our annotation approach can be repurposed to generate clojure.spec which can be \emph{run}.
    % use results to explore both the quality of the tool and to explore what the inferred specs
    % tell us about clj programs.
\end{itemize}

As a response to the results of this work, I pursued several research directions.

Part~\ref{part:autoann} presents a solution to lower the annotation burden in real-world Typed Clojure programs.
We formalize and implement a tool to automatically annotate types for top-level
user and library definitions, and empirically study the manual changes needed for the generated annotations
to pass type checking.

Part~\ref{part:spec} investigates clojure.spec, Clojure's official runtime verification library.
Released 5 years after Typed Clojure, clojure.spec represents a significant development in Clojure
verification. I investigate clojure.spec's design and compare it to Typed Clojure
by producing a formal model of clojure.spec and repurposing the automatic annotation tool 
from Part~\ref{part:autoann} to automatically produce clojure.spec annotations for real-world Clojure programs.
%
%\begin{enumerate}
%  \item The process of porting to Typed Clojure can be partially-automated, and this automation technology can be repurposed to further reveal how Clojure is used in real projects.
%    \begin{itemize}
%      \item \emph{Repurpose automation technology}
%        We describe how to automatically generate clojure.spec annotations (``specs'') for existing programs by reusing
%        most of the the infrastructure for automatic Typed Clojure annotations.
%        We present a formal model of clojure.spec (an existing and popular runtime verification tool for Clojure)
%        and implement the model in Redex.
%      \item \emph{Study how Clojure is used in real projects}
%        We conduct a study of general Clojure idioms and practices by generating, enforcing, and exercising specs
%        across hundreds of projects, as well as analyzing design choices in Typed Clojure's type system,
%        clojure.spec's features, and our automatic annotation tool.
%      \item \emph{Test effectiveness of clojure.spec annotation generation}
%        We test the effectiveness of our generated specs by generating, enforcing, and exercising specs
%        across hundreds of projects, as well as analyze design choices in Typed Clojure's type system and
%        clojure.spec's features.
%    \end{itemize}
%\end{enumerate}

% Evan Chang (Boulder)
% - symbolic exec + type checking
% David Fisher (Olin Shivers)
% - Lazy Delegation JSP paper
% Matthew (STL talk)
% - let's write a expander
% - scope sets
% add this as an alternative to clojure spec's research direction
% - sell as: Programs hard to TC
% use (comp (map inc) (map dec)) as an example

\section*{Previously published work}

Part~\ref{part:types} has been published:
%
\begin{itemize}
  \item Ambrose Bonnaire-Sergeant, Rowan Davies, Sam Tobin-Hochstadt.
        Practical Optional Types for Clojure.
        In \textit{European Symposium on Programming Languages and Systems}, 2016
\end{itemize}


\section{Technical Overview}

\section{Overview of Typed Clojure}

\label{sec:overview}

We now begin a tour of the central features of Typed Clojure,
beginning with Clojure itself. Our presentation
uses the full Typed Clojure system to illustrate key type system
ideas,\footnote{Full examples: \url{https://github.com/typedclojure/esop16}} before studying the core features in detail in
\secref{sec:formal}.

\subsection{Clojure}

Clojure~\cite{Hic08} is a Lisp that runs on the
Java Virtual Machine with support for concurrent programming
and immutable data structures in a mostly-functional
style.
%, restricting imperative updates to a limited set of
%structures each with specific thread synchronization behaviour. 
%Fast implementations of immutable vectors, and hash tables are featured,
%and a means for defining new records.
%
Clojure provides easy interoperation with existing Java libraries, with Java values being like any other Clojure value. 
However, this smooth interoperability comes at the cost of pervasive \java{null}, which leads to the possibility of null pointer exceptions---a drawback we address in Typed Clojure.

%\paragraph{Clojure Syntax}
%
%We describe new syntax as they appear in each example, but
%begin with include the essential basics of Clojure syntax.
%
%\clj{nil} is exactly Java's \java{null}.
%Parentheses indicate \emph{applications}, brackets
%delimit
%\emph{vectors}, braces
%delimit
%\emph{hash-maps}
%and double quotes delimit \emph{Java strings}.
%\emph{Symbols} begin with an alphabetic character,
%and a colon prefixed symbol like \clj{:a} is a \emph{keyword}.
%
%\emph{Commas} are always \emph{whitespace}.

\subsection{Typed Clojure}

A simple one-argument function \clj{greet} is annotated with \clj{ann} to take and return strings.

\begin{lstlisting}
(*typed ann  greet [Str -> Str] typed*)
(defn greet [n] (*invoke str "Hello, " n "!" invoke*))
(*invoke greet "Grace" invoke*) ;=> "Hello, Grace!" :- Str
\end{lstlisting}
%
Providing \clj{nil} (exactly Java's \java{null})
is a static type error---\clj{nil} is not a string.
%
\begin{lstlisting}
(*invoke greet nil invoke*) ; Type Error: Expected Str, given nil
\end{lstlisting}

\paragraph{Unions} To allow \clj{nil}, we use \emph{ad-hoc unions} (\clj{nil} and \clj{false}
are logically false).
%
\begin{lstlisting}
(*typed ann  greet-nil [(U nil Str) -> Str] typed*)
(defn greet-nil [n] (*invoke str "Hello" (when n (*invoke str ", " n invoke*)) "!" invoke*))
(*invoke greet-nil "Donald" invoke*) ;=> "Hello, Donald!" :- Str 
(*invoke greet-nil nil invoke*)      ;=> "Hello!"         :- Str
\end{lstlisting}
%
%
Typed Clojure prevents well-typed code from dereferencing \clj{nil}.
%This is important for Clojure programs---\clj{nil}
%is treated like any other distinct datum in Clojure.

\paragraph{Flow analysis} Occurrence typing~\cite{TF10}
models type-based control flow.
In \clj{greetings}, a branch ensures \clj{repeat}
is never passed \clj{nil}.
%
\begin{lstlisting}
(*typed ann  greetings [Str (U nil Int) -> Str] typed*)
(defn greetings [n i]
  (*invoke str "Hello, " (when i (*invoke apply str (*invoke repeat i "hello, " invoke*) invoke*)) n "!" invoke*))
(*invoke greetings "Donald" 2 invoke*)  ;=> "Hello, hello, hello, Donald!" :- Str
(*invoke greetings "Grace" nil invoke*) ;=> "Hello, Grace!"                :- Str
\end{lstlisting}
%
Removing the branch is a static type error---\clj{repeat} cannot be passed \clj{nil}.
%
\begin{lstlisting}
(*typed ann  greetings-bad [Str (U nil Int) -> Str] typed*)
(defn greetings-bad [n i]           ; Expected Int, given (U nil Int)
  (*invoke str "Hello, " (*invoke apply str (*invoke repeat i "hello, " invoke*) invoke*) n "!" invoke*))
\end{lstlisting}


%\subsection{Type System Basics}
%
%\cite{TF10}
%presented Typed Racket with occurrence typing,
%a technique for deriving type information from conditional control flow.
%They introduced the concept of occurrence typing 
%with the following example.
%
%\inputminted[firstline=3]{racket}{code/tr/example1.rkt}
%
%This function takes a value that is either \emph{\#f} % mintinline really hates #
%or a number, represented by an \emph{untagged} union type.
%The `then' branch has an implicit invariant
%that \rkt{x} is a number, which is automatically inferred with occurrence typing
%and type checked without further annotations.
%
%We chose to build on the ideas and implementation
%of Typed Racket to implement a type system targeting Clojure for several reasons.
%Initially, the similarities between Racket and Clojure drew us to
%investigate the effectiveness of repurposing occurrence typing
%for a Clojure type system---both languages share a Lisp heritage,
%similar standard functions 
%(for instance \clj{map}
%in both languages is variable-arity)
%and idioms.
%While Typed Racket is gradually typed and has sophisticated
%dynamic semantics for cross-language interaction, we 
%chose to first implement
%the static semantics
%with the hope to extend Typed Clojure to be gradually typed at a future date.
%Finally,
%Typed Racket's combination of bidirectional checking
%and occurrence typing presents a successful model for 
%type checking dynamically typed programs without compromising
%soundness, which is appealing over success typing~\cite{Lindahl:2006:PTI}
%which cannot prove strong properties about programs
%and soft typing~\cite{CF91}
%which has proved too complicated in practice.
%
%Here is the above program in Typed Clojure.
%\begin{exmp}
%\inputminted[firstline=5]{clojure}{code/demo/src/demo/eg1.clj}
%\label{example:conditionalflow}
%\end{exmp}
%
%The \clj{fn} macro (provided by core.typed) supports optional annotations by 
%adding
%\clj{:-} and a type after a parameter
%position
%or binding vector 
%to annotate parameter types
%and return types respectively.
%\clj{number?} is
%a Java \java{instanceof} test of \clj{java.lang.Number}.
%As in Typed Racket, \clj{U} creates an \emph{untagged union} type, which can take
%any number of types.
%
%Typed Clojure can already check all of the examples in~\cite{TF10}---the 
%rest of this section describes the extensions necessary
%to check Clojure code.


\subsection{Java interoperability}
\label{sec:overviewjavainterop}

Clojure can interact with Java constructors, methods, and fields.
This program calls the \clj{getParent} on a constructed
\clj{File}
instance, returning a nullable string.

\begin{exmp}
\begin{lstlisting}
(*interop .getParent (*interop new File "a/b" interop*) interop*)  ;=> "a" :- (U nil Str)
\end{lstlisting}
\label{example:getparent-direct-constructor}
\end{exmp}
%
Typed Clojure can integrate with the Clojure compiler to avoid expensive reflective 
calls like \clj{getParent}, however if a specific overload cannot be found based on the
surrounding static context, a type error is thrown.
%
\begin{lstlisting}
(fn [f] (*interop .getParent f interop*)) ; Type Error: Unresolved interop: getParent
\end{lstlisting}
%
Function arguments default to \clj{Any}, which is similar to a union of all types. Ascribing
a parameter type allows Typed Clojure to find a specific method.

%Calls to Java methods and fields have prefix notation
%like \clj{(.method target args*)} and \clj{(.field target)} respectively,
%with method and field names prefixed with a dot and methods taking some number of arguments.

\begin{exmp}
\begin{lstlisting}
(*typed ann parent [(U nil File) -> (U nil Str)] typed*)
(defn parent [f] (if f (*interop .getParent f interop*) nil))
\end{lstlisting}
\label{example:parent-if}
\end{exmp}

%\begin{exmp}
%\inputminted[firstline=18,lastline=19]{clojure}{code/demo/src/demo/parent3.clj}
%\end{exmp}

The conditional guards from dereferencing \clj{nil}, and---as before---removing 
it is a static type error, as typed code could possibly dereference \clj{nil}.
\begin{lstlisting}
(defn parent-bad-in [f :- (U nil File)]
  (*interop .getParent f interop*)) ; Type Error: Cannot call instance method on nil.
\end{lstlisting}

Typed Clojure rejects programs that assume methods cannot return \clj{nil}.
%
\begin{lstlisting}
(defn parent-bad-out [f :- File] :- Str
  (*interop .getParent f interop*)) ; Type Error: Expected Str, given (U nil Str).
\end{lstlisting}
Method targets can never be \clj{nil}.
Typed Clojure also prevents passing \clj{nil} as Java method or
constructor arguments by default---this restriction can be
adjusted per method.

%
%Typed Clojure and Java treat \java{null} differently.
%In Clojure, where it is known as \clj{nil}, Typed Clojure assigns it an explicit type
%called \clj{nil}. In Java \java{null} is implicitly a member of any reference type.
%This means the Java static type \java{String} is equivalent to
%\clj{(U nil String)} in Typed Clojure.
%
%Reference types in Java are nullable, so to guarantee a method call does not
%leak \java{null} into a Typed Clojure program we
%must assume methods can return \clj{nil}.
%
In contrast, JVM invariants guarantee constructors return non-null.\footnote{\url{http://docs.oracle.com/javase/specs/jls/se7/html/jls-15.html#jls-15.9.4}}
%
\begin{exmp}
\begin{lstlisting}
(*invoke parent (*interop new File s interop*) invoke*)
\end{lstlisting}
\end{exmp}


\subsection{Multimethods}

\label{sec:multioverview}

\emph{Multimethods} are a kind of extensible function---combining a \emph{dispatch function} with 
one or more \emph{methods}---widely used to define Clojure operations.

\paragraph{Value-based dispatch}
This simple multimethod takes a keyword (\clj{Kw}) and says hello in different languages.%, as
%specified by a keyword argument.

\begin{exmp}
\begin{lstlisting}
(*typed ann hi [Kw -> Str] typed*) ; multimethod type
(defmulti hi identity) ; dispatch function `identity`
(defmethod hi :en [_] "hello") ; method for `:en`
(defmethod hi :fr [_] "bonjour") ; method for `:fr`
(defmethod hi :default [_] "um...") ; default method
\end{lstlisting}
\label{example:hi-multimethod}
\end{exmp}

When invoked, the arguments are first supplied to the dispatch function---\clj{identity}---yielding
a \emph{dispatch value}. A method is then chosen
based on the dispatch value, to which the arguments are then passed to return a value.
%
\begin{lstlisting}
(*invoke map hi [*vec :en :fr :bocce vec*] invoke*) ;=> (*list "hello" "bonjour" "um..." list*)
\end{lstlisting}
%
For example, 
\clj{(*invoke hi :en invoke*)} evaluates to \clj{"hello"}---it executes
the \clj{:en} method
because \clj{(*invoke = (*invoke identity :en invoke*) :en invoke*)} is true
and \clj{(*invoke = (*invoke identity :en invoke*) :fr invoke*)} is false.

Dispatching based on literal values enables certain forms of method
definition, but this is only part of the story for multimethod dispatch.

\paragraph{Class-based dispatch}
For class values, multimethods can choose methods based on subclassing
relationships.
%
Recall the multimethod from \figref{fig:ex1}. %, reproduced here.
%\begin{minted}{clj}
%(ann pname [(U File String) -> (U nil String)])
%(defmulti pname class)
%(defmethod pname String [s] (pname (new File s)))
%(defmethod pname File [f] (.getName f))
%\end{minted}
%
%Its dispatching function is
%\clj{class}, with two methods associated with dispatch values \clj{java.lang.String} and \clj{java.io.File}
%respectively.
%\noindent
The dispatch function \clj{class}
%---associated at multimethod creation with \clj{defmulti}---
dictates 
whether the \clj{String} or \clj{File} method is chosen.
%---both installed via \clj{defmethod}
%
The multimethod dispatch rules use
\clj{isa?}, a hybrid predicate which is both a subclassing check for classes and
an equality check for other values.

%(isa? (class "STAINS/JELLY") Object) ;=> true
%(isa? (identity :en) :fr) ;=> false
%(isa? (class (new File "JELLY")) String) ;=> false
\begin{lstlisting}
(*invoke isa? :en :en invoke*)       ;=> true
(*invoke isa? String Object invoke*) ;=> true
\end{lstlisting}
%
The current dispatch value and---in turn---each method's associated dispatch value
is supplied to \clj{isa?}. If exactly one method returns true, it is chosen.
%
For example,
the call
  \clj{(*invoke pname "STAINS/JELLY" invoke*)}
picks the \clj{String} method because \clj{(*invoke isa? String String invoke*)}
is true, and
\clj{(*invoke isa? String File invoke*)}
is not.
%---\clj{(class "STAINS/JELLY")}
%is \clj{String}.
%
%The \clj{String} method body
%\clj{(pname (new File "STAINS/JELLY"))}
%chooses the \clj{File} method for opposite reasons.

%The following Typed Clojure program is semantically identical to figure~\ref{fig:ex1}.
%
%\begin{minted}{clj}
%(ann pname [(U Str File) -> (U nil Str)])
%(defn pname [x]
%  ; dispatch value calculated by applying dispatch
%  ; function `class` to argument `x`.
%  (cond
%    ; if (class x) subclasses String, but not File
%    (and (isa? (class x) String)
%         (not (isa? (class x) File)))
%    ; then choose the String method
%    (pname (new File x))
%
%    ; else if (class x) subclasses File, but not String
%    (and (isa? (class x) File)         
%         (not (isa? (class x) String)))
%    ; then choose the File method
%    (.getName x)
%    :else (throw (Exception. "No match"))))
%\end{minted}
%
%An unambiguous match leads to the corresponding method being applied to the arguments,
%giving the final result.

%\subsection{Multimethods}
%
%A multimethod in Clojure is a function with a \emph{dispatch
%function} and a \emph{dispatch table} of methods. Multimethods are created with {\clj{defmulti}}.
%\inputminted[firstline=5,lastline=6]{clojure}{code/demo/src/demo/rep.clj}
%The multimethod \clj{path} has type \clj{[Any -> (U nil String)]}, an initially empty \emph{dispatch table}
%and \emph{dispatch function} \clj{class}, a function that
%returns the class of its argument or \clj{nil} if passed \clj{nil}.
%
%We can use {\clj{defmethod}} to install a method to \clj{path}.
%\inputminted[firstline=7,lastline=7]{clojure}{code/demo/src/demo/rep.clj}
%Now the dispatch table maps
%the \emph{dispatch value} \clj{String} to the function
%\clj{(fn [x] x)}. 
%We add another method
%which maps
%\clj{File} to the function
%\clj{(fn [x] (.getPath x))}
%in the dispatch table.
%\inputminted[firstline=8,lastline=8]{clojure}{code/demo/src/demo/rep.clj}
%
%After installing both methods, the call 
%$$
%\clj{(path (new File "dir/a"))}
%$$
%dispatches to the second method we installed because
%$$
%\clj{(isa? (class "dir/a") String)}
%$$
%is true, and finally returns 
%$$
%\clj{((fn [x] (.getPath x)) "dir/a")}.
%$$

%We include the above sequence of definitions as \egref{example:rep}.
%
%\begin{Code}
%\begin{exmp}
%\inputminted[firstline=5,lastline=10]{clojure}{code/demo/src/demo/rep.clj}
%\label{example:rep}
%\end{exmp}
%\end{Code}
%
%Typed Clojure does not predict if a runtime dispatch will be successful---\clj{(path :a)} 
%type checks because \clj{:a} agrees with the parameter type \clj{Any},
%but throws an error at runtime.

%\paragraph{Multiple dispatch} \clj{isa?} is special with vectors---vectors of the
%same length recursively call \clj{isa?} on the elements pairwise.
%\begin{minted}{clojure}
%  (isa? [Keyword Keyword] [Object Object]) ;=> true
%\end{minted}
%
%\inputminted[firstline=6,lastline=23]{clojure}{code/demo/src/demo/eg7.clj}
%
%\egref{example:multidispatch}
%simulates multiple dispatch by dispatching on
%a vector containing the class of both arguments. \clj{open}
%takes two arguments which can be strings or files and returns
%a new file that concatenates their paths.
%
%We call three different \clj{File} constructors, each known at compile-time
%via type hints.
%Multiple dispatch follows the same kind of reasoning as \egref{example:incmap},
%except we update multiple bindings simultaneously.

\subsection{Heterogeneous hash-maps}

%Beyond primitives and Java objects, 
The most common way to represent compound data in Clojure 
are immutable hash-maps, typicially with keyword keys.
%
Keywords double as functions that
look themselves up in a map, or return \clj{nil} if absent.
%
\begin{exmp}
\begin{lstlisting}
(def breakfast {*map :en "waffles" :fr "croissants" map*})
(*invoke :en breakfast invoke*)    ;=> "waffles" :- Str
(*invoke :bocce breakfast invoke*) ;=> nil       :- nil
\end{lstlisting}
\label{example:breakfastcomplete}
\end{exmp}

\emph{HMap types} describe the most common usages of
keyword-keyed maps.
\begin{lstlisting}
breakfast ; :- (HMap :mandatory {:en Str, :fr Str}, :complete? true)
\end{lstlisting}
This says
\clj{:en} and \clj{:fr} are known entries mapped to strings,
and the map is fully specified---that is, no other entries exist---by \clj{:complete?} being \clj{true}.

HMap types default to partial specification, with
\clj{'\{:en Str :fr Str\}} abbreviating \clj{(HMap :mandatory \{:en Str, :fr Str\})}.
%
\begin{exmp}
\begin{lstlisting}
(*typed ann lunch '{:en Str :fr Str} typed*)
(def lunch {*map :en "muffin" :fr "baguette" map*})
(*invoke :bocce lunch invoke*) ;=> nil :- Any ; less accurate type
\end{lstlisting}
\label{example:lunchpartial}
\end{exmp}
%(:en lunch)    ; :- Str
%;=> "muffin"
%(:fr lunch)    ; :- Str
%;=> "baguette"

\paragraph{HMaps in practice} The next example is extracted from a production system at CircleCI,
a company with a large production Typed Clojure system
(\secref{sec:casestudy} presents a case study and empirical
result from this code base).

\newpage

\begin{exmp}
\begin{lstlisting}
(*typed defalias RawKeyPair ; extra keys disallowed
  (HMap :mandatory {:pub RawKey, :priv RawKey}, 
        :complete? true) typed*)
(*typed defalias EncKeyPair ; extra keys disallowed
  (HMap :mandatory {:pub RawKey, :enc-priv EncKey}, :complete? true) typed*)

(*typed ann enc-keypair [RawKeyPair -> EncKeyPair] typed*)
(defn enc-keypair [kp]
  (*invoke assoc (*invoke dissoc kp :priv invoke*) :enc-priv (*invoke encrypt (*invoke :priv kp invoke*) invoke*) invoke*))
\end{lstlisting}
\label{example:circleci}
\end{exmp}
%
%(ann enc-keypair [RawKeyPair -> EncKeyPair])
%(defn enc-keypair [{pk :priv :as kp}] ; original map is kp
%  (assoc (dissoc kp :priv)       ; remove unencrypted private key
%         :enc-priv (encrypt pk))) ; add encrypted private key
%
%\inputminted[firstline=10,lastline=22]{clojure}{code/demo/src/demo/key.clj}
As \clj{EncKeyPair} is fully specified, we remove extra keys like \clj{:priv}
via \clj{dissoc}, which returns a new map that is the first argument without the
entry named by the second argument. Notice removing \clj{dissoc} causes a type error.
%
\begin{lstlisting}
(defn enc-keypair-bad [kp] ; Type error: :priv disallowed
  (*invoke assoc kp :enc-priv (*invoke encrypt (*invoke :priv kp invoke*) invoke*) invoke*))
\end{lstlisting}

%\clj{enc-keypair} takes an unencrypted keypair and returns an encrypted keypair by
%dissociating the raw \clj{:priv} entry with \clj{dissoc}
%and associating an encrypted private key
%as \clj{:enc-priv} on an immutable map with \clj{assoc}.
%The expression \clj{(:priv kp)} shows that keywords are also 
%functions that look themselves up in a map returning the associated value or \nil{} if the key is missing.
%Since \clj{EncKeyPair} is \clj{:complete?}, Typed Clojure enforces the return type
%does not contain an entry \clj{:priv}, and would complain if the \clj{dissoc}
%operation forgot to remove it.

%\egref{example:absentkeys}
%is like \egref{example:circleci}
%except the \clj{:absent-keys} HMap option is used
%instead of \clj{:complete?},
%which takes a \emph{set literal} of keywords that do not appear in the map, written 
%with \emph{\#}-prefixed braces.
%The syntax \clj{(fn [{pkey :priv, :as kp}] ...)}
%aliases \clj{kp} to the first argument and \clj{pkey} to \clj{(:priv m)}
%in the function body.
%
%\begin{exmp}
%\inputminted[firstline=10,lastline=21]{clojure}{code/demo/src/demo/key2.clj}
%\label{example:absentkeys}
%\end{exmp}
%
%Since this example enforces that \clj{:priv} must not appear
%in a \clj{EncKeyPair}
%Typed Clojure would still complain if we forgot to \clj{dissoc} \clj{:priv}
%from the return value.
%Now, however we could stash the raw private key in another entry
%like \clj{:secret-key} which is not mentioned by the partial HMap \clj{EncKeyPair}
%without Typed Clojure noticing.

%\paragraph{Branching on HMaps} Finally, testing on HMap properties
%allows us to refine its type down branches. \clj{dec-map} takes an
%\clj{Expr}, traverses to its nodes and decrements their values by \clj{dec}, then
%builds the \clj{Expr} back up with the decremented nodes.
%
%\begin{exmp}
%\inputminted[linenos,firstnumber=1,firstline=15,lastline=27]{clojure}{code/demo/src/demo/hmap.clj}
%\label{example:decmap}
%\end{exmp}
%
%If we go down the then branch (line 4), since \clj{(= (:op m) :if)} is true
%we remove
%the \clj{:do} and \clj{:const}
%Expr's from the type of \clj{m} (because their respective \clj{:op} entries disagrees with \clj{(= (:op m) :if)})
%and we are left with an \clj{:if} Expr.
%On line 8,
%we instead strike out the \clj{:if} Expr since it contradicts \clj{(= (:op m) :if)} being false. 
%Line 9 know we can
%remove the \clj{:const} Expr from the type of \clj{m} because it contradicts \clj{(= (:op m) :do)} being true,
%and we know \clj{m} is a \clj{:do} Expr.
%Line 12 we strike out \clj{:do} because \clj{(= (:op m) :do)} is false,
%so we are left with \clj{m} being a \clj{:const} Expr.
%
%Section~\ref{sec:formalpaths} discusses how this automatic reasoning is achieved.

\subsection{HMaps and multimethods, joined at the hip}

HMaps and multimethods are the primary ways for representing
and dispatching on data respectively, and so are intrinsically linked.
As type system designers, we must
search for a compositional approach that can anticipate
any combination of these features.

Thankfully, occurrence typing, originally designed for reasoning about
\clj{if} tests, provides the compositional approach we need.
By extending the system with
a handful of rules based on HMaps and other functions, 
we can automatically cover both easy cases and those
that compose rules in arbitrary ways.

Futhermore, this approach extends to multimethod dispatch by reusing
occurrence typing's approach to conditionals
%, whose branching
%mechanism may appear complex, but
%can be understood in terms of the humble \clj{if} conditional. 
and
encoding a small number of rules to handle
the \clj{isa?}-based dispatch.
In practice, conditional-based control flow typing
extends to multimethod dispatch, and vice-versa.

We first demonstrate a very common, simple dispatch style,
then move on to deeper structural dispatching where occurrence typing's
compositionality shines.

\paragraph{HMaps and unions} Partially specified HMap's with a common dispatch key
combine naturally with ad-hoc unions.
An \clj{Order} is one of three kinds of HMaps.

%FIXME define defalias above and keyword singletons

\begin{lstlisting}
(*typed defalias Order "A meal order, tracking dessert quantities."
  (U '{:Meal ':lunch, :desserts Int} '{:Meal ':dinner :desserts Int}
     '{:Meal ':combo :meal1 Order :meal2 Order}) typed*)
\end{lstlisting}

The \clj{:Meal} entry is common to each HMap, always mapped to a known keyword singleton
type.
It's natural to dispatch on the \clj{class} of an instance---it's similarly
natural to dispatch on a known entry like \clj{:Meal}.

\newpage

\begin{exmp}
\begin{lstlisting}
(*typed ann desserts [Order -> Int] typed*)
(defmulti desserts :Meal)  ; dispatch on :Meal entry
(defmethod desserts :lunch [o] (*invoke :desserts o invoke*))
(defmethod desserts :dinner [o] (*invoke :desserts o invoke*))
(defmethod desserts :combo [o] 
  (*invoke + (*invoke desserts (*invoke :meal1 o invoke*) invoke*) (*invoke desserts (*invoke :meal2 o invoke*) invoke*) invoke*))

(*invoke desserts {*map :Meal :combo, :meal1 {*map :Meal :lunch :desserts 1 map*}, 
           :meal2 {*map :Meal :dinner :desserts 2 map*} map*} invoke*) ;=> 3
\end{lstlisting}
\label{example:desserts-on-meal}
\end{exmp}
%
The \clj{:combo} method is verified to only structurally recur
on \clj{Order}s. This is achieved because we learn the argument \clj{o} must
be of type
\clj{'\{:Meal :combo\}}
since
\clj{(isa? (:Meal o) :combo)}
is true. Combining this
with the fact that \clj{o} is an \clj{Order}
eliminates possibility of \clj{:lunch} and \clj{:dinner}
orders, simplifying \clj{o} to
\clj{'\{:Meal ':combo :meal1 Order :meal2 Order\}}
which contains appropriate arguments for both recursive calls.

\paragraph{Nested dispatch}
A more exotic dispatch mechanism for \clj{desserts}
might be on the \clj{class} of the \clj{:desserts} key.
If the result is a number, then we know the \clj{:desserts}
key is a number, otherwise the input is a \clj{:combo} meal.
We have already seen dispatch on \clj{class} and on keywords
in isolation---occurrence typing automatically understands
control flow that combines its simple building blocks.

The first method has dispatch value \clj{Long}, a subtype
of \clj{Int}, and the second method has \clj{nil}, the sentinel value for a failed map lookup.
In practice, \clj{:lunch} and \clj{:dinner} meals will dispatch to the \clj{Long}
method, but Typed Clojure infers a slightly more general type due to the definition
of \clj{:combo} meals.

\begin{exmp}
\begin{lstlisting}
(*typed ann desserts' [Order -> Int] typed*)
(defmulti desserts' 
  (fn [o :- Order] (*invoke class (*invoke :desserts o invoke*) invoke*)))
(defmethod desserts' Long [o] 
;o :- (U '{:Meal (U ':dinner ':lunch), :desserts Int}
;       '{:Meal ':combo, :desserts Int, :meal1 Order, :meal2 Order})
  (*invoke :desserts o invoke*))
(defmethod desserts' nil [o]
  ; o :- '{:Meal ':combo, :meal1 Order, :meal2 Order}
  (*invoke + (*invoke desserts' (*invoke :meal1 o invoke*) invoke*) (*invoke desserts' (*invoke :meal2 o invoke*) invoke*) invoke*))
\end{lstlisting}
\label{example:desserts-on-class}
\end{exmp}
%
%(desserts' {:Meal :combo 
%            :meal1 {:Meal :lunch :desserts 1}
%            :meal2 {:Meal :dinner :desserts 2}})
%;=> 3

In the \clj{Long} method, Typed Clojure learns that
its argument is at least of type \clj{'\{:desserts Long\}}---since
\clj{(*invoke isa? (*invoke class (*invoke :desserts o invoke*) invoke*) Long invoke*)}
must be true.
%
%Knowing \clj{o} is also an
%\clj{Order},
Here the
\clj{:desserts} entry
\emph{must} be present and mapped to a \clj{Long}---even in a \clj{:combo} meal,
which does not specify \clj{:desserts}
as present or absent.

In the \clj{nil} method,
\clj{(*invoke isa? (*invoke class (*invoke :desserts o invoke*) invoke*) nil invoke*)}
must be true---which implies \clj{(*invoke class (*invoke :desserts o invoke*) invoke*)} is \clj{nil}.
%
Since lookups on missing keys return \clj{nil}, either
\begin{itemize}
  \item \clj{o} has a \clj{:desserts} entry to \clj{nil}, like \clj{{:desserts nil}}, or
  \item \clj{o} is missing a \clj{:desserts} entry.%, like \clj{{}}.
\end{itemize}
We can express this type with the \clj{:absent-keys} HMap option
%Equivalently, \clj{o} is of type
% Note: mintedinline doesn't work with hash characters #
\begin{lstlisting}
(U '{:desserts nil} (HMap :absent-keys #{:desserts}))
\end{lstlisting}
This eliminates non-\clj{:combo} meals
since their \clj{'\{:desserts Int\}} type does not agree
with this new information (because \clj{:desserts}
is neither \clj{nil} or absent).

%simplifies to a \clj{:combo} meal, 
%\begin{minted}{clojure}
%'{:Meal ':combo :meal1 Order :meal2 Order}
%\end{minted}
%thus allowing both recursive calls to type check.

\paragraph{From multiple to arbitrary dispatch}
Clojure multimethod dispatch, and Typed Clojure's handling of it, goes
even further, supporting dispatch on multiple arguments via vectors.
%
Dispatch on multiple arguments is beyond the scope of this paper,
but the same intuition applies---adding support for multiple dispatch
admits arbitrary combinations and nestings
of it and previous dispatch rules.

%\begin{exmp}
%\inputminted[firstline=6,lastline=13]{clojure}{code/demo/src/demo/hmap.clj}
%\label{example:decleaf}
%\end{exmp}
%
%The \clj{defn} macro defines a top-level function, with syntax like the typed \clj{fn}.
%The function \clj{an-exp} is verified to return an \clj{Expr}.
%
%Here \clj{defalias} defines \clj{Expr}, a type abbreviation
%that describes the structure of a recursively-defined AST as a union of HMaps.
%Keyword singleton types are quoted---\clj{':lunch}.
%A type that is a quoted map like \clj{'{:op ':if}} is a
%HMap type with a fixed number of keyword entries of the specified types
%known to be \emph{present},
%zero entries known to absolutely be \emph{absent},
%and an infinite number of \emph{unknown} entries entries.
%Since only keyword keys are allowed, they do not require quoting.

%\paragraph{HMap dispatch} The flexibility of \clj{isa?} is key to the generality of multimethods. 
%In \egref{example:incmap} we
%dispatch on the \clj{:op} key 
%of our HMap AST \clj{Expr}.
%Since keywords are functions that look themselves up in their argument, we simply
%use \clj{:op} as the dispatch function.
%
%\begin{exmp}
%\inputminted[firstline=5,lastline=18]{clojure}{code/demo/src/demo/eg5.clj}
%\label{example:incmap}
%\end{exmp}
%
%The function \clj{inc-leaf} is like \egref{example:decmap} except the nodes are incremented.
%The reasoning is similar, except we only consider one branch (the current method) by
%locally considering the current \emph{dispatch value} and reasoning about how it relates
%to the \emph{dispatch function}.
%For example, 
%in the \clj{:do} method we learn the \clj{:op} key is a \clj{:do}, which
%narrows our argument type to the \clj{:do} Expr, and similarly for the \clj{:if}
%and \clj{:const} methods.
%
%
%\subsection{Final example}
%
%\egref{example:final}
%combines everything we will cover for the rest of the paper:
%multimethod dispatch, reflection resolution via type hints, Java method
%and constructor calls, conditional and exceptional flow reasoning,
%and HMaps. 
%
%
%\begin{figure}
%\begin{exmp}
%\inputminted[firstline=6,lastline=23]{clojure}{code/demo/src/demo/eg7.clj}
%\label{example:multidispatch}
%\end{exmp}
%\begin{exmp}
%\inputminted[firstline=6,lastline=20]{clojure}{code/demo/src/demo/eg8.clj}
%\label{example:final}
%\end{exmp}
%\caption{Multimethod Examples}
%\end{figure}
%
%We dispatch on \clj{:p} to distinguish the two cases of \clj{FSM}---for example on \clj{:F}
%we know the \clj{:file} is a file.
%The body of the first method uses type hints to resolve reflection
%and conditional control flow to prove null-pointer exceptions are impossible.
%The second method is similar except it uses exceptional control flow.


\section{A Formal Model of \lambdatc{}}

\label{sec:formal}

After demonstrating the core features of Typed Clojure, 
we link them together in a formal model called
\lambdatc{}.
%
Building on occurrence typing,
we incrementally add each
novel feature of Typed Clojure to the formalism,
interleaving presentation of syntax, typing rules, operational semantics,
and subtyping.

\subsection{Core type system}
\label{sec:coretypesystem}

We start with a review of
occurrence typing~\cite{TF10}, the foundation of \lambdatc{}.
%We build up the occurrence typing calculus for illustrative purposes, 
%and present the full syntax at the end of the section.

\paragraph{Expressions} Syntax is given in \figref{main:figure:termsyntax}. Expressions \e{} 
include variables \x{}, values \v{},
applications, abstractions, conditionals, and let expressions.
All binding forms introduce fresh variables---a subtle but important point since our type environments
are not simply dictionaries.
Values include booleans \bool{}, \nil{}, class literals {\class{}}, keywords \k{},
integers {\nat{}},
constants {\const{}}, and strings \str{}. Lexical closures {\closure {\openv{}} {\abs {\x{}} {\t{}} {\e{}}}}
close value environments \openv{}---which map bindings to values---over functions.

\paragraph{Types} Types \s{} or \t{} 
include the top type \Top,
\emph{untagged} unions {\Unionsplice {\overrightarrow{\t{}}}}, 
singletons ${\Value \singletonmeta{}}$,
and class instances \class{}.
We abbreviate the classes
\Booleanlong{} to \Boolean{}, 
\Keywordlong{} to \Keyword{},
\NumberFull{}  to \Number{},
\StringFull{}  to \String{}, and 
\FileFull{}  to \File{}.
We also abbreviate the types
\EmptyUnion{}     to \Bot{}, 
{\ValueNil}       to \Nil{}, 
{\ValueTrue}      to \True, and
{\ValueFalse} to {\False}.
%
The difference between the types
\Value{\class{}} and \class{} is subtle.
The former is inhabited by class literals like \Keyword{} and the result of 
\appexp{\classconst{}}{\makekw{a}}---the latter by \emph{instances} of classes,
like a keyword literal \makekw{a}, an instance of the type \Keyword{}.
%
Function types 
$
{\ArrowOne {\x{}} {\s{}}
             {\t{}}
             {\filterset {\prop{}} {\prop{}}}
             {\object{}}}
$
contain \emph{latent} (terminology from~\cite{Lucassen88polymorphiceffect}) propositions \prop{}, object \object{}, and return type
\t{},
which may refer to the function argument \x{}.
%Latent means they are relevant when the function is applied rather than evaluated.
They are instantiated with the
actual object of the argument in applications. % before they are used in the proposition environment.

\paragraph{Objects}
%As we saw in \secref{sec:overview},
%occurrence typing is capable of reasoning
%about deeply nested expressions.
Each expression is associated with 
a symbolic representation
called an \emph{object}.
%with respect to the current lexical environment. 
For example,
  variable \makelocal{m} has object \makelocal{m};
  $\appexpone{\ccclass{\appexp{\makekw{lunch}}{\makelocal{m}}}}$ has object ${\path{\classpe{}}{\path{\keype{\makekw{lunch}}}{\makelocal{m}}}}$; and $42$ has the \emph{empty} object \emptyobject{} since it is unimportant in our system.
%
\figref{main:figure:termsyntax} gives the syntax for objects \object{}---non-empty objects 
\path{\pathelem{}}{\x{}} combine of a root variable \x{} and a \emph{path} \pathelem{},
which consists of
a possibly-empty sequence of \emph{path elements} (\pesyntax{}) applied right-to-left from the root variable.
We use two path elements---{\classpe{}} and {\keype{k}}---representing the results
of calling \classconst{} and looking up a keyword $k$, respectively.

\paragraph{Propositions with a logical system}
In standard type systems, association lists often
track the types of variables, like in LC-Let and LC-Local.
\begin{mathpar}
\infer [LC-Let]
{ \judgementtwo {{\propenv{}}}
                {\e{1}} {\s{}}
  \\
  \judgementtwo {{\propenv{}},\x{} \mapsto {\s{}}}
                {\e{2}} {\t{}}
}
{ 
  \judgementtwo {\propenv{}} 
            {\letexp{\x{}}{\e{1}}{\e{2}}} {\t{}}
           }

\infer [LC-Local]
{ {\propenv{}}(\x{}) = {\t{}}
}
{ \judgementtwo {\propenv{}} 
            {\x{}} {\t{}}
           }
\end{mathpar}

Occurrence typing instead pairs \emph{logical formulas},
that can reason about arbitrary non-empty objects,
with a \emph{proof system}.
The logical statement {\isprop{\s{}}{\x{}}} says
variable $x$ is of type \s{}. 
%A \emph{logical system}
%must now \emph{prove}
%a variable's type.
\begin{mathpar}
\infer [T0-Let]
{ \judgementtwo {{\propenv{}}}
                {\e{1}} {\s{}}
  \\
  \judgementtwo {{\propenv{}},\isprop{\s{}}{\x{}}}
                {\e{2}} {\t{}}
}
{ 
  \judgementtwo {\propenv{}} 
            {\letexp{\x{}}{\e{1}}{\e{2}}}
            {\t{}}
           }
%\judgementtwo{\isprop{\Number{}}{\x{}}}{\appexp{\inc{}}{\x{}}}{\Number}

\infer [T0-Local]
{ \inpropenv {\propenv{}} {\isprop {\t{}} {\x{}}}}
{ \judgementtwo {\propenv{}} 
            {\x{}} {\t{}}
           }
\end{mathpar}
In T0-Local, 
$
{ \inpropenv {\propenv{}} {\isprop {\t{}}{\x{}}}}
$
appeals to the proof system to solve for \t{}.
%says under logical assumptions {\propenv{}}, object {\path{\pathelem{}}{\x{}}} is of type \t{}.
%We later define the more general T-Local.

\begin{figure}[t!]
  \footnotesize
$$
\begin{array}{lrll}
  \e{} &::=& \x{}
                      \alt \v{} 
                      \alt {\comb {\e{}} {\e{}}} 
                      \alt {\abs {\x{}} {\t{}} {\e{}}} 
                      \alt {\ifexp {\e{}} {\e{}} {\e{}}}
                      %\alt {\trdiff{\doexp {\e{}} {\e{}}}}
                      \alt {\letexp {\x{}} {\e{}} {\e{}}}
                      %\alt {\errorvalv{}}
                      &\mbox{Expressions} \\
  \v{} &::=&          \singletonmeta{}
                      \alt {\nat{}}
                      \alt {\const{}}
                      \alt {\str{}}
                      \alt {\closure {\openv{}} {\abs {\x{}} {\t{}} {\e{}}}}
                &\mbox{Values} \\
                \constantssyntax{}\\
  \s{}, \t{}    &::=& \Top 
                      \alt {\Unionsplice {\overrightarrow{\t{}}}}
                      \alt
                      {\ArrowOne {\x{}} {\t{}}
                                   {\t{}}
                                   {\filterset {\prop{}} {\prop{}}}
                                   {\object{}}}
                      \alt {\Value \singletonmeta{}} 
                      \alt \trdiff{\class{}}
                &\mbox{Types} \\
  \singletonallsyntax{}
                \\ \\
  \occurrencetypingsyntax{}\\
  \propenvsyntax{}\\
  \openvsyntax{}
  %\\
  %\classliteralallsyntax{}
\end{array}
$$
\caption{Syntax of Terms, Types, Propositions and Objects}
\label{main:figure:termsyntax}
\end{figure}

We further extend logical statements to \emph{propositional logic}.
\figref{main:figure:termsyntax} describes the syntax
for propositions \prop{},
consisting of positive and negative \emph{type propositions} 
about non-empty objects---{\isprop {\t{}} {\path {\pathelem{}} {\x{}}}}
and {\notprop {\t{}} {\path {\pathelem{}} {\x{}}}}
respectively---the latter pronounced ``the object {\path {\pathelem{}} {\x{}}} is \emph{not} of type \t{}''.
The other propositions are standard logical connectives: implications, conjunctions,
disjunctions, and the trivial (\topprop{}) and impossible (\botprop{}) propositions.
%
The full proof system judgement
$
{ \inpropenv {\propenv{}} {\prop{}} }
$
says \emph{proposition environment} {\propenv{}} proves proposition \prop{}.

Each expression is associated with two propositions---when expression
\e{1} is in test position like
\ifexp{\e{1}}{\e{2}}{\e{3}},
the type system extracts \e{1}'s `then' and `else' proposition to check
\e{2} and \e{3} respectively.
For example, in \ifexp{\makelocal{o}}{\e{2}}{\e{3}}
we learn variable {\makelocal{o}} is true in \e{2} via {\makelocal{o}}'s `then' proposition $\notprop{\falsy{}}{\makelocal{o}}$, and 
that {\makelocal{o}} is false in \e{3} via {\makelocal{o}}'s `else' proposition $\isprop{\falsy{}}{\makelocal{o}}$.

To illustrate, recall \egref{example:desserts-on-meal}.
The parameter \makelocal{o} is of type $\Order$,
%by the annotation on $desserts$
written
{\isprop{\Order}{\makelocal{o}}}
as a proposition.
%
In the ${\makekw{combo}}$ method, we know
${\appexp{\makekw{Meal}}{\makelocal{o}}}$ is ${\makekw{combo}}$,
based on multimethod dispatch rules. This is written
  {\isprop{\Value{\makekw{combo}}}{\path{\keype{\makekw{Meal}}}{\makelocal{o}}}},
pronounced ``the ${\makekw{Meal}}$ path of variable \makelocal{o} is of type
{\Value{\makekw{combo}}}''.

%\paragraph{Logical system in action} 
To attain the type of \makelocal{o}, 
we must solve for \t{} in
$
{ \inpropenv 
  {\propenv{}}
  {\isprop {\t{}} {\makelocal{o}}}}
$,
under proposition environment
$
\propenv{} = {{\isprop{\Order}{\makelocal{o}}},
    {\isprop{\Value{\makekw{combo}}}{\path{\keype{\makekw{Meal}}}{\makelocal{o}}}}}
$
which deduces \t{} to be a {\makekw{combo}} meal.
The logical
system \emph{combines} pieces of type information to deduce more accurate types for lexical
bindings---this is explained in \secref{formalmodel:proofsystem}.

%The first insight about occurrence typing is that
%logical formulas
%can be used to represent type information about our programs
%by relating parts of the runtime environment to types
%via propositional logic.
%\emph{Type propositions}  make assertions like ``variable \x{} is of type \NumberFull{}'' or
%``variable \x{} is not \nil{}''---in our logical system we write these as
%{\isprop{\NumberFull}{\x{}}}
%and {\notprop{\Nil{}}{\x{}}}. 
%
%The second insight is that we can replace the traditional 
%representation of a
%type environment (eg., a map from variables to types)
%with a set of propositions, written \propenv{}. 
%Instead of mapping \x{} to
%the type \NumberFull{}, we use the proposition {\isprop{\NumberFull}{\x{}}}.





\begin{figure*}[t]
\footnotesize
    %{\TDo}
    %{\TClass}
    %{\TIf}
    %{\TAbs}
    %\begin{array}{c}
    %  {\TSubsume}\\\\
    %  {\TNum}
    %\end{array}
  \begin{mathpar}
        {\TLocal}
        {\TAbs}
        {\TIf}
    \\
    \begin{array}{c}
    {\TKw}\\
      {\TNum}\\
    \end{array}
    \begin{array}{c}
      {\TNil}\\
      {\TFalse}\\
    {\TConst}
    \end{array}
    \begin{array}{c}
    {\TStr}\\
    {\TClass}\\
    {\TTrue}
  \end{array}
        \\

    {\TLet}
    \\

    {\TApp}\ \ 
    {\TSubsume}
    \\
  \end{mathpar}
    %\begin{array}{c}
    %  {\TSubsume}\\\\
    %  {\TStr}\\\\
    %  {\TNil}\\\\
    %  {\TFalse}
    %\end{array}
  \caption{Core typing rules}
  \label{main:figure:othertypingrules}
\end{figure*}

\begin{figure}%[t!]
  \footnotesize
  \begin{mathpar}

\SUnionSuper{}\ \ \ 
\SUnionSub{}\ \ \ 
\SFunMono{}\ \ \ 
\begin{array}{l}
\SObject{}\\
\SClass{}\\
\SSBool{}
\end{array}

\SFun{}
\begin{array}{l}
    \SRefl{}\ \ \ 
    \STop{}\\
\SSKw{}
\end{array}


  \end{mathpar}
  \caption{Core subtyping rules}
  \label{main:figure:subtyping}
\end{figure}

\begin{figure}
\begin{mathpar}
    \BIfTrue{}

    \BIfFalse{}
\end{mathpar}
  \caption{Select core semantics}
\label{main:figure:coresemantics}
\end{figure}

\paragraph{Typing judgment}

We formalize our system following Tobin-Hochstadt and Felleisen \cite{TF10}.
%(with differences highlighted in $\trdiff{\text{blue}}$)
The typing judgment 
$
{\judgementrewrite   {\propenv}
              {\e{}} {\t{}}
  {\filterset {\thenprop {\prop{}}}
              {\elseprop {\prop{}}}}
  {\object{}}
  {\ep{}}}
$
says expression \e{} rewrites to \ep{}, which
is of type \t{} in the 
proposition environment $\propenv{}$, with 
`then' proposition {\thenprop {\prop{}}}, `else' proposition {\elseprop {\prop{}}}
and object \object{}. 

We write 
{\judgementtworewrite{\propenv}{\e{}} {\t{}}{\ep{}} 
to mean 
{\judgementrewrite   {\propenv}
              {\e{}} {\t{}}
  {\filterset {\thenprop {\propp{}}}
              {\elseprop {\propp{}}}}
  {\objectp{}}
  {\ep{}}}
for some {\thenprop {\propp{}}}, {\elseprop {\propp{}}}
and {\objectp{}},
and
  abbreviate self rewriting judgements
{\judgementrewrite   {\propenv}
              {\e{}} {\t{}}
  {\filterset {\thenprop {\prop{}}}
              {\elseprop {\prop{}}}}
  {\object{}}
  {\e{}}}
to
{\judgementselfrewrite   {\propenv}
              {\e{}} {\t{}}
  {\filterset {\thenprop {\prop{}}}
              {\elseprop {\prop{}}}}
  {\object{}}}.


\paragraph{Typing rules}

The core typing rules are
given as \figref{main:figure:othertypingrules}. We introduce
the interesting rules with the complement number predicate
as a running example.
\begin{equation}
\abs{\makelocal{d}}{\Top}{\ifexp{\appexp{\numberhuh{}}{\makelocal{d}}}{\false{}}{\true{}}}
\end{equation}
%, including
%a subsumption rule T-Subsume and rules for the false values---T-Nil and T-False---encoded 
%as impossible (\botprop{}) `then' propositions.

The lambda rule T-Abs introduces \isprop{\s{}}{\x{}}} = \isprop{\Top}{\makelocal{d}}
to check the body.
With \propenv{} = \isprop{\Top}{\makelocal{d}},
T-If first checks the test \e{1} = {\appexp{\numberhuh{}}{\makelocal{d}}}
via the T-App rule, with three steps.

First, in T-App the operator \e{} = \numberhuh{} is checked with T-Const, which
uses 
\constanttypeliteral{} (\figref{main:figure:constanttyping}, dynamic semantics in the supplemental material)
to type constants.
\numberhuh{} is a predicate over numbers, and
\classconst{} returns its argument's class.

Resuming {\appexp{\numberhuh{}}{\makelocal{d}}},
in T-App the operand \ep{} = \makelocal{d} is checked with
T-Local as
\begin{equation}
\judgementselfrewrite{\propenv{}}
                     {\makelocal{d}}
                     {\Top}
                     {\filterset{\notprop{\falsy}{\makelocal{d}}}
                                {\isprop{\falsy}{\makelocal{d}}}}
                     {\makelocal{d}}
\end{equation}
which encodes the type, proposition, and object information
about variables. The proposition {\notprop{\falsy}{\makelocal{d}}}
says ``it is not the case that variable {\makelocal{d}} is of type {\falsy}'';
{\isprop{\falsy}{\makelocal{d}}} says ``{\makelocal{d}} is of type {\falsy}''.

Finally, the T-App rule substitutes the operand's object \objectp{}
for the parameter \x{} in the latent type, propositions, and object. The proposition
{\isprop{\Number{}}{\makelocal{d}}} says ``{\makelocal{d}} is of type {\Number{}}'';
{\notprop{\Number{}}{\makelocal{d}}} says ``it is not the case that {\makelocal{d}}
is of type {\Number{}}''. The object {\makelocal{d}} is the symbolic representation
of what the expression {\makelocal{d}} evaluates to.
\begin{equation}
\judgementselfrewrite{\propenv{}}
  {\appexp{\numberhuh{}}{\makelocal{d}}}
  {\Boolean{}}
  {\filterset{\isprop{\Number{}}{\makelocal{d}}}
             {\notprop{\Number{}}{\makelocal{d}}}}
  {\emptyobject{}}
\end{equation}
To demonstrate, the `then' proposition---in T-App {\replacefor {\thenprop{\prop{}}} {\objectp{}} {\x{}}}---substitutes
the latent `then' proposition of \constanttype{\numberhuh{}} with 
\makelocal{d}, giving
{\replacefor {\isprop{\Number{}}{\x{}}} {\makelocal{d}} {\x{}}} =
{\isprop{\Number{}}{\makelocal{d}}}.

To check the branches of {\ifexp{\appexp{\numberhuh{}}{\makelocal{d}}}{\false{}}{\true{}}},
T-If
introduces \thenprop{\prop{1}} = \isprop{\Number{}}{\makelocal{d}}
to check \e{2} = {\false{}},
and \elseprop{\prop{1}} = \notprop{\Number{}}{\makelocal{d}}
to check 
\e{3} = \true{}.
%
The branches are first checked with T-False and T-True respectively,
the T-Subsume premises
\inpropenv {\propenv{}, {\thenprop {\prop{}}}} {\thenprop {\propp{}}}
and
\inpropenv {\propenv{}, {\elseprop {\prop{}}}} {\elseprop {\propp{}}}
allow us to pick compatible propositions for both branches.
%$$
%\judgementselfrewrite{\propenv{},{\isprop{\Number{}}{\makelocal{d}}}}
%  {\false{}}
%  {\False{}}
%  {\filterset{\botprop{}}
%             {\topprop{}}}
%  {\emptyobject{}}
%$$
$$
\begin{array}{c}
\judgementselfrewrite{\propenv{},{\isprop{\Number{}}{\makelocal{d}}}}
  {\false{}}
  {\Boolean{}}
  {\filterset{\notprop{\Number{}}{\makelocal{d}}}
             {\isprop{\Number{}}{\makelocal{d}}}}
  {\emptyobject{}}
  \\
\judgementselfrewrite{\propenv{},{\notprop{\Number{}}{\makelocal{d}}}}
  {\true{}}
  {\Boolean{}}
  {\filterset{\notprop{\Number{}}{\makelocal{d}}}
             {\isprop{\Number{}}{\makelocal{d}}}}
  {\emptyobject{}}
\end{array}
$$
%to suit the T-If outputs \t{} = \Boolean{}, \thenprop{\prop{}}
%= {\notprop{\Number{}}{\makelocal{d}}}, \elseprop{\prop{}} = {\isprop{\Number{}}{\makelocal{d}}},
%and \object{} = {\emptyobject{}}.
%
%In T-Subsume, we can upcast \t{} = \False{} to \tp{} = \Boolean{} via the premise 
%\issubtypein{}{\t{}}{\tp{}}.
%and 
%\inpropenv {\propenv{}, {\elseprop {\prop{}}}} {\elseprop {\propp{}}}
%from 
%{\elseprop {\prop{}}} = \topprop{} to {\elseprop {\propp{}}} = {\isprop{\Number{}}{\makelocal{d}}}.

Finally T-Abs assigns a type to the overall function:
$$
{\judgementselfrewrite{}{\abs{\makelocal{d}}{\Top}{\ifexp{\appexp{\numberhuh{}}{\makelocal{d}}}{\false{}}{\true{}}}}
                    {\ArrowOne {\makelocal{d}} {\Top{}}
                                      {\Boolean{}}
                                      {\filterset {\notprop{\Number{}}{\makelocal{d}}}
                                                  {\isprop{\Number{}}{\makelocal{d}}}}
                                      {\emptyobject{}}}
                    {\filterset {\topprop{}}
                                {\botprop{}}}
                    {\emptyobject{}}}
$$

%The object \x{} over latent propositions \thenprop{\prop{}} and
%\elseprop{\prop{}}, latent object \object{}, and latent type \t{}. The
%actual argument object---\objectp{}---is substituted in at application by T-App.



%The T-App rule instantiates parameters---like \x{} in T-Abs---with their actual object.
%The expression {\appexp{\numberhuh{}}{\makelocal{d}}} replaces its parameter object with \
%\judgementselfrewrite{\isprop{\Top}{\makelocal{d}}}{\appexp{\numberhuh{}}{\makelocal{d}}}
%  {\Boolean{}}
%  {\filterset{\isprop{\Number{}}{\makelocal{d}}}
%             {\notprop{\Number{}}{\makelocal{d}}}}
%  {\emptyobject{}}

%The T-If refines each branch based on the test---a
%{\appexp{\numberhuh{}}{\makelocal{d}}} test introduces \isprop{\Number{}}{\makelocal{d}}
%for checking the `then' branch and \notprop{\Number{}}{\makelocal{d}}
%for checking the `else' branch.
%%Information from each branch is combined via subsumption for the overall type, propositions, and object.
%The T-Subsume rule 

%For example
%$
%\judgementselfrewrite{\isprop{\Top}{\makelocal{d}}}
%  {\ifexp{\appexp{\numberhuh{}}{\makelocal{d}}}{\false{}}{\true{}}}
%  {\Boolean{}}
%  {\filterset{\notprop{\Number{}}{\makelocal{d}}}
%             {\isprop{\Number{}}{\makelocal{d}}}}
%  {\emptyobject{}}
%$
%type checks because T-Subsume allows us to check both branches as
%$
%\judgementselfrewrite{\isprop{\Number{}}{\makelocal{d}}}
%  {\false{}}
%  {\Boolean{}}
%  {\filterset{\notprop{\Number{}}{\makelocal{d}}}
%             {\isprop{\Number{}}{\makelocal{d}}}}
%  {\emptyobject{}}
%$
%and
%$
%\judgementselfrewrite{\notprop{\Number{}}{\makelocal{d}}}
%  {\true{}}
%  {\Boolean{}}
%  {\filterset{\notprop{\Number{}}{\makelocal{d}}}
%             {\isprop{\Number{}}{\makelocal{d}}}}
%  {\emptyobject{}}
%$
%respectively.
%
%For example, if \e{2} is true when variable {\makelocal{y}} is a \File{}
%and \e{3} is true when variable {\makelocal{z}} is a \Number{}, then
%we use T-Subsume on both branches
%to introduce a logical disjunction
%since at least one must be true if the entire expression is true.



%Given a set of propositions, we can use logical reasoning to derive
%new information about our programs
%with the judgment \inpropenv{\propenv{}}{\prop{}}.
%In addition to the standard rules for the logical connectives, the key
%rule is L-Update, which combines multiple propositions about the same variable,
%allowing us to refine its type.
%$$
%  {\LUpdate}
%$$
%For example, with L-Update we can use the knowledge of
%\inpropenv{\propenv{}}{\isprop{\UnionNilNum}{\x{}}}
%and 
%\inpropenv{\propenv{}}{\notprop{\Nil{}}{\x{}}}
%to derive \inpropenv{\propenv{}}{\isprop{\Number}{\x{}}}.
%(The metavariable \propisnotmeta{} ranges over \t{} and \nottype{\t{}} (without variables).)
%We cover L-Update in more detail in \secref{sec:formalpaths}.
%
%Finally, this approach allows the type system to track
%programming idioms from 
%dynamic languages
%using implicit type-based reasoning based on the result of
%conditional tests.
%For instance,
%\egref{example:parent-if}
%only utilizes \clj{f} once
%the programmer is convinced it is safe to do so based whether
%\clj{f}
%is
%true or false. 
%To express this in the type system, every expression 
%is described by two propositions: a `then' proposition
%for when it reduces to a true value, and an `else' proposition
%when it reduces to a false value---for \clj{f}
%the then proposition is {\notprop{\falsy}{f}} and 
%the else proposition is {\isprop{\falsy}{f}}.
%\ref{main:figure:typingrules}




%\figref{main:figure:typingrules} contains the core typing rules.
%The key rule for reasoning about conditional control flow is
%T-If. 
%
%\begin{mathpar}
%  {\TIf}
%\end{mathpar}

%The propositions of the test expression \e{1}, \thenprop{\prop{1}} and \elseprop{\prop{1}}, are 
%used as assumptions in the then and else branch respectively.

%The let rule T-Let links inferred information about
%\x{} to the expression used to instantiate \x{}, \ep{1}, via logical implications.
%
%The T-Local rule connects the type system to the proof system over type propositions
%via \inpropenv {\propenv{}} {\isprop {\t{}} {\x{}}}
%to derive a type for a variable.
%Using this rule, the type system can then appeal to L-Update to refine the type
%assigned to \x{}.
%
%We are now equipped to type check
%\egref{example:parent-if}:
%$$
%\clj{(if f (.getParent f) nil)}
%$$
%
%With {\propenv{}} = {\isprop{\UnionNilFile{}}{f}},
%$$
%\judgement{\propenv{}}{f}{\UnionNilFile{}}{\localfilterset{f}}{f}
%$$
%via T-Local.
%
%Checking the then branch involves extending
%the proposition environment with {\notprop{\falsy}{f}}
%$$
%\judgement{{\propenv{}},{\isprop{\Number}{\x{}}}}{\x{}}{\Number{}}{\filterset{\notprop{\falsy{}}{\x{}}}{\isprop{\falsy{}}{\x{}}}}{\emptyobject{}}
%$$
%because we can now satisfy the premise of T-Local:
%$$
%\inpropenv{{\propenv{}},\isprop{\Number}{\x{}}}{\isprop{\Number}{\x{}}}.
%$$
%\judgement{{\propenv{}},\isprop{\Number}{\x{}}}{\hastype{\appexp{\inc{}}{\x{}}}{\Number{}}}{\filterset{\topprop{}}{\botprop{}}}{\emptyobject{}}
%$$
%$$
%\judgement{{\propenv{}},\notprop{\Number}{\x{}}}{\hastype{\zeroliteral{}}{\Number}}{\filterset{\topprop{}}{\botprop{}}}{\emptyobject{}}
%$$

%\inc{} has type
%$$
%{\ArrowOne{\x{}}{\Number}{\Number}
%        {\filterset{\topprop{}}{\topprop{}}}{\emptyobject{}}}
%$$
%We can now check the conditional with T-If.
%$$
%\judgement{\isprop{\Number}{\x{}}}{\hastype{\ifexp{\appexp{\numberhuh{}}{\x{}}}{\appexp{\inc{}}{\x{}}}{\zeroliteral{}}}{\Number}}{\filterset{\orprop{\isprop{\Number}{\x{}}}{\topprop{}}}{\orprop{\notprop{\Number}{\x{}}}{\topprop{}}}}{\emptyobject{}}
%$$
%Finally the function can be checked with T-Abs
%$$
%\judgement{}{\hastype{\abs{\x{}}{\UnionNilNum}{\ ...}}
%                                             {\ArrowOne{\x{}}{\UnionNilNum}{\Number}
%        {\filterset{\orprop{\isprop{\Number}{\x{}}}{\topprop{}}}{\orprop{\notprop{\Number}{\x{}}}{\topprop{}}}}{\emptyobject{}}}}
%  {\filterset{\topprop{}}{\botprop{}}}{\emptyobject{}}
%$$

\paragraph{Subtyping}
\figref{main:figure:subtyping} presents subtyping
as a reflexive and transitive relation with top type \Top. 
Singleton types are instances of their respective classes---boolean singleton types
are of type \Boolean{}, class literals are instances of \Class{} and keywords are
instances of \Keyword{}.
Instances of classes \class{} are subtypes of \Object{}. Function types 
are subtypes of \IFn{}. All types except for \Nil{} are subtypes of \Object{},
so \Top{} is similar to {\Union{\Nil}{\Object}}.
Function subtyping is contravariant left of the arrow---latent propositions, object
and result type are covariant.
Subtyping for untagged unions is standard.

\paragraph{Operational semantics} We define the dynamic semantics for \lambdatc{}
in a big-step style using an environment, following~\cite{TF10}.
We include both errors and a \wrong{} value, which is provably ruled out by the
type system.
The main judgment is \opsem{\openv{}}{\e{}}{\definedreduction{}}
which states that \e{} evaluates to answer \definedreduction{} in environment
\openv{}. We chose to omit the core rules (included in supplemental material)
however a notable difference is \nil{} is a false value, which affects the
semantics of \ifliteral{} (\figref{main:figure:coresemantics}).

%The definition of \updateliteral{} supports various idioms relating to \classpe{}
%which we introduce in \secref{sec:isaformal}.

%\begin{figure*}
%  \footnotesize
%%%   \judgbox{
%%%{\judgementrewrite   {\propenv}
%%%              {\e{}} {\t{}}
%%%  {\filterset {\thenprop {\prop{}}}
%%%              {\elseprop {\prop{}}}}
%%%  {\object{}}{\ep{}}}}
%%%           {Under proposition environment $\propenv{}$, 
%%%             expression \e{} is of type \t{}
%%%             with  \\
%%%
%%%`then' proposition {\thenprop {\prop{}}}, `else' proposition {\elseprop {\prop{}}}
%%%and object \object{} and rewrites to \ep{}.}
%  \begin{mathpar}
%    %{\TDo}
%    %{\TClass}
%    %{\TIf}
%    %{\TAbs}
%    %\begin{array}{c}
%    %  {\TSubsume}\\\\
%    %  {\TNum}
%    %\end{array}
%    \begin{array}{c}
%      {\TNum}\\\\
%      {\TConst}\\\\
%      {\TKw}\\\\
%      {\TClass}\\\\
%      {\TTrue}\\\\
%    \end{array}
%    \begin{array}{c}
%      {\TSubsume}\\\\
%      {\TNil}\\\\
%      {\TFalse}
%    \end{array}
%
%    %{\TLet}
%    %{\TLocal}
%
%    %{\TApp}
%    %{\TError}
%
%  \end{mathpar}
%  \caption{Typing rules}
%  \label{main:figure:typingrules}
%\end{figure*}

%\begin{figure}
%  \footnotesize
%  \begin{mathpar}
%    {\BLocal}
%
%    %{\BDo}
%
%    {\BLet}
%
%    \BVal{}
%
    %\BIfTrue{}

%    \BIfFalse{}
%
%    \BAbs{}
%
%    \BBetaClosure{}
%
%    \BDelta{}
%  \end{mathpar}
%  \caption{Operational Semantics}
%  \label{main:figure:standardopsem}
%\end{figure}

%\subsection{Reasoning about Exceptional Control Flow}
%\label{sec:doformal}
%
%Along with conditional control flow,
%Clojure programmers rely on \emph{exceptions}
%to assert type-related invariants.
%
%\begin{exmp}
%\inputminted[firstline=13,lastline=15]{clojure}{code/demo/src/demo/do.clj}
%\label{example:doexception}
%\end{exmp}
%
%The fully expanded increment function in~\egref{example:doexception}
%guards its final call with a number check, preventing
%a possible null-pointer exception.
%Without this check, the type system would reject the program.
%
%To check this example,
%occurrence typing 
%automatically
%assumes
%\clj{x} is a number when checking the second \clj{do} subexpression
%based on the first subexpression.
%\footnote{See \url{https://github.com/typedclojure/examples}
%  for full examples.}
%We model this formally %(section~\ref{sec:doformal}) 
%and prove
%null-pointer exceptions are impossible in typed code (section~\ref{sec:metatheory}).
%
%
%We extend our model with sequencing expressions and errors, where {\errorvalv{}}
%models the result of calling Clojure's \clj{throw} special form
%with some \clj{Throwable}.
%
%\smallskip
%$
%\begin{altgrammar}
%  \e{} &::=& \ldots \alt {\errorvalv{}} \alt {\doexp {\e{}} {\e{}}} &\mbox{Expressions} 
%\end{altgrammar}
%$
%
%\smallskip
%%
%%B-Do simply evaluates its arguments sequentially and returns the right argument.
%%Since errors are not values, we define error propagation semantics
%%like BE-Do1 (figure~\ref{appendix:figure:errorstuck} for the full rules).
%%
%%\begin{mathpar}
%%    {\BDo}
%%
%%\infer [BE-Error]
%%{}
%%{ \opsem {\openv{}} 
%%         {\errorvalv{}}
%%         {\errorvalv{}}}
%%
%%\infer [BE-Do1]
%%{ \opsem {\openv{}} {\e{1}} {\errorvalv{}} }
%%{ \opsem {\openv{}} {\doexp{\e{1}}{\e{}}} {\errorvalv{}}}
%%\end{mathpar}
%
%Our main insight is as follows: 
%if the first subexpression in a sequence reduces to a value, then it is either true or false.
%If we learn some proposition in both cases then we can use that proposition as an assumption to check the second subexpression.
%T-Do formalizes this intuition.
%
%\begin{mathpar}
%    {\TDo}  
%\end{mathpar}
%
%The introduction of errors, 
%which do not evaluate to either
%a true or false value,
%makes our insight interesting.
%
%\begin{mathpar}
%    {\TError}
%\end{mathpar}
%
%Recall \egref{example:doexception}.
%\begin{minted}{clojure}
%...
%  (do (if (number? x) nil (throw (new Exception)))
%      (inc x)) 
%...
%\end{minted}
%
%As before, checking \appexp{\numberhuh{}}{\x{}} allows us to use the proposition \isprop{\Number}{\x{}}
%when checking the then branch.
%
%By T-Nil and subsumption we can propagate this  information to both propositions.
%$$
%\judgement{\isprop{\Number}{\x{}}}{\nil{}}{\Nil{}}{\filterset{\isprop{\Number}{\x{}}}{\isprop{\Number}{\x{}}}}{\emptyobject{}}
%$$
%Furthermore, using T-Error
%and subsumption we can conclude anything in the else branch.
%$$
%\judgement{\notprop{\Number}{\x{}}}{\errorvalv{}}{\Bot}{\filterset{\isprop{\Number}{\x{}}}{\isprop{\Number}{\x{}}}}{\emptyobject{}}
%$$
%Using the above as premises to T-If we conclude that if the first
%expression in the \doliteral{} evaluates successfully, \isprop{\Number}{\x{}} must be true.
%$$
%\judgement{\isprop{\UnionNilNum}{\x{}}}
%          {\ifexp{\appexp{\numberhuh{}}{\x{}}}{\nil{}}{\errorvalv{}}}{\Boolean}
%          {\filterset{\isprop{\Number}{\x{}}}{\isprop{\Number}{\x{}}}}{\emptyobject{}}
%$$
%We can now use \isprop{\Number}{\x{}} in the environment to check the second subexpression
%{\appexp{\inc{}}{\x{}}}, completing the example.

\subsection{Java Interoperability}

\begin{figure}
  \footnotesize
  $$
  \begin{altgrammar}
    \e{} &::=& \ldots   {\fieldexp {\fld{}} {\e{}}} \alt {\methodexp {\mth{}} {\e{}} {\overrightarrow{\e{}}}}
                      \alt {\newexp {\class{}} {\overrightarrow{\e{}}}}
                      &\mbox{Expressions}\\
     &\alt& \nonreflectiveexpsyntax{} &\mbox{Non-reflective Expressions}\\

    \v{} &::=& \ldots \alt {\classvalue{\classhint{}} {\overrightarrow {\classfieldpair{\fld{}} {\v{}}}}}
    &\mbox{Values} \\

    \classtableallsyntax{}
  \end{altgrammar}
  $$
  \begin{mathpar}
    {\TNew}

    {\TMethod}

    {\TField}

    %{\TInstance}
  \end{mathpar}
 %\classtablelookupsyntax{}
 \begin{mathpar}
  \begin{altgrammar}
    \convertjavatypenil{}
  \end{altgrammar}
  \begin{altgrammar}
    \convertjavatypenonnil{}
  \end{altgrammar}
  \begin{altgrammar}
    \converttctype{}
  \end{altgrammar}
\end{mathpar}
  \begin{mathpar}
    \BField{}\ \ \ 
%
    \BNew{}

    \BMethod{}
  \end{mathpar}
  \caption{Java Interoperability Syntax, Typing and Operational Semantics}
  \label{main:figure:javatyping}
\end{figure}

\begin{figure}
  $$
\constanttypefigure{}
  $$
  \caption{Constant typing}
  \label{main:figure:constanttyping}
\end{figure}

We present Java interoperability in a restricted setting without class inheritance,
overloading or Java Generics.
%
We extend the syntax in \figref{main:figure:javatyping} with Java field lookups and calls to
methods and constructors. 
To prevent ambiguity between zero-argument methods and fields,
we use Clojure's primitive ``dot'' syntax:
field accesses are written \fieldexp{\fld{}}{\e{}}
and method calls $\methodexp{\mth{}}{\e{}}{\overrightarrow{e}}$.
%and \clj{(new class es*)} is $\newexp{\class{}}{\overrightarrow{es}}$.

In \egref{example:getparent-direct-constructor}, \clj{(*interop .getParent (*interop new File "a/b" interop*) interop*)}
translates to
\begin{equation}  \label{eq:unresolved}
  \qquad {\methodexp {\getparent{}} {\newexp {\File{}} {\makestr{a/b}}} {}}
\end{equation}

But both the constructor and method are unresolved.
We introduce \emph{non-reflective} expressions for specifying exact Java overloads.
\begin{equation} \label{eq:resolved}
\qquad {\methodstaticexp {\File} {} {\String} {\getparent{}} {\newstaticexp {\String} {\File{}} {\File{}} {\makestr{a/b}}} {}}
\end{equation}
From the left, the one-argument constructor for \File takes a \String, and the 
\getparent{} method of
\File{} takes zero arguments
and
returns a \String.

We now walk through this conversion.% from unresolved expression~\ref{eq:unresolved} to 
%resolved expression~\ref{eq:resolved}.

\paragraph{Constructors} First we check and convert {\newexp {\File{}} {\makestr{a/b}}} to {\newstaticexp {\String} {\File{}} {\File{}} {\makestr{a/b}}}.
The T-New typing rule checks and rewrites constructors.
%$$
%    {\TNew}
%$$
To check
{\newexp {\File{}} {\makestr{a/b}}}
we first resolve the constructor overload in the class table---there is at most one
to simplify presentation.
With \classhint{1} = \String,
we convert to a nilable type the argument with \t{1} = \Union{\Nil}{\String}
and type check {\makestr{a/b}} against \t{1}.
Typed Clojure defaults to allowing non-nilable arguments, but this
can be overridden, so we model the more general case.
% which erases nil
The return Java type \File is converted to a non-nil
Typed Clojure type \t{} = \File for the return type,
and the propositions say constructors can never be false---constructors
can never produce the internal boolean value that Clojure uses for \false{}, or \nil{}.
Finally, the constructor rewrites to {\newstaticexp {\String} {\File{}} {\File{}} {\makestr{a/b}}}.

\paragraph{Methods} Next we convert {\methodexp {\getparent{}} {\newstaticexp {\String} {\File{}} {\File{}} {\makestr{a/b}}} {}}
to the non-reflective expression
{\methodstaticexp {\File} {} {\String} {\getparent{}} {\newstaticexp {\String} {\File{}} {\File{}} {\makestr{a/b}}} {}}.
%The T-Method rule checks unresolved methods.
%$$
%    {\TMethod}
%$$
The T-Method rule for unresolved methods
checks {\methodexp {\getparent{}} {\newstaticexp {\String} {\File{}} {\File{}} {\makestr{a/b}}} {}}.
We verify the target type \s{} = \File is non-nil by T-New.
The overload is chosen from the class table based on \classhint{1} = \File---there is at most one.
The nilable return type \t{} = \Union{\Nil}{\String} is given, and the 
entire expression rewrites to expression \ref{eq:resolved}.
%
%We allow arguments to constructors and methods to be nilable, but not method
%and field targets.

The T-Field rule (\figref{main:figure:javatyping}) is like T-Method, but without arguments.

The evaluation rules B-Field, B-New and B-Method (\figref{main:figure:javatyping}) simply evaluate their
arguments and call the relevant JVM operation, which we do not model---\secref{sec:metatheory}
states our exact assumptions.
There are no evaluation rules for reflective Java interoperability, since there are no typing
rules that rewrite to reflective calls.


%\subsection{Paths}
%\label{sec:formalpaths}
%
%Recall the first insight of occurrence typing---we can reason
%about specific \emph{parts} of the runtime environment
%using propositions.
%We refer to parts of the runtime environment via 
%a \emph{path} that consists of a series of
%\emph{path elements} applied right-to-left to a variable
%written \path{\pathelem{}}{\x{}}.
%\cite{TF10} introduce the path elements \carpe{} and \cdrpe{}
%to reason about selector operations on cons cells.
%We instead want to reason about HMap lookups and calls to \classconst{}.
%
%\paragraph{Key path element} We introduce our first path element
%{\keype{\k{}}}, which represents the operation of looking up
%a key \k{}.
%We directly relate this to our typing rule T-GetHMap
%(\figref{main:figure:hmapsyntax}) by
%checking the then branch of the first conditional test is checked in 
%an equivalent version of \egref{example:decleaf}.
%\begin{minted}{clojure}
%  (fn [m :- Expr]
%    (if (= (get m :op) :if)
%      {:op :if, ...}
%      (if ...)))
%\end{minted}
%
%We do not specifically support \equivliteral{} in our calculus, 
%but on keyword arguments it works identically to \clj{isa?} which we model
%in \secref{sec:isaformal}.
%Intuitively, if {\judgement{\propenv{}}{\e{}}{\t{}}{\filterset{\thenprop{\prop{}}}{\elseprop{\prop{}}}}{\object{}}}
%then \equivapp{\e{}}{\makekw{if}} has the true and false propositions
%$$
%{\replacefor{\filterset{\isprop{\Value{\makekw{if}}}{\x{}}}{\notprop{\Value{\makekw{if}}}{\x{}}}}{\object{}}{\x{}}}
%$$
%where substitution reduces to \topprop{} if \object{} = \emptyobject{}.
%
%We start with proposition environment \propenv{} = {\isprop{\Expr{}}{m}}.
%Since {\Expr{}} is a union of HMaps, each with the entry \makekw{op}, we can use T-GetHMap.
%$$
%\judgement{\propenv{}}{\getexp{m}{\makekw{op}}}{\Keyword}{\filterset{\topprop{}}{\topprop{}}}{\path{\keype{\makekw{op}}}{m}}
%$$
%Using our intuitive definition of \equivliteral{} above, we know
%$$
%\judgement{\propenv{}}{\equivapp{\getexp{m}{\makekw{op}}}{\makekw{if}}}{\Boolean}{\filterset{\isprop{\Value{\makekw{if}}}{\path{\keype{\makekw{op}}}{m}}}{\notprop{\Value{\makekw{if}}}{\path{\keype{\makekw{op}}}{m}}}}{\emptyobject{}}
%$$
%Going down the then branch gives us the extended environment
%\propenvp{} = {\isprop{\Expr{}}{m}},{\isprop{\Value{\makekw{if}}}{\path{\keype{\makekw{op}}}{m}}}.
%Using L-Update we can combine what we know about object $m$ and object
%{\path{\keype{\makekw{op}}}{m}}
%to derive
%$$
%\inpropenv{\propenvp{}}{\isprop{\HMapp{\mandatoryset{{\mandatoryentrykwnoarrow{op}{\makekw{if}}}, {\mandatoryentrykwnoarrow{test}{\Expr{}}},
%                                       {\mandatoryentrykwnoarrow{then}{\Expr{}}},   {\mandatoryentrykwnoarrow{else}{\Expr{}}}}}
%                                   {\emptyabsent{}}}{m}}
%$$
%
%The full definition of \updateliteral{} is given in \figref{main:figure:update}
%which considers both keys a path elements as well as the \classconst{}
%path element described below.
%In the absence of paths, update simply performs set-theoretic operations
%on types; see \figref{main:figure:restrictremove} for details.
%
%\paragraph{Class path element} Our second path element \classpe{} is used in the latent
%object of the constant \classconst{} function. Like Clojure's \clj{class}
%function \classconst{} returns the argument's class or \nil{}
%if passed \nil{}.
%$$
%\begin{array}{lrlr}
%  \pesyntax{}   &::=& \ldots \alt {\classpe{}}
%                &\mbox{Path Elements}
%\end{array}
%$$
%\begin{mathpar}
%\constanttypefigure{}
%\end{mathpar}
%The dynamic semantics are given in \figref{main:figure:primitivesem}.
%The definition of \updateliteral{} supports various idioms relating to \classpe{}
%which we introduce in \secref{sec:isaformal}.

\subsection{Multimethod preliminaries: \isaliteral}

\label{sec:isaformal}

We now consider the \isaliteral{} operation, a core part of the multimethod dispatch mechanism. 
Recalling the examples in \secref{sec:multioverview},
\isaliteral{} is
a subclassing test for classes, but otherwise is an equality test.
%---we do not model the semantics for vectors
%
The T-IsA rule uses \isacompareliteral{}
(\figref{main:figure:mmsyntax}), a metafunction which produces the propositions for
\isaliteral{} expressions.
%\begin{mathpar}
%  \TIsA{}
%\end{mathpar}

To demonstrate the first \isacompareliteral{} case,
the expression
\isaapp{\appexp{\classconst{}}{\x{}}}{\Keyword}
is true if \x{} is a keyword, otherwise false.
When checked with T-IsA,
the object of the left subexpression \object{} = {\path{\classpe{}}{\x{}}}
(which starts with the {\classpe{}} path element)
and the type of the right subexpression \t{} = {\Value{\Keyword}} (a singleton class type)
together trigger the first \isacompareliteral{} case
\isacompare{\s{}}{\path{\classpe{}}{\x{}}}{\Value{\Keyword}}{\filterset{\isprop{\Keyword}{\x{}}}{\notprop{\Keyword}{\x{}}}},
giving propositions that correspond to our informal description {\filterset{\thenprop{\prop{}}}{\elseprop{\prop{}}}} = {\filterset{\isprop{\Keyword}{\x{}}}{\notprop{\Keyword}{\x{}}}}.

The second \isacompareliteral{} case captures the simple equality mode for non-class singleton types.
For example,
the expression
\isaapp{\x{}}{\makekw{en}} produces true 
when \x{} evaluates to {\makekw{en}}, otherwise it produces false.
Using T-IsA,
it has the propositions {\filterset{\thenprop{\prop{}}}{\elseprop{\prop{}}}} = 
\isacompare{}{\x{}}{\Value{\makekw{en}}}{\filterset {\isprop {\Value{\makekw{en}}}{\x{}}}{\notprop{\Value{\makekw{en}}}{\x{}}}}
since \object{} = {\x{}} and \t{} = {\Value{\makekw{en}}}.
%
The side condition on the second \isacompareliteral{} case ensures we are in equality mode---if \x{} can possibly be a class in 
\isaapp{\x{}}{\Object{}}, \isacompareliteral{} uses its conservative default case,
since if \x{} is a class literal, subclassing mode could be triggered.
%
Capture-avoiding substitution of objects {\replacefor {} {\object{}} {\x{}}} used in this case erases propositions
that would otherwise have \emptyobject{} substituted in for their objects---it
is defined in the appendix.

The operational behavior of \isaliteral{} is given by B-IsA (\figref{main:figure:mmsyntax}). \isaopsemliteral{} explicitly handles classes in the second case.

%The definition of \isacompareliteral{} (figure~\ref{main:figure:mmsyntax}) is deliberately conservative.
%The first line considers the case where the object of the left argument
%is a non-empty path ending in \classpe{} and the type of the right argument is a singleton class.

%\constantsemfigure{main}

\subsection{Multimethods}

\begin{figure}
  \footnotesize
$$
\begin{altgrammar}
  \e{} &::=& \ldots \alt {\createmultiexp {\t{}} {\e{}}} \alt
             {\extendmultiexp {\e{}} {\e{}} {\e{}}}
             \alt {\isaapp {\e{}} {\e{}}} &\mbox{Expressions} \\
  \v{} &::=& \ldots \alt {\multi {\v{}} {\disptable{}}}
                &\mbox{Values} \\
 \disptablesyntax{} \\
  \s{}, \t{} &::=& \ldots \alt {\MultiFntype{\t{}}{\t{}}}
                &\mbox{Types}
\end{altgrammar}
$$
  \begin{mathpar}
    \TDefMulti{}

    \TDefMethod{}

    \TIsA{}
  \end{mathpar}
  \begin{mathpar}
    \isapropsfigure{}
  \end{mathpar}
  \begin{mathpar}
    \Multisubtyping{}

    \BDefMulti{}
  \end{mathpar}
  \begin{mathpar}
    \BDefMethod{}
    %\BBetaMulti{}
  \end{mathpar}
  \getmethodfigure{}
\begin{mathpar}
  {\BIsA{}}
  {\isaopsemfigure{}}
  \\
\BBetaMulti{}
\end{mathpar}
\caption{Multimethod Syntax, Typing and Operational Semantics}
\label{main:figure:mmsyntax}
\end{figure}


\figref{main:figure:mmsyntax} presents \emph{immutable} multimethods without default methods to ease presentation.
%Syntax and semantics are given in \figref{main:figure:mmsyntax}. 
%Multimethods can error if no matching method is chosen (rules in the supplemental material).
\figref{main:figure:mmexample} translates the mutable \egref{example:hi-multimethod} to \lambdatc{}.
%\begin{minted}{clojure}
%(ann hi [Kw -> Str])
%(defmulti hi identity)
%(defmethod hi :en [_] "hello")
%(defmethod hi :fr [_] "bonjour")
%(hi :en) ;=> "hello"
%\end{minted}

\begin{figure}
$\letexp{hi_0} {\createmultiexp {\ArrowOne{\x{}}{\Keyword}{\String}{\filterset{\topprop{}}{\topprop{}}}{\emptyobject{}}} {\abs{\x{}}{\Keyword}{\x{}}}}
  {\\\text{\quad}
    \letexp{hi_1} {\extendmultiexp {hi_0} {\makekw{en}} {\abs {\x{}} {\Keyword} {\makestr{hello}}}}
      {\\\text{\quad\quad}
        \letexp{hi_2} {\extendmultiexp {hi_1} {\makekw{fr}} {\abs {\x{}} {\Keyword} {\makestr{bonjour}}}}
        {\\\text{\quad\quad\quad
          \appexp{hi_2}{\makekw{en}}}}}}
$
\caption{Multimethod example}
\label{main:figure:mmexample}
\end{figure}
%
%For convenience, examples in this section are flattened when they are really nested
%let bindings. We also elide trivial latent propositions and objects.
%The following is an abbreviation of the previous expression.
%\begin{lstlisting}
%${\createmultiexp {\ArrowTwo{\x{}}{\Keyword}{\String}} {\abs{\x{}}{\Keyword}{\x{}}}}$
%${\extendmultiexp {hi} {\makekw{en}} {\abs {\x{}} {\Keyword} {\makestr{hello}}}}$
%${\extendmultiexp {hi} {\makekw{fr}} {\abs {\x{}} {\Keyword} {\makestr{bonjour}}}}$
%$\appexp{hi}{\makekw{en}}$
%\end{lstlisting}
%
%\defmethodliteral{} returns a new extended multimethod
%without changing the original multimethod. 
%
%\begin{minted}{clojure}
%(let [hi (defmulti [Kw -> Str] identity)]
%  (let [hi (defmethod hi :en [_] "hello")]
%    (let [hi (defmethod hi :fr [_] "bonjour")]
%      (hi :en))) ;=> "hello"
%\end{minted}
%
%\paragraph{How to check}
To check 
{\createmultiexp {\ArrowTwo{\x{}}{\Keyword}{\String}} {\abs{\x{}}{\Keyword}{\x{}}}},
%
we note
{\createmultiexp {\s{}} {\e{}}} creates a multimethod with \emph{interface type} \s{}, and dispatch function \e{}
of type \sp{},
producing a value of type
{\MultiFntype {\s{}} {\sp{}}}. % with interface type {\s{}} and dispatch function type {\sp{}}.
The T-DefMulti typing rule checks the dispatch function, and
verifies both the interface and dispatch type's domain agree.
%$$
%    \TDefMulti{}
%$$
Our example checks with \t{} = \Keyword, interface type \s{} = {\ArrowTwo{\x{}}{\Keyword}{\String}},
dispatch function type \sp{} = {\ArrowOne{\x{}}{\Keyword}{\Keyword}{\filterset{\topprop{}}{\topprop{}}}{\x{}}}, and overall type
$
{\MultiFntype {\ArrowTwo{\x{}}{\Keyword}{\String}}
              {\ArrowOne{\x{}}{\Keyword}{\Keyword}{\filterset{\topprop{}}{\topprop{}}}{\x{}}}}
$.

Next, we show how to check
$
{\extendmultiexp {hi_0} {\makekw{en}} {\abs {\x{}} {\Keyword} {\makestr{hello}}}}
$.
%
The expression 
{\extendmultiexp {\e{m}} {\e{v}} {\e{f}}} creates a new multimethod
that extends multimethod \e{m}'s dispatch table, mapping dispatch value
\e{v} to method \e{f}. The T-DefMulti typing rule
checks \e{m} is a multimethod with dispatch function type \t{d},
then calculates the extra information we know based on the current
dispatch value {\thenprop{\proppp{}}}, which is assumed when checking the method
body.
%$$
%    \TDefMethod{}
%$$
Our example checks with \e{m} being of type
$
{\MultiFntype {\ArrowTwo{\x{}}{\Keyword}{\String}}
              {\ArrowOne{\x{}}{\Keyword}{\Keyword}{\filterset{\topprop{}}{\topprop{}}}{\x{}}}}
$
with \objectp{} = {\x{}} (from below the arrow on the right argument of the previous type) and \t{v} = \Value{\makekw{en}}. 
Then {\thenprop{\proppp{}}} = 
{\isprop {\Value{\makekw{en}}}{\x{}}}
from
$
\isacompare{}{\x{}}{\Value{\makekw{en}}}
{\filterset {\isprop {\Value{\makekw{en}}}{\x{}}}{\notprop{\Value{\makekw{en}}}{\x{}}}}
$
(see \secref{sec:isaformal}).
Since \t{} = \Keyword{}, we check the method body with
$
\judgement{{\isprop{\Keyword}{\x{}}},{\isprop {\Value{\makekw{en}}}{\x{}}}}
  {\makestr{hello}}
  {\String}{\filterset{\topprop{}}{\topprop{}}}{\emptyobject{}}
$.
Finally from the interface type \t{m}, we know \thenprop{\prop{}} = \elseprop{\prop{}} = \topprop{},
and \object{} = \emptyobject{}, which also agrees with the method body, above.
Notice the overall type of a \defmethodliteral{} is the same as its first subexpression \e{m}.

It is worth noting the lack of special typing rules for overlapping methods---each
method is checked independently based on local type information.

%The expression {\createmultiexp {\s{}} {\e{}}} 
%defines a multimethod
%with interface type \s{} and dispatch function \e{}.
%The expression {\extendmultiexp {\e{m}} {\e{v}}{\e{f}}}
%extends multimethod \e{m} and to map
%dispatch value {\e{v}} to {\e{f}} in an extended dispatch table.
%The value {\multi {\v{}} {\disptable{}}} is the runtime value of a multimethod
%with dispatch function {\v{}} and dispatch table {\disptable{}}.
%
%The T-DefMulti rule ensures that the type of the dispatch function
%has at least as permissive a parameter type
%as the interface type.
%
%For example, we can check the definition from our translation above of \egref{example:rep}
%using T-DefMulti.
%$$
%\judgement{}%{\propenv{}}
%{\createmultiexp 
%      {\s{}}
%      {\classconst{}}}
%  {\MultiFntype {\s{}}{\sp{}}}{\filterset{\topprop{}}{\botprop{}}}{\emptyobject{}}
%$$
%where \s{}  = {\ArrowOne {\x{}} {\Top{}} {\t{}} {\filterset {\topprop{}} {\topprop{}}} {\emptyobject{}}}
%  and \sp{} = {\ArrowOne {\x{}} {\Top{}} {\Union{\Nil}{\Class}} {\filterset {\topprop{}} {\topprop{}}} {\path{\classpe{}}{\x{}}}}.
%  Since the parameter types agree, this is well-typed.
%
%The T-DefMethod rule requires a syntactic lambda expression as the method.
%This way we can manually check the body of the lambda under an extended
%environment as sketched in \egref{example:incmap}.
%We use \isacompareliteral{} to compute the proposition for this method,
%since \isaliteral{} is used at runtime in multimethod dispatch.
%
%We continue with the next line of the translation of \egref{example:rep}.
%From the previous line we have \propenv{} = {\isprop{\MultiFntype {\s{}}{\sp{}}}{path}},
%so
%$$
%\judgement{\propenv{}}
%  {\extendmultiexp {prop} {\String}
%                   {\abs {\x{}} {\Top{}} {\x{}}}}
%  {\MultiFntype {\s{}}{\sp{}}}{\filterset{\topprop{}}{\botprop{}}}{\emptyobject{}}
%$$
%We know \emph{prop} is a multimethod by \propenv{}, so now we check the body
%of this method.
%$$
%\judgement{\propenv{},{\isprop{\Top}{\x{}}},{\isprop{\String}{\x{}}}}
%  {\x{}}
%  {\String}{\filterset{\topprop{}}{\botprop{}}}{\emptyobject{}}
%$$
%%This is checked by T-Local since {\inpropenv{\propenv{},{\isprop{\Top}{\x{}}},{\isprop{\String}{\x{}}}}{\isprop{\String}{\x{}}}}.
%The new proposition {\isprop{\String}{\x{}}} is derived by 
%$$
%  \isacompare{\Top{}}{\path{\classpe{}}{\x{}}}{\Value{\File{}}}
%             {\filterset{\isprop{\String}{\x{}}}
%                        {\notprop{\String}{\x{}}}}.
%$$
%%
%The body of the \clj{let} is checked by T-App because
%{\MultiFntype {\s{}}{\sp{}}} is a subtype of its interface type {\s{}}.

\paragraph{Subtyping}
Multimethods are functions, via S-PMultiFn,
%$$
%\SPMultiFn{}
%$$
which says a multimethod can be upcast to its interface type. 
Multimethod call sites are then handled by T-App via T-Subsume. Other rules are given
in \figref{main:figure:mmsyntax}. 

\paragraph{Semantics}
Multimethod definition semantics are also given 
in \figref{main:figure:mmsyntax}. 
B-DefMulti creates a multimethod with the given dispatch function and an empty dispatch table.
B-DefMethod produces a new multimethod with an extended dispatch table.

The overall dispatch mechanism is summarised by B-BetaMulti.
First the dispatch function \v{d} is applied to the argument \vp{} to obtain
the dispatch value \v{e}.
Based on \v{e},
the \getmethodliteral{} metafunction (\figref{main:figure:mmsyntax})
extracts a method \v{f} from the method table {\disptable{}}
and applies it to the original argument for the final result.

\subsection{Precise Types for Heterogeneous maps}
\label{sec:hmapformal}

\begin{figure}
  \footnotesize
  $$
  \begin{altgrammar}
    \e{} &::=& \ldots \alt \hmapexpressionsyntax{}
    &\mbox{Expressions} \\
    \v{} &::=& \ldots \alt {\emptymap{}}
    &\mbox{Values} \\
    \t{} &::=& \ldots \alt {\HMapgeneric {\mandatory{}} {\absent{}}}
    &\mbox{Types} \\
    \auxhmapsyntax{}\\
%    \pesyntax{}   &::=& \ldots \alt {\keype{\k{}}}
%                  &\mbox{Path Elements}
  \end{altgrammar}
  $$
  \begin{mathpar}
    {\TAssoc}

    {\TGetHMap}

    {\TGetAbsent}

    {\TGetHMapPartialDefault}
    \ \ \ 
  {\SHMapMono}
  \end{mathpar}
  \begin{mathpar}
  {\SHMapP}\ \ 
  {\SHMap}

  \end{mathpar}
  \begin{mathpar}
    {\BAssoc}\ \ 
    {\BGet}\ \ 
    {\BGetMissing}
  \end{mathpar}
  \caption{HMap Syntax, Typing and Operational Semantics}
  \label{main:figure:hmapsyntax}
\end{figure}


\begin{figure}
  \footnotesize
$$
\begin{array}{lr}
  \begin{array}{llll}
    \restrictfigure{}
  \end{array}
  \ \ 
  \begin{array}{llll}
    \removefigure{}
  \end{array}
\end{array}
$$
\caption{Restrict and remove}
\label{main:figure:restrictremove}
\end{figure}

\figref{main:figure:hmapsyntax}
presents
heterogeneous map types.
The type \HMapgeneric{\mandatory{}}{\absent{}}
contains {\mandatory{}}, a map of \emph{present} entries (mapping keywords to types),
\absent{}, a set of keyword keys that are known to be \emph{absent}
and
tag \completenessmeta{} which is either {\complete{}} (``complete'') if the map is fully specified by \mandatory{},
and {\partial{}} (``partial'') if there are \emph{unknown} entries.
%
The partially specified map of
\clj{lunch} in \egref{example:lunchpartial}
is written
\HMapp{\mandatoryset{\mandatoryentrynoarrow{\Valkw{en}}{\String}, {\mandatoryentrynoarrow{\Valkw{fr}}{\String}}}}{\emptyabsent{}}
(abbreviated \Lunch).
%
The type of the fully specified map
\clj{breakfast} in \egref{example:breakfastcomplete} elides the absent entries,
written
\HMapc{\mandatoryset{\mandatoryentrynoarrow{\Valkw{en}}{\String}, {\mandatoryentrynoarrow{\Valkw{fr}}{\String}}}}
(abbreviated \Breakfast).
To ease presentation, 
if an HMap has completeness tag \complete{} then \absent{} is elided and implicitly contains all keywords not in the domain of 
\mandatory{}---dissociating keys is not modelled, so the set of absent entries otherwise
never grows.
Keys cannot be both present and absent.
%\HMapcwithabsent{\mandatory{}}{\absent{}} is abbreviated to \HMapc{\mandatory{}}. 

The metavariable \mapval{}
ranges over the runtime value of maps {\curlymapvaloverright{\k{}}{\v{}}},
usually written {\curlymapvaloverrightnoarrow{\k{}}{\v{}}}.
We %do not model keywords as functions,
only provide syntax for the empty map literal,
however when convenient we abbreviate non-empty map literals
to be a series of \assocliteral{} operations on the empty map.
We restrict lookup and extension to keyword keys. 

\paragraph{How to check}
A mandatory lookup is checked by T-GetHMap.
$$
\abs{\makelocal{b}}{\Breakfast}{\getexp{\makelocal{b}}{\makekw{en}}}
$$
The result type is \String, and the return object is \path{\keype{\makekw{en}}}{\makelocal{b}}.
The object {\replacefor {\path {\keype{k}} {\x{}}} {\object{}} {\x{}}}
is a symbolic representation for a keyword lookup of $k$ in \object{}.
The substitution for {\x{}} handles the case where \object{} is empty.
\begin{mathpar}
\begin{array}{rcl}
{\replacefor {\path {\keype{k}} {\x{}}} {\y{}} {\x{}}} &=& {\path {\keype{k}} {\y{}}} \\
\end{array}
\ \ \ \ \ \ \ 
\begin{array}{rcl}
{\replacefor {\path {\keype{k}} {\x{}}} {\emptyobject{}} {\x{}}} &=& \emptyobject{}
\end{array}
\end{mathpar}

An absent lookup is checked by T-GetHMapAbsent.
$$
\abs{\makelocal{b}}{\Breakfast}{\getexp{\makelocal{b}}{\makekw{bocce}}}
$$
The result type is \Nil---since \Breakfast is fully specified---with return object \path{\keype{\makekw{bocce}}}{\makelocal{b}}.

A lookup that is not present or absent is checked by
T-GetHMapPartialDefault.
$$
\abs{\makelocal{u}}{\Lunch}{\getexp{\makelocal{u}}{\makekw{bocce}}}
$$
The result type is \Top---since {\Lunch} has an unknown \makekw{bocce} entry---with return object \path{\keype{\makekw{bocce}}}{\makelocal{u}}.
Notice propositions are erased once they enter a HMap type.

For presentational reasons, lookups on unions of HMaps are only supported in T-GetHMap
and each element of the union must contain the relevant key.
$$
\abs{\makelocal{u}}{\Unionsplice{\Breakfast \Lunch}}{\getexp{\makelocal{u}}{\makekw{en}}}
$$
The result type is \String, and the return object is \path{\keype{\makekw{en}}}{\makelocal{u}}.
However, lookups of \makekw{bocce} on {\Unionsplice{\Breakfast \Lunch}} maps are unsupported.
This restriction still allows us to check many of the examples in \secref{sec:overview}---in
particular we can check 
\egref{example:desserts-on-meal}, as \makekw{Meal} is in common with both HMaps,
but cannot check \egref{example:desserts-on-class}
because a \makekw{combo} meal lacks a \makekw{desserts} entry.
Adding a rule to handle \egref{example:desserts-on-class} is otherwise straightforward.

Extending a map with T-AssocHMap preserves its completeness.
$$
\abs{\makelocal{b}}{\Breakfast}{\assocexp{\makelocal{b}}{\makekw{au}}{\makestr{beans}}}
$$
The result type is
$
\HMapc{\mandatoryset{\mandatoryentrynoarrow{\Valkw{en}}{\String}, {\mandatoryentrynoarrow{\Valkw{fr}}{\String}}
        ,{\mandatoryentrynoarrow{\Valkw{au}}{\String}}}}
$,
a complete map.
T-AssocHMap also enforces ${\k{}} \not\in {\absent{}}$ to prevent badly formed types.

%for cases like \egref{example:desserts-on-meal}
%where every element in the union
%contains the key we are looking up.

\paragraph{Subtyping}
Subtyping for HMaps
designate \MapLiteral{} as a common supertype for all HMaps.
S-HMap says that HMaps are subtypes if they agree
on \completenessmeta{}, agree on mandatory entries with subtyping
and at least cover the absent keys of the supertype.
Complete maps are subtypes of partial maps
as long as they agree on the mandatory entries of the partial map via subtyping (S-HMapP).

%The typing rules for \getliteral{} consider three possible cases. T-GetHMap models a lookup
%that will certainly succeed, T-GetHMapAbsent a lookup that will certainly fail
%and T-GetHMapPartialDefault a lookup with unknown results.

%The objects in the T-Get rules are more complicated than those in T-Local---the 
%next section discusses this in detail.
%Finally T-AssocHMap extends an HMap with a mandatory entry while preserving completeness
%and absent entries, and enforcing ${\k{}} \not\in {\absent{}}$ to prevent badly
%formed types.

The semantics for \getliteral{} and \assocliteral{} are straightforward.
%If the entry is missing, B-GetMissing produces \nil{}.

\begin{figure}[t]
  $$
\begin{array}{llll}
\updatefigure{}
\end{array}
$$
\caption{Type update (the metavariable \propisnotmeta{} ranges over \t{} and \nottype{\t{}} (without variables), 
  \notsubtypein{}{\Nil{}}{\nottype{\t{}}} when \issubtypein{}{\Nil{}}{\t{}}, see
\figref{main:figure:restrictremove} for \restrictliteral{} and \removeliteral{}.
  )}
\label{main:figure:update}
\end{figure}

%\begin{figure}
%  $$
%\begin{array}{llll}
%  \restrictremovefigure{}
%\end{array}
%  $$
%  \caption{Restrict and Remove}
%  \label{main:figure:restrictremove}
%\end{figure}

\subsection{Proof system}
\label{formalmodel:proofsystem}

The occurrence typing proof system uses standard propositional logic,
except for where nested information is combined. This is
handled by L-Update:
{  \footnotesize
  $$
\LUpdate{}
$$
}

It says
under \propenv{}, if object \path{\pathelemp{}}{\x{}} is of type \t{}, and 
an extension
\path{\pathelem{}}{\path{\pathelemp{}}{\x{}}}
is of possibly-negative type \propisnotmeta{}, then
{\update{\t{}}{\propisnotmeta{}}{\pathelem{}}}
is \path{\pathelemp{}}{\x{}}'s type under \propenv{}.

Recall \egref{example:desserts-on-meal}.
%, resuming from
%\secref{sec:coretypesystem}. 
Solving
$
{ \inpropenv 
  {{\isprop{\Order}{\makelocal{o}}},
    {\isprop{\Value{\makekw{combo}}}{\path{\keype{\makekw{Meal}}}{\makelocal{o}}}}}
  {\isprop {\t{}} {\makelocal{o}}}}
$
uses L-Update, where \pathelem{} = {\emptypath{}} and \pathelemp{} = [{\keype{\makekw{Meal}}}].
$$
\inpropenv{\propenv{}}{\isprop{\update{\Order}{\Value{\makekw{combo}}}{[{\keype{\makekw{Meal}}}]}}{\makelocal{o}}}
$$
Since {\Order} is a union of HMaps, we structurally recur on the first case of \updateliteral{}
(\figref{main:figure:update}),
which preserves \pathelem{}.
Each initial recursion hits the first HMap case, since there is some \t{} such that
{\inmandatory{\k{}}{\t{}}{\mandatory{}}} and 
\completenessmeta{} accepts partial maps \partial{}.

To demonstrate,
\makekw{lunch} meals are handled by the first HMap case and
update to {\HMapp {\extendmandatoryset {\mandatory{}}{\Valkw{Meal}}{\sp{}}} {\emptyabsent{}}}
where \sp{} = {\update{\Valkw{lunch}}{\Valkw{combo}}{\emptypath{}}}
and \mandatory{} = \mandatoryset{\mandatoryentry{\Valkw{Meal}}{\Valkw{lunch}},{\mandatoryentry{\Valkw{desserts}}{\Number{}}}}.
\sp{} updates to \Bot via the penultimate \updateliteral{} case,
because \restrict{\Value{\makekw{lunch}}}{\Value{\makekw{combo}}} = \Bot
by the first \restrictliteral{} case.
The same happens to \makekw{dinner} meals,
leaving just the \makekw{combo} HMap. 

In \egref{example:desserts-on-class},
$
\inpropenv{\propenv{}}{\isprop{\update{\Order}{\Long}{[{\classpe{}}, {\keype{\makekw{desserts}}}]}}{\makelocal{o}}}
$
updates the argument in the {\Long} method.
This recurs twice for each meal to handle the {\classpe{}}
path element.

We describe the other \updateliteral{} cases.
The first \classpe{} case updates
to \class{} if \classconst{} returns \Value{\class{}}.
The second \keype{\k{}} case detects contradictions in absent
keys. % not overlapping with \Nil{}.
The third \keype{\k{}} case updates unknown entries to be mapped to \t{} or absent.
The fourth \keype{\k{}} case updates unknown entries to be \emph{present}
when they do not overlap with \Nil{}.

%$
%{\update{\Number}{\Long}{[{\classpe{}}]}}}
%$
%
%$
%{\update{\Int}{\Long}{[{\classpe{}}]}}}
%$
%
%


\section{Metatheory}
\label{sec:metatheory}

We prove type soundness following Tobin-Hochstadt and Felleisen~\cite{TF10}.  Our model is extended
to include errors \errorvalv{} and a \wrong{} value, and we prove well-typed
programs do not go wrong; this is therefore a stronger theorem than
proved by Tobin-Hochstadt and Felleisen~\cite{TF10}. 
Errors behave like Java exceptions---they can be thrown and propagate ``upwards'' in the evaluation rules
(\errorvalv{} rules are deferred to the appendix).

Rather than modeling Java's dynamic semantics, a task of daunting
complexity, we instead make our assumptions about Java explicit. We
concede that method and constructor calls may diverge or error, but
assume they can never go wrong
(other assumptions given in the supplemental material).

{\javanewassumption{main}}

%For readability we define logical truth in Clojure.

%{\istruefalsedefinitions{main}}

For the purposes of our soundness proof, we require that all values
are \emph{consistent}.  Consistency (defined in the supplemental
material) states that the types of closures are well-scoped---they do
not claim propositions about variables hidden in their closures.

%{\consistentwithonlydef{main}}

We can now state our main lemma and soundness theorem.  The
metavariable \definedreduction{} ranges over \v{}, \errorvalv{} and
\wrong{}. Proofs are deferred to the supplemental material. %\ref{appendix:lemma:soundness}.

\begin{lemma}\label{main:lemma:soundness}

  {\soundnesslemmahypothesis}
\end{lemma}



{\soundnesstheoremnoproof{main}}

%{\wrongtheoremnoproof{main}}
%
%{\nilinvoketheoremnoproof{main}}


\subsection{Typed Clojure Evaluation}
\label{sec:casestudy}

\begin{figure*}[t]

\begin{tabular}{lll}
      \toprule


  & feeds2imap & CircleCI \\
  \midrule
  Total number of typed namespaces & 11 (825 LOC) & 87 (19,000 LOC) \\
  Total number of \clj{def} expressions & 93  & 1834 \\
  \tabitem
  checked & 52 (56\%) & 407 (22\%) \\
  \tabitem
  unchecked & 41 (44\%) & 1427 (78\%) \\
  Total number of Java interactions & 32 & 105 \\
  \tabitem
  static methods & 5 (16\%) & 26 (25\%) \\ 
  \tabitem
  instance methods & 20 (62\%) & 36 (34\%) \\
  \tabitem
  constructors & 6 (19\%) & 38 (36\%) \\
  \tabitem
  static fields & 1 (3\%) & 5 (5\%) \\
  Methods overriden to return non-nil & 0 & 35 \\
  Methods overriden to accept nil arguments & 0 & 1 \\
  Total HMap lookups & 27  & 328  \\
  \tabitem
  resolved to mandatory key & 20 (74\%) & 208 (64\%) \\
  \tabitem
  resolved to optional key & 6 (22\%) & 70 (21\%) \\
  \tabitem
  resolved of absent key & 0 (0\%) & 20 (6\%) \\
  \tabitem
  unresolved key & 1 (4\%) & 30 (9\%) \\
  Total number of \clj{defalias} expressions & 18  & 95 \\
  \tabitem
  contained HMap or union of HMap type & 7 (39\%)  & 62 (65\%) \\
  Total number of checked \clj{defmulti} expressions & 0  & 11 \\
  Total number of checked \clj{defmethod} expressions & 0  & 89 \\


\end{tabular}
\caption{Typed Clojure Features used in Practice}
\label{experience:featuretable}
\end{figure*}

Throughout this thesis, we will focus on three interrelated type
system features: heterogeneous maps, Java interoperability, and
multimethods. Our hypothesis is that these features are widely used in
existing Clojure programs in interconnecting ways, and that handling
them as we have done is required to type check realistic Clojure
programs.

To evaluate this hypothesis, we will present an analysis of two existing \coretyped{}
code bases, one from the open-source community, and one from a company
that uses \coretyped{} in production (\figref{experience:featuretable}).
%For our data gathering, we
%instrumented the \coretyped{} type checker to record how often
%various features were used (summarized in \figref{experience:featuretable}).

\subsection{Automatic Annotation Approach}

We now describe our philosophy and overall approach to automatic annotations for Typed Clojure.
At a high level, there are three phases to
generating annotations: instrumentation, runtime tracking, and type reconstruction.

The first phase, \textbf{instrumentation}, involves
rewriting the code we wish to annotate such
that we can record its runtime behavior.
In this phase, we usually require the programmer to
indicate which code we wish to generate types
for in advance, with a file-level granuality.

Once instrumented, we observe our running program
via \textbf{runtime tracking}. To exercise our programs,
we usually run their unit tests, generative tests,
or just normally run the program (eg. to generate types for
a game, we can simply play the game for a few minutes).
We accumulate the results of tracking via \textbf{paths}.
If we think of types as trees and supply a label
for each branching path, our inference results
specify the type down a particular path in this tree.

Finally, the information collected during runtime tracking
is combined into annotations by our \textbf{inference algorithm}.
We first combine all inference result into a large tree of
types. If we were to convert this tree into annotations directly,
our annotations would be too specific---they would be too
deep and fine-grained.
Instead, our algorithm iterates over several passes to massage
this tree, generating good names for the nodes, compacting similar
types across the tree, and
eventually converting the tree into a directed graph by reconstructing
recursive types.

%Let's demonstrate this pipeline, with our opening example.
%
%...
%
%\begin{verbatim}
%
%\end{verbatim}

An important question to answer is ``how accurate are these annotations?''.
Unlike some previous work in this area~\cite{An10dynamicinference}, we do not aim for soundness guarantees
in our generated types.
Our main contribution is a tool that Clojure programmers
can use to help learn about and specify their programs.
In that spirit, annotations should meet several criteria.

\paragraph{Good names}
Typed Clojure %and clojure.spec 
annotations are abundant
with useful names for types. A good name often increases
readability.
Good names can sometimes be reconstructed from the program source,
like function or parameter names, and other times 
we can use the shape of a type to summarize it.

\paragraph{Compact}
Idiomatic Clojure code rarely mixes certain types in the same position,
unless the program is polymorphic. Using this knowledge---which we observed
by the annotations and specs assigned to idiomatic Clojure 
code---we can rule out certain combinations of types to compact our
resulting output, without losing information that would help us
type check our programs.

\paragraph{Recursive}
Maps in Clojure are often heterogeneous, and recursively defined.
Typed Clojure %and clojure.spec 
supplies mechanisms for the most
common case: maps of known keyword entries.
We strategically \textbf{squash} flat types to be recursive
based on their unrolled shape.
For example, a recursively defined union of maps almost always
contains a known keyword ``tag'' mapped to a keyword.
By identifying this tag, we can reconstruct a good recursive
approximation of this type.

%\subsection{Naming}
%
%For a type to be immediately useful to a programmer, it helps
%to have a great name. We explored several avenues for
%generating good names.
%
%For types that occured as function arguments, the name of
%the argument often indicated its role in the program.
%Names like \textbf{config} or \textbf{env} are often used
%for an environment being functionally threaded through
%the program.
%
%Similarly, types that occur as values in configuration
%maps often have descriptive keys. In the Star Trek 
%game, the configuration map maps \textbf{:stardate}
%to a map that contains three number entries: 
%\textbf{:start},
%\textbf{:current}, and
%\textbf{:end}.
%
%What if a type occurs in the return position of a function?
%Sometimes these are named by \textbf{let} binding the result
%of the computation.
%
%Failing these heuristics, we fall back of several approaches
%to naming.
%First, if we are naming a keyword map which is part of a tagged
%union, we use the tag as the name. For example, if the tagged entry
%maps \textbf{:op} to \textbf{:fn}, we name this map \textbf{FnOp}.
%Failing that, for maps with less than three entries, we simply
%enumerate its entries as the name.
%Finally, for large keyword maps, we give an abbreviation
%of its keyset as a name.
%
%\begin{Verbatim}
%(defn f [x] (inc x))
%\end{Verbatim}

\subsection{Automatic Annotations Formalism}

We provide a preliminary grammar for our formal treatment of automatic
annotations, and a semantics for the instrumentation phase
of our automatic annotator.
The full thesis will describe the type reconstruction algorithm in
detail.

\figref{infer:grammar} presents the grammar. Similar to Typed Clojure's
formalism, we use paths $\inferpath{}$ to represent a path through
a value. Path elements consist of the domain of a function $\dompe{}$,
the range of a function $\rngpe{}$, and the result of a map lookup 
of key ${\kw2{}}$ on a map with key set ${\overrightarrow{\kw1{}}}$
$\inferkeype{\overrightarrow{\kw1{}}}{\kw2{}}$.
Inference results $\res{}$ are collected during execution, and
associate a path $\inferpath{}$ with the type of value observed
at that path $\tau$.

\figref{infer:semantics} gives a semantics for the instrumentation
phase of the automatic annotation tool.
Most rules are standard, with B-Get and B-Assoc responsible for
looking up and associating new entries onto maps.
The B-Track rule is the entry point for tracking values,
calling $\trackmeta{}$ (\figref{infer:track}), that
rewrites a value $\val{}$ to $\val{}'$ and generates inference
results $\res{}'$ based on the input that $\inferpath{}$
is the path of $\val{}$.

Several extensions to this model are possible.
First, we will add \emph{space-efficient} runtime instrumentation,
This ensures that stack space usage is efficient by collapsing
directly stacked wrappers of the same kind. For example, wrapping
a function twice takes the same stack space as wrapping once.
The propagation of path information is then extended to support
\emph{multiple} simultaneous paths for each value, which enables
the optimization to not lose any tracking information.

Second, we can infer \emph{polymorphic} types by tracking
the pointer identity of objects. This information
can be combined with the base tracking information to generate
polymorphic type annotations that also feature concrete types
when appropriate.

\begin{figure*}
  $$
  \begin{altgrammar}
    \val{} &::=& \num
       \alt {\kw{}}
       \alt [\lambda \xvar{}. \exp{}, \rho]
       \alt \{\overrightarrow{\kw{}\ \val{}}\}
       &\mbox{Values} \\
   \exp{} &::=& \xvar{}
       \alt \val
       \alt \trackE{\exp}{\pi{}}
       \alt \lambda \xvar{}. \exp{}
       \alt \{\overrightarrow{\exp\ \exp}\}
       \alt (\exp{}\ \exp{})
       &\mbox{Expressions} \\
    \rho &::=& \{\overrightarrow{x \mapsto \val}\}
       &\mbox{Runtime environments} \\
   \pth{}
      &::=& x 
       \alt \dompe{}
       \alt \rngpe{}
       \alt \inferkeype{\overrightarrow{\kw{}}}{\kw{}}
       &\mbox{Path Elements} \\
     \pi &::=& \overrightarrow{\pth{}}
       &\mbox{Paths} \\
       \res{}
      &::=& \{\overrightarrow{\path{} : \tau}\}
      &\mbox{Inference results} \\
    \ty{}, \sigma
      &::=& \IntT{}
       \alt [\tau \rightarrow \tau]
       %\alt [\overrightarrow{\tau} \tau * \rightarrow \tau]
       \alt \{\overrightarrow{\kw{}\ \tau}\}
       \\
       &\alt& \HMaptwo{\{\overrightarrow{\kw{}\ \tau}\}}{\{\overrightarrow{\kw{}\ \tau}\}}
       \alt \tau \cup \tau
       \alt \alias{} % type alias
       \alt \UnknownT{}
      &\mbox{Types} \\
    \tenv{} &::=& \{\overrightarrow{\xvar{} : \tau}\}
      &\mbox{Type environments} \\
    \aenv{} &::=& \{\overrightarrow{\alias{} \mapsto \tau}\}
      &\mbox{Type alias environments} \\
    \atenv{} &::=& (\aenv{}, \tenv{})
      &\mbox{Combined environments} \\
  \end{altgrammar}
  $$
\end{figure*}

\begin{figure*}
\begin{mathpar}
%\infer [B-Var]
%{ track(\rho(x), [x]) = v\ ; res }
%{ \rho \vdash x \Downarrow v\ ; res}

% track is inserted manually
\infer [B-Var]
{}
{\bigstep{\rho}{\xvar{}}{\rho(\xvar{})}{\{\}}}

\infer [B-Track]
{ \bigstep{\rho}{\exp{}}{\val{}}{\res{}} \\
  \trackmeta{}(\val{}, \inferpath{}) = \val{}'\ ; \res{}' }
{ \bigstep{\rho}{\trackE{\exp{}}{\inferpath{}}}{\val{}'}{\res{} \sqcup \res{}'} }

\infer [B-App]
{ \bigstep{\rho}{\exp1}{[\lambda \xvar{}. \exp{}, \rho']}{\res1} \\\\
  \bigstep{\rho}{\exp2}{\val{}}{\res2} \\\\
  \bigstep{\rho', \xvar{} \mapsto \val{}}{\exp{}}{\val{}'}{\res3} \\
}
{ \bigstep{\rho}{(\exp1\ \exp2)}{\val{}'}{\res1 \sqcup \res2 \sqcup \res3}}

\infer [B-Clos]
{}
{ \bigstep{\rho}{\lambda \xvar{}. \exp{}}{[\lambda \overrightarrow{\xvar{}} . \exp{}, \rho ]}{\{\}}}

\infer [B-Val]
{}
{ \bigstep{\rho}{\val{}}{\val{}}{\{\}} }

%\infer [B-Map]
%{ \overrightarrow{\rho \vdash e_1 \Downarrow v_1\ ; res_1 \ \ 
%  \rho \vdash e_2 \Downarrow v_2\ ; res_2} \\
%}
%{ \rho \vdash \{\overrightarrow{e_1\ e_2}\} \Downarrow \{\overrightarrow{v_1\ v_2}\}\ ; \overrightarrow{res_1} \sqcup \overrightarrow{res_2}}

\infer [B-Get]
{ \bigstep{\rho}{\exp1}{\{ \overrightarrow{\kw{}\ \val{}} \}}{\res1} \\
  \bigstep{\rho}{\exp2}{\kw1}{\res2} \\
}
{ \bigstep{\rho}{(get\ \exp1\ \exp2)}{\{ \overrightarrow{\kw{}\ \val{}} \}[\kw1]}{\res1 \sqcup \res2}
}

\infer [B-Assoc]
{ 
  \bigstep{\rho}{\exp1}{\{ \overrightarrow{\kw{}\ \val{}}\}}{\res1} \\
  \bigstep{\rho}{\exp2}{\kw1}{\res2} \\
  \bigstep{\rho}{\exp3}{\val{}}{\res3} \\
}
{ \bigstep{\rho}{\assocE{\exp1}{\exp2}{\exp3}}{\{ \overrightarrow{\kw{}\ \val{}} \}[\kw1 \mapsto \val{}]}{\res1 \sqcup \res2 \sqcup \res3}
}
\end{mathpar}
  \label{infer:semantics}
  \caption{Runtime instrumentation semantics for the automatic annotation tool}
\end{figure*}

\begin{figure*}
\begin{mathpar}

  \begin{array}{lllll}
    \trackmeta{}(\val{}, \inferpath{}) = \val{}\ ;\ \res{}\\\\

    \trackmeta{}(\num{}, \inferpath{})
    &=&
    n\ ; \{\inferpath{} : \IntT{}\}
    \\
    \trackmeta{}([\lambda \xvar{}. \exp{}, \rho], \inferpath{})
    &=&
    [
    \lambda \yvar{}.
      \trackE{((\lambda \xvar{}. \exp{}) \trackE{\yvar{}}{\appendone{\inferpath{}}{\dompe{}}})}
             {\appendone{\inferpath{}}{\rngpe{}}}
         , \rho]
         \ ; \{\inferpath{} : [\UnknownT{} \rightarrow \UnknownT{}] \}
         \\
    &&
    \text{where}\ \yvar{} \text{ is fresh}
    \\
    \trackmeta{}(\{\overrightarrow{\val1\ \val2}\}, \inferpath{})
    &=&
    \{\overrightarrow{\val1\ \val2{}'}\}
    \ ;\ \overrightarrow{\sqcup\ \res{}}
      \sqcup
    \{\inferpath{} : \{\overrightarrow{\val1\ \UnknownT{}}\} \}
    \\
    &&
    \text{where}\ \overrightarrow{\trackmeta{}(\val2, \appendone{\inferpath{}}{\inferkeype{\overrightarrow{\val1}}{\val1}}) = \val2{}'\ ;\ \res{}}
    \\
    \\

         
    \inferupdatenoalign{\aenv{}}{\tenv{}}{\inferpath{}}{\tau}{\atenv{}}
    \\\\

    \inferupdatealign{\aenv{}}{\tenv{}}{\appendone{\inferpath{}}{\inferkeype{\overrightarrow{\kw{}'}}{\kw{}}}}{\ty{}}
            {\inferupdate{\aenv{}}{\tenv{}}{\inferpath{}}{\{\overrightarrow{\kw{}' : \UnknownT{}},\ \kw{} : \ty{} \}}}
    \\
    \inferupdatealign{\aenv{}}{\tenv{}}{\appendone{\inferpath{}}{\dompe{}}}{\ty{}}
                {\inferupdate{\aenv{}}{\tenv{}}{\inferpath{}}{\arrow{\ty{}}{\UnknownT{}}}}
    \\
    \inferupdatealign{\aenv{}}{\tenv{}}{\appendone{\inferpath{}}{\rngpe{}}}{\ty{}}
                {\inferupdate{\aenv{}}{\tenv{}}{\inferpath{}}{\arrow{\UnknownT{}}{\ty{}}}}
    \\
    \inferupdatealign{\aenv{}}{\tenv{}}{[x]}{\ty{}}{(\tenv{}[\sigma / \xvar{}] , \aenv{}')}
    \\
    && \text{where}\ \tenv{}(\xvar{}) = \ty{}', join(\aenv{}, \ty{}, \ty{}') = \sigma\ ;\ \aenv{}'
    \\
    \inferupdatealign{\aenv{}}{\tenv{}}{[\xvar{}]}{\ty{}}{(\tenv{}[\ty{} / \xvar{}], \aenv{})}
    \\
    && \text{where}\ \xvar{} \not\in dom(\tenv{})
    \\

    \\
    \joinnoalign{\aenv{}}{\tau}{\tau}{\tau}
    \\ \\

    \joinalign{\aenv{}}{(\cup\ \overrightarrow{\tau_1})}{\tau}{(\cup\ \overrightarrow{\sigma})}\\
                                                    && \text{where }
                                                    \overrightarrow{\joinnoalign{\aenv{}}{\tau_1}{\tau}{\sigma}}
    \\
    \joinalign{\aenv{}}{\tau}{(\cup\ \overrightarrow{\tau_1})}{(\cup\ \overrightarrow{\sigma})} \\
                                                    && \text{where }
                                                    \overrightarrow{\joinnoalign{\aenv{}}{\tau_1}{\tau}{\sigma}}
    \\
    \joinalign{\aenv{}}{\UnknownT{}}{\tau}{\tau}
    \\
    \joinalign{\aenv{}}{\tau}{\UnknownT{}}{\tau}
    \\
    \joinalign{\aenv{}}{[\overrightarrow{\tau_1} \rightarrow \sigma_1]}{[\overrightarrow{\tau_2} \rightarrow \sigma_2]}{[\overrightarrow{\joinexpression{\aenv{}}{\tau_1}{\tau_2}}
    \rightarrow
    \joinexpression{\aenv{}}{\sigma_1}{\sigma_2}]}
    \\
    % TODO add condition on should-join-HMap?
    %join(A, (HMap\  m_1\ o_1), (HMap\  m_2\ o_2))
    %&=& joinHMap(A, (HMap\  m_1\ o_1), (HMap\  m_2\ o_2))
    %\\
    \joinalign{\aenv{}}{\tau}{\sigma}{(\cup\ \tau\ \sigma)}

  \end{array}

\end{mathpar}
\end{figure*}


\subsection{Typed Clojure Automatic Annotations Evaluation}

Along with a manual inspection of the generated Typed Clojure annotations,
we will perform several other experiments to measure the quality of the
generated annotations.

For example, we measure the number of changes needed
needed to amend generated annotations to actually type check.
\figref{infer:amending} shows some preliminary results.

\begin{figure*}
\begin{tabular}{| l | l | l | l | l | l |}
  Library            & Lines of types  & Local annotations & Manual Line +/- Diff \\
  \hline
  startrek-clojure   & 133             & 3                 & +70 -41 \\
  math.combinatorics & 395             & 147               & +124 -120\\
  fs                 & 157             & 1                 & +119 -86\\
  data.json          & 168             & 9                 & +94 -125 \\
  mini.occ           & 49              & 1                 & +46 -26\\
  %data.xml           & \\
  %clojurescript & \\
\end{tabular}
\caption{Amending automatically generated types to type check}
  \label{infer:amending}
\end{figure*}

\subsection{Two Areas of Further Study}

The following two subsections describe
several possible research areas for the final section of this
dissertation.

At least one of the following investigations will be included in the dissertation.
The selection process will be informed by the success of early prototypes
and preliminary investigations.

\subsubsection{Option 1: Type check more Clojure programs}

A complaint from industrial users about Typed Clojure is that support for
core Clojure functionality and idioms is limited.
Furthermore, even if
a set of functionality is supported, combining them in a helper function
often give undesired results.

CircleCI summarized their experience with Typed Clojure~\cite{CircleCIBlog}, citing
several specific frustrations. One representative issue we will discuss
is that Typed Clojure cannot reliably infer usages of polymorphic higher-order functions.
The root of this problem is known as the \emph{Hard to synthesize arguments} problem~\cite{hosoya1999good}, and
has two competing forces: local type argument synthesis for polymorphic applications and
the bidirectional propagation of anonymous function argument types.

For example, consider mapping a collection over the identity function \texttt{(map (fn [x] x) [1 2 3])}.
To infer \texttt{map}'s type variables, we first type check the arguments.
Unfortunately, \texttt{(fn [x] x)} has no annotation, and so its type is \texttt{[Any -> Any]}, making
the entire invocation type \texttt{(Seq Any)} (instead of the more desirable \texttt{(Seq Int)}).

Even \texttt{(map identity [1 2 3])} has its own set of problems: both
\texttt{map} and \texttt{identity} are polymorphic, and thus cannot be inferred accurately by 
many systems based on Pierce and Turner's Local Type Inference~\cite{PierceLTI} (like Typed Clojure and Typed Racket).
This specific problem is addressed by later work in the realm of set-theoretic types~\cite{polyduce2},
by generating substitutions for the arguments of applications. However, even
there, it cannot simultaneously infer the type of the function argument of
\texttt{(map (fn [x] x) [1 2 3])}.

Anonymous functions are common in Clojure. For example, the ClojureScript compiler
uses around 40 anonymous functions in around 140 top-level definitions.
Most of these functions would require extra annotations to type check, and
over half of them are used as arguments to polymorphic higher-order functions.

To address excessive annotations and broaden the number of checkable programs,
we will design and implement an approach to mitigate the issue of \emph{Hard
to synthesize arguments} in common Clojure code, and report its the effectiveness
in reducing the number of changes in the porting process from Clojure to
Typed Clojure.

Some initial investigation in this direction has centered around building
an extensible system for specifying type rules, and enhancing symbolic
execution capabilities. For example
%
\texttt{(update m :a (fn [a] (inc a)))}
%
increments the \texttt{:a}
entry of map \texttt{m}. A custom \texttt{update} rule
effectively inlines the call to \texttt{update} as \texttt{(assoc m :a ((fn [a] (inc a)) (get m :a)))},
which allows parameter \texttt{a} to inherit the type of \texttt{(get m :a)}.

A more complicated example uses Clojure transducers. The expression 
\texttt{(comp (map (fn [a] (dec a))) (map (fn [a] (inc a))))} is a transducer that first
decrements, then increments (transducers compose left-to-right).
Since \texttt{comp}, and \texttt{map} are polymorphic higher-order
functions, and we use anonymous functions, this is hard to type check.
However, by combining custom type rules for \texttt{comp}, and \texttt{map}
with symbolic computations and adequate type propagation, we could potentially distil the original expression to
\texttt{(inc (dec input))}, which is much easier to check.

\subsubsection{Option 2: clojure.spec Automatic Annotations}

Clojure.spec is popular runtime verification system for Clojure programs included
in the core Clojure distribution.
%We will present a formal model of clojure.spec and 
We will repurpose our automatic annotation
tool to generate clojure.spec runtime specifications.

We will then conduct a larger scale investigation to evaluate the quality 
of these annotations by generating clojure.spec annotations for several hundred
projects we have sourced from the open source community.
This investigation will also serve to further analyze the assumptions made 
in designing Typed Clojure and clojure.spec.
For example, Typed Clojure and clojure.spec are designed differently
around qualified entries in maps, and by generating and enforcing specs
we can investigate whether either are compatible with actual usage.

We can empirically investigate which function spec checking semantics are 
applicable in the majority of code. Since clojure.spec provides
two distinct function checking semantics, including a surprising
``generative testing'' semantics, we can measure the likelihood
of tests still passing after generating each kind of function spec.

Other interesting questions can be asked by using clojure.spec's generative
testing features. After generating our own specs automatically for each project,
we can test to see if our annotation algorithm feeds sufficient information
to clojure.spec's value generators to create effective generative tests.
We can also use the generative tests as a ``second pass'' over the functions,
by re-instrumenting our functions and analyzing whether we get better
test coverage than just running the provided unit tests.

clojure.spec supports stubbing-out functions, useful for running unit
tests that depend on side effecting functions. For example, we might
stub-out functions that call a database that only exists in production.
Using our generated specs, we could experimentally stub-out functions
that consistently fail their specs under generative testing,
and measure how useful the chosen stubs are by manually inspecting
which functions were stubbed.
Similarly, we could experiment in stubbing out function arguments
to higher-order functions to make instrumentation with \texttt{fspec}s
more predictable.

%\subsubsection{Option 3: How Clojure is used}

There is potential for other, more general, research questions
to be answered about Clojure usage with our testing setup.
Part of the unfinished work towards this thesis will be devising research questions,
but, for now, we can imagine a spectrum of questions
from simple measurements of the frequency of particular features like variable-arguments,
keyword maps, or higher-order functions, to more ambitious questions
like detecting breaking changes between project versions
and quantifying the test coverage of unit tests versus generative tests.

\section{Related Work}

% Cite a few of the early papers here.
%http://www.cs.washington.edu/research/projects/cecil/www/pubs/
\paragraph{Typed Multimethods} 
Millstein and collaborators present a sequence of
systems~\cite{Chambers:1992:OMC,Chambers:1994:TMM,MS02} with statically-typed multimethods
and modular type checking.  In contrast to Typed Clojure, in these
system methods declare the types of arguments that they expect which
corresponds to exclusively using \clj{class} as the dispatch function
in Typed Clojure. However, Typed Clojure does not attempt to rule out
failed dispatches.

% one sentence
% TC based on TR, already covered

\paragraph{Occurrence Typing} 
Occurrence typing~\cite{TF08,TF10} extends the type 
system with a \emph{proposition environment} that represents 
the information on the types of bindings down conditional branches.
These propositions are then used to update the types associated
with bindings in the \emph{type environment} down branches
so binding occurrences are given different types 
depending on the branches they appear in, and the conditionals
that lead to that branch.

% What's diff about TC from the related work
% small summary for diesel....
% - diesel supports x
%  - calculus supports some subset of x
% we support y, which covers most of x but also foo

% eg. multiple dispatch
%     nominal vs structural

% eg. run abritrary metaprogramming over dispatch in CLOS
%  more expressive

% type systems for mm or rows
% rows vs HMap
% - no poly in HMap
% - based on subtyping
% - rows based on polymorphism

\paragraph{Record Types} Row polymorphism~\cite{Wand89typeinference,CM91,HP91}, used
in systems such as the OCaml object system, provides many of the
features of HMap types, but defined using universally-quantified row
variables. HMaps in Typed Clojure are instead designed to be used with
subtyping, but nonetheless provide similar expressiveness, including
the ability to require presence and absence of certain keys. 

Dependent JavaScript~\cite{Chugh:2012:DTJ} can track similar
invariants as HMaps with types for JS objects. They must deal with
mutable objects, they feature refinement types and strong updates to
the heap to track changes to objects.

TeJaS~\cite{TeJaS}, another type system for JavaScript,
also supports similar HMaps, with the ability to
record the presence and absence of entries, but lacks a compositional
flow-checking approach like occurrence typing.

Typed Lua~\cite{Maidl:2014:TLO} has \emph{table types} which track
entries in a mutable Lua table.  Typed Lua changes the dynamic
semantics of Lua to accommodate mutability: Typed Lua raises a runtime
error for lookups on missing keys---HMaps consider lookups on missing
keys normal.

\paragraph{Java Interoperability in Statically Typed Languages}
Scala~\cite{OCD+} has nullable references for compatibility with Java.
Programmers must manually check for
\java{null} as in Java to avoid null-pointer exceptions. 


\paragraph{Other optional and gradual type systems}
%In addition to Typed Racket, 
Several other gradual type
systems have been developed for existing
dynamically-typed languages.  Reticulated Python~\cite{Vitousek14} is
an experimental gradually typed system for Python, implemented as a
source-to-source translation that inserts dynamic checks at language
boundaries and supporting Python's first-class object system. 
Clojure's nominal classes avoids the need to support
first-class object system in Typed Clojure, however HMaps offer an alternative to
the structural objects offered by Reticulated. Similarly,
Gradualtalk~\cite{gradualtalk} offers gradual typing for Smalltalk,
with nominal classes.

Optional types
%, requiring less implementation effort and avoiding
%runtime cost, 
have been  adopted in industry, including Hack~\cite{hack}, and Flow~\cite{flow} and
TypeScript~\cite{typescript}, two extensions of JavaScript. These
systems  support  limited forms of occurrence typing,
and do not include the other features we
present.

\paragraph{Automatic annotations}
There are two common implementation strategies for such tools. The first
strategy, ``ruling-out'' (for invariant detection), assumes all invariants are true 
and then use runtime analysis results to rule out
impossible invariants. The second ``building-up'' strategy (for dynamic type inference)
assumes nothing and then uses runtime analysis results to build up invariant/type knowledge.

Examples of invariant detection tools include Daikon \cite{Ernst06thedaikon},
DIDUCE \cite{hangal2002tracking}, and Carrot \cite{pytlik2003automated}, and
typically enhance statically typed languages with more expressive types or contracts.
Examples of dynamic type inference include Rubydust \cite{An10dynamicinference},
JSTrace \cite{saftoiu2010jstrace}, and TypeDevil \cite{pradel2015typedevil},
and typically target untyped languages.

Both strategies have different space behavior with respect to representing
the set of known invariants.
The ruling-out strategy typically uses a lot of memory at the beginning,
but then can free memory as it rules out invariants. For example, if
\texttt{odd(x)} and \texttt{even(x)} are assumed, observing \texttt{x = 1}
means we can delete and free the memory recording \texttt{even(x)}.
Alternatively, the building-up strategy uses the least memory storing
known invariants/types at the beginning, but increases memory usage
as more the more samples are collected. For example, if we know
\texttt{x : Bottom}, and we observe \texttt{x = "a"} and \texttt{x = 1}
at different points in the program, we must use more memory to
store the union \texttt{x : String $\cup$ Integer} in our set of known invariants.

\paragraph{Daikon}
Daikon can reason about very expressive relationships between variables
using properties like ordering ($x < y$), linear relationships ($y = ax + b$),
and containment ($x \in y$). It also supports reasoning with ``derived variables''
like fields ($x.f$), and array accesses ($a[i]$).

Typed Clojure's dynamic inference can record heterogeneous data structures
like vectors and hash-maps, but otherwise cannot express relationships
between variables.

There are several reasons for this. The most prominent is that Daikon
primarily targets Java-like languages, so inferring simple type information
would be redundant with the explicit typing disciplines of these languages.
On the other hand, the process of moving from Clojure to Typed Clojure
mostly involves writing simple type signatures without dependencies
between variables. Typed Clojure recovers relevant dependent information
via occurrence typing, and gives the option to manually annotate necessary
dependencies in function signatures when needed.


% Inference and Evolution of TypeScript Declaration Files
% - they submit pull requests from their tool's output
% https://cs.au.dk/~amoeller/papers/tstools/paper.pdf
\paragraph{TypeScript Annotation Generation}
Kristensen and M{\o}ller~\cite{kristensen2017inference}
present TSInfer and TSEvolve that generate TypeScript annotation
files using static analysis of JavaScript code. They
submitted corrections back to libraries they found descrepancies in,
which were accepted with little to no changes in the tool's output.

NoRegrets~\cite{noregrets2018} uses dynamic analysis to learn how a program
is used, and automatically runs the tests of downstream projects to
improve test coverage. Its concept of representing a program sample as
a path paired with a type is very similar to Typed Clojure's approach.

\paragraph{How dynamic languages are used}
Several languages have seen similar investigations
into their idioms as I am proposing for Clojure.

A popular motivation is to discover which type system features to support
when retrofitting a type system.
% FIXME the is \AAkerblom but there's an error.. also in the bibliography
Akerblom et. al~\cite{Akerblom:2014:TDF:2597073.2597103} trace dynamic features in Python programs
via instrumentation. They measured the prevalence of dynamic features in startup versus
user code, and recorded usage frequencies for a set of dynamic features.
They concluded dynamism is prevalent in Python, and thus should be supported
in a retrofitted type system for Python.
A study along similar lines is also applicable to Clojure, in particular analysing Typed
Clojure's support for Clojure's dynamic features.

Calla{\'u} et al. \cite{Callau2013} also conducted a large-scale study of
dynamic Smalltalk idioms to inform future language extensions tooling support.
Notably, they further perform a qualitative analysis aiming to identify
the reasons why Smalltalk use these features in the first place, and
whether they can be replaced with more predictable features. They also 
measure which kinds of projects (e.g., testing frameworks, user-level libraries, or core system libraries) 
use particular features more frequently.
Due to the their prevalence in the open-source Clojure ecosystem,
Typed Clojure has mainly been tested on user-level libraries.
We could predict Typed Clojure's applicability to other kinds of projects
by gathering similar data on how frequently different types of Clojure libraries use
Clojure's various features.

Andreasen et. al~\cite{Andreasen2016TraceTA} developed
\emph{trace typing} to explore the design space of JavaScript type systems. 
Using runtime observations, they studied which control flow techniques
are used most often in JavaScript programs, and thus, which should
be supported by an effective type system for JavaScript.
Typed Clojure implements occurrence typing to reason about control
flow in Clojure which seems to work well in practice, but a similar
quantitative analysis could reveal further insights.

%Runtime analysis \cite{Mastrangelo:2015:UYO:2814270.2814313}

%\cite{Mastrangelo:2015:UYO:2814270.2814313} 

\paragraph{Interleaving Type Checking with Expansion, Extensible type systems and Symbolic Analysis}

Turnstile~\cite{Chang2017TSM} type checks a program during expansion
by repurposing the Racket macro system. It provides a fully extensible framework
for specifying and combining core typing rules.
On the other hand, Typed Clojure does not have the goal of allowing users to override
how language primitives type check. Instead, our goal is to provide
a simple interface to write type rules for library functions and macros
in a style that hides the necessary bookkeeping surrounding occurrence
typing and scope management.

SugarJ~\cite{Erdweg2011SJ}
adds syntactic language extensibility to languages like Java, such as pair
syntax, embedded XML, and closures.
Desugarings are expressed as rewrite rules to plain Java.
Similarly, work on \emph{type-specific languages}~\cite{omar2014safely}
adds extensible systems for the definitions of specialized syntax literals
to existing languages.
The \emph{type} of an expression determines how it is parsed and elaborated.

% this paper has a great related works section that differentiates
% the strategies of several typed metaprogramming techniques
SoundX~\cite{Lorenzen2016STS} presents a solution to a common
dilemma in typed metaprogramming: whether to desugar before
type checking, or vice-versa.
The authors present a system where a form is type checked before 
being desugared, with a guarantee that only well-typed code is generated.
Programmers specify desugarings with a combination of typing and rewriting rules, 
which are then connected to form a valid type derivation
in a process called \emph{forwarding}.
We will explore whether we can get the same effect in Typed Clojure
without requiring the user to understand typing rules.
%For example, Scala macros~\cite{Burmako2013SML} interleave type checking and
%desugaring

Ziggurat~\cite{Fisher06staticanalysis} allows programmers to define
the static and dynamic semantics of macros separately. To demonstrate its
broad applicability, they choose Scheme-like macros that generate assembly code
for the dynamic semantics.
They advocate building towers of static analyses, so
macros can be statically checked in terms the static semantics of other macros, instead
of just their assembly code expansions which would otherwise be too difficult to check.
This idea resembles our prototypes in defining custom typing rules for functions and macros in Typed Clojure,
where the dynamic semantics are defined by runtime Clojure constructs (\texttt{defn}
and \texttt{defmacro}), and towers of static semantics are progressively specified in terms of the static
analysis of other Clojure forms.

Mix~\cite{Khoo2010MTC} cleanly separates symbolic execution~\cite{King1976SEP} from type checking
in the same system, specifying a mode for (nested) regions of code.
They argue this tradeoff keeps the predictability of type checking, while preserving enough
symbolic execution to drive further checking.
In Typed Clojure, symbolic execution is managed by occurrence typing~\cite{TF10}.
Our preliminary explorations in symbolic execution for Typed Clojure, for example, type checks an
anonymous function if annotated, otherwise treats it symbolically.
As the authors envision, this is akin to automatically inserting
the mode of a code region based on its context, with a Mix-like language
becoming the intermediate language.

Type Tailoring~\cite{greenmanttailoring} is an approach to provide more information
to a host type system than it might be capable of by itself.
In particular, the authors use the host platform's metaprogramming functionality
to refine the types of calls based on the program syntax alone, as well as improve
error messages by incorporating surface syntax. Their experiments are based in Typed Racket, that fully expands
syntax before checking it. Since Typed Clojure recently changed to interleave macroexpansion
and type checking, we could extend this technique to also refine calls based on the
types of their arguments (like SoundX).


Other work is relevant to our investigations of improving the user experience
of Typed Clojure. SweetT~\cite{pombrio2018inferring} automatically infers type rules
for syntactic sugar. Helium~\cite{Heeren2003STI} provides hooks into the type inference
process for domain-specific type error messages.

\section{Research Plan and Timeline}

I have already made progress towards my thesis:

\begin{itemize}
  \item I have formalized Typed Clojure, including
    its characteristic features like hash-maps, multimethods, and Java interoperability,
    and prove the model type sound.
  \item I have conducted an empirical study of real-world Typed Clojure usage
    in over 19,000 lines of code, showing its features correspond to actual usage patterns.
  \item I have implemented and publicly released a tool that generates
    Typed Clojure and clojure.spec annotations, and started on a formalism.
  \item I have started an empirical study the of manual changes needed for the generated annotations
    to pass type checking is in progress.
  \item I have successfully run my clojure.spec annotation tool on several hundred open-source projects that
    will be used to drive further studies.
  \item I have prototyped an extensible typing rule system and symbolic execution for Typed Clojure.
\end{itemize}

To complete my thesis, I plan to follow this timeline:

\begin{itemize}
  \item \textbf{[June-July 2018 - Completed]} Fix spec generation.
    \begin{itemize}
      \item devise and implement a one or more strategies to handle clojure.spec's
        heterogeneous map spec (and intelligently register global spec aliases)
      \item test out specs generation on candidate projects and improve the tool to fix obvious
        defects
    \end{itemize}
  \item \textbf{[July-August 2018 - Completed]} Proof-of-concept extensible typing rule system
    \begin{itemize}
      \item convert core.typed to control macroexpansion, rather than type-checking
        fully-expanded code.
      \item prototype interface for defining custom type rules for usages of top-level functions and macros.
      %\item complete proof-of-concept symbolic rewriting rules to aggressively move functions into
      %  invocation target position, like lifting conditionals and beta reduction, with
      %  na\"\i ve infinite loop detection.
    \end{itemize}
  \item \textbf{[August 2018]} Finish formal model of automatic annotation tool.
    \begin{itemize}
      \item update formal model for annotation tool to include latest
        optimizations and fixes in recursive type reconstruction.
    \end{itemize}
  \item \textbf{[September-October 2018]} Devise and carry out automatic annotation experiments
    \begin{itemize}
      %\item formulate and complete larger scale study on generation of specs for hundreds of projects.
      \item complete study that quantifies the changes needed to go from automatically annotated types
        to checked code.
    \end{itemize}
  \item \textbf{[October-November 2018]} Write and submit automatic annotation paper (PLDI submission)
  \item \textbf{[November-December 2019]} Improve extensible typing rule system
    \begin{itemize}
      %\item provide tools to avoid double macroexpansion.
      \item support more core Clojure idioms using typing rules.
      \item revisit study quantifying manual type annotations using enhanced local inference
    \end{itemize}
  \item \textbf{[January-May 2019]} Write dissertation
  \item \textbf{[June 2019]} Defend
\end{itemize}

\subsection{Publications}

I plan to publish the following papers:

\begin{itemize}
  \item \textbf{ESOP 2016 - Published} \emph{Practical Optional Types for Clojure}~\cite{bonnaire2016practical}.
    A paper that provides a formal model of Typed Clojure and presents an empirical
    study of usage patterns.
  \item \textbf{Fall 2018} \emph{Squash the work: Automatic annotations for Typed Clojure and clojure.spec}
    A paper that presents a tool capable of generating Typed Clojure and clojure.spec
    annotations based on runtime observations of the program.
    We present a formal model of the tool, as well as several manual experiments in how
    accurate our annotations are.
    We conduct a larger study by sourcing several hundred projects and automatically
    generating types and specs, then using these annotations to answer various questions
    about how clojure.spec and Clojure is used.
\end{itemize}


\printbibliography

\end{document}
