The moment I touched down in Bloomington from Perth, Western Australia
to start graduate school at Indiana University,
I knew it would be my home for at least the next 5 years.
That is, until my confused cab driver informed me with a gasp
(just as I was saying ``Indiana Un\ldots'') that I had in fact arrived one
state over from my intended destination, in Bloomington, \emph{Illinois}.
The last such case was an 80-year-old woman 6 months prior (this happens pretty often apparently)
and she took a cab ride over to fix it, and so, with the airport closed,
I burnt all that extra cash my parents insisted I carry in my first
6 hours in America (thanks mum).
I will be forever grateful for that cab driver---I couldn't imagine a more
humbling way to start my graduate career.

Once I arrived on campus, from my first days around IU Computer Scientists
I felt immediately welcome.
Reconnecting with Dan Friedman and Will Byrd in Bloomington after they suggested
I attend graduate school there three years earlier at Clojure/conj 2011 was surreal.
Will has been a consistent friend and inspiration, and his commitement
to miniKanren is a model for my work on Typed Clojure.
Jason Hemann, Cameron \& Rebecca Swords, Jaime Guerrero, 
and Tori Lewis all embraced me and help me find my feet. Thank you
for your friendship over the years.

After our student cohort returned from a memorable POPL'15
in Mumbai, India, we brought back a taste for set-theoretic
types. Conversations seemed to periodically end with ``unfortunately, it's written in French.''
Andrew M.~Kent and I spent many afternoons deciphering 
terse equations from Giuseppe Castagna papers and
Zhiwu Xu's awe-inspiring dissertation.
While I couldn't quite manage to apply their ideas to Typed Clojure,
Andrew persisted, and pulled them apart piece-by-piece, patiently
reverse-engineering the last decade of insights in
set-theoretic types until he could explain it
in his own style and, finally, apply it as a new theoretical foundation
to Typed Racket. I remember how daunting such a task felt, and
I have been privileged to work closely with such a dedicated
researcher.
Throughout graduate school at IU, Andrew and I have done most things together,
from skipping class to float down a river in Eugene,~Oregon,
to finishing our dissertations in the same week.
Andrew and his family Carrie, Charlotte, Harrison, Sydney, and Dwight
quickly acquainted this Aussie with many American traditions like Thanksgiving,
(white) Christmas, Super Bowl, and generally good warm company. I don't know how I would
have done it without the Kents.

I am grateful for everyone who has helped me along my journey
with Typed Clojure, especially the hundreds of
contributors to my crowdfunding campaigns, members of the
Clojure community, and those that invited or attended my talks.

To all those that tried Typed Clojure and reported on their experience:
I'm sorry. And thank you.
Of them, the developers at CircleCI shaped this dissertation in two ways.
Having access to their large Typed Clojure
codebase was an invaluable opportunity as a researcher, and enabled the
discussion of Typed Clojure ``in practice.''
Later, I was informed of CircleCI's decision to stop using Typed Clojure
in a thoughtful email from their engineering team.
I enthusiastically requested a public post-mortem (generously left
to my discretion), and so
Marc O'Morain---then an active Typed Clojure user---blogged \emph{Why we're no longer using Core.typed}
on behalf of his team.
It blew up, making frontpage news on tech sites for days.
Many of my Clojure buddies reached out with messages of solidarity; I particularly remember 
this time one from Chas Emerick who's always looked out for me, but Brandon Bloom, David Nolen,
Ghadi Shayban, Nicola Mometto, Devin Walters, Reid D McKenzie, Nathan Sorenson, Eric
Normand, Jim Duey, Kyle Kingsbury, Colin Fleming, and Paula Gearon (to name a few!)
have pulled me out of similar holes more than once.
While I was stunned at this sequence of events, it felt right.
The post summarized my own frustrations with Typed Clojure: Clojure
deserves an amazing type system, and we're just not there yet.
Interest in Typed Clojure ground to a halt,
and all that remained was my resolve to address everything in that blog
post.

The second half of graduate school provided the structure and freedom to
realize the foundation of that vision. Having a trusting and encouraging advisor
gave me the confidence to try new directions that felt distinctly my own.
Sam directed my ideas and giddy prototypes into actionable research,
but preserved my bizarre something-or-other ideas (I can tell, because
I find my ideas hilarious and this dissertation is a hoot).
Sam helped me squeeze all the magic out of my ideas---i.e., turn bumbling feelings into facts---and 
taught me how to know \emph{when} you'd
want to do that (it's ok to indulge in uncertainty for a time).

I remember sharing my latest bumblings with Mike Vollmer at the whiteboard in Lindley Hall
on how to summarize nested cons cells by ``rolling them up'' into an equi-recursive type.
We both poked at the problem for a few minutes and agreed there was something cool there,
but it was mostly just a feeling (well, to me at least, Mike might have already solved it).
Mike has been a supporter of mine since before we met, and I look back fondly on our
many conversations, our bonkers two-shy-introverts-navigating-Mumbai experience,
and the many terrible movies we shared.
Weekly IU Gradual Typing meetings
gave me room to elaborate on that initial conversation, manifesting into
viewing types as graphs and ``squashing'' them into recursive types---thanks
Matteo Cimini, Rajan Walia, Spenser Bauman, Jeremy Siek, Mike Vitousek, Deyaaeldeen Almahallawi,
Caner Derici, Sarah Spall, and David Christiansen
for your attention and suggestions.

The idea of mixing symbolic execution with type checking was the result of my refusal
to accept that programs like \clj{(map (fn [x] x) [1 2 3])} required a type annotation
on \clj{x} under local type inference
(``but the type is \emph{right there}!'', I repeated to Sam for years).
Our talk series PL Wonks was always eager to hear my latest work,
thanks to regular attendees and speakers Aaron Hsu, Ryan Scott, Kyle Carter, 
Vikraman Choudhury, Matthew Heimerdinger, Paulette Koronkevich, Praveen Narayanan,
and
Chaitanya Koparkar for your presense and attention.
Once Andre Kuhlenschmidt, one of the most humble and meticulous computer scientists
I know, remarked (in a packed room) ``these rules make complete sense'' 
about my solution during my final PL Wonks talk,
it sparked a bout of confidence that carried
me through the final weeks before my defense.

Thanks to my committee Ryan, Larry, and Ken for your feedback, and for being
lots of fun to present to.
We didn't interact much in the context of my research committee, but outside of that
it was a privilege to take Ryan's compilers course, Larry's logic courses,
and to finally realize that Ken's neverending quest to ask the right questions
is a superpower.

Even though I'd like to think I'd never let Andrew graduate before me,
it was my wife Marcela that inspired me to wrap up my graduate career.
While our other plans changed for various reasons, we decided to stick to
my ambitious plan---it has been a tremendous grind and it's hard to believe
it's over.  Her belief in my ability helped cement my resolve, and her demands
to be referred to as Mrs.~Typed Clojure with a matching cap never
fails to make me blush.
