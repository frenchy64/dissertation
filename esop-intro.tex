\subsection{Background: Clojure with static typing}

% current situation 

The popularity of dynamically-typed languages in software
development, combined with a recognition that types often improve
programmer productivity, software reliability, and performance, has
led to the recent development of a wide variety of optional and
gradual type systems aimed at checking existing programs written in
existing languages.  These include  TypeScript~\cite{typescript} and Flow~\cite{flow} for
JavaScript, Hack~\cite{hack} for PHP, and mypy~\cite{mypy}
for Python among the optional systems, and Typed Racket~\cite{TF08}, Reticulated
Python~\cite{Vitousek14}, and Gradualtalk~\cite{gradualtalk} among gradually-typed systems.\footnote{We
  use ``gradual typing'' for systems like Typed Racket with sound
  interoperation between typed and untyped code; Typed Clojure or
 TypeScript which don't
  enforce type invariants we describe as ``optionally typed''.}

One key lesson of these systems, indeed a lesson known to early
developers of optional type systems such as Strongtalk, is that type
systems for existing languages must be designed to work with the
features and idioms of the target language. Often this takes the form
of a core language, be it of functions or classes and objects,
together with extensions to handle distinctive language features.


We synthesize these lessons to present \emph{Typed Clojure}, an
optional type system for Clojure. 
%
Clojure is a dynamically
typed language in the Lisp family---built on the Java Virtual
Machine (JVM)---which has recently gained popularity as an alternative
JVM language.  It offers the flexibility of a Lisp dialect, including
macros, emphasizes a functional style via
immutable data structures, and provides
interoperability with existing Java code, allowing programmers to use
existing Java libraries without leaving Clojure.
%
Since its initial release in 2007, Clojure has been widely adopted for
``backend'' development in places where its support for parallelism,
functional programming, and Lisp-influenced abstraction is desired on
the JVM. As a result, there is an extensive base of existing untyped
programs whose developers can benefit from Typed Clojure,
an experience we discuss in this paper.

Since Clojure is a language in the
Lisp family, we apply the lessons of Typed Racket, an existing gradual type
system for Racket, to the core of Typed Clojure, consisting of an extended
$\lambda$-calculus over a variety of base types shared between all Lisp systems.
%
Furthermore, Typed Racket's \emph{occurrence typing} has proved
necessary for type checking realistic Clojure programs.

However, Clojure goes beyond Racket in many ways, requiring several
new type system features which we detail in this paper.
%
Most significantly, Clojure supports, and Clojure developers use,
\textbf{multimethods} to structure their code in extensible
fashion. Furthermore, since Clojure is an untyped language, dispatch
within multimethods is determined by application of dynamic predicates
to argument values. 
%
Fortunately, the dynamic dispatch used by multimethods has surprising
symmetry with the conditional dispatch handled by occurrence
typing. Typed Clojure is therefore able to effectively handle complex
and highly dynamic dispatch as present in existing Clojure programs. 

But multimethods are not the only Clojure feature crucial to type
checking existing programs. As a language built on the Java Virtual
Machine, Clojure provides flexible and transparent access to existing
Java libraries, and \textbf{Clojure/Java interoperation} is found in almost
every significant Clojure code base. Typed Clojure therefore builds in
an understanding of the Java type system and handles interoperation
appropriately. Notably, \texttt{null} is a distinct type in Typed Clojure,
designed to automatically rule out null-pointer exceptions.

An example of these features is given in
\figref{fig:ex1}. Here, the \clj{pname} multimethod dispatches
on the \clj{class} of the argument---for \clj{String}s,
the first method implementation is called, for \clj{File}s, the
second. The \clj{String} method calls
a \clj{File} constructor, returning a non-nil \clj{File} instance---the 
\clj{getName} method 
on \clj{File} requires a non-nil target, returning a nilable
type.  
%Typed Clojure offers an opt-in mode that
%resolves JVM overloading, avoiding expensive runtime reflective calls.

Finally, flexible, high-performance immutable dictionaries
are the most common Clojure data structure.
Simply treating them as uniformly-typed
key-value mappings would be insufficient for existing
programs and programming styles. Instead, Typed Clojure provides a
flexible \textbf{heterogenous map} type, in which specific entries can be specified. 

While these features may seem disparate, they are unified in important
ways. First, they leverage the type system mechanisms
inherited from Typed Racket---multimethods when using 
dispatch via predicates, Java interoperation for handling
\texttt{null} tests, and heterogenous maps using union types and
reasoning about subcomponents of data. Second,
they are crucial features for handling Clojure code in
practice. Typed Clojure's use in real Clojure deployments would not be
possible without effective handling of these three Clojure features. 

\begin{figure*}[t!]
  \normalsize
\begin{lstlisting}
(*typed ann pname [(U File String) -> (U nil String)] typed*)
(defmulti pname class)  ; multimethod dispatching on class of argument
(defmethod pname String [s] (*invoke pname (*interop new File s interop*) invoke*)) ; String case 
(defmethod pname File [f] (*interop .getName f interop*)) ; File case, static null check
(*invoke pname "STAINS/JELLY" invoke*) ;=> "JELLY" :- (U nil Str)
\end{lstlisting}
\caption{A simple Typed Clojure program (delimiters: {\color{interop}Java interoperation (green)}, 
  {\color{types}type annotation (blue)},
  {\color{invoke}function invocation (black)}, {\color{red}collection literal (red)}, {\color{mygray}other (gray)})}
\label{fig:ex1}
\end{figure*}


%\subsection{Contributions}

%Our main contributions are as follows:
%
%\begin{enumerate}
%  \item We motivate and describe  Typed Clojure, an optional
%    type system for Clojure that understands existing Clojure idioms.
%  \item We present a sound formal model for three crucial type
%    system features: multi-methods, Java
%    interoperability, and heterogenous maps.
%  \item We evaluate the use of Typed Clojure features on existing
%    Typed Clojure code, including both open source and in-house systems.
%\end{enumerate}
%
%
%
%%% \begin{figure}
%%% \inputminted[firstline=5]{clojure}{code/demo/src/demo/parent2.clj}
%%% \caption{A simple Typed Clojure program}
%%% \label{fig:ex1}
%%% \end{figure}
%
%%% Figure~\ref{fig:ex1} presents a simple program demonstrating many
%%% aspects of our system, from simple type annotations to explicit
%%% handling of Java's \java{null} (written \clj{nil}) in interoperation, as well as an
%%% extended form of occurrence typing and method resolution of
%%% Java interoperability based on static type information.
%
%%% The \clj{parent} function has the type 
%%% $$
%%% \clj{['{:file (U nil File)} -> (U nil Str)]}
%%% $$
%%% which means that it takes a hash table whose \clj{:file} key maps to either
%%% \clj{nil} or a \clj{File}, and it produces either \clj{nil} or a
%%% \clj{String}. The \clj{parent} function uses the \clj{:file} keyword
%%% as an accessor to get the file, checks that it isn't \clj{nil}, and
%%% then obtains the parent by making a Java method call.
%
%\noindent
% The remainder of this paper begins with an example-driven
% presentation of the main type system features in
% \secref{sec:overview}. We then incrementally present a core calculus
% for Typed Clojure covering all of these features together in
% \secref{sec:formal} and prove type soundness
% (\secref{sec:metatheory}). We then 
% %discuss the full implementation of
% %Typed Clojure, \coretyped{}, which extends the formal model in many ways, 
% present an empirical analysis of significant code bases written
% in \coretyped{}---the full implementation of Typed Clojure---in \secref{sec:experience}. 
% Finally, we discuss related work and conclude.
%
