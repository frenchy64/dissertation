\section{Overview of Typed Clojure}

\label{sec:overview}

We now begin a tour of the central features of Typed Clojure,
beginning with Clojure itself. Our presentation
uses the full Typed Clojure system to illustrate key type system
ideas,\footnote{Full examples: \url{https://github.com/typedclojure/esop16}} before studying the core features in detail in
\secref{sec:formal}.

\subsection{Clojure}

Clojure~\cite{Hic08} is a Lisp that runs on the
Java Virtual Machine with support for concurrent programming
and immutable data structures in a mostly-functional
style.
%, restricting imperative updates to a limited set of
%structures each with specific thread synchronization behaviour. 
%Fast implementations of immutable vectors, and hash tables are featured,
%and a means for defining new records.
%
Clojure provides easy interoperation with existing Java libraries, with Java values being like any other Clojure value. 
However, this smooth interoperability comes at the cost of pervasive \java{null}, which leads to the possibility of null pointer exceptions---a drawback we address in Typed Clojure.

%\paragraph{Clojure Syntax}
%
%We describe new syntax as they appear in each example, but
%begin with include the essential basics of Clojure syntax.
%
%\clj{nil} is exactly Java's \java{null}.
%Parentheses indicate \emph{applications}, brackets
%delimit
%\emph{vectors}, braces
%delimit
%\emph{hash-maps}
%and double quotes delimit \emph{Java strings}.
%\emph{Symbols} begin with an alphabetic character,
%and a colon prefixed symbol like \clj{:a} is a \emph{keyword}.
%
%\emph{Commas} are always \emph{whitespace}.

\subsection{Typed Clojure}

A simple one-argument function \clj{greet} is annotated with \clj{ann} to take and return strings.

\begin{lstlisting}
(*typed ann  greet [Str -> Str] typed*)
(defn greet [n] (*invoke str "Hello, " n "!" invoke*))
(*invoke greet "Grace" invoke*) ;=> "Hello, Grace!" :- Str
\end{lstlisting}
%
Providing \clj{nil} (exactly Java's \java{null})
is a static type error---\clj{nil} is not a string.
%
\begin{lstlisting}
(*invoke greet nil invoke*) ; Type Error: Expected Str, given nil
\end{lstlisting}

\paragraph{Unions} To allow \clj{nil}, we use \emph{ad-hoc unions} (\clj{nil} and \clj{false}
are logically false).
%
\begin{lstlisting}
(*typed ann  greet-nil [(U nil Str) -> Str] typed*)
(defn greet-nil [n] (*invoke str "Hello" (when n (*invoke str ", " n invoke*)) "!" invoke*))
(*invoke greet-nil "Donald" invoke*) ;=> "Hello, Donald!" :- Str 
(*invoke greet-nil nil invoke*)      ;=> "Hello!"         :- Str
\end{lstlisting}
%
%
Typed Clojure prevents well-typed code from dereferencing \clj{nil}.
%This is important for Clojure programs---\clj{nil}
%is treated like any other distinct datum in Clojure.

\paragraph{Flow analysis} Occurrence typing~\cite{TF10}
models type-based control flow.
In \clj{greetings}, a branch ensures \clj{repeat}
is never passed \clj{nil}.
%
\begin{lstlisting}
(*typed ann  greetings [Str (U nil Int) -> Str] typed*)
(defn greetings [n i]
  (*invoke str "Hello, " (when i (*invoke apply str (*invoke repeat i "hello, " invoke*) invoke*)) n "!" invoke*))
(*invoke greetings "Donald" 2 invoke*)  ;=> "Hello, hello, hello, Donald!" :- Str
(*invoke greetings "Grace" nil invoke*) ;=> "Hello, Grace!"                :- Str
\end{lstlisting}
%
Removing the branch is a static type error---\clj{repeat} cannot be passed \clj{nil}.
%
\begin{lstlisting}
(*typed ann  greetings-bad [Str (U nil Int) -> Str] typed*)
(defn greetings-bad [n i]           ; Expected Int, given (U nil Int)
  (*invoke str "Hello, " (*invoke apply str (*invoke repeat i "hello, " invoke*) invoke*) n "!" invoke*))
\end{lstlisting}


%\subsection{Type System Basics}
%
%\cite{TF10}
%presented Typed Racket with occurrence typing,
%a technique for deriving type information from conditional control flow.
%They introduced the concept of occurrence typing 
%with the following example.
%
%\inputminted[firstline=3]{racket}{code/tr/example1.rkt}
%
%This function takes a value that is either \emph{\#f} % mintinline really hates #
%or a number, represented by an \emph{untagged} union type.
%The `then' branch has an implicit invariant
%that \rkt{x} is a number, which is automatically inferred with occurrence typing
%and type checked without further annotations.
%
%We chose to build on the ideas and implementation
%of Typed Racket to implement a type system targeting Clojure for several reasons.
%Initially, the similarities between Racket and Clojure drew us to
%investigate the effectiveness of repurposing occurrence typing
%for a Clojure type system---both languages share a Lisp heritage,
%similar standard functions 
%(for instance \clj{map}
%in both languages is variable-arity)
%and idioms.
%While Typed Racket is gradually typed and has sophisticated
%dynamic semantics for cross-language interaction, we 
%chose to first implement
%the static semantics
%with the hope to extend Typed Clojure to be gradually typed at a future date.
%Finally,
%Typed Racket's combination of bidirectional checking
%and occurrence typing presents a successful model for 
%type checking dynamically typed programs without compromising
%soundness, which is appealing over success typing~\cite{Lindahl:2006:PTI}
%which cannot prove strong properties about programs
%and soft typing~\cite{CF91}
%which has proved too complicated in practice.
%
%Here is the above program in Typed Clojure.
%\begin{exmp}
%\inputminted[firstline=5]{clojure}{code/demo/src/demo/eg1.clj}
%\label{example:conditionalflow}
%\end{exmp}
%
%The \clj{fn} macro (provided by core.typed) supports optional annotations by 
%adding
%\clj{:-} and a type after a parameter
%position
%or binding vector 
%to annotate parameter types
%and return types respectively.
%\clj{number?} is
%a Java \java{instanceof} test of \clj{java.lang.Number}.
%As in Typed Racket, \clj{U} creates an \emph{untagged union} type, which can take
%any number of types.
%
%Typed Clojure can already check all of the examples in~\cite{TF10}---the 
%rest of this section describes the extensions necessary
%to check Clojure code.


\subsection{Java interoperability}
\label{sec:overviewjavainterop}

Clojure can interact with Java constructors, methods, and fields.
This program calls the \clj{getParent} on a constructed
\clj{File}
instance, returning a nullable string.

\begin{exmp}
\begin{lstlisting}
(*interop .getParent (*interop new File "a/b" interop*) interop*)  ;=> "a" :- (U nil Str)
\end{lstlisting}
\label{example:getparent-direct-constructor}
\end{exmp}
%
Typed Clojure can integrate with the Clojure compiler to avoid expensive reflective 
calls like \clj{getParent}, however if a specific overload cannot be found based on the
surrounding static context, a type error is thrown.
%
\begin{lstlisting}
(fn [f] (*interop .getParent f interop*)) ; Type Error: Unresolved interop: getParent
\end{lstlisting}
%
Function arguments default to \clj{Any}, which is similar to a union of all types. Ascribing
a parameter type allows Typed Clojure to find a specific method.

%Calls to Java methods and fields have prefix notation
%like \clj{(.method target args*)} and \clj{(.field target)} respectively,
%with method and field names prefixed with a dot and methods taking some number of arguments.

\begin{exmp}
\begin{lstlisting}
(*typed ann parent [(U nil File) -> (U nil Str)] typed*)
(defn parent [f] (if f (*interop .getParent f interop*) nil))
\end{lstlisting}
\label{example:parent-if}
\end{exmp}

%\begin{exmp}
%\inputminted[firstline=18,lastline=19]{clojure}{code/demo/src/demo/parent3.clj}
%\end{exmp}

The conditional guards from dereferencing \clj{nil}, and---as before---removing 
it is a static type error, as typed code could possibly dereference \clj{nil}.
\begin{lstlisting}
(defn parent-bad-in [f :- (U nil File)]
  (*interop .getParent f interop*)) ; Type Error: Cannot call instance method on nil.
\end{lstlisting}

Typed Clojure rejects programs that assume methods cannot return \clj{nil}.
%
\begin{lstlisting}
(defn parent-bad-out [f :- File] :- Str
  (*interop .getParent f interop*)) ; Type Error: Expected Str, given (U nil Str).
\end{lstlisting}
Method targets can never be \clj{nil}.
Typed Clojure also prevents passing \clj{nil} as Java method or
constructor arguments by default---this restriction can be
adjusted per method.

%
%Typed Clojure and Java treat \java{null} differently.
%In Clojure, where it is known as \clj{nil}, Typed Clojure assigns it an explicit type
%called \clj{nil}. In Java \java{null} is implicitly a member of any reference type.
%This means the Java static type \java{String} is equivalent to
%\clj{(U nil String)} in Typed Clojure.
%
%Reference types in Java are nullable, so to guarantee a method call does not
%leak \java{null} into a Typed Clojure program we
%must assume methods can return \clj{nil}.
%
In contrast, JVM invariants guarantee constructors return non-null.\footnote{\url{http://docs.oracle.com/javase/specs/jls/se7/html/jls-15.html#jls-15.9.4}}
%
\begin{exmp}
\begin{lstlisting}
(*invoke parent (*interop new File s interop*) invoke*)
\end{lstlisting}
\end{exmp}


\subsection{Multimethods}

\label{sec:multioverview}

\emph{Multimethods} are a kind of extensible function---combining a \emph{dispatch function} with 
one or more \emph{methods}---widely used to define Clojure operations.

\paragraph{Value-based dispatch}
This simple multimethod takes a keyword (\clj{Kw}) and says hello in different languages.%, as
%specified by a keyword argument.

\begin{exmp}
\begin{lstlisting}
(*typed ann hi [Kw -> Str] typed*) ; multimethod type
(defmulti hi identity) ; dispatch function `identity`
(defmethod hi :en [_] "hello") ; method for `:en`
(defmethod hi :fr [_] "bonjour") ; method for `:fr`
(defmethod hi :default [_] "um...") ; default method
\end{lstlisting}
\label{example:hi-multimethod}
\end{exmp}

When invoked, the arguments are first supplied to the dispatch function---\clj{identity}---yielding
a \emph{dispatch value}. A method is then chosen
based on the dispatch value, to which the arguments are then passed to return a value.
%
\begin{lstlisting}
(*invoke map hi [*vec :en :fr :bocce vec*] invoke*) ;=> (*list "hello" "bonjour" "um..." list*)
\end{lstlisting}
%
For example, 
\clj{(*invoke hi :en invoke*)} evaluates to \clj{"hello"}---it executes
the \clj{:en} method
because \clj{(*invoke = (*invoke identity :en invoke*) :en invoke*)} is true
and \clj{(*invoke = (*invoke identity :en invoke*) :fr invoke*)} is false.

Dispatching based on literal values enables certain forms of method
definition, but this is only part of the story for multimethod dispatch.

\paragraph{Class-based dispatch}
For class values, multimethods can choose methods based on subclassing
relationships.
%
Recall the multimethod from \figref{fig:ex1}. %, reproduced here.
%\begin{minted}{clj}
%(ann pname [(U File String) -> (U nil String)])
%(defmulti pname class)
%(defmethod pname String [s] (pname (new File s)))
%(defmethod pname File [f] (.getName f))
%\end{minted}
%
%Its dispatching function is
%\clj{class}, with two methods associated with dispatch values \clj{java.lang.String} and \clj{java.io.File}
%respectively.
%\noindent
The dispatch function \clj{class}
%---associated at multimethod creation with \clj{defmulti}---
dictates 
whether the \clj{String} or \clj{File} method is chosen.
%---both installed via \clj{defmethod}
%
The multimethod dispatch rules use
\clj{isa?}, a hybrid predicate which is both a subclassing check for classes and
an equality check for other values.

%(isa? (class "STAINS/JELLY") Object) ;=> true
%(isa? (identity :en) :fr) ;=> false
%(isa? (class (new File "JELLY")) String) ;=> false
\begin{lstlisting}
(*invoke isa? :en :en invoke*)       ;=> true
(*invoke isa? String Object invoke*) ;=> true
\end{lstlisting}
%
The current dispatch value and---in turn---each method's associated dispatch value
is supplied to \clj{isa?}. If exactly one method returns true, it is chosen.
%
For example,
the call
  \clj{(*invoke pname "STAINS/JELLY" invoke*)}
picks the \clj{String} method because \clj{(*invoke isa? String String invoke*)}
is true, and
\clj{(*invoke isa? String File invoke*)}
is not.
%---\clj{(class "STAINS/JELLY")}
%is \clj{String}.
%
%The \clj{String} method body
%\clj{(pname (new File "STAINS/JELLY"))}
%chooses the \clj{File} method for opposite reasons.

%The following Typed Clojure program is semantically identical to figure~\ref{fig:ex1}.
%
%\begin{minted}{clj}
%(ann pname [(U Str File) -> (U nil Str)])
%(defn pname [x]
%  ; dispatch value calculated by applying dispatch
%  ; function `class` to argument `x`.
%  (cond
%    ; if (class x) subclasses String, but not File
%    (and (isa? (class x) String)
%         (not (isa? (class x) File)))
%    ; then choose the String method
%    (pname (new File x))
%
%    ; else if (class x) subclasses File, but not String
%    (and (isa? (class x) File)         
%         (not (isa? (class x) String)))
%    ; then choose the File method
%    (.getName x)
%    :else (throw (Exception. "No match"))))
%\end{minted}
%
%An unambiguous match leads to the corresponding method being applied to the arguments,
%giving the final result.

%\subsection{Multimethods}
%
%A multimethod in Clojure is a function with a \emph{dispatch
%function} and a \emph{dispatch table} of methods. Multimethods are created with {\clj{defmulti}}.
%\inputminted[firstline=5,lastline=6]{clojure}{code/demo/src/demo/rep.clj}
%The multimethod \clj{path} has type \clj{[Any -> (U nil String)]}, an initially empty \emph{dispatch table}
%and \emph{dispatch function} \clj{class}, a function that
%returns the class of its argument or \clj{nil} if passed \clj{nil}.
%
%We can use {\clj{defmethod}} to install a method to \clj{path}.
%\inputminted[firstline=7,lastline=7]{clojure}{code/demo/src/demo/rep.clj}
%Now the dispatch table maps
%the \emph{dispatch value} \clj{String} to the function
%\clj{(fn [x] x)}. 
%We add another method
%which maps
%\clj{File} to the function
%\clj{(fn [x] (.getPath x))}
%in the dispatch table.
%\inputminted[firstline=8,lastline=8]{clojure}{code/demo/src/demo/rep.clj}
%
%After installing both methods, the call 
%$$
%\clj{(path (new File "dir/a"))}
%$$
%dispatches to the second method we installed because
%$$
%\clj{(isa? (class "dir/a") String)}
%$$
%is true, and finally returns 
%$$
%\clj{((fn [x] (.getPath x)) "dir/a")}.
%$$

%We include the above sequence of definitions as \egref{example:rep}.
%
%\begin{Code}
%\begin{exmp}
%\inputminted[firstline=5,lastline=10]{clojure}{code/demo/src/demo/rep.clj}
%\label{example:rep}
%\end{exmp}
%\end{Code}
%
%Typed Clojure does not predict if a runtime dispatch will be successful---\clj{(path :a)} 
%type checks because \clj{:a} agrees with the parameter type \clj{Any},
%but throws an error at runtime.

%\paragraph{Multiple dispatch} \clj{isa?} is special with vectors---vectors of the
%same length recursively call \clj{isa?} on the elements pairwise.
%\begin{minted}{clojure}
%  (isa? [Keyword Keyword] [Object Object]) ;=> true
%\end{minted}
%
%\inputminted[firstline=6,lastline=23]{clojure}{code/demo/src/demo/eg7.clj}
%
%\egref{example:multidispatch}
%simulates multiple dispatch by dispatching on
%a vector containing the class of both arguments. \clj{open}
%takes two arguments which can be strings or files and returns
%a new file that concatenates their paths.
%
%We call three different \clj{File} constructors, each known at compile-time
%via type hints.
%Multiple dispatch follows the same kind of reasoning as \egref{example:incmap},
%except we update multiple bindings simultaneously.

\subsection{Heterogeneous hash-maps}

%Beyond primitives and Java objects, 
The most common way to represent compound data in Clojure 
are immutable hash-maps, typicially with keyword keys.
%
Keywords double as functions that
look themselves up in a map, or return \clj{nil} if absent.
%
\begin{exmp}
\begin{lstlisting}
(def breakfast {*map :en "waffles" :fr "croissants" map*})
(*invoke :en breakfast invoke*)    ;=> "waffles" :- Str
(*invoke :bocce breakfast invoke*) ;=> nil       :- nil
\end{lstlisting}
\label{example:breakfastcomplete}
\end{exmp}

\emph{HMap types} describe the most common usages of
keyword-keyed maps.
\begin{lstlisting}
breakfast ; :- (HMap :mandatory {:en Str, :fr Str}, :complete? true)
\end{lstlisting}
This says
\clj{:en} and \clj{:fr} are known entries mapped to strings,
and the map is fully specified---that is, no other entries exist---by \clj{:complete?} being \clj{true}.

HMap types default to partial specification, with
\clj{'\{:en Str :fr Str\}} abbreviating \clj{(HMap :mandatory \{:en Str, :fr Str\})}.
%
\begin{exmp}
\begin{lstlisting}
(*typed ann lunch '{:en Str :fr Str} typed*)
(def lunch {*map :en "muffin" :fr "baguette" map*})
(*invoke :bocce lunch invoke*) ;=> nil :- Any ; less accurate type
\end{lstlisting}
\label{example:lunchpartial}
\end{exmp}
%(:en lunch)    ; :- Str
%;=> "muffin"
%(:fr lunch)    ; :- Str
%;=> "baguette"

\paragraph{HMaps in practice} The next example is extracted from a production system at CircleCI,
a company with a large production Typed Clojure system
(\secref{sec:casestudy} presents a case study and empirical
result from this code base).

\newpage

\begin{exmp}
\begin{lstlisting}
(*typed defalias RawKeyPair ; extra keys disallowed
  (HMap :mandatory {:pub RawKey, :priv RawKey}, 
        :complete? true) typed*)
(*typed defalias EncKeyPair ; extra keys disallowed
  (HMap :mandatory {:pub RawKey, :enc-priv EncKey}, :complete? true) typed*)

(*typed ann enc-keypair [RawKeyPair -> EncKeyPair] typed*)
(defn enc-keypair [kp]
  (*invoke assoc (*invoke dissoc kp :priv invoke*) :enc-priv (*invoke encrypt (*invoke :priv kp invoke*) invoke*) invoke*))
\end{lstlisting}
\label{example:circleci}
\end{exmp}
%
%(ann enc-keypair [RawKeyPair -> EncKeyPair])
%(defn enc-keypair [{pk :priv :as kp}] ; original map is kp
%  (assoc (dissoc kp :priv)       ; remove unencrypted private key
%         :enc-priv (encrypt pk))) ; add encrypted private key
%
%\inputminted[firstline=10,lastline=22]{clojure}{code/demo/src/demo/key.clj}
As \clj{EncKeyPair} is fully specified, we remove extra keys like \clj{:priv}
via \clj{dissoc}, which returns a new map that is the first argument without the
entry named by the second argument. Notice removing \clj{dissoc} causes a type error.
%
\begin{lstlisting}
(defn enc-keypair-bad [kp] ; Type error: :priv disallowed
  (*invoke assoc kp :enc-priv (*invoke encrypt (*invoke :priv kp invoke*) invoke*) invoke*))
\end{lstlisting}

%\clj{enc-keypair} takes an unencrypted keypair and returns an encrypted keypair by
%dissociating the raw \clj{:priv} entry with \clj{dissoc}
%and associating an encrypted private key
%as \clj{:enc-priv} on an immutable map with \clj{assoc}.
%The expression \clj{(:priv kp)} shows that keywords are also 
%functions that look themselves up in a map returning the associated value or \nil{} if the key is missing.
%Since \clj{EncKeyPair} is \clj{:complete?}, Typed Clojure enforces the return type
%does not contain an entry \clj{:priv}, and would complain if the \clj{dissoc}
%operation forgot to remove it.

%\egref{example:absentkeys}
%is like \egref{example:circleci}
%except the \clj{:absent-keys} HMap option is used
%instead of \clj{:complete?},
%which takes a \emph{set literal} of keywords that do not appear in the map, written 
%with \emph{\#}-prefixed braces.
%The syntax \clj{(fn [{pkey :priv, :as kp}] ...)}
%aliases \clj{kp} to the first argument and \clj{pkey} to \clj{(:priv m)}
%in the function body.
%
%\begin{exmp}
%\inputminted[firstline=10,lastline=21]{clojure}{code/demo/src/demo/key2.clj}
%\label{example:absentkeys}
%\end{exmp}
%
%Since this example enforces that \clj{:priv} must not appear
%in a \clj{EncKeyPair}
%Typed Clojure would still complain if we forgot to \clj{dissoc} \clj{:priv}
%from the return value.
%Now, however we could stash the raw private key in another entry
%like \clj{:secret-key} which is not mentioned by the partial HMap \clj{EncKeyPair}
%without Typed Clojure noticing.

%\paragraph{Branching on HMaps} Finally, testing on HMap properties
%allows us to refine its type down branches. \clj{dec-map} takes an
%\clj{Expr}, traverses to its nodes and decrements their values by \clj{dec}, then
%builds the \clj{Expr} back up with the decremented nodes.
%
%\begin{exmp}
%\inputminted[linenos,firstnumber=1,firstline=15,lastline=27]{clojure}{code/demo/src/demo/hmap.clj}
%\label{example:decmap}
%\end{exmp}
%
%If we go down the then branch (line 4), since \clj{(= (:op m) :if)} is true
%we remove
%the \clj{:do} and \clj{:const}
%Expr's from the type of \clj{m} (because their respective \clj{:op} entries disagrees with \clj{(= (:op m) :if)})
%and we are left with an \clj{:if} Expr.
%On line 8,
%we instead strike out the \clj{:if} Expr since it contradicts \clj{(= (:op m) :if)} being false. 
%Line 9 know we can
%remove the \clj{:const} Expr from the type of \clj{m} because it contradicts \clj{(= (:op m) :do)} being true,
%and we know \clj{m} is a \clj{:do} Expr.
%Line 12 we strike out \clj{:do} because \clj{(= (:op m) :do)} is false,
%so we are left with \clj{m} being a \clj{:const} Expr.
%
%Section~\ref{sec:formalpaths} discusses how this automatic reasoning is achieved.

\subsection{HMaps and multimethods, joined at the hip}

HMaps and multimethods are the primary ways for representing
and dispatching on data respectively, and so are intrinsically linked.
As type system designers, we must
search for a compositional approach that can anticipate
any combination of these features.

Thankfully, occurrence typing, originally designed for reasoning about
\clj{if} tests, provides the compositional approach we need.
By extending the system with
a handful of rules based on HMaps and other functions, 
we can automatically cover both easy cases and those
that compose rules in arbitrary ways.

Futhermore, this approach extends to multimethod dispatch by reusing
occurrence typing's approach to conditionals
%, whose branching
%mechanism may appear complex, but
%can be understood in terms of the humble \clj{if} conditional. 
and
encoding a small number of rules to handle
the \clj{isa?}-based dispatch.
In practice, conditional-based control flow typing
extends to multimethod dispatch, and vice-versa.

We first demonstrate a very common, simple dispatch style,
then move on to deeper structural dispatching where occurrence typing's
compositionality shines.

\paragraph{HMaps and unions} Partially specified HMap's with a common dispatch key
combine naturally with ad-hoc unions.
An \clj{Order} is one of three kinds of HMaps.

%FIXME define defalias above and keyword singletons

\begin{lstlisting}
(*typed defalias Order "A meal order, tracking dessert quantities."
  (U '{:Meal ':lunch, :desserts Int} '{:Meal ':dinner :desserts Int}
     '{:Meal ':combo :meal1 Order :meal2 Order}) typed*)
\end{lstlisting}

The \clj{:Meal} entry is common to each HMap, always mapped to a known keyword singleton
type.
It's natural to dispatch on the \clj{class} of an instance---it's similarly
natural to dispatch on a known entry like \clj{:Meal}.

\newpage

\begin{exmp}
\begin{lstlisting}
(*typed ann desserts [Order -> Int] typed*)
(defmulti desserts :Meal)  ; dispatch on :Meal entry
(defmethod desserts :lunch [o] (*invoke :desserts o invoke*))
(defmethod desserts :dinner [o] (*invoke :desserts o invoke*))
(defmethod desserts :combo [o] 
  (*invoke + (*invoke desserts (*invoke :meal1 o invoke*) invoke*) (*invoke desserts (*invoke :meal2 o invoke*) invoke*) invoke*))

(*invoke desserts {*map :Meal :combo, :meal1 {*map :Meal :lunch :desserts 1 map*}, 
           :meal2 {*map :Meal :dinner :desserts 2 map*} map*} invoke*) ;=> 3
\end{lstlisting}
\label{example:desserts-on-meal}
\end{exmp}
%
The \clj{:combo} method is verified to only structurally recur
on \clj{Order}s. This is achieved because we learn the argument \clj{o} must
be of type
\clj{'\{:Meal ':combo\}}
since
\clj{(isa? (:Meal o) :combo)}
is true. Combining this
with the fact that \clj{o} is an \clj{Order}
eliminates possibility of \clj{:lunch} and \clj{:dinner}
orders, simplifying \clj{o} to
\clj{'\{:Meal ':combo :meal1 Order :meal2 Order\}}
which contains appropriate arguments for both recursive calls.

\paragraph{Nested dispatch}
A more exotic dispatch mechanism for \clj{desserts}
might be on the \clj{class} of the \clj{:desserts} key.
If the result is a number, then we know the \clj{:desserts}
key is a number, otherwise the input is a \clj{:combo} meal.
We have already seen dispatch on \clj{class} and on keywords
in isolation---occurrence typing automatically understands
control flow that combines its simple building blocks.

The first method has dispatch value \clj{Long}, a subtype
of \clj{Int}, and the second method has \clj{nil}, the sentinel value for a failed map lookup.
In practice, \clj{:lunch} and \clj{:dinner} meals will dispatch to the \clj{Long}
method, but Typed Clojure infers a slightly more general type due to the definition
of \clj{:combo} meals.

\begin{exmp}
\begin{lstlisting}
(*typed ann desserts' [Order -> Int] typed*)
(defmulti desserts' 
  (fn [o :- Order] (*invoke class (*invoke :desserts o invoke*) invoke*)))
(defmethod desserts' Long [o] 
;o :- (U '{:Meal (U ':dinner ':lunch), :desserts Int}
;       '{:Meal ':combo, :desserts Int, :meal1 Order, :meal2 Order})
  (*invoke :desserts o invoke*))
(defmethod desserts' nil [o]
  ; o :- '{:Meal ':combo, :meal1 Order, :meal2 Order}
  (*invoke + (*invoke desserts' (*invoke :meal1 o invoke*) invoke*) (*invoke desserts' (*invoke :meal2 o invoke*) invoke*) invoke*))
\end{lstlisting}
\label{example:desserts-on-class}
\end{exmp}
%
%(desserts' {:Meal :combo 
%            :meal1 {:Meal :lunch :desserts 1}
%            :meal2 {:Meal :dinner :desserts 2}})
%;=> 3

In the \clj{Long} method, Typed Clojure learns that
its argument is at least of type \clj{'\{:desserts Long\}}---since
\clj{(*invoke isa? (*invoke class (*invoke :desserts o invoke*) invoke*) Long invoke*)}
must be true.
%
%Knowing \clj{o} is also an
%\clj{Order},
Here the
\clj{:desserts} entry
\emph{must} be present and mapped to a \clj{Long}---even in a \clj{:combo} meal,
which does not specify \clj{:desserts}
as present or absent.

In the \clj{nil} method,
\clj{(*invoke isa? (*invoke class (*invoke :desserts o invoke*) invoke*) nil invoke*)}
must be true---which implies \clj{(*invoke class (*invoke :desserts o invoke*) invoke*)} is \clj{nil}.
%
Since lookups on missing keys return \clj{nil}, either
\begin{itemize}
  \item \clj{o} maps the \clj{:desserts} entry to \clj{nil}, like the value \clj{\{:desserts nil\}}, or
  \item \clj{o} is missing a \clj{:desserts} entry.%, like \clj{{}}.
\end{itemize}
We can express this type with the \clj{:absent-keys} HMap option
%Equivalently, \clj{o} is of type
% Note: mintedinline doesn't work with hash characters #
\begin{lstlisting}
(U '{:desserts nil} (HMap :absent-keys #{:desserts}))
\end{lstlisting}
This eliminates non-\clj{:combo} meals
since their \clj{'\{:desserts Int\}} type does not agree
with this new information (because \clj{:desserts}
is neither \clj{nil} or absent).

%simplifies to a \clj{:combo} meal, 
%\begin{minted}{clojure}
%'{:Meal ':combo :meal1 Order :meal2 Order}
%\end{minted}
%thus allowing both recursive calls to type check.

\paragraph{From multiple to arbitrary dispatch}
Clojure multimethod dispatch, and Typed Clojure's handling of it, goes
even further, supporting dispatch on multiple arguments via vectors.
%
Dispatch on multiple arguments is beyond the scope of this paper,
but the same intuition applies---adding support for multiple dispatch
admits arbitrary combinations and nestings
of it and previous dispatch rules.

%\begin{exmp}
%\inputminted[firstline=6,lastline=13]{clojure}{code/demo/src/demo/hmap.clj}
%\label{example:decleaf}
%\end{exmp}
%
%The \clj{defn} macro defines a top-level function, with syntax like the typed \clj{fn}.
%The function \clj{an-exp} is verified to return an \clj{Expr}.
%
%Here \clj{defalias} defines \clj{Expr}, a type abbreviation
%that describes the structure of a recursively-defined AST as a union of HMaps.
%Keyword singleton types are quoted---\clj{':lunch}.
%A type that is a quoted map like \clj{'{:op ':if}} is a
%HMap type with a fixed number of keyword entries of the specified types
%known to be \emph{present},
%zero entries known to absolutely be \emph{absent},
%and an infinite number of \emph{unknown} entries entries.
%Since only keyword keys are allowed, they do not require quoting.

%\paragraph{HMap dispatch} The flexibility of \clj{isa?} is key to the generality of multimethods. 
%In \egref{example:incmap} we
%dispatch on the \clj{:op} key 
%of our HMap AST \clj{Expr}.
%Since keywords are functions that look themselves up in their argument, we simply
%use \clj{:op} as the dispatch function.
%
%\begin{exmp}
%\inputminted[firstline=5,lastline=18]{clojure}{code/demo/src/demo/eg5.clj}
%\label{example:incmap}
%\end{exmp}
%
%The function \clj{inc-leaf} is like \egref{example:decmap} except the nodes are incremented.
%The reasoning is similar, except we only consider one branch (the current method) by
%locally considering the current \emph{dispatch value} and reasoning about how it relates
%to the \emph{dispatch function}.
%For example, 
%in the \clj{:do} method we learn the \clj{:op} key is a \clj{:do}, which
%narrows our argument type to the \clj{:do} Expr, and similarly for the \clj{:if}
%and \clj{:const} methods.
%
%
%\subsection{Final example}
%
%\egref{example:final}
%combines everything we will cover for the rest of the paper:
%multimethod dispatch, reflection resolution via type hints, Java method
%and constructor calls, conditional and exceptional flow reasoning,
%and HMaps. 
%
%
%\begin{figure}
%\begin{exmp}
%\inputminted[firstline=6,lastline=23]{clojure}{code/demo/src/demo/eg7.clj}
%\label{example:multidispatch}
%\end{exmp}
%\begin{exmp}
%\inputminted[firstline=6,lastline=20]{clojure}{code/demo/src/demo/eg8.clj}
%\label{example:final}
%\end{exmp}
%\caption{Multimethod Examples}
%\end{figure}
%
%We dispatch on \clj{:p} to distinguish the two cases of \clj{FSM}---for example on \clj{:F}
%we know the \clj{:file} is a file.
%The body of the first method uses type hints to resolve reflection
%and conditional control flow to prove null-pointer exceptions are impossible.
%The second method is similar except it uses exceptional control flow.
