\chapter{Background}

% - Problem
%   - many common idioms cannot be checked
%   - limitations of local type inference
%   - made harder by occurrence typing
%   - want general solutions available to all users
%   - preliminary investigation of several techniques
% - Possible solutions
%   - symbolic analysis
%     - symbolic closures
%       - deal with "obvious" local function annotations
%     - inlining
%   - directed local type inference
%     - derive data flow from polymorphic types for more aggressive local type variable inference
%   - custom typing rules
%     - interface for describing how to check an unexpanded macro call
%       - or complex functions
%     - custom errors
% - Constraints
%   - some speculation of how well they compose together
%   - small models without rigorous proofs

\chapter{Symbolic Analysis}

% - Solution
%   - "obvious" function annotations
%     - can be derived from usage context
%   - introduce "symbolic" closure types
%     - a function's type is its code + typed local scope
% - Constraints
%   - wildcard "?" type
%     - needed to provide argument types while inferring body
%     - from Colored LTI
%   - Infinite loops
%     - subtyping
%     - type generalization
%     - term reduction limits
%   - user-level story
%     - symbolic closures enabled by flag
%     - users cannot write a symbolic closure
%     - that way, global annotations cannot contain
%       a symbolic closure
%       - helps with polymorphism story
%         - constraint solving
%           - hypothesis/goal: only one side of contraint solving can have a
%             symbolic closure
%             - one side is from global annotation, other side from local inference
%   - reporting errors
%     - suggesting types
%     - avoid showing inlining to users
%   - 0-n checks to same function
%     - avoid double expansion
%       - many copies of symbolic closures are made, could be expanded at different times
%         - how to synchronize?
%     - skipping unreachable functions
%     - performance
%   - consistent evaluation results
%     - how to ensure correct inlining?
%     - relationship between inlined and evaluated code?
%       - do we want to "undo" the inlining when finally evaluating?
%   - when to use a closure type?
%     - partial annotations
%   - polymorphism
%     - postpone discussion to next chapter
%   - applying symbolic analysis to infer loop/recur annotations
%     - similar issues
%     - different type generalization story?
%   - comparison to let-polymorphism
%     - expressiveness
%     - performance
%   - help check macros?
%     - not directly applicable, since too much context would be lost
%       - would help check *more* of an expansion, but error messages
%         are still unrelated to original code
%   - is this a sound strategy?
%     - faithfully simulates beta reduction
%   - case studies
%     - criteria:
%       - good errors?
%       - predictable behavior?
%       - performance?
%     - simple eta expansions
%       - (+ 1 2)
%       - ((fn [x y] (+ x y)) 1 2)
%     - let-bound functions
%       - (+ 1 2)
%       - (let [plus (fn [x y] (+ x y))]
%           (plus 1 2))
%     - y-combinator
%       - stress test
%     - let-polymorphism worst case (exponential) comparison
%       - stress test
%     - completely inlined transducers
%       - case study: inlining map + comp
%         - why: non recursive polymorphic functions
%           - common idiom
%         - how to report errors?

\chapter{Directed Local Type Inference}

% - Solution
%   - extend colored LTI with directed inference
%   - introduce constrained types
%   - derive data flow from (variances of) polymorphic variable occurrences 
%   - simple example
%     - (identity 1)
%     - demonstrate how this is checked with colored LTI
%     - compare to directed LTI:
%       - 2/----v
%         [x -> x]  Int
%          ^--------/1
%
%         1. Int flows to contravariant position
%         2. contravariant position flows to covariant position (because it's on the other side of ->)
%       - no loop, because variables not under different numbers of function types
%         - (we don't know the precise rule yet)
%   - complex example
% - Constraints
%   - advantages over colored LTI
%     - aids symbolic analysis
%       - because we derive potential dataflows, we don't need to over-approximate,
%         and thus trigger unneeded symbolic analysis
%         - which might then fail because of not enough contextual information
%   - disadvantages over colored LTI
%     - significant deviation from LTI
%       - constrained types
%       - aggressive local inference based on data flows
%     - not obvious how to prove soundness
%   - infinite loops
%     - how to manage cycles in inferred data flow 
%   - flow diagrams
%     - see symb.tex
%   - relationship to colored LTI model
%     - see symb.tex
%   - related work
%     - ML_sub
%     - see: symb.tex

\chapter{Custom Typing rules}

% - Solution
%   - allow users to provide custom typing rules
%   - 
% - Constraints
%   - wildcard type from colored LTI useful
%   - custom error messages
%     - propagation via expected types
%       - outer-most wins
%   - using clojure.spec to conform/unform
%     - to rip apart and put syntax back together
%   - differences with Turnstile
%     - in Turnstile, the macro *is* the rule
%       - here, we separate the two
%       - we preserve the macro call until evaluation
%       - use the typing rule to expand "under" the macro as many times as we want
%         - can do this 0-n times, thus compatible with directed LTI & symbolic analysis
