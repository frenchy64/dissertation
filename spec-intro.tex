%\subsection{Overview of clojure.spec}
%
%This thesis claims that we can automatically generate annotations to other verification
%systems similar to Typed Clojure.
%We add a second usecase to strengthen our claim: generating annotations for \texttt{clojure.spec},
%a runtime verification library recently added to Clojure's core library.
%It resembles common approaches to runtime verification, such as Racket's contract
%system, but is different in several important ways.
%
%Firstly, \texttt{clojure.spec} is designed to treat most values as ``data at rest''. That is,
%at verification sites, values are eagerly traversed without waiting to see
%if or how the program actually uses them.
%When we consider that \texttt{clojure.spec} treats infinite streams
%and functions as data at rest, we begin to see the tradeoffs that have been
%made.
%
%Secondly, specifications (called ``specs'') are not enforced by default. Users must
%opt-in to enforcing specs via an explicit instrumentation phase.
%This is also different than most contract systems, many of which are enforced
%by default. There is no standard way to integrate spec enforcement into a
%test suite, so it is difficult to tell whether specific specs are primarily 
%unchecked documentation, or actually used for runtime verification.
%
%Since \texttt{clojure.spec} has a unique feature set amongst runtime verification
%libraries, it is interesting to consider how programmers use \texttt{clojure.spec}
%in practice. For example, do programmers find the semantics of treating functions
%as data at rest useful?
%
%%Unfortunately since specs are opt-in, it is difficult to
%%correlate someone writing a spec with that person \emph{using} the spec,
%%implying spec's semantics as being useful.
%%Nevertheless, in the following sections we attempt to draw conclusions about
%%spec's common usage based mostly on the frequency of spec annotations.
%
%\subsection{Function specifications in clojure.spec}
%
%We now give a brief introduction to what using clojure.spec is like,
%focussing on the semantics of the different kinds of function specifications
%it supports.
%
%(From here, we map the namespace prefix \texttt{s} to \texttt{clojure.spec.alpha},
%and \texttt{stest} to \texttt{clojure.spec.test.alpha}.)
%
%\begin{verbatim}
%(require '[clojure.spec.alpha :as s])
%(require '[clojure.spec.test.alpha :as stest])
%\end{verbatim}
%
%There are two kinds of function checking semantics in \texttt{clojure.spec}.
%We use \texttt{intmap}, a higher-order function that maps a function over 
%a collection of ints, to demonstrate both semantics.
%
%\begin{verbatim}
%(defn intmap
%  "Maps a collection of ints over a function."
%  [f c]
%  (map f c))
%\end{verbatim}
%
%If the programmer wants to write a higher-order function spec to
%verify \texttt{intmap}, they might write the following spec.
%
%\begin{verbatim}
%(s/fdef intmap
%  :args (s/cat :f (s/fspec :args (cat :x int?) :ret int?)
%               :c (s/coll-of int?))
%  :ret (s/coll-of int?))
%\end{verbatim}
%
%The \texttt{s/fdef} form signals we are annotating a top-level
%function, in this case \texttt{intmap}. Argument specs are
%provided with the \texttt{:args} keyword option
%in the form of the ``tagged'' heterogeneous collection spec
%\texttt{s/cat}---here 2 arguments are allowed, tagged as
%\texttt{:f} for the function and \texttt{:c} as the collection.
%
%The \texttt{s/fspec} spec is another kind of function spec,
%specifically for non-top-level functions (such as function arguments
%to top-level functions). It has a similar syntax to \texttt{s/fdef},
%but a function name is not provided.
%
%In a nutshell, \texttt{s/fdef} provides traditional proxy-based
%verification semantics while \texttt{s/fspec} uses eager \emph{generative testing}
%to exercise a function before letting it pass the spec boundary, bare (without a proxy).
%
%We will now demonstrate how the following call gets checked.
%
%\begin{verbatim}
%(intmap inc [1 2 3])
%;=> (2 3 4)
%\end{verbatim}
%
%First, the programmer instruments \texttt{intmap} with:
%
%\begin{verbatim}
%(stest/instrument `intmap)
%\end{verbatim}
%
%This mutates the top-level binding associated with \texttt{intmap}, wrapping a function
%proxy around the original value.
%
%Now, when checking \texttt{(intmap inc [1 2 3])}, the \texttt{inc} function is
%called several hundred times with generated values conforming to \texttt{int?},
%and checks each call returns an \texttt{int?}.
%Then, \texttt{[1 2 3]} is eagerly checked against \texttt{(s/coll-of int?)}.
%The original \texttt{intmap} function is then called with the original arguments,
%yielding a value \texttt{(2 3 4)}. Instrumentation does not check return value specs,
%so \texttt{(s/coll-of int?)} is ignored, and the original return value is passed to the calling
%context.
%
%\subsection{Automatic Annotations for clojure.spec}
%
%Having introduced \texttt{clojure.spec}, we now give an overview
%of how to repurpose our automatic annotation technology to generate specs.
%At a glance, the problems of generating annotations for Typed Clojure
%and \texttt{clojure.spec} are similar, and indeed we can reuse much
%of the machinery from our approach to generating Typed Clojure annotations.
%The differences lie mostly in the finally type reconstruction phase.
%
%There are several important components in \texttt{clojure.spec}'s
%philosophy and design that complicate its annotation story.
%First, \texttt{clojure.spec} does not support local key-type pairs
%in its heterogeneous map specification--instead you are forced
%to globally define map entries as \emph{spec aliases}.
%
%For example, the Typed Clojure type \clj{'\{:a Int\}} must be
%written \clj{(s/keys :req-un [:my-ns/a])}, along with the following
%global definition of the spec alias \clj{:my-ns/a}.
%\begin{verbatim}
%(s/def :my-ns/a int?)
%\end{verbatim}
%When \clj{(s/keys :req-un [:my-ns/a])} is checked against a value,
%it first locates the spec under \clj{:my-ns/a}, and (because
%we have declared the key to be \clj{:req-un}, that is, required
%but unqualified), we use the located \clj{int?} spec
%to check against the \clj{:a} entry of the value.
%Furthermore, if the key was declared as \clj{:req} instead of \clj{:req-un},
%the fully qualified keyword entry \clj{:my-ns/a} would be checked
%against \clj{int?}.
%This means that there is \emph{exactly one} spec for each fully
%qualified keyword entry.
%
%This raises several interesting design issues related to reusing
%specs that are very different from the simpler problem of generating
%Typed Clojure \clj{HMap} types. 
%How do we decide which namespace to register unqualified map entries?
%Should we assume similar looking keyword entries are in fact the same, and register their specs under the same namespace?
%How do we handle fully qualified keyword entries?
%
%The second component of
