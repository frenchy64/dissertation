\Dchapter{Related work\either{ to Automatic Annotations}{}}

%The field of dynamic analysis has a rich history.
%Ball~\cite{ball1999concept}
%introduces frequency spectrum analysis,
%an approach that observes a running program
%that is similar to our instrumentation approach.
%Mock~\cite{mock2003dynamic}
%makes the case for efficient profiling of programs
%to better facilitate usage of instrumentation.
%
%Value Profiling is another related area which characterises
%programs based on their running entities.
%
%Daikon~\cite{ernst2001dynamically}
%uses dynamic analysis to recover likely program invariants
%in C programs.
%
%Dynamic type inference has been attempted in many different
%areas.
%Rubydust~\cite{An10dynamicinference}
%infers static types for Ruby. They prove the generated types
%sound, which we do not. 
%\begin{verbatim}
%Conversely, they do not generate
%recursive types, but recursive types in ruby are probably
%nominal, so how different are we?
%\end{verbatim}
%
%In the context of JavaScript, several usages of this technique
%can be found.
%JSTrace~\cite{saftoiu2010jstrace}
%generates types for (?).
%Separately, work has been done to generate JSDoc-like annotations~\cite{odgaard2014}.
%TypeDevil~\cite{pradel2015typedevil}
%uses dynamic analysis to warn JavaScript programmers of possible inconsistencies
%in their programs.
%
%Work in recovering context-free grammars is most related to our algorithm
%to recover recursive types.
%% TODO Shamir~\cite{shamir1962remark} notes that it is impossible
%% TODO \cite{knobe1976method}
%
%In the context of machine learning, 
%this area is called grammar induction or language learning.  % according to vcrepinvsek2005inferring
%% TODO Wang\cite{wang1998grammar} summarises 
%{\v{C}}repin{\v{s}}ek et. al~\cite{vcrepinvsek2005inferring}
%use genetic programming to infer context-free grammars
%for domain-specific languages.
%Most work in this area assume both positive and negative
%examples. We cannot distinguish between these two in
%our system, so we assume all examples are positive.
%
%Chen~\cite{chen1995bayesian} uses Bayesian inference to converge
%on a suitable grammar, given examples.
%
%There has been recent interest in approximate type inference.
%
%Pluquet et. al~\cite{marot2009fast} investigate heuristics
%to quickly infer types in dynamic programs.
%So does Milojkovi{\'c}
%\cite{milojkovic2016exploring}.
%Spasojevi{\'c} et. al~\cite{spasojevic2014mining}
%compare types across a cross section of projects to improve
%inference.
%
%Adamsen et. al~\cite{adamsen2016analyzing} verify test suite completeness using a hybrid approach of lightweight dependency analysis, static type checking and dynamic instrumentation.
%
%% Inference and Evolution of TypeScript Declaration Files
%% - they submit pull requests from their tool's output
%% https://cs.au.dk/~amoeller/papers/tstools/paper.pdf
%
%% Automatic TS annotations from JSON (including recursive types)
%% https://github.com/shakyShane/json-ts

\paragraph{Automatic annotations}
There are two common implementation strategies for automatic annotation tools. The first
strategy, ``ruling-out'' (for invariant detection), assumes all invariants are true 
and then use runtime analysis results to rule out
impossible invariants. The second ``building-up'' strategy (for dynamic type inference)
assumes nothing and uses runtime analysis results to build up invariant/type knowledge.

Examples of invariant detection tools include Daikon~\infercitep{ernst2001dynamically},
DIDUCE~\infercitep{hangal2002tracking}, and Carrot~\infercitep{pytlik2003automated}, and
typically enhance statically typed languages with more expressive types or contracts.
Examples of dynamic type inference include our tool, Rubydust \infercitep{An10dynamicinference},
JSTrace~\infercitep{saftoiu2010jstrace}, and TypeDevil~\infercitep{pradel2015typedevil},
and typically target untyped languages.

Both strategies have different space behavior with respect to representing
the set of known invariants.
The ruling-out strategy typically uses a lot of memory at the beginning,
but then can free memory as it rules out invariants. For example, if
\texttt{odd(x)} and \texttt{even(x)} are assumed, observing \texttt{x = 1}
means we can delete and free the memory recording \texttt{even(x)}.
Alternatively, the building-up strategy uses the least memory storing
known invariants/types at the beginning, but increases memory usage
as more the more samples are collected. For example, if we know
\texttt{x : Bottom}, and we observe \texttt{x = "a"} and \texttt{x = 1}
at different points in the program, we must use more memory to
store the union \texttt{x : String $\cup$ Integer} in our set of known invariants.

\paragraph{Daikon}
Daikon can reason about very expressive relationships between variables
using properties like ordering ($x < y$), linear relationships ($y = ax + b$),
and containment ($x \in y$). It also supports reasoning with ``derived variables''
like fields ($x.f$), and array accesses ($a[i]$).
%
Typed Clojure's dynamic inference can record heterogeneous data structures
like vectors and hash-maps, but otherwise cannot express relationships
between variables.

There are several reasons for this. The most prominent is that Daikon
primarily targets Java-like languages, so inferring simple type information
would be redundant with the explicit typing disciplines of these languages.
On the other hand, the process of moving from Clojure to Typed Clojure
mostly involves writing simple type signatures without dependencies
between variables. Typed Clojure recovers relevant dependent information
via occurrence typing~\infercitep{TF10}, and gives the option to manually annotate necessary
dependencies in function signatures when needed.


% Inference and Evolution of TypeScript Declaration Files
% - they submit pull requests from their tool's output
% https://cs.au.dk/~amoeller/papers/tstools/paper.pdf
\paragraph{Other Annotation Tools}
Static analyzers
for JavaScript
(TSInfer~\infercitep{kristensen2017inference}) and for Python (Typpete~\infercitep{typette18}
and Pytype~\infercitep{pytype})
automatically annotate code with types.
Pytype and Typpete inferred \texttt{nodes} in \figref{fig:infer:nodes}
as \texttt{(? -> int)}
and \texttt{Dict[(Sequence, object)] -> int}, respectively---our tool 
infers it as \clj{[Op -> Int]} by also generating a compact recursive
type (\figref{fig:infer:nodestype}).
Similarly, a class-based translation of \figref{fig:infer:nodes}
inferred both \texttt{left} and \texttt{right}
fields
as \texttt{Any} by Pytype, and as \texttt{Leaf} by Typpete---our tool
uses \clj{Op},
a compact recursive type containing \emph{both} \clj{Leaf} and \clj{Node}.
This is similar to our experience with TypeWiz in \Dchapref{infer:chapter:intro}.
(We were unable to install TSInfer.)

NoRegrets~\infercitep{noregrets2018} uses dynamic analysis to learn how a program
is used, and automatically runs the tests of downstream projects to
improve test coverage.
Their \emph{dynamic access paths} represented as
a series of \emph{actions} are analogous to our paths of path elements.

% distinguishes public/private API

% Python
% - MaxSMT-Based Type Inference for Python 3
%  - cites other python based projects
%  - https://link.springer.com/content/pdf/10.1007%2F978-3-319-96142-2_2.pdf
% - pytype
%  - static analysis to generate python annotations
%  - https://github.com/google/pytype
% - pyannotate
%   - dynamic analysis
%   - https://github.com/dropbox/pyannotate

% A Survey of Dynamic Analysis and Test Generation for JavaScript
%  - http://mp.binaervarianz.de/js_survey_2017.pdf
